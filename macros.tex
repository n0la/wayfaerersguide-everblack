\newcommand\litem[1]{\item{\bfseries#1.\space}}

\definecolor{light-grey}{gray}{0.85}

\newcommand*\cleartoleftpage{%
  \clearpage
  \ifodd\value{page}\hbox{}\newpage\fi
}

% Notes for DM's
%
\newenvironment{note}{
  \begin{mdframed}[roundcorner = 5pt,
                   skipabove=0.3cm,
                   leftmargin=0.2cm,
                   rightmargin=0.2cm]
    \begin{centering}
      \textbf{\large{Notes for Dungeon Masters}}
    \end{centering}
    \par
}{
  \end{mdframed}
  \par
}

% That one song quote
%
\newcommand\songquote[2]{
  \begin{center}
    ``#2'' \\ - \emph{#1}
  \end{center}
}

% Environment for 3.5e rules
%
\newenvironment{35e}[1]{%
  \begin{mdframed}[roundcorner = 5pt,
                   skipabove=0.3cm,
                   leftmargin=0.2cm,
                   rightmargin=0.2cm]
    \begin{centering}
      \textbf{\large{#1 (3.5)}}
    \end{centering}
    \par
}{
  \end{mdframed}
  \par
}

\newcommand{\featref}[1]{%
  \hyperref[feat:#1]{#1}
}

\newenvironment{ebtable}[1]{%
  \begin{table*}[!htb]
    \captionsetup{labelformat=empty,font={large,bf},position=top}
    \caption{#1}
    \rowcolors{1}{white}{light-grey}
}{
  \end{table*}
}

%% The 3.5e materials have a style where they start with
%% with a bold face header text, following by a colon and
%% then the description text. \srditem{header}{description}
%% simulates exactly that.
%%
\newcommand{\srditem}[2]{%
  \par \noindent \textbf{#1}: #2
}

\newcommand{\srdbackgroundfeat}[0]{
  \srditem{Background Feat}{This feat is a background feat and reflects the
    environment, culture, and society you grew up in. It can only be taken at
    level 1, and does not count towards your maximum allowed feats of level
    1. But only one background feat may be taken at level 1, unless racial
    bonuses allow for additional background feats.}
}

%% Begin a new feat block. Multiple \srditem should be used to recreate
%% the look of the original 3.5e books
%%
\newenvironment{35efeat}[1]{
  \vspace*{0.3cm}
  \begin{centering}
    \textbf{\large{#1}}\label{feat:#1}
  \end{centering}
}{
  \par
}
