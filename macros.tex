\newenvironment{35e}[1]{%
  \begin{mdframed}[roundcorner = 5pt,
                   skipabove=0.3cm,
                   leftmargin=0.2cm,
                   rightmargin=0.2cm]
    \begin{centering}
      \textbf{\large{#1 (3.5)}}
    \end{centering}
    \par
}{
  \end{mdframed}
  \par
}

\newcommand{\featref}[1]{%
  \hyperref[feat:#1]{#1}
}

%% The 3.5e materials have a style where they start with
%% with a bold face header text, following by a colon and
%% then the description text. \srditem{header}{description}
%% simulates exactly that.
%%
\newcommand{\srditem}[2]{%
  \par \hspace*{0.2cm} \textbf{#1}: #2
}

\newcommand{\srdheritagefeat}[0]{
  \srditem{Heritage Feat}{This feat is a heritage feat and reflects the
    environment, culture, and society you grew up in. It can only be taken at
    level 1, and does not count towards your maximum allowed feats of level
    1. But only one heritage feat may be taken at level 1, unless racial
    bonuses allow for additional heritage feats.}
}

\newcommand{\srdeducationfeat}[0]{
  \srditem{Education Feat}{This feat is a \emph{education feat} and represents
    what you learned during your childhood. It can only be taken at level 1,
    and this feat does not count towards your maximum allowed feats of level
    1. But only one education feat may be taken at level 1, unless racial
    bonuses allow for additional education feats.}
}

%% Begin a new feat block. Multiple \srditem should be used to recreate
%% the look of the original 3.5e books
%%
\newenvironment{35efeat}[1]{
  \vspace*{0.3cm}
  \begin{centering}
    \textbf{\large{#1}}\label{feat:#1}
  \end{centering}
}{
  \par
}
