\subsection{Ilians}
\label{sec:Ilians}

The \emph{Ilians} are a monstrous, humanoid and giant race that live in vast
and great cities deep underground. They are masters psionics, and live in a
strictly ordered society.

\subsubsection{Physiology}

The appearance of Ilians varied across the castes, but they all had a few
features in common. Their skin was dark grey or light blue, were often above
two metres tall, and had no bodily hair. All Ilians have at least some
inherent psionic capabilities.

Ilians - specifically female Elmek - bore live young. Aquatic snake like
creatures without limbs or head, which represent the basic, original and
unaltered form all Ilians stem from. These young were then fed variations of
Ramesk, which force the young to grow into the various castes of Ilian
society. Ilian spawns that were fed Ramesk turned into Elnak, Karak or Elmek
depending on the specific recipe used. This allowed Ilian societies to
engineer exact populations and demographics of each caste. Spawns that were
fed regular food (such as meat, vegetables) would turn into Arnaks. Arnak's
however were also capable of producing their own spawns, that could only
mature into other Arnaks. In many Ilian societies the dead were fed to a new
generation of spawns, allowing them to contribute one last time to the
society. Larger Ilian societies had growth chambers, large glass tanks filled
with Ramesk.

\subsubsection{History}

Not much is known about their early history, except that Ilians already dwelt
in their great underground cities when the dwarves, elves and humans went
underground during the \nameref{sec:Schism}. The Ilians did not take kindly to
what they perceived as intruders upon their land, and attempted to drive the
humanoid races back to the surface.

Although the Ilians had a massive advantage, as their cities and war
infrastructure was already built, their slow reproductive cycles and strict
and unmoving societal structure ultimately caused them massive problems in
the war against the humanoid races.

During the strife the dark elves, dwarves and deepkin fought the Ilians
relentlessly in an attempt to establish themselves in the deep. And although
early centuries of the war often ended in favour for the Ilians, the tide
turned against them when the deepkin began building \hyperref[sec:Everblack
  Golem]{everblack golems} that could withstand the psionic powers of the
Ilians. These constant battles and conflicts cost both sides heavily, and
ultimately culminated in the downfall of the Ilian culture and society across
most of Aror. The victory was devastating for the humanoids as well, and it
lead to a \hyperref[sec:Exodus from the Depths]{great exodus from the depths}
for many deep dwelling humanoids.

Nowadays only ruins remain where once proud Ilian cities stood. Burying their
psionic machinery, artefacts and riches beneath them. Some communities and
cities have survived and guard their treasures still, while some retreated
deeper underground into solitude and isolation.

\subsubsection{Society}

The Ilian society was strictly ordered into castes, where the members of each
caste had distinct biological traits suited for the work they were required to
perform. There were four main casts: \emph{Arnak}, \emph{Elnak}, \emph{Karek},
and \emph{Elmek}.

Ilian also have their own language, called \emph{Ilian} which had no writing
system. Ilians inscribed their thoughts, feelings or knowledge as psionic
energy into \hyperref[sec:Everblack]{everblack} crystals, which could then be
accessed by any psionic race. The language ``Ilian'' was used in combination
with telepathy to allow individual Ilians to talk to one another. Most of the
time the psionic powers were used to command, or even force, lower caste
members, thus forming a strict hierarchical society that did not allow
dissenting individuals.

They are one of the great societal architects, city builders and engineers of
Aror. Often their machinery, artefacts and contraptions powered by psionic
magic work to this day, even though many of their cities have fallen to ruin.

\subsubsection{Ramesk}
\label{sec:Ramesk}

Ramesk is a blueish, thick fluid that the Ilians drank as nourishment. It has
almost no smell, but tastes salty, due to the high mineral content of the
liquid. It was made out of various roots, herbs and mushrooms that grew in the
depths, and was the sole food provided to Arnaks, the Ilian worker caste.
Although the Ilians still had humanoid mouths and could eat and digest other
food, Ramesk turned into the main source of nourishment when food became scarce
during the aeon of strife.

\subsubsection{Arnak}
\label{sec:Arnak}

Arnaks are the labourer and thus lowest caste of Ilian society. They stand
roughly two and a half metres tall, their bodies are extremely muscular, and
they have sharp claws and teeth they used for hunting and defending the Ilian
cities and outlying farms. They are blind, and rely on their fine sense of
smell and their ability to sense small tremors to move about in the dark caves.

The least intelligent of all Ilians, and were only used as front line troops,
for carving new tunnels, constructing new buildings or tending to underground
farms. Arnaks were often overseen by a few \emph{Karek}, who could force them
into hibernation if their labour was not needed, or food had to be rationed or
preserved.

The most numerous of all the Ilians they can still be found underground today,
often leaderless and organised into small groups. Their fierce strength and
uncanny senses make them expert hunters of anything that ventures or lives down
below.

The fall of major Ilian civilisations and cities have left them cut off from
their supply of Ramesk, and have thus resorted to drinking humanoid blood,
which is similar in composition to Ramesk. If no suitable sources of nourishment
are available, Arnaks hibernate beneath stalactites. The mineral rich water
that drops from the stones nourishes them enough while they hibernate.

Arnaks are now a major scourge of anyone who ventures underground, and many
cultures use Arnaks as the ``boogey man'' to frighten and scare children away
from caves.

\subsubsection{Elnak}
\label{sec:Elnak}

Elnak are the tinkerer and crafter caste of the Ilians. Highly intelligent,
although shorter than any other Ilian. They stood roughly two metres tall, and
had extra-ordinarily sharp eyesight and dexterity to allow them to work on
intricate psionic machinery. They were leaner than other Ilians, but made up
for their lack of strength with advanced psionic powers and abilities. They
were responsible for forging all psionic artefacts and machinery used by the
Ilians.

\subsubsection{Karek}
\label{sec:Karek}

Karek are the elite warrior and fighter caste of the Ilians. They stood between
two and a half, to three metres tall, and had four arms. Highly intelligent
tacticians, powerful psionics, and above all else fierce fighters. There were
but a few Karek per city or community, as they used Arnaks as the main fighting
force. Karek were often high ranking military leaders, akin to captains or
generals.

While the Arnak were mostly the front line troops of the Ilians, the Karek were
the specialised fighters, tacticians and generals. Some Karek also lead small
communities, often by proxy for an Elmek.

\subsubsection{Elmek}
\label{sec:Elmek}

Above all others stood the elmek, the ruling caste of the Ilians. Highly
intelligent, and powerful psionics. They lead their cities administratively as
well as spiritually. Elmek were also active in psionic research, and were
brilliant engineers and responsible for leading and building entire Ilian
communities and societies. They towered over three metres and a half tall,
so that they could easily be identified as the rulers by all other Ilians.

Elmek used their psionic powers to levitate and wield a ceremonial
quarterstaff that is a representation of their power within society. Elmek are
incredibly powerful psykers, capable of reshaping reality if it so suits
them. Female Elmek are the only ones capable of giving birth to an higher
Ilian spawn. Higher spawns, as compared to the Arnak spawns, were the stock
out of other Elnak, Karek or Elmek were bred.
