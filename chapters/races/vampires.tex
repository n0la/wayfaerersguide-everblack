\subsection{Vampires}
\label{sec:Vampires}

Vampires are undead that were once humanoid creatures such as elves and
humans, that now mostly coexist peacefully with their mortal brethren.

There are two main types of vampires: those that drink \nameref{sec:Ramesk} or
monstrous blood and thus retain their mortal knowledge and character (called
``civilised vampires''), and those that have fallen for humanoid blood and
have turned feral (``feral vampires''). Both types of vampires retain their
physical appearance from when they were mortal (such as skin colour or body
shape and proportions), but gain retractable, elongated fangs as well as
retractable hardened finger nails that can act as claws.

A feral vampire's iris turn white (blinding them), and looses all abilities to
grow hair, or repair brittle or wounded skin, giving them a horrible, feral
and animal-like appearance compared to civilised vampires. Those vampires that
consume Ramesk regain their ability to see (their iris' are red), grow blue
hair and blue nails (which take the pigment from the blue Ramesk). But most
importantly the Ramesk allows them to suppress the hunger for blood, freeing
their minds and souls to follow other thoughts and feelings, without having an
obsessive and oppressive urge to seek out their next meal. Feral vampires' skin
is damaged by direct exposure to the suns, and they will avoid it if possible.
Although the skin of a civilised vampire is also damaged by the suns, the
effect is less severe, and many civilised vampires walk around during the day
by covering their skin with clothing.

Civilised vampires can fall feral if they begin feasting on humanoid blood,
and likewise feral vampires can return to a civilised behaviour with the help
of monstrous blood or Ramesk. This process is not instantaneous, but takes
place over the course of weeks. So even a civilised vampire may drink humanoid
blood once or twice without experiencing immediate deterioration.

Vampires are under direct threat by many militant religious organisations such
as the knights of \nameref{sec:Lor} or the \nameref{sec:Nightwatch}, but have
their own part of a kingdom in \nameref{sec:Helmarnock} where they might live
in peace together with other humanoids and vampires.

\subsubsection{History}

During the \nameref{sec:Aeon of Strife} many humanoid clans prayed to what
they believed was the greater deity \nameref{sec:Morana}. She in turn revealed
to her followers a path to great power to vanquish their foes, and many of her
followers accepted it. They were betrayed by their goddess and turned into
vampires.

These followers were turned undead, enhanced with special abilities, and were
given an insatiable hunger for monstrous blood. The vampires' hunger for
monstrous blood grew with every taste of the crimson liquid, perpetuating a
cycle of hunting, killing and feasting to satiate their thirst. Most humanoid
tribes that had people turned into vampires began locking themselves away at
night while allowing the vampires to roam the surrounding forests and area to
feast on those monstrous creatures unfortunate enough to remain outdoors
during the night. While in some areas the vampires spread terror and death
among the monstrous creatures that would attack the humanoid settlements, in
others, the lack of alternate food sources forced the vampires to turn on
those they were supposed to protect. These vampires then learned of yet
another betrayal by their mother: Vampires that feasted on the blood of the
core humanoid races (human, elf, dwarven, halfling and half-breeds) blood
would turn feral, and lose the last remaining modicum of control they had left
over their inner hunger. This lead them to be ostracised, hunted, and even
killed by their own people, the humanoids they were supposed to protect from
harm.

Many of these old vampires turned from their ``mother'' and were then
threatened with death by Morana in turn. This direct threat to her children,
and her outburst of anger revealed Morana to be a lesser deity. Not
intimidated by the threat, many vampires turned away from her and were then
killed by Morana for their insolence. Those that survived did so by turning
toward lesser deities, such as the \nameref{sec:Silent Queen}, who protected
them from Morana's wrath. This incident, known as the \emph{great betrayal},
has cemented Morana's downfall as a grand deity, and she fell in favour not
only amongst vampires, but also among the mortals.

The surviving vampires, often without a way to turn back to a mortal form,
then continued what they originally set out to do: aid the humanoids in
fighting the monstrous races, while also fighting their inner thirst for
control over their own lives. Humanoid settlements often protected their
vampires during the day, while the vampires fought and feasted upon the
monstrous races, such as the bugbears, goblins, gnolls and hobgoblins, during
the night. Although still a threat to the very humanoids these vampires
attempted to protect, many earned themselves respect, friendship and often a
noble title and land for their efforts during the \nameref{sec:Aeon of
  Strife}.

After the Aeon of Strife ended, most vampires held on to their titles, lands,
and riches, trying to live peacefully together with the mortal humanoid races.
They often feasted on animals, or only drank blood sparingly often willingly
donated by their subjects. Some vampires however instead turned on the
humanoids, feasting on their blood and thus became hated, feared and hunted as
they turned into feral vampires.

\subsubsection{Ramesk and Vampires}

In \emph{MI:310} the recipe for \nameref{sec:Ramesk}, a mineral rich blueish
liquid, was discovered by a \nameref{sec:Deepkin} expedition in the depths of
the \nameref{sec:Cnamh Mountains}. It took a few years of experimentation to
adapt the liquid to work as nourishment for vampires, and there were many
troubles in growing the ingredients on farmland on the surface. But now the
liquid is widely used by the vampires as a substitute to monstrous blood. Only
a cup full of Ramesk is required for a vampire to be fed for a few days.

Ramesk has a few other benefits for vampires. It makes their nails and hair
grow again, albeit in the same shade of light blue of the liquid. Although
nail colours and hair dyes exist, blue hair has become an identification mark
for vampires to show to the mortal races that they are civilised and pose no
threat. Ramesk also restores the vampire's eyesight, giving them red
irises. It further reduces their primal longing for humanoid blood, giving
them the ability to explore other feelings such as love, anger or grief which
would otherwise be suppressed by their hunger. Another important aspect of
Ramesk is that vampires become less sensitive to sunlight. Vampires that drink
Ramesk can endure the daylight, as long as they wear clothing that shades them
from direct exposure to the suns.

Ramesk also has social benefits, as several inns and taverns make normal meals
containing Ramesk allowing vampires to dine alongside their mortal friends.
This shows both vary strangers that the vampire actively consumes Ramesk, and
allows them to socialise with friends or business associates. Ramesk is served
by most major inns and taverns, as it has a long shelf life, and requires
little storage space.

\subsubsection{Born Vampires}
\label{sec:Born Vampires}

The vampire nobility of \nameref{sec:House deVar} heavily researched the magic
that turned the early humans and elves into vampires. In \emph{MI:755}, they
restored the ability for female vampires to birth children on their own, thus
furthering the vampire race without having to turn other mortals to become
undead. These new vampires are often called \emph{born vampires}, and are less
powerful that the original vampires. They retain the physical appearance of
the mortal races of their parents. Born vampires reach maturity within a few
years, but often go out into the world to seek adventure, learn skills and
even professions like mortals do.

\subsubsection{Feral Vampires}
\label{sec:Feral Vampires}

Those vampires that actively hunt, feed and kill humanoid races such as
humans or elves, on a regular basis are called \emph{feral vampires}. The
taste of humanoid blood makes them savage, primal and often primitive hunters
and killers. They are far and few, but often cause great harm to towns, cities
and kingdoms. Feral vampires are outcasts among the humanoids, monstrous races
and even their civilised brethren, who see them as a threat to the fragile
peace between the mortal and undead races. Feral vampires often return to a
worship of their mother \nameref{sec:Morana}. Much like the original vampires,
feral vampires, are blind and hunt by scent only.

Although feral vampires could be turned from their savage ways by feeding them
Ramesk while keeping them away from blood, very few undead hunters or
civilised vampires bother to do so. Feral vampires are often hunted and
destroyed without prejudice.

\begin{35e}{Feral Vampire}
  A vampire that drinks blood of the core humanoid races (elves, dwarves,
  humans and halflings) or their half-blood descendants, must make a will save
  DC 10 + victims HD, or lose 1 intelligence to ability drain. If the vampire
  reaches 2 intelligence (and thus becomes an animal), he or she turns into a
  feral vampire. \emph{Restoration} can cure this ability drain, so can Ramesk
  or monstrous blood which can heal 1 ability damage per day.
\end{35e}

\begin{35e}{Vampire}
  Vampires speak whatever language their given by their region or heritage,
  and are not evil by default. Feral vampires have been driven feral are
  considered animals (as per game rule) and are thus ``neutral''. All
  vampires, regardless of type, gain the \emph{Scent} ability. Furthermore
  feral vampires are blind.

  Vampires are not afraid of holy symbols, garlic, mirrors, but still take
  penalties from direct exposure to sunlight. Those that drink Ramesk can
  venture into the broad daylight, given that they wear clothing that shields
  their skin from direct exposure. Vampires may also cross running water
  without problems, and may enter any house without being invited. There is
  also need for them to sleep in coffins, and a wooden spike to the heart does
  not kill them, as reducing them to zero hit points kills them normally.

  Vampires further lose the following abilities: \emph{Children of the Night},
  \emph{Dominate}, \emph{Gaseous Form}, and \emph{Alternate Form}. But gain
  retractable claws that they can deploy as a free action and with which they
  are proficient. These claws do one melee damage dice step higher, than the
  original slams. Further more the level adjustment is reduced to +3.

  Normal vampires thus gain the \emph{Scent (Ex)} ability, as well as the
  \emph{Feral Blindness (Ex)} ability.

  \textbf{Born Vampires}:\\
  \begin{itemize}[noitemsep]
    \item Medium: Born vampires are medium creatures, and thus have no special
      bonuses or penalties due to their size.
    \item Type: Undead. Born vampires gain all undead traits, and thus have no
      constitution score.
    \item Feral Blindness (Ex): A vampire turns blind if he or she turns feral.
    \item Blood Drain (Ex): A born vampire can suck blood from a living victim
      with its fangs by making a successful grapple check. If it pins the foe,
      it drains blood, dealing 1d4 points of Constitution drain each round the
      pin is maintained. On each such successful attack, the vampire gains 5
      temporary hit points.
    \item Energy Drain (Su): Living creatures hit by a born vampire's slam
      attack (or any other natural weapon the vampire might possess) gain one
      negative levels. For each negative level bestowed, the born vampire
      gains 5 temporary hit points. A vampire can use its energy drain ability
      once per round.
    \item Scent (Ex): Scent lets a creature detect approaching enemies, sniff
      out hidden foes, and track by sense of smell.
    \item Turn Resistance (Ex): A born vampire has +4 turn resistance.
    \item Level Adjustment: +2
  \end{itemize}
\end{35e}
