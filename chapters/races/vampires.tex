\ifimages
\clearpage
\incgraph[
  overlay={\node[black] at ([xshift=0cm,yshift=+1.2cm] page.south)
    (main)[text width=0.9\paperwidth]{
      \large \centering
      \textbf{``Priest of \nameref{sec:Ishtar} curing a feral. The wounds
        trigger the feral's ability regenerate, which in turn makes it
        hungry. The more \nameref{sec:Ramesk} it drinks, the faster the feral
        state is cured. Cruel, but effective.''} -- Aren Fel
    };
  }
]{media/vampires-feral.\imagesuffix}
\clearpage
\fi

\subsection{Vampires}
\label{sec:Vampires}

Vampires (or ``strigoi'') are undead that were once humanoid creatures such as
elves and humans that were turned to a life of undeath. They coexist
peacefully with their mortal brethren, and are a staple of civilised society
in many regions and city kingdoms.

There are two main types of vampires: those that drink \nameref{sec:Ramesk} or
monstrous blood and thus retain their mortal knowledge and character (called
``civilised vampires'', or ``strigoi''), and those that have fallen for
humanoid blood, and have turned feral (``feral vampires'', ``ferals''). Both
types of vampires retain their physical appearance from when they were mortal
(such as skin colour or body shape and proportions), but gain retractable,
elongated fangs as well as retractable hardened finger nails that can act as
claws.

A feral vampire's iris turns white (blinding them), and they lose all hair and
the ability to grow new hair, and the ability heal brittle or wounded skin,
giving them a horrible, feral and animal-like appearance compared to civilised
vampires. Those vampires that consume Ramesk regain their ability to see
(their iris' turn red), and grow blue hair and nails (which take the pigment
from the blue Ramesk). But most importantly the Ramesk allows them to suppress
the hunger for blood, freeing their minds and souls to follow other thoughts
and feelings. The skin of a feral vampires is easily damaged by direct
exposure to the suns, and they will avoid it if possible. Although the skin of
a civilised vampire is also damaged by the suns, the effect is less severe,
and many civilised vampires walk around during the day by covering their skin
with clothing.

Civilised vampires can fall feral if they begin feasting on humanoid blood,
and feral vampires can return to a civilised behaviour with the help of
monstrous blood or Ramesk. This process is not instantaneous, but takes place
over the course of weeks, or even months. So even a civilised vampire may
drink humanoid blood once or twice without experiencing immediate
deterioration.

Vampires are under direct threat by many militant religious organisations such
as the knights of \nameref{sec:Lor}, or the \nameref{sec:Nightwatch}, but are
usually well integrated into civilised nations. A line of vampire nobility
rule a part of the kingdom of \nameref{sec:Helmarnock}. Vampires, like any of
the core races, are free to chose their own lives, and become knights, smiths,
artisans, artists, adoptive mothers and fathers, as well as spiritual and
political leaders.

\subsubsection{History}

During the \nameref{sec:Strife} many humanoid clans prayed to what
they believed was the greater deity \nameref{sec:Morana}. She in turn revealed
to her followers a path to great power to vanquish their foes, and many of her
followers accepted it. They were betrayed by their goddess, and turned into
vampires against their will.

These followers were turned undead, enhanced with special abilities, and were
given an insatiable hunger for monstrous blood. The vampires' hunger for
monstrous blood grew with every taste of the crimson liquid, perpetuating a
cycle of hunting, killing and feasting to satiate their thirst. Most humanoid
tribes that had people turned into vampires began locking themselves away at
night while allowing the vampires to roam the surrounding forests and area to
feast on those monstrous creatures unfortunate enough to remain outdoors
during the night. While in some areas the vampires spread terror and death
among the monstrous creatures that would attack the humanoid settlements, in
others, the lack of alternate food sources forced the vampires to turn on
those they were supposed to protect. These vampires then learned of yet
another betrayal by their mother: Vampires that feasted on the blood of the
core humanoid races (human, elf, dwarven, halfling and half-breeds) blood
would turn feral, and lose the last remaining modicum of control they had left
over their inner hunger. This lead them to be ostracised, hunted, and even
killed by their own people, the humanoids they were supposed to protect from
harm.

Many of these old vampires turned from their ``mother'' and were then
threatened with death by Morana in turn. This direct threat to her children,
and her outburst of anger revealed Morana to be a lesser deity. Not
intimidated by the threat, many vampires turned away from her and were then
killed by Morana for their insolence. Those that survived did so by turning
toward lesser deities, such as the \nameref{sec:Silent Queen}, who protected
them from Morana's wrath. This incident, known as the \emph{great betrayal},
has cemented Morana's downfall as a grand deity, and she fell in favour not
only amongst vampires, but also among the mortals.

The surviving vampires, often without a way to turn back to a mortal form,
then continued what they originally set out to do: aid the humanoids in
fighting the monstrous races, while also fighting their inner thirst for
control over their own lives. Humanoid settlements often protected their
vampires during the day, while the vampires fought and feasted upon the
monstrous races, such as the bugbears, goblins, gnolls and hobgoblins, during
the night. Although still a threat to the very humanoids these vampires
attempted to protect, many earned themselves respect, friendship and often a
noble title and land for their efforts during the \nameref{sec:Strife}.

After the Aeon of Strife ended, most vampires held on to their titles, lands,
and riches, trying to live peacefully together with the mortal humanoid races.
They often feasted on animals, or only drank blood sparingly often willingly
donated by their subjects. Some vampires however instead turned on the
humanoids, feasting on their blood and thus became hated, feared and hunted as
they turned into feral vampires.

\subsubsection{Ramesk and Vampires}

In \emph{MI:310} the recipe for \nameref{sec:Ramesk}, a mineral rich blueish
liquid, was discovered by a \nameref{sec:Deepkin} expedition in the depths of
the \nameref{sec:Cnamh Mountains}. It took a few years of experimentation to
adapt the liquid to work as nourishment for vampires, and there were many
troubles in growing the ingredients on farmland on the surface. But now the
liquid is widely used by the vampires as a substitute to monstrous blood. Only
a cup full of Ramesk is required for a vampire to be fed for a few days.

Ramesk has a few other benefits for vampires. It makes their nails and hair
grow again, albeit in the same shade of light blue of the liquid. Although
nail colours and hair dyes exist, blue hair has become an identification mark
for vampires to show to the mortal races that they are civilised and pose no
threat. Ramesk also restores the vampire's eyesight, giving them red
irises. It further reduces their primal longing for humanoid blood, giving
them the ability to explore other feelings such as love, anger or grief which
would otherwise be suppressed by their hunger. Another important aspect of
Ramesk is that vampires become less sensitive to sunlight. Vampires that drink
Ramesk can endure the daylight, as long as they wear clothing that shades them
from direct exposure to the suns.

Ramesk also has social benefits, as several inns and taverns make normal meals
containing Ramesk allowing vampires to dine alongside their mortal friends.
This shows both vary strangers that the vampire actively consumes Ramesk, and
allows them to socialise with friends or business associates. Ramesk is served
by most major inns and taverns, as it has a long shelf life, and requires
little storage space.

\subsubsection{Becoming a Vampire}

Popular belief might tell you that being bitten by a vampire is enough to turn
into a vampire, or perhaps a lesser version of one. That is false however, and
in fact the process requires the complete opposite: Those that wish to seek to
turn to a life of undeath must drink the blood of a vampire. However the
transformation to a vampire only works on the core humanoid races, and the half
breeds that they produce amongst each other. The blood of a vampire is poisonous
to all monstrous races, including the \nameref{sec:Gorgons} and half-orcs.

The giants attempted to create monstrous vampires, but even millennia of
necromancy did not yet lead to success. The giants then attempted to turn
their slave race of \nameref{sec:Diarim} (which were bred out of the core
humanoid races) to vampires. The transformation worked but backfired on the
giants as the vampire Diarim turned on their masters and their kin, driven by
an inner lust to kill and feed upon monstrous creatures.

\subsubsection{Born Vampires}
\label{sec:Born Vampires}

The vampire nobility of \nameref{sec:House deVar} heavily researched the magic
that turned the early humans and elves into vampires. In \emph{MI:755}, they
restored the ability for female vampires to birth children on their own, thus
furthering the vampire race without having to turn other mortals. These new
vampires are often called \emph{born vampires}, and are equally as powerful
as the original vampires. They retain the physical appearance of the mortal
races of their parents. Born vampires reach maturity within a few years, but
often go out into the world to seek adventure, learn skills and even
professions like mortals do.

\begin{35e}{Living Undead (Subtype)}
  Vampires can also be born, in which case they have the \emph{Living Undead
  Subtype}. It further gains the additional natural attacks (two claws, one
  bite in grapple), special attacks, and special qualities, ability bonuses,
  and level adjustment of the \emph{Vampire}  template (see below).

  \begin{itemize}[noitemsep]
    \item{A living undead derives its Hit Dice, base attack bonus progression,
      saving throws, and skill points from the class it selects.
    }
    \item{Unlike other undead a living undead has a constitution score.}
    \item{Unlike other undead, living undead are subject to nonlethal damage,
      stunning, death affects or necromancy effects.
    }
    \item{A living undead respons slightly differently to other living
      creatures when reduced to 0 hit points. A living undead with 0
      hit points is disabled, but does not risk further injury when
      performing strenuous activity. When the hit points are less than
      zero, a living undead is unconscious and helpless like any living
      creature, but does not lose additional hit points unless more damage
      is dealt to them.
    }
    \item{As a living undead they can be raised, and resurrected. A resurrection
      spell does not restore the vampire back to its mortal form, and they stay
      vampires after a resurrection.
    }
    \item{Living undead need to eat, and breathe; but do not require sleep.}
  \end{itemize}
\end{35e}

\subsubsection{Feral Vampires}
\label{sec:Feral Vampires}

Those vampires that actively hunt, feed and kill humanoid races such as
humans or elves, on a regular basis are called \emph{feral vampires}. The
taste of humanoid blood makes them savage, primal and often primitive hunters
and killers. They are far and few, but often cause great harm to towns, cities
and kingdoms. Feral vampires are outcasts among the humanoids, monstrous races
and even their civilised brethren, who see them as a threat to the fragile
peace between the mortal and undead races. Feral vampires often return to a
worship of their mother \nameref{sec:Morana}. Much like the original vampires,
feral vampires, are blind and hunt by scent only.

Although feral vampires could be turned from their savage ways by feeding them
Ramesk while keeping them away from blood, very few undead hunters or
civilised vampires bother to do so. Feral vampires are often hunted and
destroyed without prejudice.

\begin{35e}{Vampire (Template)}
  Those that are turned into vampires later in life acquire the ``Vampire''
  template. Only the core humanoid races, as well as their half-offspring
  (except Gorgons and Half-Orcs), and \nameref{sec:Diarim} can gain this
  template. A transformed vampire uses all of the base creature's statistics
  except as noted here.

  \srditem{Size and Type}{The creature's type changes to \emph{living undead}
    (augmented humanoid). Base attack bonus, skill points, and size are
    unchanged. See the ``born vampire'' section for details on living undead.
    Vampires require nourishment in the form of monstrous blood
    or \nameref{sec:Ramesk}, and may suffer starvation without nourishment
    like any living being.
  }
  \srditem{Abilities}{Vampires gain a +4 bonus to strength, and a +2 bonus to
    dexterity, and constitution scores.
  }
  \srditem{Armour Class}{The base creature's natural armour improves by +4}
  \srditem{Attack}{Vampires retain all attacks of the base creature, but gain
    two natural weapons: It can deal a Claw attack with each hand. The claw
    is of course unavailable should a vampire wield a weapon, or shield in
    that hand.
  }
  \srditem{Damage}{Vampires have two claw attacks (one with each hand), that
    deal 1d6 points of damage for medium base creatures (adjust damage
    according to size changes). The bite does 1d4 points of damage for medium
    creatures (adjust according to base creature's size).
  }
  \srditem{Special Attacks}{A vampire retains all special attacks of the base
   creature but gains those listed below. Saves have a DC of $ 10
   + \frac{1}{2} $ vampire's HD + vampire's CHA modifier unless noted
   otherwise.

  \emph{Blood Drain} (Ex): A vampire can suck blood from a living victim
   through its fang whenever it makes a successful hit with its \emph{Bite}
   attack. On each successful attack with its \emph{Bite} it regains 1d8 hit
   points if the creature belongs to the monstrous races. This natural attack
   is only available in grapple, and deals 1d8 non-lethal damage to a victim.
   If the target is of the core humanoid races, the vampire takes 1 point of
   intelligence drain instead, and if the target is any other creature type
   (e.g. animal) the vampire gains no benefits.

  \emph{Energy Drain} (Ex): Living monstrous creatures hit by a vampire's
   claw attacks (or any other natural weapon the vampire might possess), gain
   two negative levels. For each negative level bestowed, the vampire gains
   1d8 hit points. A vampire can use its energy drain ability once per round.
  }

  \srditem{Special Qualities}{A vampire retains all the special qualities of
    the base creature, and gains those listed below.

    \emph{Turn Resistance (Ex)}: Non-feral vampires have gain a +8 bonus to
    turn resistance, a feral vampire gains +4 bonus to turn resistance.

    \emph{Track (Ex)}: Vampires gain the benefits of the \emph{Track} feat.

    \emph{Scent (Ex)}: Vampires gain the benefits of the \emph{Scent} ability.

    \emph{Feral Decay (Ex)}: A vampire that feasts on the blood of the core
    humanoid races loses 1 point of intelligence to \emph{ability drain}, down
    to a minimum of 2 points of intelligence. Nothing can cure this ability
    drain, except \nameref{sec:Ramesk}, or the blood of monstrous creatures,
    at the rate of 1 point healed per day.

    \emph{Feral Blindness (Ex)}: A vampire that has been reduced to a feral
    state (by having the intelligence reduced to 2 through Feral Decay),
    becomes blind. This blindness can be cured once the intelligence drain
    induced by Feral Decay has been cured.
  }

  \srditem{Level Adjustment}{+2}
\end{35e}
