\subsection{Percht}
\label{sec:Percht}

\emph{Percht} (plural ``perchten'') are ancient gestalts that inhabit the
great divide, as well as \nameref{sec:Dirgewood}. They are ancient souls that
roam the forest and the mountains, that can possess humanoid bodies and
figures when they wish to interact with other mortal beings.

Generally perchten can be categorised into two groups: \emph{Schiachperchten}
which is old Teranim or Dirgewood slang for ``ugly perchten''. They often
appear with horribly distorted, ugly, animal-like appearance and are generally
considered evil and cruel. While on the other hand are the \emph{Schenperchten},
which means ``beautiful perchten'', that are considered good, helpful and
supportive. Schenperchten always appear in normal humanoid form (for example as
an old human crone living by herself in a wood), and prefer to remain
unnoticed.  Both kinds of perchten have incredibly powerful souls, and thus
any \hyperref[sec:Soul Magic]{soul magic} that reveals souls, will also
identify actual perchten in disguise.

Perchten are not a race per se, but a collection of powerful individual souls
that roam the forests, caves, mountains, and valleys. While the evil ones might
cause mischief, harm and grief, the beautiful ones might actually help and aid
travellers and villages.

\subsubsection{Wild Hunt}
\label{sec:Wild Hunt}

The \emph{Wild Hunt} (``Wudjogt'' in Dirgewood) are a collection of Perchten,
that roam the forests of the Dirgewood to hunt. They hunt anything, even
humanoid creatures, and are thus considered evil Schiachperchten. They hunt
mostly by bow and arrow, and sound their very loud and distinct hunting horns
when setting out for hunts. The wild hunt will continue until the Perchten are
satisfied with their prey or until they are driven away by the bell ringers.
The hunters cannot be reasoned with, but may be defeated, and will simply hunt
any worthy as prey.

\subsubsection{Bell Ringers}
\label{sec:Bell Ringers}

The \emph{Bell Ringers} (or ``Gleckler'' in Dirgewood) are a small group of
good Perchten (``Schenperchten'') who only appear when the evil Perchten
of the Wild Hunt begin to set out to hunt. They are free souls that shine
in a bright yellow light, and wander through towns and cities in the Dirgewood
warning the living inhabitants with a bell and rhyme to stay safe while
the wild hunt are afoot. The bell ringers are also known to reside in town
squares until the wild hunt is officially over, or even attempt to bring the
wild hunt to a premature end by chasing of the Schiarchperchten hunters. The
bell ringers only speak cryptic warnings in old Teranim warning anyone about
the hunt, and disappear again as soon as the hunt is over. Many shamans of the
\nameref{sec:Old Ways} believe that bell ringers where once killed by the
hunters, and now wish to spare others of that fate.  Worship, honouring and
giving sacrifices to the bell ringers is considered in an integral part of the
faith of \nameref{sec:Marwaid}.

\subsubsection{Habergoas}
\label{sec:Habergeiss}

\aren{Many know these by their new Teranim word ``Krampus''.}

The Habergoas, which combines the old Teranim words for ``he goat'' and ``she
goat'' into one name, is a horrific huge creature with a goat head, and horse
like legs. It has often either white or black fur, and always walks on two
legs while caring a huge woven basket on the back. Its visage is often a
horribly disfigured face of a goat, and it is considered an evil
Schiachperchten. Habergoas is often armed with a long walking and fighting
stick on which hangs a bell that it ring loudly to announce its arrival. It
will wander into towns and villages, often at the beginning of winter, and
demand sacrifices from the villagers and towns folk or it will start abducting
people and children. It will not stop terrorising villages and people until it
finds the sacrifices to be sufficient.

The Habergoas is a respectable fighter, and its half-goat body has
considerable bodily strength and resilience. Fighting off a Habergoas might
just work, or it will just be angered even more doubling and tripling the
sacrifices required to appease it.
