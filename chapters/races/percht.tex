\subsection{Percht}
\label{sec:Percht}

\emph{Percht} (plural ``perchten'') are ancient gestalts that inhabit the
land surrounding the great divide, such as the \nameref{sec:Dirgewood}. They
are ancient souls that roam the forest and the mountains, that can manifest
humanoid bodies and figures when they wish to interact with other mortal
beings.

Generally perchten can be categorised into two groups: \emph{Schiachperchten},
which is old Teranim or Dirgewood dialect for ``ugly perchten''. They often
appear with horribly distorted, ugly, animal-like appearance and are generally
considered evil and cruel. \emph{Schenperchten}, which means ``beautiful
perchten'', on the other hand are considered good, helpful and
supportive. Schenperchten always appear in normal humanoid form (for example
as an old human crone living by herself in a wood), and prefer to remain
unnoticed. Both kinds of perchten have incredibly powerful souls, and thus
any \hyperref[sec:Soul Magic]{soul magic} that reveals souls, will also
identify actual perchten in disguise.

Perchten are not a race per se, but a collection of powerful individual souls
that roam the forests, caves, mountains, and valleys. While the evil ones might
cause mischief, harm and grief, the beautiful ones might actually help and aid
travellers and villages.

\subsubsection{Bell Ringers}
\label{sec:Bell Ringers}

The \emph{Bell Ringers} (or ``Gleckler'' in Dirgewood) are a small group of
good Perchten (``Schenperchten'') who only appear when the evil Perchten
of the Wild Hunt begin to set out to hunt. They are free souls that shine
in a bright yellow light, and wander through towns and cities in the Dirgewood
warning the living inhabitants with a bell and rhyme to stay safe while
the wild hunt are afoot. The bell ringers are also known to reside in town
squares until the wild hunt is officially over, or even attempt to bring the
wild hunt to a premature end by chasing away the Schiarchperchten hunters. The
bell ringers only speak cryptic rhymes in old Teranim warning anyone about
the hunt, and disappear again as soon as the hunt is over. Many shamans of the
\nameref{sec:Old Ways} believe that bell ringers where once killed by the
hunters, and now wish to spare others of that fate. Worship, honouring and
giving sacrifices to the bell ringers is considered in an integral part of the
faith of \nameref{sec:Marwaid} and the \nameref{sec:Old Ways}.

\subsubsection{Bärbele}
\label{sec:Barbele}

The \emph{Bärbele} is a perchten with a distinct female appearance. It often
wears ragged, torn clothing, a colourful apron, and a beautifully carved
wooden mask. They are often ``armed'' with a broom and a woven basket, and
announce their arrival by ringing bells. They arrive at villages in small
groups near the end of the year and begin to clean the streets taking anything
with them that they consider litter. They then carry those items away and
build a shrine to \nameref{sec:Marwaid} near the village. While some might be
annoyed at the theft of their belongings, many see the arrival of these
perchten as a sign that worship Marwaid has been neglected in the last
year. They are generally harmless unless they are attacked first, or someone
attempts to steal from them.

\subsubsection{Frau Perchta}
\label{sec:Frau Perchta}

Mother Perchta (``Frau Perchta'' in dialect and old Teranim) is the supposed
mother and patron of all perchten gestalts that roam the region. She is
generally described as a good spirit that appears in the form of an old woman
or crone. Several tales about her exist in the stories of the \nameref{sec:Old
  Ways}, but most depict her as an older lady that punishes those that are
lazy and rude, while giving to those that work hard and are humble in spirit.
Throughout the stories and traditions of the \nameref{sec:Old Ways} she is
considered an aspect of \nameref{sec:Nyddwr}, as Frau Perchta rewards those
that work hard to earn their good future. She often visits towns, farmsteads,
villages and hammocks in disguise, or can be found living deep in the mountains
or woods as an old lone crone.

\subsubsection{Habergoas}
\label{sec:Habergoas}

\aren{Many know these by their mystical ``leader'' called \emph{Krampus}.}

The Habergoas, which combines the old Teranim words for ``he goat'' and ``she
goat'' into one name, is a horrific huge creature with a goat head, and legs
of a horse. It has often either white or black fur, and always walks on two
legs while carrying a huge woven basket on the back. Its visage is often a
horribly disfigured face of a goat, and it is considered an evil Schiachperchten.
Habergoas is often armed with a long walking and fighting stick on which hangs
a bell that it ring loudly to announce its arrival. It will wield that
stick and its heavy bell much like one would a heavy flail. It will wander
into towns and villages, often at the beginning of winter, and demand
sacrifices from the villagers and towns folk or it will start abducting people
and children. It will not stop terrorising villages and people until it finds
the sacrifices to be sufficient.

The Habergoas is a respectable fighter, and its half-goat body has
considerable bodily strength and resilience. Fighting off a Habergoas might
just work, or it will just be angered even more doubling and tripling the
sacrifices required to appease it. The Habergoas will however recognise when
humanoid males, especially young males, wish to brawl and will use non lethal
force in combat. To many young humanoid males it is a great source of pride
to have fought a Habergoas, regardless of whether that attempt was successful
or not.

\subsubsection{Schnabelperchten}
\label{sec:Schnabelperchten}

The Schnabelperchten (lit. beak or pecker perchten, but often simply called
``crow spirits'') are evil perchten (Schiachperchten) that arrive often in
groups at the beginning of a new year. They have humanoid bodies, often
covered in thick linen clothes. Their body shape suggests that there are male
and female crow spirits. Their heads are distinct by being absolutely
featureless, except for a huge (half a metre) white beak as a mouth. They do
not speak, but will herald their arrival with loud and rather distinct
crowing. Schnabelperchten further carry four items with them: A broom, a
dustpan, a large woven pannier that is carried on their backs, as well as an
oversized pair of scissors. They will knock on doors and windows, demanding
entrance. If entrance is not granted they will simply let themselves in, and
walk around the house inspecting its cleanliness. Should Schnabelperchten find
dust or dirt (apart from what they carried in themselves) they will clean it
up, gut the owner of the house with their scissors, and replace his entrails
with the dirt and stones before sewing up his stomach again.

Among all the Schiachperchten the crow spirits are the most feared and hated,
and have in recent years been actively hunted and killed by the knights and
priests following the \nameref{sec:Order} as well as the church of
\nameref{sec:Lor}. It is unknown why they act in this manner, especially since
their punishment is draconian compared to the offence of having a dirty house.

\subsubsection{Wild Hunt}
\label{sec:Wild Hunt}

The \emph{Wild Hunt} (``Wüdjogt'' in Dirgewood) are a collection of Perchten,
that roam the forests of the Dirgewood to hunt. They hunt anything, even
humanoid creatures, and are thus considered evil Schiachperchten. They hunt
mostly by bow and arrow, and sound their very loud and distinct hunting horns
and drums when setting out for hunts. The wild hunt will continue until the
Perchten are satisfied with their prey or until they are driven away by the
bell ringers. The hunters cannot be reasoned with, but may be defeated, and
will simply hunt anything that is worthy prey.
