\subsection{Humans}
\label{sec:Humans}

\emph{Humans} are one of the most prolific of the humanoid races of
\emph{Aror}, and also the dominant race of the planet. Human artefacts have
been found dating back thousands of of years, outmatching the other humanoid
races. There are two separate ``races'' of humans: those inhabiting the
southern part of the hemisphere who usually have darker skin, and the
``northerners'' who usually have fair skin. This distinction is superficial
only. Apart from the tone in skin colour, there is no other biological
difference between the various human tribes and civilisations. Humans live up
to 80 years, and are known for being statesmen, diplomats, farmers,
adventurers, explorers and scientists.

\subsubsection{Language}

Humans speak \emph{Teranim}, either as their primary language, or as their
secondary language together with their local language. Teranim has become the
de-facto language of \emph{Aror} and is usually spoken almost everywhere, even
among the beast races. Teranim is written in its own alphabet of the same
name. \emph{Old Teranim} and \emph{Ancient Teranim} exist, spoken by the
humans of old, and are the root of many other languages and various local
dialects. Although \emph{ancient teranim} is no longer actively spoken by
humans, various books, poems, stories and songs still exist in that language.

Humans living in the southern hemisphere often speak a language that is stuck
halfway between old and new Teranim, called \emph{Kalest}. While the people of
\emph{Forsby}, and the surrounding Toralian higlands, have their own distinct
dialect, which can be so hard to understand that it has received its own
classification and name: \emph{Reatham}.

Albeit many local dialects exist, almost all humans, and the other races
living with them are capable of speaking Teranim. It is, after all, the
official language of many governments, the language that is printed and used
in official capacity as well as in inter-kingdom cultural exchange and trade.

\begin{35e}{Common in Aror}
  The language of Teranim is equal to \emph{Common} of D\&D. See also
  the \nameref{sec:Speak Language} section for a list of available languages
  and their alphabet.
\end{35e}

\subsubsection{Human Lands}

Humans can be found everywhere on Aror. But history indicates that they
originated on the southern continent of \emph{Arania} together with the other
core humanoid races, and migrated to all other continents during the last ice
age tens of thousands of years ago. Wherever humans settle they build
villages, cities, and large kingdoms and often become the dominant culture and
social structure. Human kingdoms have often endured for hundreds of years,
and have bested many difficulties that had driven other societies to ruin. The
ingenuity of the humans, their stubborn attitude and their uncanny ability to
adapt to any difficulty makes them the dominant race of \emph{Aror}.

\subsubsection{Human Culture}

Like most races, humans have no inherent global culture, tradition or customs.
Instead their believes and customs are ever evolving, and specific to the
realm they live in. But there is one trait that the average human has: ingenuity
in the face of adversity. No other species has managed to settle every corner
of the world and endure, and even build lasting civilisations out of the
hostile environment they found themselves in.

\begin{35e}{Human Traits}
  \begin{itemize}[noitemsep]
  \item Medium: As Medium creatures, humans have no special bonuses or
    penalties due to their size.
  \item Human base land speed is 30 feet.
  \item 1 extra feat at 1st level.
  \item 4 extra skill points at 1st level and 1 extra skill point at each
    additional level.
  \item Automatic Language: Teranim. Bonus Languages: Any (other than secret
    languages).
  \item Favoured Class: Any. When determining whether a multiclass human takes
    an experience point penalty, his or her highest-level class does not count.
  \end{itemize}
\end{35e}
