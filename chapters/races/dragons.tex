\subsection{Dragons}
\label{sec:Dragons}

Dragons are a race of highly intelligent, winged creatures with reptilian
traits. They come in a variety of shapes, sizes and colours. They grow bigger
and more powerful they more they age and mature.

The size of dragons ranges from small (roughly one metre), up to gigantic (20
metres), depending on their age. They have lightweight, yet protective scales
that cover most of their bodies. And their skin colour range from black, blue,
green, red white, to brass, copper, gold or silver. Dragons with more than
one colour are very rare, but have been observed as well. All dragons are
capable of breathing fire, and some even master secondary or even tertiary
elementary breath as they grow older.

The dragons of Aror are immortal but not invulnerable. Furthermore each
hatchling is born with the genetic knowledge of its two parents. This knowledge
is not immediately available to the young dragon, but is instead unlocked
over the course of decades, or even centuries, as the dragon gains new
experiences, impressions and learns about the surrounding world. Young dragons
thus study among their own adults, in an attempt to learn new things, as well as
to unlock their own hidden powers and knowledge. Dragon eggs that successfully
hatch are a rarity, and so only a few young dragons are born in a century.

Dragons, with their aeons of study of the sciences and the arcane arts, are
highly advanced engineers, smiths, and spell casters. They have made many of
the arcane artefacts that can be found on Aror, such as the
\hyperref[sec:Dragon Teleporter]{dragon teleporters}.

Most dragons on Aror live on a continent called \nameref{sec:Draigynus}, which
is named after them. Dragons are not a native species of Aror, and have settled
this continent after arriving on the planet from another plane. It is unknown
when, or from where they came.

In \emph{MI:1916} the \hyperref[sec:Giants]{giants} arrived on the northern
continent of \nameref{sec:Farlar}, to wage war against the dragon colony. The
dragons were joined by the halflings and elves of the city of Nen-Hilith,
until the city was destroyed by the giants. Over two hundred dragons have lost
their lives in the war, and by \emph{MI:2008} the war is still continuing.

About a hundred dragons now remain on the continent, and have organised
themselves into a small clan lead by the eldest, and most wisest of the
dragons, called the \emph{elder}. This elder is supported by an apprentice
dragon, that is referred to as the \emph{aspirant}. The dragons are known to
keep to themselves, and prefer not to mingle with what they call ``younger
races''. To them the continued existence of their rather small clan is the
utmost importance, and they will do anything to protect themselves, and
especially their young and unborn. Threatening, harming a young or unborn is
the worst crime anyone can inflict in dragon society, is punishable by death.

Although the dragons have no use for gold or money, they have began hoarding
wealth by selling or trading arcane weapons, armour and artefacts to the
humanoid races. In exchange humanoid and monstrous mercenaries fight for them,
deepkin and dwarven engineers build siege engines and golems, and elven and
halfling raiders strike their old homeland in daring raids to cripple the
efforts of the giants.

\begin{35e}{Dragons}
  Overall dragon society is considered \emph{neutral good}, albeit you can
  find dragons ranging in all sorts of alignments.

  All dragons breath fire foremost, and once they reach the \emph{young adult}
  stage of development (so after 50 years of age) they gain the ability to
  breath another element. A \emph{mature dragon} (200 years or older) gains a
  tertiary element. Roll these at random from the list of available elements:
  \textbf{acid}, \textbf{sonic}, \textbf{electricity}, \textbf{cold}, or
  \textbf{force}. The dragon can decide which form the breath attack takes,
  before the attack as a free action. The dragon also gains immunity against
  the elements it breathes.

  Dragons gain no spell-like abilities, but all dragons gain the \emph{Water
    Breathing} extraordinary ability.
\end{35e}
