\subsection{Elves}
\label{sec:Elves}

Elves are tall - often between 1.8 and 2.4 metres - and slender race, with
long and pointy ears. The style of the elven ears varies, with some having
smaller pointy ears facing backwards, while others have longer and sharper
ears that follows the contours of their face. The variety within the elven
ears is vast, but is but a minor cosmetic difference. Their physique and build
is slender as well, with long skinny legs and arms. Some elves live up to be
200 years of age, and thus often pick up professions that take longer to
master, such as wizardry, artistry or the sciences. Although elves live long,
they often shift focus in their goals.

Unlike humans, elves do have distinct sub races that differ from each other in
various physical and biological aspects. These arose because elves are highly
adaptable towards their environment, adopting new traits over the course of a
dozen generations. The \emph{snow elves} for example have a natural resistance
to cold like no other species have, while the \emph{dark elves} have adapted
to see better in the dark caverns they inhabit. However they have not diverged
so far from one another to not produce viable offspring. As with humans, and
deepkin the race of the offspring is always that of the mother.

\aren{There are some old racist jokes about elves: If drown enough elves you
  will eventually produce an ``aquatic elf''.
}

\graham{We should stop now, lest we offend our elfish readers.}

There are four major elven races that are recognised across the world of Aror.
Two of these are long lived (high-, and wood elves) and live up to 200
years. Those elves require longer to mature, and are usually considered adults
when they turn 40 years old. They thus have longer to learn and study, and
thus often pick up several trades or trades that are difficult to master.
Snow- and dark elves on the other hand live similar life spans as humans, and
also follow humanoid epochs throughout their life (i.e. being adults at 18).

\subsubsection{High Elves}
\label{sec:High Elves}

The most numerous race of the elves are the \emph{high elves} of
\nameref{sec:Avenfjord}. High elves have fair skin with a hint of yellow and
gold. Their hair ranges from blond, fiery red to brown and black. They stand
between 1.8 and 2.4 metres high, towering with their slender appearance over
most other humanoid races with ease. They have pointed, elongated ears which
often point skyward, following the slender contour of their face.

High elves are the most intelligent, and curious of the elves, and often live
within human city kingdoms, mixing and integrating well with other cultures
and societies. Even though they have their own kingdom, the vast majority of
high elves live outside the kingdom of Avenfjord. High elves are the only
kingdom building elven race, but do prefer to live with together with their
distant human cousins within baronies, kingdoms and city states.

The high elves are also the longest living of all the elven races, easily
reaching ages close to two-hundred years. This gives them ample time to study
and learn several fields and areas of expertise during their early childhood.

High Elves speak \hyperref[sec:Speak Language]{Enro'ad}, a variant of
\emph{Old Teranim}, but use the Taavid, halfling alphabet to write it.

\begin{35e}{High Elf Traits}
  \hyperref[sec:Speak Language]{Enro'ad} is elvish, albeit the elven alphabet
  is now written with Taavid the \emph{halfling} script.

  \begin{itemize}[noitemsep]
    \item Medium: As Medium creatures, elves have no special bonuses or
      penalties due to their size.
    \item High elves gain a +2 bonus to \emph{charisma}, and
      \emph{intelligence}.
    \item Low-Light Vision: An elf see twice as far as a human in starlight,
      moonlight, torchlight, and similar conditions of poor illumination. She
      retains the ability to distinguish colour and detail under these
      conditions.
    \item Automatic languages: Enro'ad, Teranim. Bonus Languages: Any (except
      secret languages)
    \item Favoured Class: Any. When determining whether a multi class takes an
      experience point penalty, his or her highest-level class does not count.
  \end{itemize}
\end{35e}

\subsubsection{Dark Elves}
\label{sec:Dark Elves}

The \emph{dark elves} live mostly underground, have black to blue skin, and
their hair ranges from a faint hint of blue, silver to snow white. Their eyes
are often red, blue or a green. They are the smallest of all elven races, and
range from 1.60 to 1.90 metres in height. In terms of bodily physique they
also more closely resemble humans. Their rounder contours, and their stronger
and more muscular physique sets them apart from their slender high elven
cousins.

Dark elves once followed the dwarves underground, when the ancient battles
against the sentient non-humanoid creatures and the fey turned deadly and
dangerous. They have since adapted to that life underground, both physically
and culturally. They can see well in the dark, and their skin turned dark
letting them blend in with the dark surroundings of the depths. They also only
live roughly 80 years, and are thus not as long lived as their high elven
cousins.

In the depths below the dark elves often lead primitive, nomadic lives, which
allow them to survive the harsh realities of the land below. They have
survived the dangerous environment of the deep caverns by remaining on the
move, hiding from threats or by employing hit-and-run tactics. They value
their small communities and family above all else.

Many dark elves also live on the surface, and much like their surface cousins,
they prefer to integrate into already existing humanoid kingdoms and baronies.
Dark elves are rather common all over the world, and can be found on almost all
continents.

\begin{35e}{Dark Elf Traits}
  \textbf{Dark Elf Traits (Ex)}: The following traits are in \emph{addition}
  to the high elf traits, except when noted.
  \begin{itemize}[noitemsep]
    \item Dark elves are considered smaller and more nimble than the other
      elven races, and thus gain a +2 bonus to dexterity, which replaces the
      high elven bonus to intelligence.
    \item Dark vision out to 120 feet, albeit they cannot distinguish colour
      in the dark.
    \item Dark Elves do not gain any additional disadvantages if suddenly
      exposed to bright light.
    \item Automatic languages: Doresh, Enro'ad, Teranim. Bonus Languages: Any
      (except secret languages)
  \end{itemize}
\end{35e}

\subsubsection{Snow Elves}
\label{sec:Snow Elves}

\aren{Snow Elves are the pinnacle of beauty...}

Far to the north, and south live the \emph{snow elves}, nomadic hunter-gatherer
elves with white to silver, or blue skin; and white, silver, grey or blue
hair. Their eyes are often bright blue, green or yellow. Male snow elves are
capable of growing facial hair, while many female snow elves have light red
freckles in their faces. Snow elves are as tall as their high elven brethren,
ranging from 1.8 to 2.4 metres, and also share most of their facial features
with high elves. They have adapted well to the colder climates, and can
withstand the cold far easier than any other humanoid race. This has also come
with a major drawback: With a life expectancy of 80 years, they are among the
elven race with shortest lifespan. Many \emph{snow elves} live in smaller
families and tribes, content with surviving the harsh realities of the polar
north and south by becoming fierce hunters, fishers or even raiders
themselves.

These snow elves from northern and southern tribes are generally known to be
calm, and quiet, preferring the solitude of a small group or town over the
vast masses and stretches of cities. They rarely leave the frozen north and
south, and are thus exotic, as in, many other humanoids haven't seen a tribal
snow elf in person. They do not build cities or kingdoms, but smaller tribes
often join to form larger ones (several hundred individuals) if required.

To tribal snow elves the community, family and their own tribe is everything, as
it ensures survival and in the frozen tundras. The harshest punishment within
these communities, reserved for major crimes, is exile which is often equal to
a death penalty. Since these exiles are then also shunned by other snow elven
tribes, they sometimes wander away from their homes and join city kingdoms or
baronies.

Snow elves rarely leave their icy domain and tribal lives willingly, and thus
the ancestors of the snow elves in the city kingdoms were either exiles, or
were once captured and brought there as slaves. Those civilised snow elves do
not harbour any animosity any more about what happened to their ancestors
thousands of years ago, and prefer to remain in the places and cultures they
now call their homes. To the tribal snow elves these city snow elves are equal
to exiles, and are thus not welcome in the frozen north or south. Most larger
civilisation have a well established, and flourishing population of snow elves,
although they are often the minority.

\begin{35e}{Snow Elf Traits}
  \textbf{Snow Elf Traits (Ex)}: The following traits are in \emph{addition}
  to the high elf traits, except when noted.
  \begin{itemize}[noitemsep]
    \item \textbf{Pale Wastes (Su)}: A pale elf can live comfortably in
      conditions of extreme cold, even with barely any clothing or external
      sources of warmth. This ability functions like a continuous \emph{Endure
      Elements} but for cold conditions only.
    \item Snow elves are known for their great hunting skill which allows them
      to perceive hidden things and small details, and thus have a +2 bonus to
      wisdom, which replaces the high elven bonus to intelligence.
    \item Automatic languages: Enro'ad, Teranim. Bonus Languages: Any (except
      secret languages)
  \end{itemize}
\end{35e}

\subsubsection{Wood Elves}
\label{sec:Wood Elves}

Wood elves have light brown or greenish skin, and green or red hair. Their
faces are often covered in light or red freckles, and their eyes are often
brown, blue or green. They are as tall as their high elven counter parts,
often ranging from 1.9 to 2.4 metres, and also share the facial features of
the other high elves. Although they are very close to their high elven
cousins, they are counted as their own race based on their ability to easily
build muscle mass. This often makes them the strongest elven race, and a wood
elf can easily be compared to \nameref{sec:Half-Orcs} in terms of bodily
strength.

There are three major communities of wood elves. The first live in the vast
temperate and boreal forest of \nameref{sec:Eilean Mor}, known as the
\nameref{sec:Dirgewood}, while the second lives in the \nameref{sec:Toralian
  Highlands}. These wood elves live together with humans, halflings of the
Dirgewood, as well as the dark elves, deepkin and the dwarves of the
\nameref{sec:Great Divide}. They are followers of the \nameref{sec:Old Ways},
and prefer to build small settlements, towns and perhaps tiny cities of their
own. These wood elves are expert trackers, hunters, farmers, as well as
shamans and priests of the old ways.

And in the jungles \nameref{sec:Yuacata} live the \emph{savage elves}, a loose
collection of tribes of cannibalistic, and demon worshipping wood elves. They
rarely wander beyond the confines of their jungle and are one of the rarest
elven culture to meet in the civilised areas of \emph{Aror}. They live a
primitive life, adapted to the harsh and dangerous rain forest.

\begin{35e}{Woold Elf Traits}
  \textbf{Wood Elf Traits (Ex)}: The following traits are in \emph{addition}
  to the high elf traits, except when noted.
  \begin{itemize}[noitemsep]
    \item Wood elves have it easier to gain muscle mass than the other elvish
      races, and thus have a +2 racial bonus to strength, which replaces the
      high elven intelligence bonus.
    \item Wood elves have a long history of being crafty, handy and versatile
      and gain 4 extra skill points at 1st level and 1 extra skill point at
      each additional level.
    \item Automatic languages: Enro'ad. Bonus Languages: Any (except secret
      languages)
  \end{itemize}
\end{35e}
