\subsection{Medusa}
\label{sec:Medusa}

Medusa are an all-female, half-humanoid half-reptilian offspring of the three
members of the so called \nameref{sec:Triumvirate}, after they were cursed
with a monstrous appearance by \nameref{sec:Forneus}. And although the
triumvirate is considered humanoid, their offspring are not. The modern medusa
are the children of the three sisters, and even though only one of them was
called ``Medusa'', the name stuck for all of their offspring.

The medusa are known to be excellent wizards, sorceress and arcane
scholars. Medusa build large towers, castles and underground complexes that
act as their arcane laboratories, libraries and studies. There they research
the arcane, build magical golems, and craft artefacts. Lone medusa also join
monstrous or humanoid tribes as wizards, arcane smiths, witches and spiritual
leaders. While many medusa have tried to cure their mothers in the decades
after their great sleep, most modern medusa have abandoned that task, and now
focus on their own personal gain and power.

Since there are no male medusa, they rely on the males of another humanoid
races for reproduction. The male child of a medusa are always a snow elf,
reminding them of their heritage from the Triumvirate, in the case of humanoid
male partner, or a half-orc if the father happened to be an orc. Female children
are almost always Medusa at birth. Almost. Sometimes the union of a medusa and
a male humanoid (such as an elf, halfling, human or dwarf) results in a
half-medusa called \hyperref[sec:Gorgons]{Gorgon}. While Medusa will usually
wish to keep Medusa children to grow their tribe, male or Gorgon children are
usually left in the care of the father. This is why many Gorgon's live among
the monstrous, and humanoid cities and tribes.

\begin{35e}{Medusa}
  Most Medusa can be found living in small groups, within in elaborate, well
  stocked and equipped arcane laboratories. They may have any alignment, but
  avoid the extremes, such as \emph{chaotic evil} or \emph{lawful good}.
\end{35e}

There are three distinct types of medusa, each of them tracing their lineage
directly back to one of the three members of the triumvirate. Children of Stheno
retain a lower humanoid torso, but have wings. They are most commonly found in
the jungles of \nameref{sec:Yuacata}.

\begin{35e}{Stheno's Children}
  These medusa have wings, and gain \emph{fly speed 30 ft. (good)}. They also
  gain an additional +2 bonus to intelligence.
\end{35e}

Children of Medusa herself have a snake like torso, but are generally more
intelligent that the others. Her children are most commonly found in
\nameref{sec:Iafandir}, where they have used their cunning to become
influential leaders among both monstrous and humanoid tribes.

\begin{35e}{Medusa's Children}
  Medusa's children gain an additional \emph{+4 racial bonus} to intelligence
  and charisma, as well as an additional secondary attack with their tail.
  This tail slap deals 1d6 points of damage, plus $ \frac{1}{2} $ their strength
  bonus.
\end{35e}

The children of Euryale have an additional set of arms, and prefer to weaken
opponents from afar with magic, before striking them down in close combat.
Most children of Euryale are both capable arcane wielders, and deadly
fighters. They live mostly in the vast plains and jungle north of Goban
mountain.

\begin{35e}{Euryale's Children}
  Euryale's children have an additional \emph{+10 racial bonus} to strength
  (increasing their strength to 20), and have the \emph{Multi-Weapon Fighting}
  feat instead of \emph{Weapon Finesse}. Their CR increases by 1.
\end{35e}
