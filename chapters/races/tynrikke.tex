\subsection{Týnríkke}
\label{sec:Tynrikke}

The \emph{Týnríkke} (Ancient Teranim for the ``forgotten kingdom''), often
shortened to simply ``týn'', is a large tribe of giant humanoids that live in
the north western part of \nameref{sec:Eilean Mor}, and all over
\nameref{sec:Iafandir}.

Their outward appearance is very similar to humans, but their average size
ranges between 2.5 to 3 metres. Although most other races call them giants
because of their sheer size, they are closer to the core humanoid races, than
to the actual giants and giant races of \hyperref[sec:Aror]{Aror}. The týn
have grey-blue skin, green or blue eyes, often blond or light brown hair.
Much like the other humanoid races, the male týn can grow facial hair.

It is unclear how they are related to the core humanoid races, but due to the
fact that they speak ancient Teranim, and their similarities to humans,
scholars agree that the týn and humanoids may share a common ancestor.

\subsubsection{History}

The ancestors of the týn built magnificent cities and temples hundreds of
thousands of years ago. This knowledge is now lost to the descendants of the
týn, and those great cities and structures are now nothing but ruins, tombs
and dungeons that dot the valleys, forests, and mountains of
\nameref{sec:Eilean Mor}, \nameref{sec:Goltir} and
\nameref{sec:Iafandir}. Ancient legends of the týn say that their ancestors
fought against the fey and beast races during the \nameref{sec:Strife}, and in
the process lost their lives, their civilisation and ultimately their future.

At some point in history the ancient týn connected their cities and temples
with teleportation magic. This magic is still active in some ruins, allowing
easy travel between the continents. That is, if the týn grant one access to
their holy sites.

\subsubsection{Society}

The týn believe in tradition, family and personal honour above all else, and
practise a heavily ritualised ancestor worship. Their ancestors built
magnificent cities and temples thousands of years ago, which are now nothing
but ruins, but these sites are still holy to them. The týn often live near
such ruins in tribes, or small villages. Since their ancestors do not grant
divine magic, most skalds (poets, and story tellers), witches and shamans
of the týnríkke are often bards and sorcerers.

In the eyes of the týn their ancestors lost their great civilisation and
future, to aid the other humanoid races in establishing themselves during
the \nameref{sec:Strife}. So their descendants, who hold that
sacrifice in great honour, often see the other humanoid races as ungrateful
and arrogant. A belief that is often strengthened when humanoid races attempt
to break into and loot the ancient tombs and catacombs of the týn. Many tribes
protect their holy sites fiercely against any intruders, and only allow those
they trust to visit the ruins.

For a týn his personal honour is his highest value in life. Honour is gained
through the hunt, battle or a life in service to family, tribe or protecting
the ancient sites from intruders. Tokens of honour are often prominently
displayed on the týn's clothing. Such tokens are, for example, teeth of an
animal the týn has hunted, items taken from slain enemies, or gifts from
other týn given to them for services rendered. Within a tribe the most
honoured leads a tribe, and he or she swears to serve the clan and the
ancestors. Should a týn lose his honour, for example by committing a crime,
they are often banished from the tribe, and move permanently into their
ancestor's ruins to protect them from looting and destruction. It is there
that a dishonoured týn can restore his honour, and return to the tribe, by
serving his ancestors. Those that die in honour are embalmed and buried in the
holy sites, next to their ancestors.

The týn are capable of producing iron and steel weapons, as well as advanced
weaponry such as crossbows. Furthermore they are fierce, strong, skilled
warriors and cunning tacticians. The týn fight in formations, and often know
how to use the terrain to their advantage, but consider hit-and-run tactics
dishonouring and thus prefer small skirmishes and open battles.

\subsubsection{Relations}

\graham{Thinking it wise to try to drive the týnríkke off their land, is a
  sure way to ruin your barony.}

Although almost all týn tribes allow other humanoids from visiting their
villages, all foreigners are forbidden from entering their holy sites and
ruins. The týn are willing to trade and interact with humanoids, but hold
fierce hatred against any beast races. Since the týnríkke will not integrate
or mingle with other humanoid villages or cities, and are notoriously hard to
drive off their land, most baronies and kingdoms simply ignore the týnríkke.

\begin{35e}{Týn}
  The týn speak \emph{ancient Teranim}. See the section \nameref{sec:Speak
    Language} for details on available languages and their alphabets.

  \begin{itemize}[noitemsep]
    \item +4 Strength, +2 Constitution
    \item Large size. -1 penalty to Armor Class, -1 penalty on attack rolls,
      -4 penalty on Hide checks, +4 bonus on grapple checks, lifting and
      carrying limits double those of Medium characters.
    \item Space/Reach: 10 feet/5 feet.
    \item An týn's base land speed is 40 feet.
    \item Ancestor Defence (Ex): +4 racial bonus to attack and damage while
      defending sacred ruins and holy sites. As well as a +4 dodge bonus
      against enemies while fighting in these holy sites or ruins.
    \item Fierce Appearance (Ex): +2 racial bonus to intimidate and perform
      checks.
    \item Level adjustment: +2
    \item Favoured class: Any.
  \end{itemize}
\end{35e}
