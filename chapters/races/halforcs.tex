\subsection{Half-Orcs}
\label{sec:Half-Orcs}

The rarest half breed of them all is the \emph{half-orc}, the offspring of a
dwarven mother and an orc. Any union including an orc and a dwarven mother,
will always produce a half breed commonly called a ``half-orc''. They share
distinct traits of their orcish heritage, such as greenish to grey skin, large
protruding eyetooth (often dismissively called ``tusks''), and a large,
towering and muscled physique.

Half-orcs are sterile, and thus have no inner need or desire to settle down or
start families. They often live amongst their humanoid parents offering their
brute strength and enduring physique as heavy labourers or fighters. A dwarven
mother giving birth to a half-orc has a high change of dying due to
complications, albeit birthing a half orc is less dangerous than giving birth
to a \nameref{sec:Mul}.

Due to the deep divide between monstrous and the core races, almost all
half-orc children are not the result of a voluntary union. This sad reality,
combined with a short temper, less than flattering appearance, and social
hardship, often elicit condescending or outright demeaning behaviour from the
other core races towards half-orcs.

\begin{35e}{Half-Orc Traits}
  \begin{itemize}[noitemsep]
    \item Their orcish blood makes it easier for half-orcs to gain muscle mass
      and thus have +2 racial bonus to Strength.
    \item Medium: As Medium creatures, half-orcs have no special bonuses or
      penalties due to their size.
    \item Half-orc base land speed is 30 feet.
    \item Orc Blood: For all effects related to race, a half-orc is considered
    an orc.
    \item Automatic Languages: Teranim, Orc. Bonus Languages: Any (except secret
      languages)
    \item Favoured Class: Any. When determining whether a multi class takes an
    experience point penalty, his or her highest-level class does not count.
  \end{itemize}
\end{35e}
