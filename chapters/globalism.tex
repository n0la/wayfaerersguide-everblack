\section{A Global World}

The world of \hyperref[sec:Aror]{Aror} has a few major city kingdoms that are
vastly populated centres of society and civilisation. These immense cities are
the major power players in the world, and often have power over the fates of
many million people. Not only in their own realm, but also in the lands,
baronies and smaller kingdoms they project their power onto.

Knowledge about the other city kingdoms is wide spread, and almost all people
have a rough understanding on how big the world is, and what culture lives
where. Although travel is only feasible for the middle and upper classes, word
about foreign lands and the intricate politics of the city kingdoms spreads
easily to every corner of the land.

\subsection{Dragon Teleporter}
\label{sec:Dragon Teleporter}

These large city kingdoms are also connected with each other with large
teleporters, that were originally invented by the dragons to inter connect
places of interest on their continent of \nameref{sec:Draigynus}. A dragon
teleporter has three large pillars, that sharpen to a point at their end. They
are arranged in a circle that arch inward, touching each other at the top, to
form a sort of arch over a small area beneath them. Once activated arcane
energy flows from their stem to the tip, where they form a glowing, floating
ball of energy beneath the arch, that acts as the horizon for the
teleportation magic. Once a living person touches the horizon, they are
instantly transported near one of the pillars of the remote teleporter. A
teleporter that receives a person, can still teleport one other person away at
the same time.

Draconic runes are inscribed into the pillar and help with selecting a target
for the teleportation, and can be used to program new targets into an existing
teleporter. A dragon teleporter requires considerable arcane power to operate,
and can only teleport one person at a time to a another, preprogrammed, dragon
teleporter. It may still receive another visitor at the same time as it sends
another traveller to a distant teleporter.

In \emph{MI:782} one of these dragon teleporters was found in the depths of
the \nameref{sec:Great Divide} by a mining expedition. It was reverse
engineered by arcane scholars and wizards and then installed in all city
kingdoms that could afford to buy and maintain one. Now they are often seated
centrally within the kingdom, and everyone can purchase tickets to be
transported to another city kingdom of their choosing. Prices for tickets
often range from five to twenty \hyperref[sec:Shin]{shins}, depending on the
prices of the city kingdom. For almost all city kingdoms the teleporters are a
net loss commercially, but they are aware that the trade and commerce (and
thus taxes) they help facilitate is invaluable to the economy of the kingdom.

This lead to the large city kingdoms to become more and more interconnected.
It helped spread the clean, high version of \emph{Teranim} to be spoken all
over the world, as a lingua franca was required to facilitate trade and
commerce.

The dragon teleporters were designed to transport living creatures (dragons)
across great distances, and are almost incapable of transporting material
goods. One teleport can transport one person plus their light equipment, or
perhaps ten kilograms of innate material (such as armour, weapons and
clothes) at a time. This design choice was deliberate by the dragons, as they
do not wear or own that much innate objects; and thus have designed their
teleporters to transport themselves safely and efficiently across vast
distances.

Many have feared that the trade by sea or land would become obsolete since
teleporters were introduced. But the dragons did not share their blueprints,
and the knowledge about the dragon teleporters is still actively researched
and reverse engineered to this day. Their design limitation and relatively
high power consumption has made them a bad economical alternative for
transporting goods.

\subsection{Global Trade}
\label{sec:Trade}

All city kingdoms and the majority of baronies mint their own coins, but they
are often only used locally. Although their value is held by precious metals
they contain, they have a strong variance in terms of metal purity, weight and
thus actual value. This made conversion of one local coin or currency to
another difficult. To make global trade and commerce easier, two new
currencies were introduces: the \emph{shin} and the \emph{shard} which are
made out of \hyperref[sec:Everblack]{everblack}.

Many banks offer a service to weigh, assess and convert local coins and
gemstones to shin and shard. And almost all baronies and kingdoms accept shins
and shards as payment method. Large purchases, such as properties or ships, are
mostly done in shards, or perhaps in bars of pure gold. Although crystal
everblack is more easily shattered and destroyed than gold, it is lighter
to carry and it cannot be as easily diluted or forged. Manipulating everblack
takes a highly skilled arcane wielder or scientific scholar, and thus cannot
be forged or manipulated like other ore or metals.

Thus many traders might simply refuse to take foreign coin, or even local
coins, preferring to deal with the more save shins and shards of everblack.
Assessing the purity of large amounts of gold or silver coins can be costly
and time consuming, and thus traders that deal with expensive items might
prefer the safety and convenience of the everblack based currency.

\subsubsection{Shin}
\label{sec:Shin}

Small chips of \hyperref[sec:Everblack]{everblack}, roughly a centimetre in
diameter - also referred to as \emph{shins} - are a common currency accepted
everywhere on the world of Aror. These are often useless for magical
applications due to their small size, but still hold material value. They are
roughly equivalent to one copper coin and are used to make everyday purchases.

\subsubsection{Shard}
\label{sec:Shard}

Bigger sticks of everblack, roughly five centimetres long, one centimetre thick
and perhaps one centimetre wide are called \emph{shards}. They are used in the
global economy of Aror for expensive purchases such as building projects,
property and artefacts. Shards are roughly equivalent to ten coins of gold.
