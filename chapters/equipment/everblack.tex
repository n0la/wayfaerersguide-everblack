\ifimages
\clearpage
\incgraph[
  overlay={\node[black] at ([xshift=0cm,yshift=+1cm] page.south)
    (main)[text width=0.9\paperwidth]{
      \large \centering
      \textbf{``Enchantment of a ceremonial blade with magical properties.''}
    };
  }
]{media/everblack.\imagesuffix}
\clearpage
\fi

\section{Everblack}
\label{sec:Everblack}

There is one thing that is unique to the world of Aror: \emph{Everblack}. It is
a pitch black crystal, almost as resilient as adamantine but harder to work
than the metal.

It is found in small quantities all over the world of \hyperref[sec:Aror]{Aror},
but especially in deep soil and embedded in the stones of the mountains. Some
places are richer in everblack than others, and entire economies are built
around mining the metal. For example the dwarves \nameref{sec:Kesmar} mine the
\nameref{sec:Cnamh Mountains} for everblack, and then sell most of it to the
other city nations. Another large quantity has been found beneath the city of
\nameref{sec:El-Fayam}.

\subsection{Mining}

It can be mined easily, as untreated and unheated it is rather brittle. However
the everblack dust that is whirled up during the process is highly toxic when
inhaled. This makes mining the brittle crystal rather dangerous for all miners
and workers involved. Early symptoms include coughing, temporary blindness,
dizziness and diarrhoea. Prolonged exposure can lead to a bloody cough,
permanent loss of the ability to perceive colours, perceived symptoms of
hypothermia, such as being cold, shivering, blue limbs and lips, and an
increased risk of heart failure. Very few miners risk working these mines
voluntarily, and thus either enslaved labour or only work these mines for very
high pay voluntarily.

\subsection{Arcane Battery}

Everblack is capable of holding and storing magical power, and is also able to
release it in a controlled matter. This makes everblack invaluable in arcane
and divine research, as well as making arcane machinery and artefacts. all
artefacts, wands and even scrolls made on Aror have trace amounts of
everblack, that holds the arcane or divine energy required to make these
magical devices work and function. Everblack is capable of holding arcane,
divine, psionic and even soul energy making it highly sought after all around
the world.

Charged chunks and pieces of the crystal are embedded in magical weapons and
armour, as well as wands and other divine and arcane artefact.

A charged everblack crystal or charged composite everblack is warm to the
touch.  It gets warmer and warmer the more power is stored within it. If it is
charged to its capacity it will begin to glow in a low, and orange light. Once
charged beyond its ability to safely store power, it will begin to emanate a
low-pitched, drone, and vibrate softly. Everblack that is overcharged may
explode if handled roughly, shattering the crystal to dust and damaging
everything and everyone caught in the explosion. However if an overcharged
crystal is left alone it will release excess magical power in the form of
light and warmth until it returns to maximum capacity.

The excess storage capacity of everblack is very high, and even small
everblack shards require almost three times the power that would make them
full to cause an explosion. The explosion of a small shard is barely enough to
damage a normal sized humanoid. Although highly expensive, everblack is
sometimes fashioned into bigger explosive devices with devastating results.

\aren{Making everblack explosives is like making catapult ammunition out of
  platinum.}

\subsection{Everblack Ink}
\label{sec:Everblack Ink}

Everblack Ink is made by crushing the crystals, mixing them with water and
boiling the resulted mixture down. It is used in the inscription of magical
scrolls, as well as runes and seals. Everblack ink is poisonous if consumed
directly, and one must be careful to avoid prolonged exposure of everblack
ink as it can be absorbed through the skin.

\subsection{Power Dampening}

An area filled with natural or artificial everblack crystal acts as a magical
dampening field. Such areas impose a natural and environmental potential for
spell failures upon everyone who seeks to cast spells within them. The
crystals redirect the magic and absorb it, often nullifying the power.
Devices and places are often fashioned deliberately out of everblack, so that
powerful witches and wielders of the arcane arts can be robbed of their power.
A direct application of this power dampening power of the everblack crystals
are \hyperref[sec:Null Stone]{null stones}.

\subsection{Composite Everblack}
\label{sec:Composite Everblack}

The crystal itself can also be hardened to incredible strengths by melting
everblack in a blast furnace to remove impurities, and then adding trace
amounts of carbon and iron. This everblack alloy, known as \emph{composite
everblack}, is then harder and denser than adamantine. This alloy does not lose
the ability to store magic, and can be used to build larger everblack crystals
and structures, as well as golems and everblack weapons.

Working with already smelted composite everblack bears no risk of poisoning the
smith. However the smelting process releases gases that are toxic if inhaled
or absorbed through the skin.

\begin{35e}{Composite Everblack as a Material}
  Weapons, armour, and shields can be fashioned out of composite everblack.
  Weapons, armour and shields made out of the composite have half more hit
  points than normal, and 50 hit points per inch of thickness as well as
  hardness 30. Composite everblack materials are always costly enough that all
  weapons, armour and shields are always made of masterwork quality.  Only
  weapons, armour and shields normally made of metal can be fashioned from
  composite everblack.

  Light armour made out of composite everblack grant a spell resistance of 14
  while worn, but costs 10.000 gp more, medium armour grant a spell resistance
  of 16 while worn but costs 15.000 gp more, while heavy armour made out of
  composite everblack, grant spell resistance 18 while worn but costs 20,000
  GP more.

  Any shield made out of composite everblack can be called upon to store
  spells that would normally target the wearer of the shield. An composite
  everblack light shield can store up to two spell levels of spells but cost
  2000 gp more, a heavy shield can store up to four levels of spells but costs
  4000 gp more, and a composite everblack tower shield can store up to 6 spell
  levels but costs up to 6000 gp more.

  Bludgeoning weapons which have a heavy steel head (such as maces), can have
  their head made out of \emph{composite everblack}, and are then especially
  effective. Their damage dice they make are then doubled. For example a
  \emph{Heavy Composite Everblack Mace} does \emph{2d8} damage instead of
  \emph{1d8}.

  Any weapon made out of composite everblack can store one spell of spell
  level three or lower within itself. Upon the next successful attack with
  that weapon the spell is released upon the target, as if it were cast on
  the target of the attack. Weapons made out of composite everblack cost
  3,000 gp more to make.
\end{35e}

\subsection{Everblack Golem}
\label{sec:Everblack Golem}

Even though everblack consumes all arcane power, it is possible to construct
golems out of composite everblack. The technique of their construction is a
closely guarded secret, known only to a few highly skilled golem engineers of
\nameref{sec:Stenheim}. They are marvels of engineering and arcane wonders but
are expensive to make. They absorb all magical energy directed at them, and
can then release it when they strike their attackers.
