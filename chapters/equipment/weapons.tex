\section{Weapons}
\label{sec:Weapons}

The people of Aror are exceptional sword makers, and thus there is a wide
variety of new weapons that you will find on offer at the smithies. Although
spears and halberds are the most common weapons used in warfare, swords are
still with both professional soldiers and civilians alike.

\begin{table*}[!htb]
  \small
  \captionsetup{labelformat=empty,font={large,bf},position=top}
  \caption{Overview of Weapons}
  \rowcolors{1}{white}{light-grey}
  \begin{tabular}{p{2.5cm}     l               p{1cm}             p{1cm}             p{1.5cm}            p{2cm}          p{3.0cm}}
    \textbf{Weapons} & \textbf{Cost} & \textbf{Dmg (S)} & \textbf{Dmg (M)} & \textbf{Critical} & \textbf{Weight} & \textbf{Type} \\
    \multicolumn{7}{l}{\textbf{Light Weapons}} \\
    Basler       & 1 shard  & 1d4  & 1d6  & 19-20x2 & ~1kg (2lbs)   & simple, finesse, piercing or slashing \\
    Rondel       & 3 shards & 1d6  & 1d8  &    20x3 & ~0.8kg (1lbs) & simple, finesse, piercing \\
    \multicolumn{7}{l}{\textbf{One-Handed Weapons}} \\
    Arming Sword & 2 shards & 1d6  & 1d8  & 19-20x2 & ~1.1kg (2lbs) & martial, slashing, or half-swording, or mordhau \\
    Long Sword   & 4 shards & 1d8  & 1d10 & 19-20x2 & ~1.4kg (3lbs) & martial, two handed, slashing, or half-swording, or mordhau \\
    Side Sword   & 3 shards & 1d4  & 1d6  & 18-20x2 & ~1.1kg (2lbs) & martial, slashing, or piercing \\
    \multicolumn{7}{l}{\textbf{Two-Handed Weapons}} \\
    Estoc        & 5 shards & 1d10 & 2d6  & 18-20x2 & ~2.0kg (4lbs) & martial, finesse, one handed, piercing \\
    Goodendag    & 1 shard  & 1d8  & 1d10 & 20x2    & ~1.8kg (3lbs) & simple, bludgeoning or piercing \\
    Man Catcher  & 2 shards & 1d4  & 1d6  & 20x2    & ~1.8kg (3lbs) & simple, non-lethal, piercing, grapple \\
    Zweihänder   & 5 shards & 1d10 & 2d6  & 19-20x2 & ~2.5kg (5lbs) & martial, slashing, or half-swording, or mordhau \\
  \end{tabular}
\end{table*}

A \emph{basler} (also baselard, Katzbalrger or short sword) is sword with
shorter blade, between 40 or 60 centimetres. It is often carried as a side arm
by soldiers who would use a pole arm or a ranged weapon as their main
armament.

A \emph{rondel} dagger (also misericorde, or stiletto) is a long knife or
dagger with a sharp and hardened tip. It is mainly used against knights wearing
full plate armour, either in the heat of a grapple or as a mercy kill to a
seriously wounded knight. The weapon is commonly used as a side weapon for
knights and soldiers, but also favoured by many assassins.

An \emph{arming sword} is a one handed sword with a blade length between 70
and 80 centimetres. It is the main armament of many knights (and thus often
called ``knightly sword''). It can be used to do half swording, or a mordhau.

The \emph{long sword} (or bastard sword) is a sword that can either be wielded
one handed, or two handed. It usually has a blade length of 80 to 110
centimetres and is the favoured weapon of many advanced swordsmen and -women,
as well as veteran soldiers and mercenaries. Much like the arming sword it
can be used for half-swording or performing a mordhau.

A \emph{side sword} is a transitional sword between an arming sword and a
rapier. It has both a wider blade and an edge for cutting, and a hardened
point for thrusting. This kind of sword is very popular in cultures with
extensive focus on fencing, such as \nameref{sec:Avenfjord}.

An \emph{estoc} is a long two handed sword that is used for thrusting. Many
estocs do not have an edge, and its primary role is to defeat full plate
armour. It is often used with two hands, although it can also be wielded with
one.

A \emph{goodendag}, is a hefty two handed club made out of wood. The end of
the club is reinforced with a metal cap, which also secures a large iron spike
that protrudes vertically from the end of the club. It can thus be swung with
both hands as a club, or used as a makeshift spear. The spike and the metal
cap at the end of the goodendag can also be used to parry blows. The weapon
was invented by the militias of \nameref{sec:Tredegar} who patrolled the
kingdom's many rural roads, and greeted travellers either with a hearty
``goodendag'' (good day in local dialect), or a massive hit from the end of
their weapons. The weapon quickly became highly popular with bandits, militias
or recruits as it was easy and cheap to make, but was also easy to use, and
highly effective against both armoured and unarmoured opponents.

\begin{35e}{Goodendag}
  The goodendag deals 1d10 points of bludgeoning damage, and can also be used
  as a makeshift spear in which case it deals 1d6 points of piercing damage.
\end{35e}

The \emph{Man Catcher} is a long pole weapon with two metal prongs at the end.
Some have a spring-based contraption that allowed the limbs or necks to enter
the prongs, but not leave it, while others have a rope based contraption that
allows the wielder to tighten a noose. The main purpose of the weapon is to
catch humanoids or humanoid-like monsters without seriously harming them. Most
man catchers also feature sharp edges or spikes to allow the wielder to coerce
the victim into moving. They are mostly used in a law enforcement role to
catch violent criminals, transfer prisoners or subdue feral animals. Man
Catchers also see use in castle defence where they are especially useful
against attackers that scale castle walls. Slavers also favour this weapon to
catch, subdue and move their victims.

\begin{35e}{Man Catcher}
  A man catcher can deal non-lethal damage without the wielder having
  incurring a -4 penalty to his or her attack roll. After a successful hit
  with a man catcher, the attacker can opt to grapple his opponent (with a +4
  bonus) as a free action, without needing to make a touch attack or provoking
  an attack of opportunity from the defender.
\end{35e}

A \emph{Zweihänder} (also bihander, great sword or claymore) is a huge sword,
between 1.4 and 2 metres, that can only be wielded with two hands. It is a
dedicated weapon of war, and is often used by front line troops to break
through shield walls defended by pole weapons. Much like its smaller cousins
it can still be used with half-sword techniques, and to perform a mordhau.

\emph{Half-Swording} refers to a technique in which the off hand grips the
central blade (or just beyond the cross guard) to allow for greater force and
accuracy while thrusting. The main purpose of this technique is to defeat
armoured opponents who would otherwise be well protected against cuts.

The \emph{Mordhau} (literally ``death hit'') is a technique where sword is
held inverted with bonds on the blade, and striking the opponent with either
the pommel or the cross guard. This technique allows the wielder to use the
sword to deal blunt damage to an opponent. It is especially effective against
opponents wearing strong armour (such as thick natural armour, plate) that
would otherwise be well protected against cuts.

\begin{note}
  Half-swording attacks deals piercing damage, and a mordhau attack deals
  bludgeoning damage. You can switch between normal stance, half-swording, and
  mordhau stance as part of a move action. If you have the \emph{Quick Draw}
  feat, you can move between these stances as a free action.

  Mordhau deals less damage, as if the sword was one size category
  smaller. For example a medium sized long sword would deal 1d8 points of
  bludgeoning damage, compared to 1d10 points of damage when cutting.
\end{note}
