\section{Demon Bone}
\label{sec:Demon Bone}

Demon bone (or ``scourge bone'') is the organic, yet sturdy and strong
material the \nameref{sec:Scourge} produces to add structural integrity to its
structures and growth patterns. The bones of the demons spawned by the scourge
also contain this material, hence the name ``demon bone''. Demons often shape
weapons out of such bones, which they scrounge from the corpses of other
demons. The material of the demon bone is very hard to shape and brittle, but
can easily be sharpened and always has a rough and jagged edge, that causes
horrific wounds and cuts, that are difficult to heal and treat.

Demon bone cannot be made, created (even by spells), or mined, and must thus
be harvested either from dead demons, or by deconstructing the growths of the
scourge.

\begin{35e}{Demon Bone}
  Weapons can be fashioned out of demon bone. The material is sadly too brittle
  to make armours and shields out of it.

  Slashing weapons can have their blade made out of demon bone, and are
  especially effective. Their damage dice is then doubled. For example a
  \emph{Longsword} made out of demon bone does \emph{2d8} damage instead of
  \emph{1d8}. On a critical hit these weapons cause a lasting wound that does
  2 points of damage each round until the wound is treated with a healing
  spell, or with a DC 15 \emph{Heal} check. Making a demon bone weapon adds
  2,000 gp to the crafting cost, and the weapon must be of masterwork quality.
\end{35e}
