\section{Armour}
\label{sec:Armour}

Aror has some specific armours that you might find here, but not anywhere
else.  Typically armour on Aror is crafted to be lightweight, protective, but
also cost effective. Armour comes in many shapes and sizes, depending on the
culture where it was made, the skill of the armourer as well as the depth of
the owners pockets. Higher grade armour, such as breast plates or full sets of
plate armours are also status symbols, and often lavishly embellished and
engraved.

\begin{table*}[!htb]
  \small
  \captionsetup{labelformat=empty,font={large,bf},position=top}
  \caption{Overview of Armours}
  \rowcolors{1}{white}{light-grey}
  \begin{tabular}{p{2cm} l p{1cm} l l l p{1cm} p{2.5cm} l}
    \textbf{Armour} & \textbf{Cost} & \textbf{Armour Bonus} & \textbf{MDB} & \textbf{ACP} & \textbf{ASF} & \textbf{Speed Penalty} & \textbf{DR} & \textbf{Weight} \\
    \multicolumn{9}{l}{\textbf{Light Armour}} \\
    Gambeson        &   2 shards  & 2 & 8 &  0 &  5\% & -                  & 2/bludgeoning                  &  2kg  (5 lbs.) \\
    Jack of Plates  &   5 shards  & 3 & 7 & -1 & 10\% & -                  & 2/bludgeoning, and 2/piercing  &  4kg  (9 lbs.) \\
    Mail Shirt      &  10 shards  & 4 & 5 & -2 & 15\% & -                  & 5/slashing                     &  8kg (17 lbs.) \\
    \multicolumn{9}{l}{\textbf{Medium Armour}} \\
    Hide Armour     &   1 shard   & 3 & 4 & -3 & 20\% & -                  & -                              & 12kg (25 lbs.) \\
    Coat of Plates  &  10 shards  & 5 & 3 & -4 & 25\% & -                  & 4/bludgeoning, and 2/piercing  & 15kg (33 lbs.) \\
    Brigandine      &  30 shards  & 6 & 4 & -3 & 30\% & -                  & 4/bludgeoning, and 4/piercing  & 14kg (30 lbs.) \\
    \multicolumn{9}{l}{\textbf{Heavy Armour}} \\
    Breastplate     &  60 shards  & 7 & 3 & -4 & 25\% & -10 ft. / -5 ft.   & 5/slashing, and 5/piercing     & 20kg (44 lbs.) \\
    Plate Armour    & 150 shards  & 9 & 1 & -6 & 35\% & -10 ft. / -5 ft.   & 10/slashing, and 5/piercing    & 25kg (55 lbs.) \\
  \end{tabular}
\end{table*}

A \emph{gambeson} (aketon, padded jacket or arming doublet) is heavily padded
defensive jacket, which was either worn on its own, or in combination with
mail or plate armour. It serves as a padding to soften incoming blows and
offers some resistance against cuts. While meant as a padding beneath actual
armour, it is also often used a stand alone armour especially with those that
cannot afford better armour.

\emph{Jack of Plates} is a lightweight padded vest (or a converted gambeson)
into which thin metal plates have been sewed. It is often fashioned from a
gambeson to give it better resistance against thrusts and cuts. Since a jack
of plates is rarely made by a professional armourer, they are of lesser quality
and thus provide less protection. However is extremely popular with civilians,
militias and bandits due to its low cost.

A \emph{Mail Shirt} (also chain mail, chain shirt, or hauberk) is a shirt,
vest or jacket made out of small interlocking metal rings. It provides
excellent protection against cuts, but is also more expensive to
manufacture. They do come in a wide variety, as some feature protection for
the groin, upper arms, or sometimes even have long sleeves, while others only
protect the central torso. It is often worn over a gambeson.

\emph{Hide Armour} is a catch-all term for all sorts of home made armours
that are used by various brigands, barbarians, raiders, and bandits. It is often
fashioned out of leather, fur, linen and hide. Since barbarian armours often
feature the hide and fur of trophy animals (such as bears or wolves) these
armours are often simply referred to as ``hide armour''.

\emph{Coat of Plates} is a professionally crafted jack of plates, whose steel
plates are thicker, and cover all of the torso and the groin. While jack of
plates tend to have gaps in their plating due to limited craftsmanship, a coat
of plate was professionally crafted to carry several large and heavy steel
plates to protect the torso, and also sometimes the groin and the upper legs. A
coat of plates can also be worn over a gambeson.

A \emph{Brigandine} is a refined version of the coat of plates, which instead of
a few large steel plates uses many small oblong steel plates riveted into the
fabric for enhanced manoeuvrability. It is in all aspects superior to a coat of
plates, but also more expensive to manufacture. A brigandine can also be worn
over a gambeson.

\emph{Breastplate} (chest plate or cuirass) is a heavy armour worn at the
torso. It is a part of a full steel plate armour set, covering the front and
back of the upper body. Many also feature tassets that protect the groin and
the upper legs. The breast plate acts as additional defence of the torso,
in combination with mail which is worn beneath the plate.

\emph{Plate Armour} is a full suit of armour entirely encasing the wearer. It
features a breast plate for the torso, and several smaller pieces protecting
the legs, arms, hands and feet. Special round discs called rondel were affixed
to vulnerable spots in the armour (such as joints) to ward off thrusts and
piercing attacks. It is often worn over simple garments, and clothing. It is
by far the most expensive armour, and is often only reserved for nobility,
knights, as well as kings and queens. Although the protection from a full suite
of plate is superior to any other armour, it does have weak spots (e.g. at
joints) and is susceptible to bludgeon attacks.

\begin{note}
  If an armour states that it can be worn over another, that is exactly what
  the players can do. So a player can buy a gambeson, and then wear a hauberk
  over it. The effects do not stack, but the player can magically enhance each
  of these armours with different enhancements and enchantments.

  Normal restrictions regarding medium armour, as well as restrictions regarding
  plate armour (which is equivalent to a ``Full plate'') apply as normal.

  Damage reductions that affect the same damage source (e.g. slashing) do not
  stack, only the highest applies. Damage reductions for different damage types
  do indeed stack.
\end{note}
