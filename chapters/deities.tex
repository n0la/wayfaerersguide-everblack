\chapter{Religions and Deities}

\section{Religion on Aror}
\label{sec:Religion}

The deities of Everblack have been divided into two major groups by most of
the divine scholars: Actual deities, and powerful extra-planar or
extra-dimensional creatures that either pose as such, or have enough power to
enforce their will as if they were deities.

\textbf{True deities} are manifestations of abstract ideas and higher
metaphysical concepts, that grant those that follow their ideals with
power. These deities \emph{never} speak to their followers directly, cannot be
``visited'' on some obscure plane, will never visit the world in turn as an
``avatar'', and can also not be reasoned with. And you may either chose to
align yourself with their ideals, question their ideals and power, or you
choose to ignore them completely.

However it is a strange fact of live that those that do align their lives,
ideals, and value systems according to specific dogma that manifests these
abstract ideas and concepts in the world are granted seemingly divine favour
and power. Some become paladins, holy knights with an arcane power to heal,
while others are granted a vast array of powerful arcane spells at their
disposal. Deviate from the true deity's path, and you risk losing that
power. Followers of these gods have collected thousands of years of history,
rituals, teachings and dogmas and often brought them under one roof as a
religious order or church devoted to said true deity. While following the
proven and existing dogma will lead to reward from the true deity, it is never
the \emph{only way} to elicit favour from said deity. Often new
interpretations of the ideals and value systems are discovered to be as
favoured by the true deity as the tried and true older dogma. This often leads
to schisms within religious orders and institutions, as well as new orders
being founded based off the new set of dogma. \emph{Nyddwr}, a goddess of
olden times, is an example of an abstract ideal manifest as a religion.

On the other hand \emph{Aror} is under the influence of \textbf{lesser
  deities}, a few powerful extra dimensional or extra planar entities, that
pose as gods. Most of these are malevolent and often twist their followers
into doing evil. While the true deities seem eternal, these entities are prone
to disappearing, or having their power usurped by other entities. Although
followers of the true gods often look down on the follower of these entities,
the divine power granted by these entities cannot be denied. It often rivals,
or even surpasses the powers granted by true deities. \emph{Aria} is a prime
example of a powerful extra dimensional entity that is worshipped as a god
among many people on Aror.


\section{True Deities}
\label{sec:True Deities}

%% Forun
\ifimages
\clearpage
\incgraph[
  overlay={\node[black] at ([xshift=0cm,yshift=-1cm] page.north)
    (main)[text width=0.9\paperwidth]{
      \large \centering
      \textbf{``Two halfling priestesses of Forun preach to their followers during the Day of Candles.''}
    };
    \node[black, below=of main,yshift=1cm]{
      \nameref{sec:Helmarnock} circa GT:2102
    };
  }
]{media/forun-dayofcandles.\imagesuffix}
\clearpage
\fi

\subsection{Forun}
\label{sec:Forun}

A true deity, \emph{Forun} represents warmth, kindness, fire, beauty, and
youth. She represents the warmth in cold places, both physically and
spiritually, as well as the destruction that fire brings and the ashes it
leaves behind that aid the rebirth. Forun thus also represents the concept
that sometimes, something has become so old, unmovable, or even corrupt that
it has to be burned down to allow for something new to grow. She is closely
associated with spring.

Forun is also known as \emph{Elora} (``lady of fire'') among the
\nameref{sec:Inua}, and as the \emph{young mother} among the followers of
the \nameref{sec:Old Ways}.

Many followers and priests personify Forun as a woman with fiery, waving, and
flaming red hair. She is often depicted as a loving and caring mother giving
warmth to her children, as well as the ever burning fire within everyone that
is capable of love, and passion.

\subsubsection{Prevailing Dogma}

The main church of Forun builds small temples, centred around a large brazier
that must be lit at all times. Priests and priestess of Forun spend their
lives serving others, offering their divine power to aid healing of the sick
and wounded, as well as offering shelter and warmth to those that have
neither. The church of Forun can be found almost everywhere on Aror, and their
followers as numerous as they are liked and loved by the people.

The church of Forun has entrenched itself as a main source for culture and
tradition in many places of Aror. Forun's holy days are celebrated in places
such as Forsby or Helmarnock. The church celebrates two major holy days: The
Day of the Winter's Flower, and the Day of Candles.

\subsubsection{Day of Winter's Flower}
\label{sec:Day of Winters Flower}

The \emph{Day of December Flower} is celebrated on the day of first frost or
snow in the coming winter, and thus varies from region to region. A large
bonfire is built and lit, and people are encouraged to dance and celebrate one
last time before the harshness of winter covers the land. The festivity is
officially over when the bonfire no longer burns, and thus heralds the final
arrival of the cold season.

\subsubsection{Day of Candles}
\label{sec:Day of Candles}

The \emph{Day of Candles} is an unspecified day in spring where the community
is encouraged to light torches, candles and oil lamps in their windows. The
day is announced by the priest, and people bring their candles and lamps to be
blessed in a morning mass. Each lamp or candle is supposed to welcome the
spring, as well as remember any family member or friend which has not survived
the recent winter. These lights are then affixed to windows, walls or balconies
for all to see and kept lid for the entire day and night, often completely
illuminating the night until the morning.

In Forsby the lights are then hung outside the cliff side houses, and can then
be seen from the bay, illuminating the stone wall of the cliff. In Helmarnock
the lights are attached to the bridges connecting the islands, making the
central forum and bridges dance in soothing orange light.

Whereas in Helmarnock they are fixed to the bridge that connects the islands
together. During the night the bridge is the beautifully illuminated, and many
people visit it to pray and remember their day.

\subsubsection{Relations}

The goddess itself, as well as the main church of Forun are popular all around
the globe, due to their caring and selfless attitude. In many large city
kingdoms the church of Forun has been a mainstay of society and culture for
hundreds of years.

\begin{35e}{Forun}
  Forun is considered neutral or chaotic good, and her favoured weapon is the
  unarmed strike. Her domains are good, chaos, fire and destruction.
\end{35e}

%% Nyddwr
\clearpage
\incgraph[
  overlay={\node[black] at ([xshift=-5cm,yshift=+0.2cm]page.south east)
    {\large \textbf{Temple to Nyddwr in \emph{Forsby}, circa GT:2101}};
  }
]{media/nyddwr.png}
\clearpage

\subsection{Nyddwr}
\label{sec:Nyddwr}

Nyddwr is a true deity, and the goddess of fate, history, and time. She is the
patron of historians, archivists, and everyone who seeks to interpret the
past, the present and the future. She is considered the most ancient of all
the true deities, and depictions of her date back hundreds of thousands of
years to the earliest human civilisations. She is closely associated with
summer.

\subsubsection{Personification}

Many personifications of Nyddwr exist, but the most predominant is that of a
six armed female. Her arms are coloured by dried paint to represent the three
stages of time: the lower arms are coloured black, and stand for the distant,
often horrible past. The middle pair of arms are coloured red to symbolise the
often dangerous present, while the upper pair of arms are coloured white to
represent a bright future.

\subsubsection{Prevailing Dogma}

She favours anyone who is interested in analysing and learning from the past
to enact positive change in the present that ultimately lead toward a better
future. Nyddwr also favours people that value history, and those that share
the wisdom learned from it with others. This often puts followers of Nyddwr
in direct conflict with those of \emph{Aria}.

\subsubsection{Seers}

Priestesses of Nyddwr are called \emph{seers}. Seers only accept female
applicants, and there are always three in one group or temple. Some seers
travel the world, while others attend shrines and temples within large
cities or in secluded places of contemplation. Much like the trinity of
their goddess, each seer represents one aspect of time. They are also
required to carry five holy possessions at all times:

\begin{itemize}[noitemsep]
  \item Either red, black and white powder to use as face and body paint. One
  seer represents the past (black), one the present (red) while the other
  represents the future (white).
  \item A small dagger or knife, used as a tool and weapon to defend themselves
  and others.
  \item Either a black, red and white ceremonial robes a seer has to craft
  herself. This does not bar her from wearing more clothes beneath, if the
  weather demands it.
  \item Ornament necklace that also acts as a divine focus and prayer bead.
  \item Mortar and pestle used to crunch the coloured powder with which the
  seers must mark the people they granting their wisdom.
\end{itemize}

The three seers are required to always remain at each others side. They will
enter a town together and offer their services and wisdom to everyone that
seeks it. It is customary to offer seers of Nyddwr food and shelter in return,
which they will accept in exchange for sharing their wisdom. However seers of
Nyddwr are not allowed to amass wealth.

When performing the ritual of guidance, the prospect must kneel in privacy
before the seers, and then explain his past to the black seer. She will
identify events and emotions that might linger still, barring the prospect
from moving onward in his life. She will give guidance on how to overcome
these unresolved issues of the past. Once she has done so, she will mark the
prospects head with black paint. Then the prospect may ask the red seer about
guidance about current problems and troubles. In consolidation with what she
has heard about the prospects past, she will outline immediate changes the
prospect can affect in his or her life to improve it. She will then mark the
prospect's head with red paint. Last but not least the white seer, often the
most wise and intelligent, will attempt to give the prospect both a reading of
the future, as well as outlining possible goals the prospect should be working
towards. Once the ritual of guidance is complete, she will mark the prospect's
head with white paint.

\subsubsection{Relations}

The goddess itself, and her followers are well respected among most
civilisations. Most city kingdoms have a temple dedicated to her, and in almost
all it is a major offence to attack wandering seers. Her seers are even welcome
among the more savage tribes of Iâfandir. Among fighters and paladins of
\emph{Lor}, \emph{Order} and even the \emph{Three Kings} it is considered an
honour to escort seers on a pilgrimage to their destination.

\begin{35e}{Nyddwr}
  Nyddwr is considered Neutral Good, and her favourite weapon is the dagger.
  She is favoured by bards, skalds, scholars, archivists and wizards.
\end{35e}

%% Marwaid
\subsection{Marwaid}
\label{sec:Marwaid}

\emph{Marwaid}, is a true deity, and represents both abstract and literal
sacrifice, family and community. Along with \nameref{sec:Nyddwr} and
\nameref{sec:Forun}, who are often described as sisters in the old stories,
Marwaid is among the oldest true deities worshipped by the humanoid races. She
is associated with autumn.

\subsubsection{Personification}

Like her sisters she is often depicted as a woman, especially as a mother who
was willing to accept that her grown children would sacrifice themselves in an
effort to make life better for everyone. This is often represented as a
wailing mother smothering her dead child who appears to have died in
battle. But in the tribes that still follow the old ways, she is rarely
directly personified, but instead worshipped through specially made stone
altars.

\subsubsection{Dogma}

Most followers of Marwaid follow the old ways, meaning they live in small
villages and tribes far away from civilisation. In these hostile and dangerous
regions - where strife against monstrous races, food shortages, war, monsters
and disease are common - Marwaid favours anyone who is willing to sacrifice
themselves for others and the common good. She favours hunters and farmers
that go hungry to feed the children, warriors that hold their ground to let
the weak, young and elderly escape, and those that sacrifice the now for a
better future, for example by stockpiling food, and use vital resources
sparingly to ensure there is enough for future generations.

She is often explained as having an erratic will and often tests her trusted
followers. Those that were forced to sacrifice - for example by losing loved
ones to sickness or famine - often see no pattern or purpose in their own
suffering and then attribute it to Marwaid's fickle and unpredictable nature.

\subsubsection{Shrines of Marwaid}

Most shamans of the old ways follow her, and build shrines to her worship.
These are often situated in clearings or in the centre of small towns, and are
large painted standing stones adorned with personal belongings that the
followers sacrificed. Shamans and priests of Marwaid perform ceremonies where
followers either offer either abstract sacrifices, in form of promises and
pledges, or literal sacrifices, in the form of personal belongings, food, life
stock and - albeit rarely - humanoid sacrifice to these stone shrines. These
sacrifices are then accepted by the shaman on her behalf, and are then added
and standing stone as ornaments and decoration.

This often gives the shrines of Marwaid a rather grim appearance, as they are
adorned with skulls, bones, spoiled food, and perhaps even the remains of
humanoid bodies, alongside with personal affects such as weapons, necklaces,
tools, and clothing. Those that take from the shrines are said to be cursed
until they sacrifice something of importance to the very shrine they stole
from.

\subsubsection{Relations}

The goddess of Marwaid is often said to be related to the other three
female primordial true deities, \nameref{sec:Forun} and \nameref{sec:Nyddwr}.
Her followers are well respected among the followers of the old ways, and
those living on the country side. However worship of her have waned in the
large city kingdoms were resources are in abundance and thus ritualised
sacrifice are rarely required to ensure a prosperous future. This often gives
her followers grounds to berate the ``city folk'' for being spoiled and having
lost their strength that comes with the struggle and replaced it with comfort
and security.

\begin{35e}{Marwaid}
  Marwaid is considered as Chaotic Good or Chaotic Neutral, and her favourite
  weapon are the unarmed attack and the quarterstaff. She's favoured by rangers,
  barbarians, shamans and those that have a tormented life, such as slaves.
\end{35e}

\ifimages
\clearpage
\incgraph[
  overlay={\node[black] at ([xshift=-8cm,yshift=+0.5cm]page.south east)
    {\large \textbf{Shrine to Marwaid somewhere in the \emph{Dirgewood}, circa GT:2102}};
  }
]{media/marwaid.\imagesuffix}
\clearpage
\fi

%% Old Ways
\subsection{Old Ways}
\label{sec:Old Ways}

The \emph{Old Ways} are not a god or deity, but instead a set of believes,
traditions, and stories that represent how the ancient humanoids worshipped
the three major deities of Aror. A few minor deities also appear in this
religion, such as \nameref{sec:Morana}.

\subsubsection{Three Mothers}
\label{sec:Three Mothers}

The core of the religion are three female deities that are worshipped as the
\emph{three mothers} of all humanoid races. Each of them represents one stage
of a woman's development. \nameref{sec:Forun} represents the young, hopeful,
and beautiful woman that raises her children with warmth, compassion and
love. She represents youth, beauty, fire and fertility. \nameref{sec:Marwaid}
represents the ageing mother, that has lost children, and sacrificed
everything that she has for her young. She is seen as the hardened, stern and
often embittered woman that raises her children to withstand the harsh
realities of life. \nameref{sec:Marwaid} also represents the strong fighter
within each person, that would fight to the bitter end to protect their
children and family. \nameref{sec:Nyddwr} represents the old woman, the crone,
that offers her immense wisdom to aid her already fully grown children and
family. She represents the tempered, wise, yet hardy old woman that survived
against all odds, and now shares her wisdom with others.

For a while there was a fourth mother, \nameref{sec:Morana}, who stood for the
dead mother, that still protect their children from beyond the afterlife.
After it was revealed that Morana had gained her status among the true deities
through lies and deception, she was cast from her role as one of the main
mothers. She is now vilified in the stories of the Old Ways, as the
\emph{great betrayer}.

\subsubsection{Stories}

\graham{I always had nightmares of being told ``The Eyeless Man of the Cave''
  by my mother.}

\aren{We had the same story, but called it ``The Lone Ilian''. The stories of
  the Old Ways survive in many cultures, and will continue to scar and
  traumatise children for centuries to come.}

Another important part of the Old Ways are how the knowledge, values propagate
and continue to live on: stories. A collection of mystic stories, parables,
and myths that are passed orally from one generation to the other. They often
have several purposes, but most stories are told to children to explain to
them the dangers, beauties, and also the horror stories of the world. The
collection of all these stories and myths is called the \emph{old prose}.

The stories also contain heroic accounts of heroes of old. Bold tales about
people that ventured forth to slay beasts, save the innocent, become kings, or
return a powerful magical artefact back to save the village from certain doom.
But the stories also contain horror stories that end badly, such as the new
mother that goes to the woods only to be eaten by a werewolf.

The most important part of these stories however is to teach the core values of
the Old Ways. The most important being dedication, loyalty and honour to your
own clan or tribe. Each member of a clan should give what he can, and take as
little as he requires. This not only extends to love, life, family but also
to nature, with which every follower of the Old Ways must strive to respect.

Some of these stories also tell of failings, crimes, and their appropriate
punishment. While most of these stories tell of compassion for minor offences,
they lay out brutal punishments for serious crimes and even include the death
penalty or slavery for severe crimes such as murder or rape. The same stories
also tell of just rulers, their heroic behaviour and their favourable
personality that made them beloved by their followers.

However these old stories also lay down a strict social construct that often
portraits women as responsible for the household and family, while the man is
supposed to hunt, fight and protect. These structures are then reinforced by
heroic hero stories - who are more often than not - male, and by the stories
of the three mothers whose motherly attributes serve as role models for women.
While heroines and thus precedent for a balanced and fair society exist - such
as the great huntress Eigyr that slew more beasts than she could eat, or the
powerful witch Gweneth that saved her village from a terrible plague - many
tribes that follow the old ways still see the woman's place with the family.

\subsubsection{Incantations}

The Old Ways do not have priests, but witches and shamans as spiritual
leaders. Also the \emph{old prose} contains incantations that may be cast by
anyone versed well enough in the old stories and \emph{Ancient Teranim}. These
incantations are divine magic, and quite powerful. However they often require
hours of preparations, several people performing the incantation together,
complicated chants and songs in \emph{Ancient Teranim}, and sometimes even
live sacrifices to the three mothers to succeed. Practising witches and
witchers of the old ways are often the spiritual centre of a clan or
community, and tasked with gathering, teaching and performing these old
incantations when required.

%% Order
\subsection{Order Above Chaos}
\label{sec:Order}

The \emph{Order Above Chaos} or simply the \emph{Order} is a true deity, that
represents order and justice, regardless of prevailing law. It is seen as the
\emph{higher order of all things} intrinsic and true to all societies. The
fundamental laws and rules that govern all species, from which no one can
escape, and that which thrones above everyone and everything, even kings.

\subsubsection{Personification}

The Order Above Chaos is often depicted as some form of being hovering over
everything that lives below in the earthly domain. More often than not the
Order is not personified at all, and instead represented by a dagger pointing
downward to form a cross. This dagger pointing downward is also worn by all
members and believers as a tattoo, and has become a widely recognised symbol
of the religion.

\subsubsection{Prevailing Dogma}

The \emph{Order Above Chaos} has two major churches and dogmas: The
\nameref{sec:Five Holy Orders}, and the \nameref{sec:Holy Church of Aleaste}.
Both follow the same basic tenets of an order above chaos, but differ
substantially in the amount they are allowed to interfere in local affairs.

The dogma states that all things must submit to a higher state of order. All
social structures created by intelligent creatures (such as humanoids or
intelligent monstrous races) will only work in the long term if they orient
themselves towards that higher order which regulates a peaceful and productive
together. Corrupt, heavily fascistic, collectivised or socially unfair social
structures stray from that ideal and will thus inevitably fail and fall
apart. All members believe that only a fair and open system of government, which
is lead by a fair, competent and just ruler that listens to concerns of his
subjects, will ultimately succeed in bringing long term stability by keeping
chaos at bay. This higher order is above all things, such as kings, emperors
or even lesser deities.

This higher order not only applies to social systems as a whole, but also to
individuals. The followers believe that every individual being should strife
towards that higher order in their personal life, and will thus, inevitably
also contribute to the higher order of the social structure they are a part
of by aligning themselves towards its virtues. Individual virtues valued by
the Order Above Chaos are conscientiousness, honesty, modesty, strength in
both mind and body and forgiveness.

Most churches and interpretations vary greatly by the tools available to those
that wish to seek out and destroy corruption. Some argue that only the virtuous
themselves are allowed to counter chaos, while others argue that some amount of
chaos is required to fight chaos itself. Discord among the believers is so
great in this regard that it lead to a schism, which split off the Holy Church
of Aleaste from the Four Holy Orders.

\subsubsection{Rivalry with Lor}

The followers of the Order are in a dialectic conflict with the followers of
the lesser deity of \nameref{sec:Lor}. Their argument is that the entity known
as Lor represents chaos disguised as a just crusade, and does nothing to
maintain the order of things in the long term. In return the followers of Lor
accuse the followers of the Order to indulge in vengeance, and vigilante
justice instead of seeking true and lasting order. Other followers of Lor
claim that he is the very embodiment of the ``just ruler'' that the Order
Above Chaos predicts, while followers of the Order counter that ``true order''
stands even above powerful entities such as \nameref{sec:Lor}.

\begin{35e}{Order}
  The \emph{Order Above Chaos} is considered Lawful Neutral, and their favoured
  weapon is the dagger.
\end{35e}

\FloatBarrier

\clearpage
\incgraph[
  overlay={\node[black] at ([xshift=0cm,yshift=-1cm] page.north)
    (main)[text width=0.9\paperwidth]{
      \large \centering
      \textbf{``The concept of the \emph{Order Above Chaos} as illustrated in
        a mural in the cathedral of \emph{Stenheim}.''}
    };
    \node[black, below=of main,yshift=1cm,xshift=-6cm]{
      \nameref{sec:Stenheim} circa GT:2101
    };
  }
]{media/order-colour.png}
\clearpage

%% Sea Priestess
\subsection{Sea Priestess}
\label{sec:Sea Priestess}

The \emph{Sea Priestess} is a proposed true deity that was perceived by
\nameref{sec:Graham Balance}, to be a new goddess of death and decay. She is
often portrayed as a pale, white haired woman with blue lips, inhabiting
bodies of water.

After studying the ancient texts, stories, and lore about the \nameref{sec:Old
  Ways} Graham concluded that the position of goddess of death was usurped by
Morana, and there had always been a fourth mother since the ancient times.
Graham did not learn the name of the ancient goddess, but instead proposed a
new one: ``The Sea Priestess''.

The Sea Priestess is a mythological woman that dwells deep beneath the
\hyperref[sec:Soul Well]{sea of souls}, where she shepherds the dead to
return to the endless sea as rain does to the ground water. She also speaks
mystical warnings, reminding the living about their own mortality, and how
careless acts may jeopardise others. According to Graham's treatise, she
accepts the dead that are properly buried, either in the soil, on water, or
through fire. She opposes most soulless undead such as skeletons, zombies or
ghouls, but accepts intelligent undead and soul magic but warns caution in
those areas.

After Graham had published the treatise on his new proposed religion, very
few people took it seriously. In the early days, and during the rest of his
lifetime worship of the sea priestess was limited to him, and his closest
friends. He continued to publish books, songs, and poetry about the sea
priestess, expanding her lore by adding much of his personal philosophy into
her teachings. For much of the late decades of GT, and early decades of MI
after Graham's death, the followers of the ``sea priestess'' steadily grew.
Many saw her as a ``goddess for disbelievers'', while some flocked to the old
ways after the tragedy surrounding the lesser deity \nameref{sec:Griannar}.
In MI:210, five hundred years after Graham's death, the first priest following
his practices in worship of the Sea Priestess received holy power through
divine magick.

Ever since it has been unclear whether the Sea Priestess truly is a true deity
of Aror, or whether yet another lesser deity saw its chance to impersonate
one. So far all indications point to her being a true deity, while many remain
sceptical, especially since \nameref{sec:Morana}'s great betrayal.

\begin{35e}{Sea Priestess}
  The Sea Priestess is considered neutral good, and her favourite weapon is
  the long bow. She accepts soul casters, and intelligent undead as followers.
\end{35e}


\section{Lesser Deities}
\label{sec:Lesser Deities}

%% Aria
\subsection{Aria}
\label{sec:Aria}

\aren{Traitor...}

\emph{Aria} is the lesser deity of secrets, intrigues, knowledge and hidden
things. She is often depicted as a woman clad hiding her face from onlookers.

\subsubsection{History}

Aria was once a powerful priestess of the \nameref{sec:Silent Queen}, before
she challenged the queen's reign during the conflict against
\nameref{sec:Griannar}.  During that challenge it is said that she was
responsible for killing the Silent Queen and usurping her domain, power and
rule over the extra-planar realm where the Silent Queen resided.

\subsubsection{Well of Truth}
\label{sec:Well of Truth}

She openly encourages hiding dangerous knowledge in hidden libraries and
archives. She, and her followers, believe that some knowledge is just too
powerful to be left in the hands of mere mortals. The followers of Aria
that go out and seek such knowledge to lock away, are called the \emph{Well of
  Truth}. This knowledge may include powerful artefacts and magical
techniques, that are deemed to dangerous. This puts her followers often in
direct conflict with most \hyperref[sec:Soul Magic]{soul magic} practitioners,
as well as those following the \hyperref[sec:Runemaster]{runemaster}. Followers
of the well do not research knowledge themselves, but instead see themselves
as guards against dangerous knowledge in the wrong hands. They are known for
stealing research from scientists and wizards, as well as killing researchers
so that their secrets may die with them.

\subsubsection{The Puppeteer}

Not only does she encourage the gathering of knowledge and information, she
also openly encourages her followers to use said knowledge for personal gain.
Many of her followers are thus often those that seek to control society from
the shadows, while amassing wealth and power in secrecy.

\subsubsection{Relations}

Aria is in direct conflict with the \nameref{sec:Runemaster}, as he gifts
powerful magic and teachings to mortals. She also openly opposes anyone who
seeks to investigate unethical practices such as necromancy. The
uncompromising methods of her followers, such as theft or outright
assassination, brings her and her followers often in direct conflict with the
\nameref{sec:Order} or the knights of \nameref{sec:Lor}.

\begin{35e}{Aria}
  She considered Lawful Evil and her domains are knowledge, travel, magic and
  trickery. She considered the patron of thieves, wizards, librarians,
  archivists, and researchers. Her favoured weapon is the short sword.

  There are two feats associated with Aria: \featref{Adept of Aria} and
  \featref{Well of Truth Agent}.
\end{35e}

%% Forneus
\subsection{Forneus}
\label{sec:Forneus}

Forneus is the name given to an otherwise unnamed, extraordinarily powerful
devil from the hellish planar realms. He is often often worshipped as the god
of magic, arcane craftsmanship, and the study of languages and forgotten texts.

His followers are often wizards, scholars of the arcane arts, as well as those
who wish to craft arcane weaponry, artefacts and machinery. He often strikes
deals with mortals through his minions, and in exchange for powerful magical
artefacts he offers arcane, and spell casting services.

He is often depicted as a huge winged arch-duke of hell, but in reality he
rarely leaves his domain, preferring to send minions (often erinyes) to
strike deals.

Even though many other religions, such as \nameref{sec:Lor}, consider him evil
very few of his minions are considered evil. He does not tempt them to do
evil, and more often than not simply seeks to gain magical artefacts in
exchange for spells, arcane knowledge and arcane services. His followers openly
embrace all forms of magic, including necromancy, which often brings them in
conflict with other religions and deities.

\begin{35e}{}
  Forneus himself is \emph{lawful evil}, but he accepts anyone as a follower
  that seeks to simply improve their own magical powers and prowess. His
  followers may be of alignment. His favoured weapon is the dagger, and his
  domains are arcane, knowledge, travel and fire.
\end{35e}

%% Griannar
\subsection{Griannar}
\label{sec:Griannar}

\emph{Griannar} was once the lesser deity of light, piety, forgiveness and
repentance, until he was killed by the \nameref{sec:Silent Queen} in
\emph{MI:0}. He was often portrayed as a sunbeam, or as the two suns rising on
the horizon.

Worship of Griannar stretched back thousands of years, and before his death,
the \nameref{sec:Church of Light} was one the dominant religion in many
human city kingdoms. His church was among the most powerful institution on
\hyperref[sec:Aror]{Aror}, and at its peak, counted millions of followers.
The church held vast and unparalleled influence, and political power. His
church was not only an institution to further his worship, but also included a
knight order as a military wing, that could rival many armies in terms of
manpower, training and equipment.

\subsubsection{Holy Crusade}
\label{sec:Holy Crusade}

Griannar was always an open rival of the \nameref{sec:Silent Queen}. This
rivalry existed over centuries, and lead to the occasional skirmishes, violent
disputes, and armed clashes between the two religions and their
followers. Over the years the power of the Holy Church grew, and began to
entrench itself deeply in the political landscape of the major city
kingdoms. Since it openly tried to enforce a policy of zero tolerance against
corruption, impurity, debauchery, evil and the undead (regardless on whether
they were evil or not), many noble houses began to secretly support the
followers of the Silent Queen.

The Queen's followers, who where often rich thieves, smugglers, corrupt barons
and nobles, began to fund mercenaries and assassins, to drive the followers of
Griannar of their land, or to assassinate powerful priests and bishops of the
Holy Church. The church retaliated by sending her knights to defend churches,
pilgrims and protect the bishops. As the open engagements between the Queen's
mercenaries and the knights became frequent, more and more noble houses, who
saw the Church as a threat to their power, began funding the Queen's war
campaign.

In \emph{GT:3391} the holy church openly called for a holy crusade against the
evil that is the Silent Queen and her followers. The declaration was met by a
rival declaration by the \nameref{sec:House deVar}, who openly supported the
Silent Queen in the crusade. This plunged \nameref{sec:Helmarnock} into war
against other city nations that openly supported Griannar, including
\nameref{sec:Hraglund} and \nameref{sec:Forsby}.

The Holy Crusade lasted for almost twenty years, and reached its conclusion
at the decisive siege of \nameref{sec:Hraglund} in \emph{GT:3408}. The
siege lasted three years, and finally ended when the archbishop of Griannar
tried to flee the city in secret. He was betrayed by the nobles of the city,
and delivered to be executed by the high priestess \nameref{sec:Aria} of the
Silent Queen.

\subsubsection{Death}

Scholars still debate Griannar's death to this day, but in \emph{MI:0}, all
priests of Griannar lost their divine power, and their prayers went
unanswered. Alongside him, his rival the \nameref{sec:Silent Queen} disappeared
(or died) as well, giving rise to a new religion surrounding the high priestess
\nameref{sec:Aria} a few decades later.

\subsubsection{Legacy}

Over the course of many decades the Holy Church of His Divine Light slowly
lost influence and power, until it slid into obscurity. Ruined temples and
churches of the church can still be found all over the world, while the major
sites of worships within the city kingdoms were either demolished or have been
taken over by other faiths.

The death of such a major deity was a major event, and the scholars of
\nameref{sec:Fes al-Bashir} tried for a long time to understand the
intricacies of such a world shattering event. The sad occasion was
immortalised in the dawn of the new aeon of \emph{Midaris} that began with
year zero in the year of Griannar's death.

\begin{35e}{Griannar}
  Griannar was considered lawful-good, his favoured weapon was the arming sword.
  Griannar is considered dead, and his followers and priests no longer receive
  divine power.
\end{35e}

%% Isamir
\subsection{Isamir}
\label{sec:Isamir}

\emph{Isamir} is a lesser deity who claims sovereignty over the sea, sea
creatures, storms, rains and the weather. He is often portrayed as a deep
sea dragon that waits and lurks beneath the surface.

\subsubsection{Inua}

The \nameref{sec:Inua} worship him as their patron deity, and it is from
them that he became known to the wider world. The Inua see him as malevolent
deep sea creature that must be appeased with prayer and sacrifice, otherwise
he sends storms and thunder in retaliation. He grants power to those that
worship him, and is said to conjure storms and against those that displease
him.

He preaches respect against the sea, its creatures, and that you should not
defile or pollute his seas, lakes or rivers. His followers should never take
more from the waters than they require, and must not allow buildings that
alter lakes and rivers, such as damns.

Isamir openly encourages the tribes of the Inua to raid and attack foreigners,
as well as their own tribes that stray from the path that he has laid out for
them. He also supports the Inua's more ancient tradition of converting their
dead to undead to allow them to further serve their tribes. Ever since the
other city kingdoms have come to the \nameref{sec:Kanaria Archipelago} his
worship is threatened by the other religions of Aror. He especially takes a
great dislike against any tribe that would worship \nameref{sec:Forun}.

The Inua are the only people who build large temples, shrines and places of
worship in his name. These are often hidden deep in the jungles of the
archipelago.

\subsubsection{Sailors}

Isamir is also a popular deity to sailors, who pray to him for safe passage
over the sea, lakes and rivers. Sailors that follow Isamir often sacrifice
to him, by throwing provisions or even money overboard to feed the beast
that sleeps beneath the surface. Like with the Inua, he demands that his
lakes, rivers and the sea are respected, and demands that they be not
polluted or even destroyed by damns and diversions.

\subsubsection{Relations}

He openly denounces anyone who would destroy, damage or overfish lakes,
rivers and the sea. And he is a fierce opponent of anyone who follows
\nameref{sec:Forun}.

\begin{35e}{Isamir}
  Isamir is considered \emph{chaotic neutral} or \emph{neutral evil},
  and his favoured weapon is the long spear. His domains are water,
  strength, war and chaos.
\end{35e}

%% Ishtar
\subsubsection{Ishtar}
\label{sec:Ishtar}

Ishtar (or \emph{Inanna} in some regions) is a lesser deity. She is in truth a
female succubus that is often revered as the patron of doctors, freed slaves,
conventional healers, surgeons and those that study medicine or biology.
Ishtar's symbol is a snake coiling around thin dagger without a cross guard.

Her role in the layers of hell are to free \nameref{sec:Demons} from the
scourge, and then aid them in their recovery process and integration into
devil society. She is aided in this role by her master and teacher
\nameref{sec:Asmoday}. While she is a devil, many see her role within the
layer of abyss as one of a healer, freer of the enslaved and patron of studies
of medicine and biology. Ishtar teaches that conventional healing is both art
and science, and must be practised with great care and great
responsibility. Ishtar also represents vanity and beauty as she performs
surgery on the disfigured spawns of the \nameref{sec:Scourge}. Her followers
also practice plastic surgery on both the deformed, and those whose beauty is
fading due to old age.

Generally she accepts both the good natured healer that attempts to aid and
heal those that are sick, wounded or disfigured by illness and accident, as
well as the vain surgeon that performs plastic surgery on the rich nobility
for large amounts of money. And even though her religion is mostly practised
by a small minority of expert surgeons, healers and doctors, they are well
respected all across Aror. Some elements of her faith are questionable, as she
is vague on topics on whether healers and doctors require prior consent and
authorisation for treatments or experiments.

Ishtar can be summoned, in which case she will send a succubus or incubus
minion to make deals with mortals. She is interested in granting patronage to
those that research medicine (especially surgery) and biology. She will often
ask for research results, as well as gained knowledge in exchange for favours,
knowledge and artefacts.

Her followers are mostly well liked and well received, especially by those who
cannot afford divine healing magic to cure illnesses and treat wounds. Some of
Ishtar's more morally grey followers will run afoul with local authorities, or
members of the \nameref{sec:Order} if they conduct treatments and experiments
without consent. \nameref{sec:Lor} bans devil worship outright, and so also
prosecutes those that follow Ishtar.

\begin{35e}{Ishtar}
  All in all \emph{Ishtar} is considered \emph{neutral} (with a slight tilt
  towards \emph{neutral good} when it comes to treat her own fellow devils),
  and her favoured weapon is the kukri.
\end{35e}

% Leszy
\subsection{Leszy}
\label{sec:Leszy}

Leszy (or ``leshiy'' or ``leshy'') is a \hyperref[sec:Daemons]{daemon} and
deity that inhabits the vast forest in the \nameref{sec:Dirgewood}. He often
appears as a monstrous, gigantic humanoid male with green skin and plant
growths covering his body. He may change his appearance, height and physique,
as any daemon can, and will show himself often to travellers that get lost in
the vast forest east of \nameref{sec:Forsby}

It is unclear whether he is good or evil, as some reports portray him as a
helpful spirit guiding people out of the forest, while others have confirmed
that he sometimes abducts children from villages. Regardless of his alignment,
he loathes tempering with forests and will attack anyone that seeks to damage
the forest on a massive scale.

Leszy's origins are unclear. While the corrupted druids claim that Leszy
poisoned their minds and souls, the old ways tell stories about how Leszy is
simply a manifestation of the combined corrupted will of the druids. Regardless
of his origin, many druids still follow him and see him as their deity.

\begin{35e}{Leszy}
  Leszy is chaotic neutral, and his favoured weapon is the quarter staff.
\end{35e}

%% Lilith
\subsection{Lilith}
\label{sec:Lilith}

Lilith is one of the generals of the \hyperref[sec:Devils]{devils}, and is
thus also worshipped on Aror as a goddess. She is the goddess of sex, lust,
debauchery, and hedonism. She is there often called the \emph{scarlet whore},
as she encourages people to frown in promiscuous, lavish and excessive
endeavours.

She has no official churches, or even dogma, but she is revered among the
wealthy that can frown in all sorts of excesses, those of sexual perversions
but also among sex workers and those sold to sexual slavery. Lilith's followers
indulge in food and sexual orgies, and follow their short-sighted pleasures
and indulgences to wherever they might take them. Lilith's followers range
from harmless but sexually curious, to the downtrodden and exploited sexual
workers and slave, to deranged and perverted. Since the few spoil the apple
basket, and the nature of the deity, most worship of Lilith happens in secret.

Since she has no dogma, no official rules or code to follow, she sends succubi
and incubi to her followers to hand out instructions, demands, as well as
answer prayers and grant gifts. Lilith is known to tempt powerful mortals to
fall for her ways, using her incubi and succubus as tools of temptation. Lilith
uses those debaucheries then also to leverage power and influence among powerful
and wealthy mortals.

\begin{35e}{Lilith}
  Lilith herself is \emph{lawful evil}, but she welcomes anyone to her religion
  that prefers indulgence over restraint. From the neutral good sex worker just
  trying to make a living, to the chaotic neutral gigolo that just lives for
  pleasures.

  Her domains are Passion, Charm, Evil and Trickery.
\end{35e}

%% Lor
\subsection{Lor}
\label{sec:Lor}

Lor is a lesser deity of justice, law, discipline, good and the fight against
all evil. He is often depicted as an angelic humanoid creature that kneels
down with his two handed sword buried in the soil.

Lor himself is the current patriarch of \nameref{sec:Aurelis}, and is thus
among the younger lesser deities of Aror. His followers are always in direct
conflict with evil elements of the world. Lor's worship is well established
all across the world, and many welcome his knights and paladins as a means to
establish law and order, as well as a way to fight evils, such as daemons and
undead. After the death of \nameref{sec:Griannar} many children of the light
joined the church of Lor, with which the \nameref{sec:Church of Light} shared
many similar dogmas. This catapulted the church of Lor from being a small
knight order, to a global religion with churches and followers all across
Aror.

\subsubsection{Dogma}

The followers of Lor have one prevailing dogma, which is strictly controlled
by the church of Lor, called the \nameref{sec:Knight Order of Tavos}. The
church, and its leader the reigning patriarch or matriarch, reside in the city
kingdom of \nameref{sec:Hraglund}. The church preaches austerity, monogamy and
a strict life that rewards those that help the weak and wounded. Daily prayers
are required, as well as weekly attendance to masses. The priests of Lor often
guide a community of followers either alone, or with an adept they teach. The
followers of Lor are meant to seek out and destroy evil, even the evil that
resides within themselves. He instructs his followers to confess and repent,
for their signs and purge evil within them through fasting, prayer and devotion
to the common good.

Lor, and his followers, consider undead, devils (including tieflings), daemons,
demons, necromancers, and fey evil, and wicked. Lor openly encourages
anyone to seek out and destroy such evil, wherever it may lurk and hide. Lor
also abhors the manipulation of souls, and thus sees anyone that practise
\hyperref[sec:Soul Magic]{soul magic} as evil. This includes the shamans and
witches of the \nameref{sec:Old Ways}, who use soul magic through rituals and
incantations. Lor also abhors the evil practices of most druids, but tends to
stay clear of the vast rural areas (such as the Dirgewood) and thus rarely
comes in contact with either the old ways or druids. Lor, and his followers,
are suspicious about anyone who wields either psionic or arcane powers, and
often seeks policies to strictly regulate both forms of magic. Lor also opposes
widely practised evils, such as chattel slavery, or justice systems that rely
on an ``eye-for-an-eye'' mentally.

Most followers that dedicate themselves to Lor become priests, knights,
paladins or holy crusaders. Lor values dedication, hard work as well as
spiritual strength and resolve. The dogma is preserved and cultivated in the
\nameref{sec:Knight Order of Tavos}.

\subsubsection{Relations}

Lor is in direct conflict with pretty much all major evil or neutral gods and
religions, especially with \nameref{sec:Three Kings}, the \nameref{sec:Order},
as well as the major gods of the Old Ways. Lor sees these as heretic and
archaic forms of worship, practising or allowing practise and acts he considers
evil, such as slavery, humanoid sacrifice or soul magic rituals.

\begin{35e}{Lor}
  Lor is considered \emph{lawful neutral}, and he is the patron of knights,
  paladins, fighters, and anyone who seeks to destroy evil. His favoured
  weapon is the two handed sword.
\end{35e}

%% Morana
\subsection{Morana}
\label{sec:Morana}

\songquote{COIL}{
  A sleeping explorer \\
  {[her]} wandering mind, \\
  crossed over the border. \\
  A mind like a cemetery, \\
  where the corpses are turning \\
  where the bodies twist deep, \\
  in the frozen grip \\
  of a dreamless sleep.
}

\emph{Morana}, often called \emph{lady death} or the \emph{great betrayer}, is
the lesser deity of death and the undead. In the old manuscripts she is often
depicted as a black veiled female humanoid with pale skin. In newer works of
art and literature, made after the her deceit was discovered, she is often
shown as a blue, translucent female humanoid ghost, stealing or shepherding
souls.

\subsubsection{Great Betrayer}
\label{sec:Great Betrayer}

Originally she was believed to be a greater deity of death, and thus was often
seen as the fourth sister to \nameref{sec:Forun}, \nameref{sec:Marwaid} and
\nameref{sec:Nyddwr}. She was seen as the last stage of motherhood: the old
woman that died, but still holds her protective hand over her children from
the afterlife. Welcoming, and beckoning her children to her side once their
time had come. She was thus a major part of the \nameref{sec:Old Ways} once,
before she was almost unilaterally rejected, and now holds the role of a
villain in the religion.

During the \nameref{sec:Aeon of Strife} some of her followers prayed to her to
give them strength against the monsters and monstrous races that threatened
the humanoid races. As an answer she revealed to the early humanoids the
knowledge on how to create \nameref{sec:Vampires}. She did so secretly and
indirectly, as to not arouse suspicion that she was not in fact a greater
deity.

Her followers were promised great power, strength and eternal life, but Morana
did not reveal the many drawbacks and sacrifices that came with the undead
life. Many of her followers accepted her gift. Upon realising that their new
form was savage, evil and animalistic in nature, and a danger to the very
humanoids they sought to protect, the majority of her followers turned away
from her. This angered Morana greatly, and in retaliation she openly
threatened the vampires and high priests with death should they abandon her
faith.

Since higher deities do not speak to their followers, as they are abstract
concepts and not extra planar entities that live and breathe, her deception
was brought to the light. This deception and betrayal angered almost all of
her followers who in turn abandoned her. Morana however made good on her
threat, and smote and killed most of her undead followers and arch priests.

This betrayal was never forgotten and she's now simply known as the
\emph{great betrayer} among most of the humanoid species. Her name is never
spoken directly, as it is considered too much honour for a being so petty,
deceitful and evil. Her remaining followers call her either \emph{lady Death}
or \emph{Morana}.

\songquote{COIL}{
  Then the lowest comes up \\
  like a wreck from the depths. \\
  {[She]} hears night calling \\
  and has dreams of waking. \\
  Here in this darkness \\
  That burns like slow lighting.
}

\subsubsection{Followers}

Most baronies and city kingdoms ban her worship. The duties of burying the
dead have been taken over by the church of \nameref{sec:Forun} or the church
of \nameref{sec:Lor}. Although she has followers among the \nameref{sec:Inua},
as well as less civilised undead, such as feral vampires or liches, her worship
is rejected among the civilised undead such as \nameref{sec:Vampires} and
\nameref{sec:Umgeher}.

\subsubsection{Teachings}

Morana's modern teachings openly encourages the creation and spreading of
evil undead, and she often helps powerful necromancers to achieve lichdom.
She also welcomes anyone who wishes to practise necromancy, and sees all
undead as her children, even if they reject her.

\subsubsection{Relations}

She now holds poor relations with most other deities and their followers,
especially the true deities that are still worshipped in the
\nameref{sec:Old Ways}. Her direct enemy is \nameref{sec:Lor}, who openly
opposes her followers for creating and summoning undead.

Morana and \nameref{sec:Isamir} appear to be on good terms, as they are both
worshipped together among the \nameref{sec:Inua} of the \nameref{sec:Silver
  Isles}.

\begin{35e}{Morana}
  She is considered \emph{neutral evil}, and her favoured weapon is the
  morning star. Her domains are evil, death, destruction and knowledge.
\end{35e}

%% Runemaster
\subsection{Runemaster}
\label{sec:Runemaster}

The Runemaster is an unnamed, extraordinarily powerful
\hyperref[sec:Devils]{devil} from the hellish planar realms, that entices
mortals with promises of great power and knowledge in exchange for sacrifices.

\subsubsection{Dogma}

The Runemaster has no church or concrete following, but his minions can
be summoned through arcane means and will strike deals with mortals. In these
dealings his followers often accept living humanoid sacrifices, powerful
artefacts, or knowledge in exchange for arcane knowledge and power.

\subsubsection{Runes}

He gained his name from offering a special variant of arcane magic to those
that summon his followers: \emph{rune magic}. With this technique he enables
those that cannot cast magic themselves an easy way to inscribe powerful
arcane runes into the flesh and skin of their own bodies. The magic
incantations to inscribe these runes often require gruesome ingredients, such
as body parts of animals or even humanoids to cast. Although the runes of
lower power require just animal organs, the ingredients become more gruesome as
they grow in power; ultimately peaking at ritual humanoid sacrifice in the
name of the Runemaster.

Most that practise rune magic stop at the lower tiers, never going beyond what
society or standing laws would allow them to acquire in terms of
ingredients. Some however are corrupted by the lust for power, are desperate
or perhaps morally challenged to begin with, and then proceed to fulfil the
Runemaster's ultimate goal: humanoid sacrifices in his name.

\subsubsection{Runic Lexicon}

He sometimes chooses one of his most loyal followers to write a book about how
to craft these runes, called the \emph{Runic Lexicon}, in the hope that this
spreads his influence among the mortal races. These books are priced
artefacts, as very few of these actually exist at any given time. Many are
destroyed by the various enemies of the Runemaster, and those that do
exist are often shared and traded in secrecy.

\subsubsection{Relations}

The Runemaster is in direct conflict with most good deities that seek
to reign in his powers, and wish to stop his followers from becoming so corrupt
that they'd perform humanoid sacrifice. His enemies include followers of
\nameref{sec:Forun}, \nameref{sec:Lor}, the church of the \nameref{sec:Order},
but especially \nameref{sec:Aria}, as she sees everything he does as an
affront to her teachings.

\begin{35e}{Runemaster}
  The Runemaster is, as most of the devils, \emph{lawful evil} and his
  favoured weapon is a ritual knife (a dagger) that is used to inscribe the
  runes.

  Rune magic is described later in this book.
\end{35e}

%% Silent Queen
\subsection{Silent Queen}
\label{sec:Silent Queen}

The \emph{Silent Queen} was a powerful lesser deity of the night, shadows,
theft, magic and subterfuge, until she was killed by her high priestess
\nameref{sec:Aria} in \emph{MI:4}. She was often depicted as a hooded woman,
who had her mouth sewn shut.

She was once the patron of all people that worked in the shadows, such as
thieves, spies, smugglers and assassins, but also those that sought knowledge,
as well as those that practised forbidden arcane arts in secret, such as
necromancers or conjurers of devils. She did find a few large congregations
among the ancient \hyperref[sec:Dark Elves]{dark elves} and
\hyperref[sec:Deepkin]{deepkin}.

As the goddess of night, she was always the opposite of \nameref{sec:Griannar},
and the two religions often clashed in violent disputes and skirmishes. These
clashes escalated to a full fledged war, which the followers of Griannar simply
called the \nameref{sec:Holy Crusade}. Although the queen emerged victorious
from the war, she was ultimately slain and usurped by her then high priestess

Although knowledge of her existence is widespread among scholars, she has
drifted into obscurity amidst the general populace. Most of her rites,
teachings and knowledge are all but lost. Depictions of her are still prevalent
in the deep, were the ancient dark elves erected statues in her honour. Although
the followers of Aria were diligent in their erasure of the Queen's history,
they have not yet erased everything, nor killed all of her remaining followers.

\graham{I shall let you know that my co-author is a devout follower of the
  Queen, and some biases might thus be present within the book.}

\subsubsection{Gathering of Silence}
\label{sec:Gathering of Silence}

The ``Gathering in Silence'' was the informal head-council of the church, in
which all high ranking members of the various aspects of the queen would come
together to elect a high priestess. This priestess then lead the followers,
and would ensure that the various organisations within the faith, such as the
\nameref{sec:Well of Truth} or the \nameref{sec:Scions of Silence} would work
together. The gathering was also responsible for collecting a tithe, or tax,
from its member, and distribute the money to fund the churches many
endeavours.

\begin{35e}{Silent Queen}
  She was considered \emph{neutral evil}, but since her death, her followers
  and priests no longer receive divine power.
\end{35e}

%% Three Kings
\subsection{Three Kings}
\label{sec:Three Kings}

The \emph{Three Kings}, are three lesser deities that represent conquest, war
and tyranny. The three kings are personified as humanoid knights clan in
armour and heavily armed that raise their weapons into the sky to cross their
swords at the blade.

The kings are either worshipped individually, or together as a pantheon. Each
king represents one aspect of war and tyranny. \emph{Aruim} represents the
conquest of war and the rule through might and power. He values cunning,
fierceness and strategy in war. \emph{Miator} represents the chaos and
unpredictability of skirmishes, and values anyone who shows no mercy towards
their enemy. He favours landslide victories, and spurns their followers to crush
the weak, plunder, pillage and rape. \emph{Karor} represents the tyrannical
rule of the intelligent over the strong, and the strong over the weak. He
favours ruling conquered lands with an iron and tyrannical fist.

Albeit they are often worshipped together within a region, many soldiers and
other followers pick one of the three specifically for worship. Their
followers are earls, kings, tribal rulers and counts who seek to dominate
their enemies by war and submission. And are often worshipped by knights,
soldiers and barbarians that pray to them for strength in battle.

Many of the warring sentient monstrous races follow the Three Kings, as do
humanoids that live and die for battle. Their worship is widespread in
Norbury, and among the hobgoblin, ogres and troll clans of
\nameref{sec:Iafandir}.

With a specific ritual any worthy follower can challenge one of the three
kings to single combat. This requires the follower to make humanoid, or
monstrous sacrifices, and to openly taunt and challenge the respective
king. If the specific king deems the challenger worthy, they will appear and
fight the challenger in a fair one on one duel. This single combat is to the
death, and should the challenger prevail he or she may assume the role of that
king among the other two. This follows the basic principle that only the
strongest, most worthy should be allowed to rule as kings

The \emph{Three Kings} stand in direct conflict with the followers of most
other gods and true deities, however their followers harbour resentment to
anyone who would help the weak, including \emph{Forun}, \emph{Lor} and the
\emph{Order}.

\begin{35e}{Three Kings}
  As a pantheon the Three Kings are considered neutral evil, and their domains
  are war, death, destruction and evil.

  Aruim is considered neutral evil and their favoured weapon is the battle axe.
  Miator is considered chaotic evil and their favoured weapon is the bastard
  sword. Karor is considered lawful evil, and their favoured weapon is the
  war hammer.
\end{35e}

%% Xir
\subsection{Xir}
\label{sec:Xir}

\emph{Xir}, often called the \emph{beast below}, is fabled powerful being that
sleeps beneath the rivers of \nameref{sec:Muut}. It is often represented as a
large, finned and gilled reptilian with large and sharp teeth.

It is unknown if this creature lurks beneath the rivers, or if it is simply an
aspect of \nameref{sec:Isamir}. Nevertheless the creature is worshipped by the
\hyperref[sec:Savage Elves]{savage elves}, and their tribal shamans and clerics
receive divine aid and favour.

\subsubsection{Dogma}

The creature finds all monstrous and humanoid races filthy, repugnant and
unworthy. It displays a deep seated hatred and aversion to humanoid and
monstrous creatures, but states that anyone may be redeemed if they allow
themselves to be ``cleansed'', or aid others in being cleansed. Its followers
thus raid and capture other humanoid and monstrous races, and drown them in
the rivers in a holy ceremony in Xir's worship.

In truth Xir takes the souls and bodies into the river, and uses them to create
abomination to serve it. These abominations, half-humanoid half-fish creatures
are called \emph{Sahuagin}, and serve their master without question or will on
their own.

\begin{35e}{Xir}
  Xir is considered \emph{chaotic evil}, and its favoured weapon is the long
  spear. Its domains are water, evil, chaos and destruction.
\end{35e}

