\section{Arcane Marks}
\label{sec:Arcane Marks}

Arcane marks are permanent arcane sigils, inscriptions or marks specifically
crafted onto a person's body. They are to be treated as magical items, in terms
of mechanics (such as dispel). Since arcane marks bear no mechanical benefit,
there is no limit on how many arcane marks a person my carry on his body.
If the bodily area where the mark was inscribed is lost or damaged the mark
simply moves to another area in the body. Arcane marks can still be revealed
even on the dead, making it easier to identify them after they have passed.

\subsection{Citizen Mark}
\label{sec:Citizen Mark}

A more expensive way to mark citizens is the arcane \emph{citizen mark}. It is
based upon \hyperref[sec:Deepkin]{deepkin} blood magic, and infuses itself
with the citizens body. At will, the citizen may then produce or hide an
arcane tattoo anywhere on their own body, which contains all the necessary
information to identify him as a citizen of a city kingdom or nation. These
usually cost up to 500 shins to inscribe, but cannot be stolen, misplaced or
so easily forged.

\begin{35e}{Citizen Mark}
  \srditem{Citizen Mark}{These magical runes are inscribed onto citizens of
    a specific city kingdom or nation, and are used to identify them as
    rightful citizens. They show themselves as tattoos upon the wearers skin,
    and often show the kingdom's or nations banner along with a few words,
    not more than seven, to help identify the person. Common inscriptions are
    name, address, race or the persons status within the nation or kingdom.
    These tattoos can be shown and hidden at will by the wearer.}
  \srditem{Crafting}{Caster Level: 3rd, Prerequisites:
    \emph{Craft Wondrous Item}, \emph{Arcane mark}, Price: \emph{500 shins}}
\end{35e}

\subsection{Nobility Mark}
\label{sec:Nobility Mark}

Nobility marks work just the same as citizen marks, except that they are
customised toward the house of nobility that issues them. Instead of
identifying that person as a citizen of a kingdom, these marks identify the
person to be of noble blood line. These marks are scribed usually at birth of
a new member of a noble family, and act as a proof of nobility across the
known world of Aror.

\begin{35e}{Nobility Mark}
  \srditem{Nobility Mark}{These marks work just the same as a
    \nameref{sec:Citizen Mark}, would, except are hardened against magical
    tampering and forgery, and cost more to inscribe.}
  \srditem{Crafting}{Caster Level: 7th, Prerequisites: \emph{Craft Wondrous
      Item}, \emph{Arcane mark}, Price: \emph{100 shard}}
\end{35e}

\subsection{Guild Mark}
\label{sec:Guild Mark}

Guild marks are equivalent to citizen marks, but instead show the wearers
affiliation to an organisation, company, guild, or military force. Almost all
guilds of the crafts give out guild marks to their members, to allow them to
identify themselves as guild members. Since guilds vouch for the skill of
their members, the marks also service as proof for an artisans' skill, and
proficiency. Such guild marks may encode ranks, special skills, years of
service, or other useful information related to the specific organisation or
guild. These marks are also often used by hierarchical organisations, such as
armies to identify a soldier's rank, years of service, or the specific branch
of the military force.

\begin{35e}{Guild Mark}
  \srditem{Guild Mark}{These marks work just the same as a
    \nameref{sec:Citizen Mark}, would, except are somewhat hardened against
    magical tampering and forgery, and cost more to inscribe.
  }
  \srditem{Crafting}{Caster Level: 5th, Prerequisites: \emph{Craft Wondrous
      Item}, \emph{Arcane mark}, Price: \emph{50 shard}}
\end{35e}

\subsection{Slave Mark}
\label{sec:Slave Mark}

Slave marks function similar to \nameref{sec:Citizen Mark}. They are arcane
marks that are permanently crafted onto the slaves body as a tattoo. Often
behind the ears, or onto the slaves neck. But unless with the citizen marks,
they cannot be made invisible by the wearer. These marks often encode the
slave's name registration number, owner, and the city or nation of the
owner. Slave marks are bonded to \nameref{sec:Master Ring} just like slave
collars are, and can thus be used as a target for certain magical spells and
powers that will then target the slave.

Slave marks are expensive, and are thus rarely used unless the slave is prone
to escape, or is extremely dangerous or valuable to the owner.

\begin{35e}{Slave Mark}
  \srditem{Slave Mark}{These magical arcane marks are made to permanently
    identify a person as a slave. They attach permanently to a person's body,
    and can encode simple messages. A slave mark is keyed to one or more
    \hyperref[sec:Master Ring]{Master Rings}. Once inscribed the slave mark
    cannot be removed except by the owner of the Master Ring. A bearer of
    slave mark is subject to the damage, and any divination spell (with a -4
    penalty to saves) from the wearer of the keyed master ring, and can
    exchange messages with the wearer of the keyed master ring three times per
    day as if using the \emph{Sending} spell. The wearer of the slave mark
    cannot directly touch or use the keyed master ring, and suffers 3d6 points
    of damage in such an attempt.
  }
  \srditem{Crafting}{Caster Level: 7th, Prerequisites: \emph{Craft Wondrous
      Item}, \emph{Crushing Despair}, Price: \emph{1000 shards}.}
\end{35e}
