\section{Base Classes}
\label{sec:Base Classes}

\subsection{Barbarian}
\label{sec:Barbarian}

Most barbarians can be found among the more savage tribes of
\nameref{sec:Iafandir}, \nameref{sec:Dirgewood} as well as the Toralian
heights. Barbarians there are an integral part of many smaller tribes, being
warlords, warriors, hunters and chieftains. Many barbarians follow the example
of \nameref{sec:Eigyr}, the mighty huntress, who is often seen as a positive
example on how a mighty warrior shall serve his community and the greater
good. Her deeds inspire young warriors of any race, background, gender or
religion to achieve ever great feats and heroics. Unlike other worlds, there
are no tribes solely made up of barbarians, instead these mighty warriors are
always embedded in existing villages and tribes.

The monstrous races also favour the reckless yet courageous style of battle
the barbarians do, often following the example set by \nameref{sec:Three Kings}.
These warriors are often called ``beast warriors'' or ``ravagers'', particularly
known for their savagery and unparalleled strength in battle.

\subsection{Bards}
\label{sec:Bards}

Bards can be found in all across Aror, in a variety of roles, shapes and
professions. Many artists, musicians, actors and performers of
\nameref{sec:Avenfjord} are bards, enticing large crowds with their artistic
skill and enchanting performances. But bards can also be found among the
\nameref{sec:Old Ways}, either telling or performing plays about the stories
so deeply entrenched within that religion. Many rituals of the old ways
include chanting and music, and so many shamans and spiritual leaders of
communities that follow the old way are bards.

\subsection{Clerics}
\label{sec:Clerics}

Clerics, priests, child of the gods, acolytes, shamans, and many witches are
all the same: followers of either the true or lesser gods. They highly valued,
if not the most important, member of any civilisation. They share the wisdom,
and power of their gods with their community, and lead them through spiritual
crisis and hardship. Clerics take on my roles, often depending on the
spiritual journey their god or deities has put them on. Some are healers, some
are torturers. Some guide their flock towards peace, love and understanding,
while others chant their message of hatred, war and destruction to drive their
communities towards hostility against others.

\subsection{Druids}
\label{sec:Druids}

The druids where once the spiritual leaders of the old ways, but are now a
community facing imminent extinction. Druids split from the shamans and priests
of the old ways, by abandoning the worship of the holy mothers, and instead
worshipping \nameref{sec:Daemons}, in particular \nameref{sec:Leszy} and the
\nameref{sec:Percht}. A few druidic circles, including the ``Circle of
Rebirth'' remained loyal the holy mothers, and continued to worship
\nameref{sec:Morana}, which ultimately, also lead them fall from grace in the
eyes of the followers of the old ways. Although many druidic circles followed
practices and traditions inspired by their often evil deities, their lives
deep within forests and marshes of the \nameref{sec:Dirgewood} and Toralian
Heights isolated them from other humanoid civilisations. They were seen as
religious fanatics, albeit few in numbers and generally considered harmless.

This all changed during the age of battle against the beast races, when both
the humanoid races, and the intelligent beast races, attempted to outbid each
other in a rush for resources and industrial growth. Forests were cut down,
marshes drained, lakes fishes dry, and lush grassland burnt down to make way
for farmland. This enraged both humanoid and beast race druids, causing them
to solidify in their religious fanaticism. They began to work together to
keep both humanoids and beasts out of their forests, often devising evil,
sadistic and cruel means with the help of their equally deranged deities.

Druids are the sole reason \nameref{sec:Fey} and \nameref{sec:Lycanthropes}
exist, and terrorise the humanoid and beast population to this day. Many
druids were not above acts of domestic terrorism against large baronies and
kingdoms, poisoning wells and food stores, turning populations against their
will into fey, causing farmland to wither and die, and abducting the young to
bolster their own numbers. In turn, druids were hunted to near extinction.

Those few that remain, hide in the depths of the forest or marshes and hold on
to their believes and fanaticism. These remaining circles are the aforementioned
``Circle of Rebirth'', ``Nightowls'' and ``Ondtrad''. The ``divine'' magic they
wield stems from the daemons they worship, and continues to poison them and all
they create. Those druids that escape from the clutches of these fanatic cults
find they are still not welcome among either their humanoid or monstrous
brethren. Any appearance and terror from fey is blamed on them, as are any
failings in the harvest. Knowledge about the history of druids, and their
misdeeds are common, as they are often used as a boogeyman to frighten
children away from forests. Almost all city kingdoms and baronies ban druidic
worship of the daemons, and will arrest, exile or even execute druids.

\graham{If you are a druid attempting to flee your cult and join a modern
  civilisation, know this: Unless you lie about your heritage, you might be
  welcomed with aversion and hostility.
}

\begin{35e}{Druids}
  Druids on Aror have a predisposition towards being evil against their
  civilised brethren. As always, not all are evil, and many do not have a
  choice as the very daemons they worship poison their minds and judgements.
  Knowledge about who and what druids are, and what they have done in the
  past is wide spread among most people and monstrous races on Aror.
\end{35e}

\subsection{Fighter}
\label{sec:Fighter}

The backbone of any and all civilisation. If you can defend your family, your
land, your king and your country you will be welcomed anywhere. Fighter,
warriors, archers, two-handed wielding Landsknechte, mercenaries, shield
maidens, master fencers or ruthless thugs are but a few of the multiple
facets that these brave men and women play in your society.

\aren{There is inherent strength in being able to wield a weapon effectively.
  Limiting yourself by preconceptions on their inferiority against magic,
  makes you much, much weaker than you are.
}

\begin{note}
  You should consider giving Fighters more skill points.
\end{note}

\subsection{Monk}
\label{sec:Monk}

Monks, as in mystical warriors of eastern lands wielding some sort of ``chi''
in balance with their lives, do not exist on Aror. If you do find one, he is
probably an inter-planar traveller.

While monastic orders exist, they usually wear grey robes, live in poverty,
and gain fulfilment in their lives by studying the sciences, philosophy, or
the rule of law. Very few monks study martial arts, and those that do become
knights or warriors instead.

\begin{35e}{Monk}
  The 3.5e style of eastern monk is not found on, and is atypical to Aror.
\end{35e}

\subsection{Paladin}
\label{sec:Paladin}

Those that you might know and call ``paladin'' and might see as holy avengers
and servant of the gods, are just a small subset of the wide variety of
fighters, warriors and that gain divine power through servitude. All of them
have one theme in common: a strongly held set of believes, and the willingness
to fight those that would oppose said believes. While the classical, almost
stereotypical, holy crusader that follows \nameref{sec:Lor} or
\nameref{sec:Order}, who wears heavy plate armour and smites evil foes do
exist, and might even be the majority, they are not the only servants that are
granted divine power.

\nameref{sec:Marwaid} sees servitude as a form of sacrifice to be held
precious, holy and divine even if the servitude is not to a god or church. Many
paladins work for the ideals represented by factions, organisations, their own
set of believes, serve powerful individuals in a \nameref{sec:Life Bond}, or
even long lost, and dead gods. All of these are actually following the ideals
of Marwaid and are thus granted divine power.

\begin{35e}{Paladin}
  There are no longer any alignment restrictions on paladins, and you no longer
  have to directly serve a deity to be a paladin. The mere act of acknowledging
  strength through servitude (to a person, ideal, church, organisation etc.) is
  seen as honourable to Marwaid and thus makes you eligible to be a paladin,
  regardless of alignment.

  Classical 3.5e paladins can choose the Order, Lor, Marwaid, devils or
  daemons (among others) as their patron deities.
\end{35e}

\subsection{Sorcerer}
\label{sec:Sorcerer}

Sorcerers, as they are called by the arcane institutions, are people that wield
arcane magic without formal training. They often lack formal arcane training,
and simply just ``do it''. Many sorcerers are part of tribal societies,
villages, or cities in rural areas, where they are often called ``shaman'' or
``witch'', and offer their services towards the community. Although many wizards
look down on these free spirits, their power is undeniable.

\graham{Never look down on a witch, or she just might hex you.}

\subsection{Ranger}
\label{sec:Ranger}

Rangers are people of the forest, the swamp or the mountains. They are either
defenders of nature, hunters, but above all else, followers of the
\nameref{sec:Old Ways} that live in harmony with the teachings of the old
mothers, nature, and their people. In many tribes, villages and cities rangers
are a cornerstone of the community, often serving as a link between the people
and nature. Some wayward rangers also live in the city kingdoms, often serving
as hunters, trackers, priests, and in some rare cases, as slave hunters and
soldiers.

\aren{Rangers are most typically spiritual leaders of smaller tribes and
  villagers, and might often bear titles such as ``priestess'', ``shaman'' or
  ``elder''.
}

\subsection{Rogue}
\label{sec:Rogue}

Rogues are as multi faceted in their occupation within as their skills. Some
are spies, thieves, burglars, diplomats, crime lords, pirates or smugglers.
But most of them lead some sort shady life at the edge of legality. They are
so ubiquitous, that is hard to say where to start looking for one.

\graham{Perhaps among the authors of this book.}

\subsection{Warlocks}
\label{sec:Warlocks}

Warlocks have made a pact with a powerful entity in exchange for power. The
basic premise of these pacts is that the power helps the warlock grow in
power and strength, and once the warlock dies his empowered soul is already
promised the powerful entity. The powers that warlocks wield were invented
by \nameref{sec:Forneus}, and thus many of his disciples (be they infernal
or from Aror) follow him. The devils are not the only ones that grant powers
such as these to their disciples. Many \nameref{sec:Daemons} have started
to foster their own disciples.

No warlock is particularly welcomed on Aror, as resentment towards devil
worshippers has increased after the devil siege against Forsby. Especially if
you also have infernal traits, or are a \hyperref[sec:Tieflings]{tiefling},
the authors advise keeping your powers secret, and your appearance hidden
behind illusion.

\subsection{Wizards}
\label{sec:Wizards}

Wizards are the rare view that have shown both talent and dedication to study
the arcane arts. Albeit wizards are extra-ordinarily rare on Aror (very few
have the mental capacity to understand, and wield the logic required to cast
spells), they are highly regarded in almost all nations and regions. Wizards
make up most of the high elite that rule \nameref{sec:Fes al-Bashir} through
the Hall of Knowledge, they study the arcane arts in the \nameref{sec:Magistrata
  Arcanum}, craft and create wondrous machines in \nameref{sec:Stenheim} and
scry the northern sea of \nameref{sec:Norbury} for immediate raids.

Very few wizards ever reach the pinnacle of their power, as natural death
takes them before they could unlock their higher mysteries of arcane
knowledge. Those that reach these unfathomable heights of arcane power are
granted the title ``Grand Magus'' by the \nameref{sec:Hall of Knowledge},
of which none have ever surpassed the power of \nameref{sec:Graham Balance}.

In recent years the both the Hall of Knowledge and the Magistrata Arcanum
have confirmed that \nameref{sec:Taras} has achieved the same power as a
Grand Magus, but both institutions have denied him that title.

\begin{note}
  High level wizards (>15) are extra-ordinarily rare on Aror, with only a few
  living at any given time. Graham Balance was the most powerful wizard ever
  known to live (level 19), with Taras coming in close second (level 17).

  Most wizards are around level 5 or 9, with a few select leaders and professors
  of arcane institutions reaching level 10 to 14. The title of ``Grand Magus''
  is bestowed upon anyone who can proof that they can cast level nine spells.
\end{note}
