\section{Base Classes}
\label{sec:Base Classes}

\subsection{Barbarian}
\label{sec:Barbarian}

Most barbarians can be found among the more savage tribes of
\nameref{sec:Iafandir}, \nameref{sec:Dirgewood} as well as the Toralian
heights. Barbarians there are an integral part of many smaller tribes, being
warlords, warriors, hunters and chieftains. Many barbarians follow the example
of \nameref{sec:Eigyr}, the mighty huntress, who is often seen as a positive
example on how a mighty warrior shall serve his community and the greater
good. Her deeds inspire young warriors of any race, background, gender or
religion to achieve ever great feats and heroics. Unlike other worlds, there
are no tribes solely made up of barbarians, instead these mighty warriors are
always embedded in existing villages and tribes.

The monstrous races also favour the reckless yet courageous style of battle
the barbarians do, often following the example set by \nameref{sec:Three Kings}.
These warriors are often called ``beast warriors'' or ``ravagers'', particularly
known for their savagery and unparalleled strength in battle.

\subsection{Bards}
\label{sec:Bards}

Bards can be found in all across Aror, in a variety of roles, shapes and
professions. Many artists, musicians, actors and performers of
\nameref{sec:Avenfjord} are bards, enticing large crowds with their artistic
skill and enchanting performances. But bards can also be found among the
\nameref{sec:Old Ways}, either telling or performing plays about the stories
so deeply entrenched within that religion. Many rituals of the old ways
include chanting and music, and so many shamans and spiritual leaders of
communities that follow the old way are bards.

\subsection{Clerics}
\label{sec:Clerics}

Clerics, priests, child of the gods, acolytes, shamans, and many witches are
all the same: followers of either the true or lesser gods. They highly valued,
if not the most important, member of any civilisation. They share the wisdom,
and power of their gods with their community, and lead them through spiritual
crisis and hardship. Clerics take on my roles, often depending on the
spiritual journey their god or deities has put them on. Some are healers, some
are torturers. Some guide their flock towards peace, love and understanding,
while others chant their message of hatred, war and destruction to drive their
communities towards hostility against others.

\subsection{Druids}
\label{sec:Druids}

The druids where once the spiritual leaders of the old ways, but are now a
community facing imminent extinction. Druids split from the shamans and priests
of the old ways, by abandoning the worship of the holy mothers, and instead
worshipping \nameref{sec:Daemons}, in particular \nameref{sec:Leszy} and the
\nameref{sec:Perchten}. A few druidic circles, including the ``Circle of
Rebirth'' remained loyal the holy mothers, and continued to worship
\nameref{sec:Morana}, which ultimately, also lead them fall from grace in the
eyes of the followers of the old ways. Although many druidic circles followed
practices and traditions inspired by their often evil deities, their lives
deep within forests and marshes of the \nameref{sec:Dirgewood} and Toralian
Heights isolated them from other humanoid civilisations. They were seen as
religious fanatics, albeit few in numbers and generally considered harmless.

This all changed during the age of battle against the beast races, when both
the humanoid races, and the intelligent beast races, attempted to outbid each
other in a rush for resources and industrial growth. Forests were cut down,
marshes drained, lakes fishes dry, and lush grassland burnt down to make way
for farmland. This enraged both humanoid and beast race druids, causing them
to solidify in their religious fanaticism. They began to work together to
keep both humanoids and beasts out of their forests, often devising evil,
sadistic and cruel means with the help of their equally deranged deities.

Druids are the sole reason \nameref{sec:Fey} and \nameref{sec:Lycanthropes}
exist, and terrorise the humanoid and beast population to this day. Many
druids were not above acts of domestic terrorism against large baronies and
kingdoms, poisoning wells and food stores, turning populations against their
will into fey, causing farmland to wither and die, and abducting the young to
bolster their own numbers. In turn, druids were hunted to near extinction.

Those few that remain, hide in the depths of the forest or marshes and hold on
to their believes and fanaticism. These remaining circles are the aforementioned
``Circle of Rebirth'', ``Nightowls'' and ``Ondtrad''. The ``divine'' magic they
wield stems from the daemons they worship, and continues to poison them and all
they create. Those druids that escape from the clutches of these fanatic cults
find they are still not welcome among either their humanoid or monstrous
brethren. Any appearance and terror from fey is blamed on them, as are any
failings in the harvest. Knowledge about the history of druids, and their
misdeeds are common, as they are often used as a boogeyman to frighten
children away from forests. Almost all city kingdoms and baronies ban druidic
worship of the daemons, and will arrest, exile or even execute druids.

\graham{If you are a druid attempting to flee your cult and join a modern
  civilisation, know this: Unless you lie about your heritage, you might be
  welcomed with aversion and hostility.
}

\begin{35e}{Druids}
  Druids on Aror have a predisposition towards being evil against their
  civilised brethren. As always, not all are evil, and many do not have a
  choice as the very daemons they worship poison their minds and judgements.
  Knowledge about who and what druids are, and what they have done in the
  past is wide spread among most people and monstrous races on Aror.
\end{35e}

\subsection{Fighter}
\label{sec:Fighter}

The backbone of any and all civilisation. If you can defend your family, your
land, your king and your country you will be welcomed anywhere. Fighter,
warriors, archers, two-handed wielding Landsknechte, mercenaries, shield
maidens, master fencers or ruthless thugs are but a few of the multiple
facets that these brave men and women play in your society.

\aren{There is inherent strength in being able to wield a weapon effectively.
  Limiting yourself by preconceptions on their inferiority against magic,
  makes you much, much weaker than you are.
}

\subsection{Monk}
\label{sec:Monk}

Monks are rather uncommon to the world of Aror, as in, not many have actually
met or even seen a monk. Almost all of the warrior monks study their art in
solitude, or join monastic organisations such as the \nameref{sec:Fifth Order}.
Although often monks follow a strict traditional life of order, mental training
and disciple, it is not required to learn the martial arts.

\begin{35e}{Monk}
  Monks no longer have any alignment restrictions, and the longsword, rapier,
  spear, bows (of any form), kuhkri and scimitar are added to the list of monk
  weapons.
\end{35e}

\begin{note}
  See the monk more as a warrior who learns and practices martial arts on a
  spiritual level. This can include ``western style'' martial arts, which are
  often referred to as ``HEMA''. On Aror a monk is no longer strictly
  ``eastern'' martial arts, especially since the 3.5e monk was a very poor
  representation of that in the first place.
\end{note}
