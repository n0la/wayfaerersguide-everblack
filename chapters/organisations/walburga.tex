\clearpage
\incgraph[
  overlay={\node[black] at ([xshift=0cm,yshift=-1cm] page.north)
    (main)[text width=0.9\paperwidth]{
      \large \centering
      \textbf{``The burning of the wicker queen, to drive away winter.''}
    };
    \node[black, below=of main,yshift=1cm]{
      \nameref{sec:Dirgewood} circa GT:2100
    };
  }
]{media/walpurgawitches.png}
\clearpage

\section{Walburga}
\label{sec:Walburga}

The Walburga (Walpurga, Valborg, or Valborga) is an all-female coven of
witches, who practice soul magic, and worship \nameref{sec:Morana}. They have
their home in the \nameref{sec:Toralian Highlands} and the
\nameref{sec:Dirgewood}. The coven consists of female soul-witches, whose soul
powers have a distinct red hue, and of servants who cannot cast soul magic
that are called handmaidens. Members were often lightly clothed, displaying
the colours red, and black on their bodies while using crow feathers, antlers,
bones and carved bone as jewellery. The carrion crow, which represents Morana,
is a holy animal to the coven.

As servants and handmaidens of the goddess of winter, and death since the dawn
of time, they are a deeply religious and spiritual sect within the
\nameref{sec:Old Ways}. The coven is not a uniform organisation, but is
split into smaller covens that hold a close-nit relationship with one another.
After the great betrayal they remained fiercely loyal to their mother, denying
her fall from grace as one of the great mothers. As children of the Old Ways
they also revere the other main mothers, and are generally favourable towards
everyone that follows the old traditions and teachings.

The witches have a wide range of tasks, and responsibilities that their mother
bestowed upon them. They watch over, and tend to graveyards, take care of
intelligent undead such as vampires, but also use their soul powers to create
new undead. Since their mother is also the goddess of winter, the witches are
mostly active during that time, performing rituals, tending shrines and graves,
and bringing sacrifices for their mother. Witches of the coven often have a
deep spiritual interest in the songs and stories of old, often performing them
with vocals, instruments and in play during their rituals and incantations.
The Walburga also aid in the burial of the dead, or help burying the dead of
battle fields or plagues.

As soul witches they also aided local villagers and travellers with curing
ailments of the soul, often in exchange for common goods, food, tools and
utensils. Soul incantations are difficult to prepare and perform, and if they
fail they often cause a backlash, harming, or even mutilating the caster.
Together with a difficult life in the forest, and tending to undead that are
often hostile, many older witches have a scared, hardened, and rough
appearance. The initiation ritual, which leads to \hyperref[sec:Soul
  Awakening]{soul awakening} is equally dangerous for both the initiate and
the witch performing it. Those members that succeeded their initiation ritual
will be inducted in the order as full witches. For those initiates for which
the ritual failed to awake their souls, become ``handmaidens'' instead. Witches
may at any time leave the order, but are always considered members of the coven,
and welcomed back as wayward sisters at any time.

The most important ritual and holiday associated with the witches is the burning
of the wicker witch in spring. During a night - called Walburga's night - in
early spring, after the snow has already melted and the flowers and trees have
begun blooming, a large wicker figurine is burnt on a huge bonfire. The figurine
is supposed to represent Morana, who is burnt away by fire to make way for
\nameref{sec:Forun}, who represents the spring. The witches often organise the
festivities, while surrounding villagers bring food and wine to celebrate.

Before the great betrayal, the coven was influential and powerful. It could
easily recruit new members by mingling (often in an excessive fashion during
Walburga's night) with the local towns folk. Boys would be handed back to the
villages to raise, while girls would be raised within the coven. After the
great betrayal the coven was shunned and excluded, and became more desperate
for new members. While some will openly ask for female children and newborns
in exchange for services, others have resorted to threats or even abduction.

This shift in the coven's ethics for the worse, gave the knights and priests
of Lor within the \nameref{sec:Knight Order of Tavos} a reason to launch a
crusade against the witches. After this crusade and culling ended around
GT:580, most covens of Walburga have been driven into hiding and in some
regions even into obscurity. The crusaders also spread negative rumours about
the covens, including supposed cases of ritual sacrifice of male children,
lavish consumption of hallucinogenic mushrooms, hellish incantations,
ritualistic self mutilation, or hedonistic orgies to please devil patrons.
Although unfounded, untrue and merely a way to discredit the witches and
justify the crusade, these rumours have entrenched in public awareness,
especially with those not well versed with the history and teachings of the
\nameref{sec:Old Ways}.

\graham{Lies all of it! Except for Walburga's night orgies and the mushrooms.}

After their devastation at the hands of the \nameref{sec:Third Crusade}, many
witches left their now dysfunctional covens behind, and instead joined the
towns and villages. The covens were no longer powerful enough to protect their
witches, and thus relied on the common folk for protection. In return they
offered their services, knowledge and magic to further the well being of the
communities they joined.

\aren{A step towards healing the wound left behind by the great betrayal.}

\begin{35e}{Walburga}
  The Walburga only accept women, and are soul casters. They are of course free
  to learn other trades, or professions but are witches of Morana first. Most
  covens move within the ``neutral'' range of alignments, with most showing a
  deep loyalty to each other, Morana, and their traditions; but being allowed
  vast personal freedoms when not serving the coven. While some covens may be
  evil, these are indeed rare.
\end{35e}
