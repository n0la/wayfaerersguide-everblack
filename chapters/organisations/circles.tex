\section{Druidic Circles}
\label{sec:Druidic Circle}

Druidic circles are a group of druids who share a common tradition, aptitude
for magic, as well as a common deity and philosophy. Druidic circles are by
their very nature mysterious, hidden, and secretive, as their individual
members often wield incredible magical power, and their core philosophy
opposes that of the civilised races.

During the early ages of the civilised races, druids lived in harmony with
the other races, being powerful and influential preachers, teachers, shamans,
and philosophers of the \nameref{sec:Old Ways}. However as the age of
\nameref{sec:Strife} began (nearly 11 thousand years ago), the druids realised
the destruction of nature both the monstrous, and core humanoid races were
willing to commit to fuel their engines of war. Over the course of millennia
the druids withdrew from societal live, and began to defend nature by any
means necessary.

After millennia many druids had turned fanatical, aggressive, dangerous, or
outright corrupted, and possessed by \nameref{sec:Daemons}. Many of their
creations, such as the \nameref{sec:Fey}, \nameref{sec:Dire Creatures}, and
\nameref{sec:Lycanthropes}, had brought immense destruction, death, and
devastation to the civilised races. The conflict between the druids, and
the civilised races came to a violent conclusion in the \nameref{sec:First
  Crusade}, where all but hand full of druids were killed.

Many druid circles do not care about race, and are often mixed, even between
core humanoid, and monstrous races. Especially in modern times, druids will
recruit anyone willing to join, as their way of life struggles to survive.
Modern druids are still a scourge upon the lands, commanding fey, attacking
and destroying towns, villages and even baronies in an attempt to allow
nature to reclaim the land.

\begin{note}
  Hatred for druids is still a reality all across Aror, as they powers, and
  their creations are still a threat to civilisations to this day. Most
  people however do not actively crusade against druids anymore, unless the
  druid gives them a reason. Druid players best lie about their heritage,
  by pretending to be priests or clerics.
\end{note}

\subsection{Circle of Rebirth}
\label{sec:Circle of Rebirth}

The \emph{Circle of Rebirth}, or \emph{Morana's Children}, are the only
druidic circle confirmed to still operate to this day. They live secluded
lives in the jungles of \nameref{sec:Goban}, south of the Torainn
Mountains. Druids of this circle have not yet fully abandoned the
\nameref{sec:Old Ways}, yet believe that mankind (and the monstrous races)
have cheated \hyperref[sec:Morana]{lady death}, with supernatural abilities,
and through the use of magical powers. Druids of the circle believe in a
natural circle of life and death, which must not be mocked by careless
application of healing magic, undeath, or resurrections. Druids of the circle
of rebirth abhor undead, as well as restoration magic to heal the weak, and
resurrection magic and thus vow to use them sparingly. Much like any druidic
circle, the druids of rebirth abhor the civilised races and their creations,
and are more than willing to defend nature with violence should it be
threatened by the advancements of civilisation.

The circle was responsible for fusing daemons together with animals, leading
to the creation of \nameref{sec:Dire Creatures}. Since they were the last
circle to remain somewhat close the philosophy of the Old Ways, they were
sheltered from the brunt of the crusade by their spiritual brethren. A number
of circles thus still remain, hidden away in the jungles of Goban, protecting
it, and its inhabitants, by all means necessary. A weed used as a recreational
drug called \nameref{sec:Sarelis} grows in the circle's domain. Many druids
use it, but forbid outsiders from planting, or harvesting it. This brings them
into conflict with farmers, and traders that seek to earn a profit from the
weed.

\begin{35e}{Circle of Rebirth}
  The Circle of Rebirth is considered \emph{lawful evil}, and their patron
  goddess is \nameref{sec:Morana}. A druid of the circle of rebirth can fall
  from favour by creating, helping, or curing undead, or by curing the weak
  and the frail, or by attempting resurrection magic.
\end{35e}

\subsection{Ynar}
\label{sec:Ynar}

The Ynar were a druidic circle that was mostly active in the
\nameref{sec:Dirgewood}, and the \nameref{sec:Toralian Highlands}. The circle
of Ynar was solely responsible for the creation of the \nameref{sec:Fey}, and
where eradicated for it during the first crusade. Those few that still remain
hidden, and keep their allegiance a secret. Druids of Ynar are also often
called \emph{nightowls}, as they prefer to roam, hunt, work and practise their
magic during the night.

Compared to the other druidic circles, the circle of Ynar was initially
relatively peaceful, preferring trickery, and fear to keep the civilised races
away from their woods and marshes. However the millennia of dabbling with the
creation magic that resulted in the fey, they became corrupted, evil and
twisted. Many of the Ynar worshipped powerful fey their ancestors had created,
or worshipped other \hyperref[sec:Percht]{Schiarchperchten} that had helped
them create the fey in the first place.

Not a lot is known about the rituals of the Ynar, as you are hard pressed to
find any active practitioners. Those that do still roam the forests, marshes
and grasslands of \nameref{sec:Eilean Mor} and \nameref{sec:Goltir}, vary
greatly in temper, and personality. Some are ashamed what their ancestors had
done, while others actively scour the land for the lost knowledge of how to
conjure, control and create more fey.

\begin{35e}{Ynar}
  The Ynar may have any alignment, yet their protection of nature against
  anyone and everyone, bars them from bearing a \emph{good} alignments. As any
  druid, if they leave their domain for too long, they will fall and loose
  their powers.
\end{35e}

\subsection{Bodmar}
\label{sec:Bodmar}

The \emph{Bodmar} were a druidic circle mostly found in \nameref{sec:Iafandir},
\nameref{sec:Dirgewood} and the northern stretches of \nameref{sec:Goltir}.
They were responsible for creating true \nameref{sec:Lycanthropes}, and where
thus culled during the first crusade. Due to harsh terrain and dangerous fauna
of the savage lands, many smaller cells were able to hide, and so a few of the
Bondmar survive to this day.

Reports from the crusaders describe the Bodmar druids as aggressive,
dangerous, barely-clothed, and feral in all aspects of live. They live to
emulate the primitive, and feral spirit of the many beasts that roam the wild.
Some of these reports are supported by tales of the monstrous races from the
savage lands that claim that druids of Bodmar are aggressive and feral
tribes, that seek to unite their soul with their shape shifted spirit animals.

The tales surrounding the Bodmar have been exaggerated beyond belief during the
ages, and tales about their savagery have shaped a cultural stereotype that is
applied to all druids. The phrase ``becoming a Bodmar'' is used in may cultures,
describing someone who has gone crazy, or hysteric.

\begin{35e}{Bodmar}
  The Bodmar try to emulate feral, wild, and dangerous creatures (who are
  neutral due to their lack of sentience), living only by the ``laws of the
  wild'' and are thus considered \emph{chaotic evil}. They expert weavers of
  all things shape shifting, and worship the very essence of nature itself. It
  is unclear how they receive their powers.
\end{35e}

\begin{note}
  While the Circle of Rebirth, Ynar, and Selanir are meant for druid players,
  Bodmar are mostly meant to be adversaries.
\end{note}

\subsection{Selanir}
\label{sec:Selanir}

The Selanir druids were once the largest druidic circle of all of Aror, with
members counting in the tens of thousand. They live all across Aror, and
worshipped \nameref{sec:Daemons}, especially \nameref{sec:Leszy}. While not
directly responsible for the creation of any evil creatures, the ancient
Selanir druids used their powers, and numbers to assault baronies, cities, and
even kingdoms in an attempt to stem the tide of civilisation. The Selanir
druids of old used all the tools their brethren had created with deadly
effectiveness, allowing them to topple even the mightiest of baronies. For
millennia of warfare, death and destruction there were cleansed during the
\nameref{sec:First Crusade}.

While the Selanir circles could be found all across Aror in ancient times,
they are all but extinct except for a few remaining surviving circle in
the jungles of \nameref{sec:Yuacata}. Hidden away in the feral jungles they
remain true to the teachings of old, using their magically bound fey servants
to drive away any explorer, treasure hunter, prospector or surveyor that
seeks to exploit the jungle.

They mostly worship daemons, such as \nameref{sec:Leszy}, the Perchten, or
in some cases even \nameref{sec:Xir}.

\begin{35e}{Selanir}
  The Selanir are the only druidic circle that still continue to fight the
  encroaching threat of civilisation with magical powers, enslaved fey,
  and their shape shifting powers. They are considered \emph{neutral evil}.
\end{35e}
