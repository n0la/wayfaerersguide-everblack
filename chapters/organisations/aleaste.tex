\section{Holy Order of Aleaste}
\label{sec:Holy Order of Aleaste}

In \emph{MI:890} the high priestess of the
\hyperref[sec:Four Holy Orders]{Fourth Order}, a dark-elven woman named
Aleaste of \nameref{sec:House Eseriel} questioned the basic dogma of local
interference of the Five Holy Orders. She saw that many city kingdoms rejected
the authority of the five orders, since they often convicted and sentenced
wrong doers even though that openly opposed local rulings, laws and customs.
Aleaste's Order follows a reformed dogma of the \nameref{sec:Order}, that
prefers indirect and more permanent solutions over direct involvement and
interference.

She postulated that instead of fighting local law enforcement, the orders
should instead focus on changing local laws to more closely resemble the laws
and justice sought by the orders. After years of internal strife and struggle
between the orders on how to proceed, Aleaste took the majority knights, priests
and inquisitors of the Fourth Order, of which she was high priestess, and
declared them separate from the five orders.

Aleaste's Order, instead of sending inquisitors throughout the lands, offers
their knowledge, highly educated and experienced judges and prosecutors to
large kingdoms and baronies as advisers. These clerics and priests then aid
and advise the ruling monarchs on new laws, building a fair justice system and
on how to enforce these new laws in a just manner.

Although tainted by the bad reputation of the Third Order, the Holy Order
gained a lot of influence over the centuries. Especially in smaller baronies
that are plagued with chaos, the highly educated advisers and judges are
welcomed to restore order. Thus many of these smaller baronies and earldoms
rely on the Holy Order to provide a framework for laws and security.

Those that join Aleaste's Order are trained as judges, advisors, diplomats, as
well as knights, paladins, inquisitors and investigators.

\begin{35e}{Holy Order of Aleaste}
  All in all the order is considered \emph{lawful neutral}. Most members of
  of the order are bards (yes, they may be lawful), clerics, paladins, and even
  rogues (as investigators and diplomats). Their favoured weapon is the morning
  star, and their domains are protection, law, good and strength.
\end{35e}
