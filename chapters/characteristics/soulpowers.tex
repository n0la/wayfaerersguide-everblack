\section{Soul Powers}
\label{sec:Soul Powers}

\subsection{Overview of Rules}

Soul powers are manifestations of soul energy, and have been shaped by a soul
caster to be useful as spells or auras. Soul powers follow many of the basic
rules of magic, including aspects such as allowing saving throws, casting
duration, verbal or somatic components, or resulting effects. Soul powers,
much like spells or psionic powers, have a power level, and differ wildly in
terms of overall power and flexibility.

Like psionic powers, each soul power costs a given amount soul power points to
cast. However there are no meta-magic feats for soul powers, and spells can be
warped simply by spending more soul power points during casting. One aspect is
true for most soul spells: spells can be cast as a higher power level, simply
by spending the power points required for that spell power level. Whether this
is possible for a particular spell is marked with the ``can be raised''
attribute in the spell's description block. Soul power points regenerate after
an eight hour rest.

Soul auras work differently. They can be activated and last until they are
dismissed. Instead of requiring soul powers to cast, they instead reduce the
maximum amount of power points you have available. Auras always cost twice the
amount of power points that a spell would cost of the same level.

Although soul casters are not limited in the amount spell powers they can
learn, they are limited in the highest power level they may know. For every
HD a soul caster gains, this ceiling - in what the highest power level is he
or she can cast - is raised, and the soul caster learns one additional spell.
Soul powers are not something that are just ``gained'' like arcane, divine
or psionic powers, but should be unlocked, or even created during the course
of a campaign.

If a saving throw is allowed, soul powers have a DC to resist the effect of 10
+ \emph{soul power level} + the caster's \emph{charisma} modifier. A soul
casters caster level is likewise the amount of HD that the soul caster gained,
since awakening, that gave him more than 1 soul power point (thus excluding
classes that give you 0 power points from your soul caster level).

\begin{table}[!htb]
  \captionsetup{labelformat=empty,font={large,bf},position=top}
  \caption{Soul Power Point Cost by Power Level}
  \rowcolors{1}{white}{light-grey}
  \begin{tabular}{p{2cm} p{3cm}}
    \textbf{Power Level} & \textbf{Power Point Cost} \\
    0 &  1 \\
    1 &  3 \\
    2 &  5 \\
    3 &  7 \\
    4 &  9 \\
    5 & 11 \\
    6 & 13 \\
    7 & 17 \\
  \end{tabular}
\end{table}

\begin{table}[!htb]
  \captionsetup{labelformat=empty,font={large,bf},position=top}
  \caption{Maximum Soul Power Level}
  \rowcolors{1}{white}{light-grey}
  \begin{tabular}{p{2cm} p{3cm}}
    \textbf{Level} & \textbf{Highest Power level} \\
    1-3   & 1st \\
    4-6   & 2nd \\
    7-9   & 3rd \\
    10-12 & 4th \\
    13-15 & 5th \\
    16-19 & 6th \\
    >20   & 7th \\
  \end{tabular}
\end{table}

Some soul powers are \emph{soul sundering}. Meaning that they have a chance
to break the target's soul, and thus forcing him to soul awaken. Surviving a
soul power is by far the most common way to awaken to have one's own soul
awoken. It is rarely a pleasant way to awaken, and comes with a risk of
serious long term harm, or even death.

\subsection{Individual Soul Powers}

\begin{table*}[!htb]
  \captionsetup{labelformat=empty,font={large,bf},position=top}
  \caption{Soul Powers}
  \rowcolors{1}{white}{light-grey}
  \begin{tabular}{p{4cm} p{11cm}}
    \textbf{Level 0} & \nobreak \\
    Shining Aura     & You emanate shining light, that illuminates your surroundings. \\
    \textbf{Level 1} & \nobreak \\
    Mend Soul        & You cure a target soul, or creature with a soul, for 1d8+1 points of damage. \\
    War Fuel         & You power your physical prowess with your soul's energy \\
    Aflame           & You set others or objects aflame using soul fire. \\
    \textbf{Level 2} & \nobreak \\
    Soul Fire Ray    & You conjure a ray of soul fire to damage your enemies. \\
    Disjoint         & Leave your body behind and free your soul \\
    \textbf{Level 3} & \nobreak \\
    Fearsome Presence& Your mere presence strikes fear and dread in to the heart of your enemies \\
    Soul Blade       & You fuel your weapons with soul fire, causing them to deal additional damage on hit. \\
    \textbf{Level 4} & \nobreak \\
    Inconspicuous Presence & You go unnoticed, undetected and easily slip the minds of those that do see you. \\
    Violent Disjoint & You force your enemy's soul out of their body \\
    \textbf{Level 5} & \nobreak \\
    \textbf{Level 6} & \nobreak \\
    Defile           & You kill and destroy to gain soul power \\
    Preserve         & You sacrifice your own soul power to restore a defiled area \\
  \end{tabular}
\end{table*}

\begin{soulpower}{Mend Soul}
  \level{1}
  \components{V, S}
  \castingtime{1 standard action}
  \rangetouch
  \instantaneous
  \raising{Yes}
  \savingthrow{Will half (harmless)}

  When laying your hand upon a creature with a soul, or a soul, you grant it
  some of your power to cure 1d8+1 points of damage. For every power level you
  raise the spell, you cure an additional 1d8+1 of damage (maximum 9d8+9).
\end{soulpower}

\begin{soulpower}{Soul Fire Ray}
  \level{2}
  \components{S}
  \castingtime{1 standard action}
  \rangeclose
  \instantaneous
  \raising{Yes}
  \sundering{Yes, near death experience from damage from this spell}

  You conjure a sharp, fiery ray of raw soul energy that you can direct at
  your enemies. Each ray requires a ranged touch attack to hit, and deals 3d6
  points of soul damage. For every power level you raise the power, you gain
  an additional 1d6 points of soul damage (maximum 11d6).
\end{soulpower}

\begin{soulpower}{War Fuel}
  \level{1}
  \components{V}
  \castingtime{1 swift action}
  \rangepersonal
  \raising{Yes}
  \duration{2 rounds (+1 per raise)}

  You use your soul power to overcharge your physical power, gaining +2 soul
  bonus to strength and constitution. This spell can be raised, and for each
  additional level you gain an additional +2 soul bonus to strength and
  constitution (maximum +14), and the spells lasts for an additional round.
\end{soulpower}

\begin{soulpower}{Soul Blade}
  \level{3}
  \components{V}
  \castingtime{1 swift action}
  \range{Personal}
  \target{Any (or all) weapon you wield, or your unarmed strikes}
  \raising{Yes}
  \duration{1 round (+1 per raise)}

  Your weapons or unarmed attacks become imbued with soul damage, and deal
  1d8+3 soul damage on a successful it. For every level this power is raised
  above 3 your weapon deal an additional +1 soul damage (maximum +7), and the
  power lasts one round longer.
\end{soulpower}

\begin{soulpower}{Aflame}
  \level{1}
  \components{S}
  \castingtime{1 standard action}
  \range{Touch}
  \target{Object or person touched}
  \raising{Yes}
  \duration{1 round (+1 per raise) or until doused}
  \savingthrow{Special, see text}

  You conjure a small flame of soul fire and set ablaze a creature or object
  you touch. If your touch attack hits you damage your target for 1d6
  \emph{soul damage}. You can raise the spell, and with each additional power
  level you gain an additional 1d6 damage (maximum 7d6), and the fire lasts
  one round longer.  Souls, or creatures with souls, must succeed a fortitude
  saving throw or be set aflame by the fire. A creature set a flame by your
  soul fire can spent a full round to douse the flames engulfing them.
\end{soulpower}

\begin{soulpower}{Disjoint}
  \level{2}
  \components{-}
  \castingtime{1 standard action}
  \rangepersonal
  \raising{No}
  \duration{1 hour per caster level}

  Your soul steps out of your own body and appears as a spectral ghost. You
  leave your body behind as a soulless husk, that cannot move, think, walk,
  feed or fend for itself. Your body is considered helpless.
  Your spectral image has the same stats, abilities as you, but gains the
  ``Incorporeal'' sub type. You may return to your body by ending the spell
  as a free action, or you automatically return to it when the spell ends.
  If your soul dies, your body remains as a soulless husk, and if you body
  dies you remain as a free soul.
\end{soulpower}

\begin{soulpower}{Violent Disjoint}
  \level{4}
  \components{S}
  \castingtime{1 standard action}
  \rangetouch
  \raising{Yes}
  \duration{1 minute per cast level}
  \savingthrow{Will to resist}

  Works just like ``disjoint'', except you can force another creature's soul
  out of their body. If the saving throw fails, the soul of your target is
  forced out its body, and incapable of returning for the duration. Raising
  this soul power increases your DC for +1 for each level you raise it.
\end{soulpower}

\begin{soulpower}{Defile}
  \level{6}
  \components{S, V}
  \castingtime{1 full round}
  \rangeclose
  \raising{No}
  \savingthrow{None}

  You leech power from the souls surrounding you. From the trees, the birds,
  the insects, the animals and even other people to fuel your own power. But
  everything you draw upon suffers, withers and dies. Every creature, plant
  and wildlife within the radius must succeed a will save or take 1d4+1 negative
  level, and give you the same amount of soul power points in return.

  It is always up to the terrain how many creatures around to draw upon.
  Deserts (either sand or snow) have nothing to draw upon, while lush forests
  or jungles have plenty draw upon. Defilement will always leave a patch of
  land that is barren.
\end{soulpower}

\begin{soulpower}{Preserve}
  \level{6}
  \components{S, V}
  \castingtime{1 full round}
  \rangeclose
  \raising{No}
  \savingthrow{Will save (harmless)}

  You sacrifice your own soul power (and health) to restore life to a
  previously defiled area. You can restore 1d4+1 negative level of everyone in
  your surrounding area and take the same amount soul power damage. If you
  don't have enough soul power points the damage is dealt to your health
  instead.
\end{soulpower}

\subsection{Soul Auras}

\begin{soulpower}{Shining Aura (Aura)}
  \level{0}
  \components{-}
  \castingtime{Free action}
  \rangepersonal
  \raising{No}
  \duration{Until dismissed}

  The light from your soul becomes visible to everyone surrounding you, and
  your aura clearly illuminates everything within a 30 ft. radius, and
  provides a shadowy illumination out to 60 ft.
\end{soulpower}

\begin{soulpower}{Fearsome Presence (Aura)}
  \level{3}
  \components{-}
  \castingtime{Free action}
  \range{Personal}
  \target{Enemies within 30 ft. of you}
  \raising{Yes}
  \duration{Until dismissed}
  \savingthrow{Special, see text}

  An aura of dread and horror emanates from you, causing any creature within
  30 ft. that have a soul (or are souls) to make a will save. If the will save
  fails and the creature's HD are less than 3, the creature is frightened for
  4d6 rounds. If the save fails and the creature's HD are more than 3 the
  creature is shaken for 3d6 rounds instead. You can raise the aura, adding an
  additional one HD to your threshold by doing so (maximum +7).
\end{soulpower}

\begin{soulpower}{Inconspicuous Presence (Aura)}
  \aren{My favourite tool to remain unnoticed in large and crowded places.}

  \level{4}
  \components{-}
  \castingtime{Free action}
  \rangepersonal
  \raising{Yes}
  \duration{Until dismissed}

  You emanate an aura of inconspicuousness, sliding from other people's notice
  and memory easily, if they notice you at all. While the aura is active you
  gain a +4 soul bonus to bluff, hide and move silently. You can raise the
  aura, giving you an additional +2 bonus to those skills for each power level
  (maximum +10 soul power). Creatures that fail the save against your aura
  forget they ever saw you after 1d4 rounds.
\end{soulpower}
