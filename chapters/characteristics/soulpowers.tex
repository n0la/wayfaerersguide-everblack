\section{Soul Powers}
\label{sec:Soul Powers}

\subsection{Overview of Rules}

Soul powers are manifestations of soul energy, and have been shaped by a soul
caster to be useful as spells or auras. Soul powers follow many of the basic
rules of magic, including aspects such as allowing saving throws, casting
duration, verbal or somatic components, or resulting effects. Soul powers,
much like spells or psionic powers, have a power level, and differ wildly in
terms of overall power and flexibility.

Like psionic powers, each soul power costs a given amount soul power points to
cast. However there are no meta-magic feats for soul powers, and spells can be
warped simply by spending more soul power points during casting. One aspect is
true for most soul spells: spells can be cast as a higher power level, simply
by spending the power points required for that spell power level. Whether this
is possible for a particular spell is marked with the ``can be raised''
attribute in the spell's description block. Soul power points regenerate after
an eight hour rest.

Soul auras work differently. They can be activated and last until they are
dismissed. Instead of requiring soul powers to cast, they instead reduce the
maximum amount of power points you have available. Auras always cost twice the
amount of power points that a spell would cost of the same level.

Although soul casters are not limited in the amount spell powers they can
learn, they are limited in the highest power level they may know. For every
HD a soul caster gains, this ceiling - in what the highest power level is he
or she can cast - is raised, and the soul caster learns one additional spell.
Soul powers are not something that are just ``gained'' like arcane, divine
or psionic powers, but should be unlocked, or even created during the course
of a campaign.

If a saving throw is allowed, soul powers have a DC to resist the effect of 10
+ \emph{soul power level} + the caster's \emph{charisma} modifier. A soul
casters caster level is likewise the amount of HD that the soul caster gained,
since awakening, that gave him more than 1 soul power point (thus excluding
classes that give you 0 power points from your soul caster level).

\begin{table}[!htb]
  \captionsetup{labelformat=empty,font={large,bf},position=top}
  \caption{Soul Power Point Cost by Power Level}
  \rowcolors{1}{white}{light-grey}
  \begin{tabular}{p{2cm} p{3cm}}
    \textbf{Power Level} & \textbf{Power Point Cost} \\
    0 &  0 \\
    1 &  1 \\
    2 &  3 \\
    3 &  5 \\
    4 &  7 \\
    5 &  9 \\
    6 & 11 \\
    7 & 13 \\
    8 & 15 \\
    9 & 17
  \end{tabular}
\end{table}

\begin{table}[!htb]
  \captionsetup{labelformat=empty,font={large,bf},position=top}
  \caption{Maximum Soul Power Level}
  \rowcolors{1}{white}{light-grey}
  \begin{tabular}{p{2cm} p{3cm}}
    \textbf{Level} & \textbf{Highest Power level} \\
    1-2   & 1st \\
    3-4   & 2nd \\
    5-6   & 3rd \\
    7-8   & 4th \\
    9-10  & 5th \\
    11-12 & 6th \\
    13-14 & 7th \\
    15-16 & 8th \\
    17-20 & 9th
  \end{tabular}
\end{table}

Some soul powers are \emph{soul sundering}. Meaning that they have a chance
to break the target's soul, and thus forcing him to soul awaken. Surviving a
soul power is by far the most common way to awaken to have one's own soul
awoken. It is rarely a pleasant way to awaken, and comes with a risk of
serious long term harm, or even death.

\subsection{Individual Soul Powers}

\begin{soulpower}{Mend Soul}
  \level{1}
  \components{V, S}
  \castingtime{1 standard action}
  \rangetouch
  \instantaneous
  \raising{Yes}
  \savingthrow{Will half (harmless)}

  When laying your hand upon a creature with a soul, or a soul, you grant it
  some of your power to cure 1d8+1 points of damage. For every power level you
  raise the spell, you cure an additional 1d8+1 of damage (maximum 9d8+9).
\end{soulpower}

\begin{soulpower}{Soul Fire Ray}
  \level{1}
  \components{S}
  \castingtime{1 standard action}
  \rangeclose
  \instantaneous
  \raising{Yes}

  You conjure a sharp, fiery ray of raw soul energy that you can direct at
  your enemies. Each ray requires a ranged touch attack to hit, and deals 2d6
  points of soul damage. For every power level you raise the power, you gain
  an additional 1d6 points of soul damage (maximum 10d6).
\end{soulpower}

\begin{soulpower}{Overcharge}
  \level{1}
  \components{V}
  \castingtime{1 swift action}
  \rangepersonal
  \raising{Yes}
  \duration{2 rounds (+1 per raise)}

  You use your soul power to overcharge your power, gaining +2 soul bonus to
  strength and constitution. This spell can be raised, and for each additional
  level you gain an additional +1 soul bonus to strength and constitution
  (maximum +10), and the spells lasts for an additional round.
\end{soulpower}

\begin{soulpower}{Soul Blade}
  \level{3}
  \components{V}
  \castingtime{1 swift action}
  \range{Personal}
  \target{One or two weapons you wield or touch, or your unarmed attacks}
  \raising{Yes}
  \duration{1 round (+1 per raise)}

  Your weapons or unarmed attacks become imbued with soul damage, and deal
  1d8+1 soul damage on a successful it. For every level this power is raised
  above 2 your weapon deal an additional +1 soul damage (maximum +7), and the
  power lasts one round longer.
\end{soulpower}
