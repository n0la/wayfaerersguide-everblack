\section{Skills}
\label{sec:Skills}

A few skills have new uses, and some additional rules apply to them in
\emph{Everblack}.

\subsection{Craft (Runes)}
\label{sec:Craft Runes}

The new craft skill, \emph{Craft (Runes)}, is used craft stylised runes and
sigils, embellished shapes and complex ritualistic patterns. Some of these
runes and shapes might be used in religions such as the \nameref{sec:Old
  Ways}, especially when writing or deciphering \emph{Ancient
  Teranim}. However this skill is also used to carve runes into the bodies of
sacrifices in combination with \nameref{sec:Rune Magic}.

\subsection{Knowledge}
\label{sec:Knowledge}

The knowledge family of skills have two new entries: \emph{Knowledge (Old Ways)}
and \emph{Knowledge (Soul Magic)}.

\emph{Knowledge (Old Ways)} encompasses the knowledge about rituals, practices,
believes and customs of the religion of the \nameref{sec:Old Ways}. It mixes
knowledge about the religion itself, as well as parts of soul magic, which are
presented in that religion in the form of ancient rituals and incantations.

\emph{Knowledge (Soul Magic)} functions much like \emph{Knowledge (Arcana)}
except for spells, magic and creatures related to soul magic.

\subsection{Speak Language}
\label{sec:Speak Language}

The common languages of \emph{Everblack} are summarized in the following
\hyperref[tbl:Languages]{table}.

\begin{table*}[!htb]
  \small
  \captionsetup{labelformat=empty,font={large,bf},position=top}
  \caption{Languages of Aror} \label{tbl:Languages}
  \rowcolors{1}{white}{light-grey}
  \begin{tabular}{l p{8cm} l}
    \textbf{Language} & \textbf{Typical Speakers} & \textbf{Alphabet} \\
    Ancient Teranim & \nameref{sec:Tynrikke}      & Ancient Teranim \\
    Danvark         & \nameref{sec:Morkan}, \nameref{sec:Norbury} & Old Teranim \\
    Doresh          & Deep races, such as \nameref{sec:Deepkin} & Old Teranim \\
    Draconic        & Dragons, mages and scholars & Draconic, Teranim \\
    Druidic         & Druids (secret language)    & Ancient Teranim \\
    Enro'ad         & Elves, Halflings            & Taavid \\
    Goblin          & Goblins, Bugbears           & Giant \\
    Giant           & Giants, ogres, trolls       & Giant \\
    Gnoll           & Gnolls                      & Giant \\
    Ilian           & Ilians                      & Old Teranim \\
    Inua            & \nameref{sec:Inua}          & Inua \\
    Kalest          & People of \nameref{sec:Arania} & Teranim \\
    Old Teranim     & People of the \nameref{sec:Dirgewood} and by followers of the \nameref{sec:Old Ways} & Old Teranim \\
    Old High Teranim& local dialect of\nameref{sec:Tredegar} & Old Teranim \\
    Orcish          & Orcs                        & Giant \\
    Reatham         & People of \nameref{sec:Forsby} & Teranim \\
    Rutari          & Dwarves                     & Rutari \\
    Senari          & Fey                         & - \\
    Taavid          & Halflings                   & Taavid \\
    Teranim         & Humans, Halflings, Elves    & Teranim \\
    Tolarn          & Hobgoblins                  & Giant \\
  \end{tabular}
\end{table*}

\begin{note}
  Draconic is Lojban, or at least heavily based on it, and thus follows
  different pronunciation rules. The apostrophe in words is a ``[h]'' sound,
  while an ``x'' is an unvoiced velar fricative, like Spanish Jose or German
  Bach.
\end{note}

\subsection{Forgery}
\label{sec:Forgery}

Anyone who can cast \emph{Arcane Mark}, can attempt to forge a
\hyperref[sec:Citizen Mark]{citizen mark} or a \hyperref[sec:Slave Mark]{slave
mark} using the \emph{Forgery} skill while casting the spell. This special
use of forgery and the spell stretches the casting time to ten minutes. The
character takes the \emph{Forgery} check as normal, taking a -10 penalty on
the check. The same procedure can also be used to forge \hyperref[sec:Nobility
Mark]{nobility marks} but the \emph{Forgery} check incurs an additional -10
penalty, for a total of a -20 penalty on the skill check.

\subsection{Open Lock}
\label{sec:Open Lock}

Anyone who can also use the skill \emph{Spellcraft} can attempt to remove
a \nameref{sec:Slave Band} from a slave by using the \emph{Open Lock} skill.
This attempt requires a basic arcane lab, arcane ingredients worth fifty
shards which are consumed in the process, and takes one our to perform.
The DC of the \emph{Open Lock} and \emph{Spellcraft} is the caster level
of the slave band plus 30.

\FloatBarrier
