\twocolumn
\section*{Welcome to the Everblack campaign setting}

The \emph{Everblack} campaign setting is a D\&D setting for both 3.5e.
It is written from the view of the two most prominent NPCs of the world:
\emph{Graham Balance} an influential archmagi and bard, and \emph{Aren Fel},
also known as the \emph{Undying Witch}. But more about them later on.

It is in fact a labour of love of one person, with input from many different
people, players, and carried together over the course of many years.

\section*{Tone of Everblack}

One of the main defining features of \emph{Everblack} is the tone and attitude
of the campaign setting. It combines traditional fantasy settings with the love
for horror. It also aims to introduce grit, moral ambiguity, shades of moral
gray and an everlasting fear of death. Although \emph{Dark Sun} already does
many of these aspects extra ordinarily well already, \emph{Everblack} does not
really on a post-apocalyptic surrounding. It's just that in \emph{Everblack}
everything tries to eat, kill or enslave you while you can also sniff pretty
flowers on the side of the road.

The world of \emph{Everblack} is old, and rich in history. Filled with old
legends, gods long dead, and civilisations that have long fallen. Those that
venture outside the save walls of the major city kingdoms will find death and
peril, but also adventures and great riches.

Magic is a major part of the setting, and is augmented by a new, and fourth
form of magic: \emph{soul magic}. While in \emph{Everblack} both arcane and
divine magic come from the gods, \emph{soul magic} is inherent to every living
being, much like psionic powers. However \emph{soul magic} is often
misunderstood, shunned and even forbidden by the followers of the gods.

But still the world of \emph{Everblack} is a place where adventurers set forth
to slay the beast and bring back the gold. A world full of magic, ancient
artifacts, but also of eldritch horrors, great strife and despair. The normal
folk.

\section*{Six things to you should know}

\begin{enumerate}
 \litem{Not everything of D\&D fits} Many creatures, concepts, classes and
 races that are part of the basic D\&D do not fit into \emph{Everblack}. Some
 monsters may be extinct, classes might be rarer in \emph{Everblack} than
 elsewhere, and some concepts do not directly fit within the world
 itself. While this might bar you from certain concepts and mechanics, this
 was done deliberately to enhance the tone and attitude of the campaign
 setting.

 \litem{Overall tone} \emph{Everblack} attempts to blend horror and evil with
 D\&D. It aims to convey a world where ancient malevolent beings rule from the
 shadows. Where horrible and sadistic creatures just lurk in the dark forest
 beyond your village waiting for the perfect opportunity to strike. Where the
 world itself is dark, horrible, cruel, unfair, dirty and filthy. A world were
 slavery is as much accepted, as an evil Inquisition that enacts vigilante
 justice across the face of the world. Where \emph{soul witches} are
 meticulously persecuted and burnt at stake, and entire city kingdoms move
 forth with a continent wide genocide of what they deem \emph{lesser} races.
 \emph{Everblack} is not a nice neighbourhood.

 \litem{Soul Magic} The setting introduces a new, fourth kind of magic to the
 world: \emph{soul magic}. It postulates that every living being has a soul,
 from which the being can draw power from. This power might enhance martial
 abilities or may be directly shaped into magic spells. Arcane and divine
 magic comes directly from the realm of the gods, and they have used both as a
 tool to manipulate the mortal races for eons. They see soul magic as an
 aberration, and thus manipulate their followers into seeking out and
 destroying soul magic everywhere they can find it. To those untrained in the
 arts of the soul, a soul afflicted person might just another undead creature,
 fuelling misconceptions and ignorance among the general populace.

 \litem{Aror is a global place} The planet of Everblack, known as Aror, has a
 few very important and big city kingdoms which are connected with each other
 with an arcane teleporter network. Although the cities charge money for the
 usage of these teleporters it has made the world smaller. Commoners, traders
 and adventurers use them regularly to visit other city kingdoms, to travel,
 do trade and exchange cultures. These teleporters are limited to large cities
 as the network requires constant care of magi and wizards to be kept
 operational.

 \litem{Everblack} The world is named after the \emph{Everblack} crystal. Pitch
 black crystals that can store any magical energy. They are often used to aid
 casters, fuel arcane machinery. \emph{Everblack} is extremely rare, so much so
 that tiny shards of it have replaced platin as the major currency for large
 purchases. The crystals are used in \emph{soul magic} research, but also in
 arcane experiments and arcane machinery. It his highly sought after all across
 the world, and entire societies depend on it for mining, trade and arcane
 constructions.

 \litem{New Races} In addition to the common player races, \emph{Everblack}
 introduces two new races: the \emph{Deepkin} as well as the \emph{Umgeher}.
 \emph{Deepkin} are the cavern dwelling cousins of the human races, with dark
 vision, bright white skin and flowing red hair. While \emph{Umgeher} are a
 half undead species of humans that were created by powerful vampires a long
 time ago as servants.

\end{enumerate}

\pagebreak

\onecolumn
\section*{Foreword by Graham Balance}

The following book should serve you well as an introduction to the
place you might only know as the place \emph{where everblack comes
  from}. I, Archmagi Graham Balance, would like you welcome you to
our corner of the multiverse: ``Aror''.

This book will serve you as a guide to Aror's many places, cultures,
customs, people and creatures. I am fortunate to have counted this
place as my home for many years, and it holds a special place in my
heart. Great was the dismay when I had to discover on my frequent
travels and journyes to other planes, that our tiny little speck of
the multiverse was not very well know, or perhaps only known as the
``place where Everblack crystals come from''. While true, it saddens
me deeply to see our fine planet, and its people, reduced to the mere
fact that we export black crystals charged with arcane energy. That
experience was the sole motivator on why I am wrote this book: To bring
our way of life to you, dear reader.

I have interviewed many that did come to Aror, and most of them only
remembered the bad things that had happened to them in their short
visit: unfriendly locals speaking weird tongues, stale ale, and the
occasional blind subterranean humanoid that tried to eat them. While
many of these aspects are true, they are warped exaggerations of the
truth. My sincere hopes are that, after reading through this book, you
have arrived at the conclusion that Aror is a place worth visiting,
that our ale is as good as any - if not better - and that the
cannibalistic blind humanoid creatures of the deep are too far and few
between to be a real threat.

You will have heard our most treasured places, such as the holy lake
of \emph{Mu'ut} in the south; the lone yet towering \emph{Goban}
mountain in the east; the endless treasures, archipelagos and beaches
that the \emph{Silver Isles} have to offer; or the vast mountains,
lush forests, and friendly locals of \emph{Eilean Mor}. If that does
not impress you, I am sure the grandious library and arcane college of
\emph{Fes el-Bali}, the towering cliff side castle of \emph{Forsby} or
the endless riches in art and music of \emph{Nen-Hilith} surely will
make a lasting impression.

\section*{Additional Foreword by Aren Fel}

My good old friend Graham is right in one aspect: Aror has an insidious
way of sneaking into your heart, embedding itself so deeply that not
even a Dryad could convince the roots to leave you. But he is wrong in
so many other things. The ale has not gotten better, and the people
are still unfriendly. But much has changed since the almost 4000 years
this book was originally published. Gods have vanished, civilisations
have fallen, generations have been born and have died; and most important
of all - cultures, languages and entire epochs have been lost only to be
rediscovered centuries later.

I, Aren Fel, have done my best to adapt, change and translate the
book, so that the work of my good friend continues as he originally
intended: as an useful guide to our world, both past and present. In
many ways the book has now exceeded it's original purpose. It was once
a mere introduction to Aror, but it is now also a chronicle of our
history. A lens into our past, and perhaps a warning for our future.

Be safe adventurer, and fasten this book to your belt for easy access,
in case any blind cannibalistic humanoids chase you throw a cavern.
