\section{Games}
\label{sec:Games}

If you ever wander into a local establishment you might be interested in
joining the locals for a game of cards. This section of the book will explain
the rules of the most commonly played card games on Aror, but be aware that
local variants and house rules are common as well. Also be aware of cheaters,
and those that seek to exploit your inexperience to pluck your money right
out of your pocket.

\subsection{Kratzen}
\label{sec:Kratzen}

Kratzen (Reatham for ``scratch'') is a card game popular in
\nameref{sec:Forsby}, as well as the \nameref{sec:Toralian Highlands}. It is
played with 4 to 6 players.

The game is played with a deck of 32 cards, that come in four distinct
colours: hearts (red heart), diamonds (bells), spades (green leaf), and clubs
(acorns). The cards ranking is: Mother / Sow > King > Queen / Ober > Jack /
Unter > Ten > Nine > Eight > Seven. However the seven of diamonds (bells),
also called the \emph{Weli}, is considered the second highest card, just beneath
the mother of the trump.

At the very beginning each player pays the minimum bet. The dealer must pay
the minimum bet again before dealing any cards. The dealer then deals out two
cards to each player, before giving himself an open third card that determines
the colour of the trump. If the open third card is the \emph{Weli}, then the
dealer must make a choice: Either they bless it with another card (whose
colour becomes the trump colour), and they automatically take on the role of
the \emph{striker}, or they forgo blessing in which case the \emph{Weli} turns
into a regular seven of diamonds, losing its status of the second highest
card. After the trump colour has been determined, the dealer gives two
additional cards to other players, and themselves.

Now the next player in line must make a statement regarding their role in the
game: \emph{Pass} means that the player does not wish to make a
statement. \emph{Or} means that the player will beat the game with a different
colour as trump, and \emph{strike} declares that the player is capable of
winning the game. A player who wishes to bid must outbid the players before
them, for example a player can declare ``or'', but could be outbid by the next
player who declares a \emph{strike}. If every player passes then the cards are
shuffled again, and the next player deals. Don't forget to pay the minimum bet
as a dealer.

If no one overrules an \emph{or}, then the dealer must openly deal cards until
a new trump colour has been chosen. This new card is given to the one who
declared the \emph{or}, and they become the \emph{striker}. Now that a
\emph{striker} has been chosen (either by an \emph{or} or because one player
declared a strike), all others have a chance to either \emph{come along} (play
against the striker), or \emph{stay at home} (skip the game). All those that
\emph{come along} must make one trick, and the \emph{striker} must take two
tricks. If no one plays against the \emph{striker}, then the \emph{striker}
automatically wins the pot.

Then any players still in the game, starting with the \emph{striker}, must
reduce their hand to four cards, and may exchange up to three cards from the
talon. The \emph{striker} is allowed to dismiss three cards, and buy four new
ones from the talon by openly presenting the Mother of the trump colour should
they have it in hand. Once all players have reduced their cards to four, the
\emph{striker} plays the first card, and the other players must play the same
colour if they have it (Farbzwang). If they don't have that colour they must
beat it with a higher trump card. They may add any card if they don't have the
colour, or can't beat the highest card in play. The player that makes the
trick is next in turn to play the first card, and game continues until four
rounds in total have been played.

Any player who \emph{came along} must score one trick to win, and the
\emph{striker} must secure two tricks to win. After all tricks have been made
the pot is divided into four parts, and each player with a trick receives a
part of the pot. The remainder goes to the \emph{striker}. If a player has
\emph{fallen} (i.e. failed to gain at least one trick), they must pay current
amount of the pot into the next pot. If the striker has \emph{fallen} (i.e.
failed to secure at least two tricks), then they must pay double of the
current pot into the next pot.

\aren{And now you know why people call this game ``one step above robbery''
  as the pot has a tendency to grow exponentially.}

Many variant rules exist, for example in some games a dealer can opt to simply
reshuffle, and hand over dealership to the next person should their trump card
be a seven of hearts, spades or clubs. Other games allow players to purchase
an ``arse'', in which they discard three cards, present the forth openly, and
receive three new hidden cards. A fifth card is then dealt to them openly, and
should that card be a trump, more open cards are dealt to them until they
receive a non-trump card. Many friendly, and low-stakes games simply limit the
amount of money losers have to pay into the pot. Other games have a
\emph{pass-or}, which is ranked lower than a regular \emph{or}, in which only
one card is revealed and its colour becomes the new trump colour. This of
course has the risk of leaving the trump colour unchanged.
