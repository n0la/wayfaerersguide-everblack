\section{Geography of Aror}
\label{sec:Geography}

The land of Aror encompasses eight great continents, of which all but one are
inhabited by sentient races. The planet has two polar ice caps, as well as lush
rainforests around the equator. Both the northern and southern hemispheres
experience moderate to cold climate depending on the latitude. Latitude 0 runs
through what has been determined to be central island in the middle of the
archipelago known as the \nameref{sec:Silver Isles}.

\subsection{Arania}
\label{sec:Arania}

To the west of Karnak, and in the south western corner of Aror lies the
continent of \emph{Arania}. It is considered the birth place of the ancestors
of all sentient humanoid races, such as the elves, halflings, dwarves and
humans. The continent is a huge and vast steppe, dotted with the occasional
mountain ranges, deserts and fertile oasis. Arania is home to many native and
nomadic tribes, wild animals such as elephants, lions as well as three of the
largest and oldest city kingdoms: \nameref{sec:Fes al-Bashir} to the north
east, and \nameref{sec:Esmayar} and \nameref{sec:El-Fayam}.

In the centre of the continent lies a vast sand desert called the
\nameref{sec:Suam Desert}, which is home to many nomadic tribes of humanoids,
gnolls, hobgoblins, naga, and medusa. South of the desert lies a huge lake,
which feeds many steppes, oasis and other fertile regions surrounding it.

\subsection{Draigynus}
\label{sec:Draigynus}

\emph{Draigynus} is an elongated continent that encircles Farlar to the south
in a half-moon shape. It is close to the southern pale, and thus features a
temperate continental climate to the north, and a subarctic climate to the
south. Very few humanoid races live there, however a small colony of
\hyperref[sec:Dragons]{dragons} has made this continent their home. It is
mostly inhabited by a few sturdy tribes of \hyperref[sec:Snow Elves]{snow
  elves}, and various subarctic beast races.

\subsection{Eilean Mor}
\label{sec:Eilean Mor}

Far to the west, and north of \emph{Arania} lies the continent of \emph{Eilean
  Mor}. It is split in half by a large mountain range called the
\nameref{sec:Great Divide} that runs from all the from the north east down to
the south west. South of the Great Divide lies a vast boreal forest called the
\nameref{sec:Dirgewood}. The continent features subarctic climate in the
north, and temperate climate in the middle, and warmer climate in the
south. Home to many sentient humanoid races, but most predominantly elves,
humans, halflings, and dwarves, it houses four large city kingdoms:
\nameref{sec:Norbury} off the coast in the far north, \nameref{sec:Hraglund}
to the east, \nameref{sec:Helmarnock} to the south east, and
\nameref{sec:Tredegar} to the south west. Eilean Mor is the first continent
that was settled by the humanoid races after the great ice age, and often
referred to as the \emph{old world}.

\subsubsection{Eafiadir}
\label{sec:Eafiadir}

North of Eilean Mor lies a massive dormant volcano called \emph{Eafiadir}. It
is the source of many rivers, and is surrounded by many poisonous sulphur pits
and lakes. Although the area around the volcano is more sparsely populated
there are many smaller baronies and kingdoms on the shores of the north and
north west of the mountain.

\subsection{Farlar}
\label{sec:Farlar}

South of Goltir lies the smaller continent of \emph{Farlar}. Known for its
diverse biome and hilly country side in the north, and cold tundra in the
south. It is now home to mostly \hyperref[sec:Giants]{giants}, giant races
such as ogres and trolls, and also \hyperref[sec:Diarim]{diarim}. The north is
covered with thick and lush forests, which are watered by several large
streams and lakes. These lakes and streams are fed by the fresh rainfall that
falls against the mountain range called \emph{Lias'wa}. The mountain range
splits the continent in half, and allows for a colder climate in the southern
part of the continent. Among all the continents Farlar is by far the smallest,
and once housed the sprawling city kingdom of \emph{Nen-Hilith} before the
giants forcefully took the land as their own.

\subsection{Goltir}
\label{sec:Goltir}

The largest and westernmost continent of Aror, named \emph{Goltir} stretches
from the northern polar ice caps beyond the equator. It is separated in two
large landmasses by a mountain range called \emph{Torainn}. Although the
entire continent is called Goltir, it is often divided into north and south
Goltir, split east to west by the Torainn mountain range.

North Goltir is the home of the city kingdoms of \nameref{sec:Forsby},
\nameref{sec:Kesmar} and \nameref{sec:Stenheim}. North Goltir also known as
the ``land of lakes'' as its entire central landmass is dotted with millions
of smaller lakes, rivers and springs. The central area between Kesmar and
Forsby is called the \emph{Toralian} highlands, and is ruled by various smaller
baronies, earldoms and individual tribes.

A mountain range called the \emph{Alparan} mountains stretches from the centre
to north east, reaching into the polar ice cap of the northern pale. In the
west, the continent stretches out a land tongue towards the Silver Isles,
which was responsible for allowing the early humanoids to spread to the
archipelago. To the south the continent features a vast sand desert, called
the \emph{Walusian Desert} that reaches the foot of the mountain chain
\emph{Torainn}.

North Goltir is home to many sentient humanoid as well as beast races. Humans,
halflings, elves and dwarves live in small baronies in the moderate climate of
the central regions. They share their land with tribes of orcs, bugbears,
hobgoblin, goblins, ogres, trolls, and gnolls who prefer the desert region of
the south.

\subsubsection{Cnámh Mountains}
\label{sec:Cnamh Mountains}

The Cnámh mountain range resides on the eastern shores of North Goltir. They
run from the central regions of the Torlian highlands all the way east towards
the great sea. The mountain range is famous for housing the
\hyperref[sec:Kesmar]{Blackhammer Clan} the biggest and oldest dwarven clan on
all of Aror. South of the mountain range, along a river fed by the mountains,
lies the city of \nameref{sec:Kesmar}.

\subsubsection{South Goltir}
\label{sec:South Goltir}

South of the mountain range of \emph{Torainn} lies the continent \emph{South
  Goltir}. From the mountain range southward it is mostly steppe and grassland
but turns into a thick rainforest around the equator. The most prominent
feature is the almost 9000 metre high \emph{Goban} mountain that peeks forth
from the impenetrable rainforest. The mountain is also the source of the
\emph{Al'ahri} river that runs all the way south through the rainforest, the
southern desert into the sea. The desert and jungle are mostly inhabited by
native and nomadic tribes of elves, monstrous races, as well as certain beast
races such as harpies and gorgons. Gnolls and nomadic humanoid tribes roam the
desert, while the far south hosts the city kingdom of \emph{Avenfjord}.

\subsubsection{Goban}
\label{sec:Goban}

Goban is a large mountain peak that towers above a smaller mountain range in
the centre of southern Goltir. With a size of roughly 9000 metres, it is the
largest mountain on all of Aror. It is the origin of many rivers that worm
their way through southern Goltir, making the land surrounding the mountain
fertile and humid.

Just north of the mountain, and south of Torainn, is also a vast grassland and
jungle called \emph{Goban Jungle}. It is fuelled by the many rivers and their
side arms that spring forth from both the Torainn and Goban mountain range.
The jungle is home to many exotic animals, various tribes of both humanoid and
monstrous races, as well as many elusive and exotic beast races.

South of the mountain is a vast sand desert, called the \emph{Goban desert}.
It features a few fertile oasis and grass lands, but is mostly an inhospitable
sea of sand and heat. Many nomadic humanoid tribes (especially halflings), as
well as many nomadic monstrous tribes (such as gnolls, hobgoblins and ogres)
call this desert their home.

\subsection{Iâfandir}
\label{sec:Iafandir}

Far to the north west lies the continent of \emph{Iâfandir}. It is often
called the \emph{savage lands}, because of its harsh subarctic climate, cold
winters, and the many beasts and beast races that call it home. Most of the
civilisations and tribes there live on the shores along the southern bank, as
the land becomes a frozen tundra further north. It is home of many
\hyperref[sec:Snow Elves]{snow elves} and the \nameref{sec:Tynrikke}, as well as
various monstrous races such as ogres, trolls and hobgoblins, it became
synonymous with a cold climate, harsh conditions and a primal, more savage way
of living.

Although only populated by smaller tribes, villages and perhaps the occasional
snow elven or hobgoblin city, the continent was the birth place of the youngest
city kingdom on Aror: \nameref{sec:Morkan}.

\subsection{Karnak}
\label{sec:Karnak}

West of the land of the dragons is the massive continent of \emph{Karnak}.The
continent itself features a massive lake in its centre, called \emph{Mu'ut}.
The lake is encased in huge mountain ranges, and mostly inaccessible by
land. The lake itself is considered holy to many of the native tribes of the
continent, and is also the home of an ancient evil god called
\nameref{sec:Xir}.

The northern, eastern and western parts of the continent are covered in a vast
and untouched rain forest called \nameref{sec:Yuacata}. The continent is
mostly inhabited by a vast amount of small tribes of wood elves, beast races
such as gorgons, harpies and many different and often dangerous animals such
as apes, primates, tigers and jaguars.

It has a vast chain of large islands off the northern coast, which are mostly
covered in rain forests, called the \emph{Kanaria Archipelago}. It houses many
native tribes but is also home to many smaller outposts of various trading
guilds, mining corporations, slavers and pirates, that exploit the archipelago
for its resources.

\subsection{Silver Isles}

In the centre of the vast sea that separates all the continent, stretching in
from the land tongue of \emph{North Goltir}, lies the archipelago called the
\nameref{sec:Silver Isles}. A huge network of thousands of smaller islands and
inlets, which are mostly covered with rain forests and jungles. It is the
native home of many beasts, native tribes as well as newly settled villages
and expeditions that seek to find what the islands are named after: mineral
riches.
