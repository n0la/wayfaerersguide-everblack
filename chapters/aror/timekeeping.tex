\section{Time Keeping}

A year on \emph{Aror} takes roughly 385 days. Years are prefixed with an
\emph{Aeon}, which are only changed in case of a world-shattering event
or shift, or in case the year numbers become unwieldy. As of the last
edit of this book, the current Aeon prefix is \emph{MI} which stands for
the Aeon of Midaris. In official records you will also find prefixes for
which calendar is used. For example \emph{S/20/05 MI:2002} denotes the
twentieth day of the fifth month in the year 2002 in the Aeon of
Midaris, as described by the \emph{Storst} calendar.

The last \emph{aeon} is now called ``gamla tiden'' (or the ``old age'' in old
teranim), and it began almost three and half thousand years ago (3411 to be
precise). Its prefix is \emph{GT}.

\subsection{Old Calendar}
\label{sec:Old Calendar}

If you use the orbit of the bigger moon \emph{Lilest} to create a
calendar, you get twelve (12) months, with thirty-two (32) days
each. A month is then often broken down to four (4) weeks with eight
(8) days each. Although the \emph{Lilest} based calendar was used
preliminary in the norther hemisphere, it has fallen out of favour and
has been replaced with the \emph{Storst} calendar by MI:1000. Many
cultures still use the old calendar, but most cultures of relevance
have made the switch to the new calendar.

\subsection{New Calendar}
\label{sec:New Calendar}

A calendar utilising the rotation of the moon \emph{Storst} is favoured
in the southern hemisphere. Since most arcane and scientific studies are
done in the southern hemisphere, it has been the de facto calendar of all
of Aror since a few decades.

It separates the year into eight (8) months, with forty-eight (48) days
each, which is then again subdivided into four months with twelve (12)
days each.
