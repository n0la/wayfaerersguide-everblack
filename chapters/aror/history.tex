\section{History of the World}
\label{sec:History}

\subsection{Early History}

Roughly sixty thousand years ago an ice age covered most of the world
of \emph{Aror} in vast sheets of ice and snow. Continents that were separated
by sea, where now suddenly accessible through thick sheets of ice. This
allowed the \textbf{four core humanoid races}, the
\hyperref[sec:Humans]{humans}, \hyperref[sec:Elves]{elves},
\hyperref[sec:Dwarves]{dwarves} and \hyperref[sec:Halflings]{halflings}
to migrate away from \nameref{sec:Arania} and settle the entire known
world. Although the humanoid races were successful in adapting to the new
lands and challenges, they had to face fierce opposition.

\subsection{Aeon of Strife}
\label{sec:Aeon of Strife}

When the humanoid races arrived in the other continents around a thousand
years before the aeon of \emph{GT}, they found that they were already
inhabited by sentient races. \textbf{Monstrous races} such as ogres,
minotaurs, trolls, hobgoblins, gnolls, bugbears already called these
continents their home, alongside non-sentient monstrous species such as
hydras, manticores and wyverns. Through ancient stories of the dwarves, early
writings, ancient stone tablets and even cave paintings, it was revealed that
the humanoid ancestors were in a constant state of conflict with these
monstrous races. Wars, skirmishes, and often the destruction of entire early
villages and even cities was a common occurrence in early history. This early
time of constant fighting, turmoil and wars against the sentient beast races
is referred to as the \emph{aeon of strife}.

The aeon saw the birth of many new races, such as
\hyperref[sec:Vampires]{vampires}, \hyperref[sec:Fey]{fey} and
\hyperref[sec:Lycanthropes]{lycanthropes}. It also saw a vast destruction of
nature, as both sides of the war cut down forests, dried swamps, and forever
corrupted landscapes and areas with dark and foul magic.

Even though the aeon of strife is now thousands years past, it is still
vividly remembered in both humanoid and monstrous cultures through stories,
song, and tradition.

\subsection{Schism}
\label{sec:Schism}

The ultimately successful survival strategy to deal with such a hostility
during the strife ingrained itself as the cultural core of the humanoid
races. The dwarves went underground and organised themselves in strict
hierarchical clans and cities, that allowed them to optimise their economies to
the harsh realities and lack of resources of the depths. Some humans and elves
followed the dwarves underground but ultimately failed to replicate the
dwarven's success. With a few notable exceptions such as \nameref{sec:Stenheim}.
Although the \nameref{sec:Deepkin}, the underground dwelling cousins of the
humans, built large civilisations underground they were ultimately defeated by
the sentient races of the depths and driven to the surface. The dark elves
instead relied on small clans, families and by being constantly on the move to
ensure the survival of their species.

Meanwhile on the surface elves and halflings sought to settle as far away from
the monstrous races as possible, leading them to the continent of \emph{South
Goltir} and \emph{Farlar}. Other elves settled in the frozen north and south,
becoming known as the \nameref{sec:Snow Elves}. Humans on the other hand used
their ingenuity and skill to build great civilisations and cities that could
potentially withstand the skirmishes and sieges of the sentient monstrous
races. Through many iterations over the course of thousands of years, which
resulted in countless destroyed and ransacked cities and fallen civilisations,
humans have now achieved the unthinkable: dethrone the monstrous races as the
predominant species across all of Aror.

\subsection{Exodus from the Depths}
\label{sec:Exodus from the Depths}

Now many elves and halflings have joined the human effort of building large
centres of civilisation, enriching the predominantly human city kingdoms that
dot the world of \emph{Aror}. Their struggle against the sentient monstrous
races is far from over, especially in the central regions of the continents,
or from the still predominantly monstrous continent of \emph{Iâfandir}. The
majority of dwarves have remained underground, continuing their strict ways of
life as it has served them for aeons. A success that was not shared by the
dark elves and deepkin, who have mostly abandoned the deep and returned to the
surface.

\subsection{First City Kingdoms}

The tide of the aeon long war against the monstrous humanoids turned in favour
for the humanoid races when the first large city kingdoms were founded. These
huge and vast cities provided defence, security and above else a place where
society, culture, and means of production could grow and flourish. The first
city kingdom to be founded was \nameref{sec:Fes al-Bashir}, and its success
in the battles during the aeon of strife, was soon copied across the world.
The early years of what is now the the aeon of \emph{GT}, saw the formation
of many powerful nations, such as \nameref{sec:Hraglund}, \nameref{sec:Esmayar}
and \nameref{sec:Helmarnock}, which helped cement the humanoid victory and
conquest with brick and stone.
