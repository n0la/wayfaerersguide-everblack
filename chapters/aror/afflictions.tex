\section{Afflictions}
\label{sec:Afflictions}

Aror is not only home to various dangerous races, but also to various
maladies, sicknesses, and diseases that may be dangerous to you. Should you be
an inter-planar traveller and show any symptoms listed here, you best consult
a priest, cleric or a follower of \nameref{sec:Ishtar}. The authors of this
book are not competent enough to give medical advise.

\subsection{Rot}
\label{sec:Rot}

The ``rot'' or ``black rot'' is a disease that most people of Aror get during
their early childhood. Its first stage is harmless, with sore, and flaking
skin around the toes, fingers, and the nose and mouth. If the child is strong
enough the disease will never progress any further, and disappear after one or
two weeks. In such a case the child has been immunised against the black rot
for the rest of its life.

In rare cases however, especially in weak, malnourished or children that
already suffer another disease at the same time, the disease will advance to
stage two, and begin to fester and spread. The skin of the diseased will begin
to peel, the flesh turns black, and ultimately rot, causing immense pain, and
suffering. The severity of the disease varies greatly. In mild cases only one
hand or foot is affected, while severe cases have it in all or most limbs,
allowing them to live for decades before the disease becomes fatal. Those
unfortunate enough to have the rot fester in the face have only a few years to
life, as nourishing yourself becomes ever more difficult. The last stage of
the black root is ultimately fatal, as over the course of years the disease
will spread from the extremities towards the torso, ultimately infecting major
organs.

A herbal medicine exists for stage one rot, which is readily available in
alchemy stores. But no natural medicine or treatment exists for the second
stage of the disease. Those afflicted with the advanced stage of the black rot
are usually invalid, relying on others to provide for them, as motor functions
are severely limited in the affected limbs.

\begin{35e}{Rot}
  The disease \emph{Rot} has DC12, and causes 1d4 points of \emph{dexterity}
  and \emph{charisma} damage. If untreated the disease will advance to stage
  two within two weeks, causing 1d6 \emph{dexterity} drain, and 1d6
  \emph{charisma} drain.

  Stage one rot can easily be cured by succeeding a \emph{Survival} check DC15
  to find the correct herb, and a \emph{Craft Alchemy} check DC13 to brew a
  tea from it. The herb is readily available in well-sorted alchemy stores.
\end{35e}


\subsection{Black Blight}
\label{sec:Black Blight}

The \emph{Black Blight} (black pest, or miner's disease) is a living,
poisonous organism that is capable of infecting any living creature. The
blight exists as a black gooey liquid beneath the surface, and once it has
sensed heat emanating from a living being, it will slowly move towards the
source in an attempt to infect the creature. Its most dangerous when dissolved
in underwater lakes, as only a few drops are sufficient to infect a
human-sized creature, or when stuck to cave ceilings as it can drop itself
down onto unsuspecting cave explorers.

Once a creature has been infected it will show symptoms of the common flu:
head ache, cough, sore limbs, fever and running nose. The blight is thus
often mistaken for the common cold, and not treated any further. At this
stage of the blight, strict bed rest, and a strict medicinal regimen from
a doctor can defeat the blight before it causes any more damage.

However if it remains untreated the blight moves on to stage two within ten
days or a week. Those infected will begin to lose their grasp on memory,
become delusional, experience hallucinations, until ultimately, they become
feral and aggressive within two weeks. Once this stage has been reached the
infected become ``blight bearers'', who can easily transmit the disease to
others with scratches, bites, or exchange of fluids. Blight bearers are
exceedingly dangerous within the confines of cities, as they can easily infect
hundreds of people by contaminating wells with their blood. A blight bearer
can only be stopped by killing them, and burning their bodies to destroy
the disease.

In \emph{MI:1680} a single blight bearer was responsible for the death of
roughly 12 million people in, and surrounding the city of
\nameref{sec:Hraglund}. Lessons were learned from the tragedy, as major
centres of civilisation now mandate regular health checks for miners,
adventurers, and other professions of risk at becoming infected with the
blight. Most city kingdoms offer the cure of stage one blight for free,
should it be diagnosed by a doctor, or a priest.
