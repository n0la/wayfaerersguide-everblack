\subsection{Toralian Highlands}
\label{sec:Toralian Highlands}

The \emph{Toralian Highlands} (``Toral'' or the ``Highlands'') are a vast
stretch of land in the centre of \hyperref[sec:Goltir]{North Goltir}. It
stretches from the borders of \nameref{sec:Forsby} in the west, all the way east
towards the \nameref{sec:Cnamh Mountains}. North east of Toral lies the
mountain range of Alpara, and to the south west the region borders the Walusian
desert. Most of the region is boreal or temperate, with many places covered
in ancient forests, harsh swamps and marches, but also vast stretches of
fertile grass land and plains. The Highlands also contain two of the biggest
lakes in Aror: at the centre lies \emph{Ledava}, and to the north west lies
lake \emph{Ferto}.

Much like the \nameref{sec:Dirgewood}, the difficult terrain, and the ever
constant threat of the beast races (such as orcs, goblins and hobgoblins)
makes settling Toral a dangerous endeavours. Very few humanoid baronies exist
within the region, and most of the Highlands is settled by smaller tribes,
villages, or perhaps the odd city. Life in the Highlands is harsh and
unforgiving, as the monstrous and humanoid races still fight over land,
resources, and culture to this very day. Many wild and fantastical creatures
call the region their home, and fey and druids still roam the secluded places
of the Highlands.

Much like the Dirgewood, many of the Highlanders are staunch followers of
the \nameref{sec:Old Ways}, and form small settlements lead by warlords,
shamans, and village elders. Within Toral, the people still believe that
\nameref{sec:Morana} is the fourth mother, and her worship as a goddess of
winter, death and rebirth never waned. Many shamans of Toral practice
necromancy, and vampires are a common sight within the tribes and villages.
They welcome any humanoid, half-humanoid or vampire that follows the Old Ways,
or at least one of the mothers. The lesser deities have little power or
following within the region.

Culture and tradition is particularly kept well in line between the Dirgewood
and the Highlands, although scholars did not know how for quite a while. Until
it was discovered that the ancient \hyperref[sec:Tynrikke]{Týn} connected most
of their ruins through teleportation magic, allowing seamless travel between
the Dirgewood and the Highlands. This was a secret kept by shamans of both
sides, and was used to facilitate that, but also to allow for an escape route
in case villages or tribes were threatened.

\aren{It's not ``until it was discovered'', it is ``until I, Graham Balance,
  discovered''...}
\graham{Now, now. Let's avoid bragging, and keep the book professional.}

The politics of the Highlands is ever shifting, with many villages, towns
and cities either aligned or at odds with each other. The constant battles
and skirmishes between the monstrous and humanoid races contribute to an
ever shifting landscape, into which even the most powerful kingdoms of the
regions, namely \nameref{sec:Forsby}, \nameref{sec:Kesmar} and
\nameref{sec:Stenheim}, rarely interfere.

Common male names are: Aleks, Anton, Bogdan, Boris, Dejan, Dusan, Emil,
Gal, Janosh (Janos, Jan), Jaromir (Jaro, Mir), Milan, Miklos, Nikola
(Nikolaus, Nikolai), Radek, Soma, Stanislav (Standa), Viktor, Zdisa

Common female names are: Anna, Abigel, Bianka, Brana, Elisabeth (Elsa, Lisa),
Elena, Emma, Emelie (Amalie), Helene, Johanna, Lucia, Magda, Malina (Melina,
Lina), Melinda (Linda), Mira, Nada (Nadia, Nadezhda), Radmila (Mila), Rosalie
(Rosa), Sobena, Svetlana (Lana), Vera, Vesna, Zvesdana (Zvesda)
