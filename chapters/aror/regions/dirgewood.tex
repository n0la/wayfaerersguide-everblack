\subsection{Dirgewood}
\label{sec:Dirgewood}

The \emph{Dirgewood} is a vast forest south east of the \nameref{sec:Great
  Divide}, a large mountain chain that splits the continent of
\nameref{sec:Eilean Mor} in half north by south. It runs along most of the
southern side of the Great Divide, reaching up the mountains until the tree
border, but does not extend to the shores. It is a vast temperate and boreal
forest and marshland, that is split by a few major rivers that have their
origins in the Great Divide. The thick woods, harsh and unwelcoming marches,
and hilly terrain has made it almost impossible to build large castles in the
Dirgewood. However it is dotted with thousands of smaller towns and villages.

The Dirgewood is mostly settled by humans, wood elves, snow elves, and
halflings. The hilly regions of the Dirgewood are also home to deepkin, and
dark elves. Many monstrous tribes remain, especially hobgoblins, bugbears,
kobolds and goblins as well as many tribes and packs of lycanthropes. The
Dirgewood is also known for still housing many faeries and fey. The Dirgewood
is one of the few areas in which the \nameref{sec:Strife} is still
actively fought between humanoid and beast races.

Albeit the humanoid villages are as diverse as any city they have two things
in common: Most villages know of each other, and even know people from other
villages. They trade with each other often, intermarry, and also come to each
other's aid in case of an attack. Most of the time they fight monstrous races
and beasts, but have also killed bandits that settled in their land or halted
the expansionist dreams of baronies that board the Dirgewood. Skirmishes,
feuds and even wars between the villages are known to happen, but are far and
few in between.

The other thin in common is a staunch belief and adherence to the tradition of
the \nameref{sec:Old Ways}. Almost all villages are lead by a village elder, a
shaman, and a war chief. Most villages and small hamlets celebrate the rituals,
incantations, spells and sacrifice demanded by them by the three (or four)
mothers, and view outsiders that do not believe in the old ways as
suspect. They harbour an open animosity against anyone that would follow a
lesser deity. This animosity is rarely violent, except to those that would
come to the Dirgewood as missionaries.

Although the old traditions are part of their religion and culture, there
exist a great variety in how they are followed among the various tribes. Some
are more strict than others, and while others only revere three mothers, other
revere four and in turn those are split on who the fourth mother truly is. But
most tribes have forgone with the oldest, most brutal traditions. For example
very few tribes still practice humanoid sacrifices to \emph{Marwaid}, or do so
extremely rarely. The Old Ways, and their interpretation within the Dirgewood
and Golian Heights is the birth place for modern equality among men and
women. During the aeons of conflict with the beast races everyone is
encouraged to follow his or her inner calling (or ``true will of the mothers''
as it is often referred to), and thus contribute to the family, village or
town to the best of their ability.

The many rivers that flow from the mountains to the south-eastern shore lines of
\emph{Eilean Mor} are used to transport people and goods in and out of the
Dirgewood. The villagers use the river network to trade with each other, raid
monstrous encampments and tribes, as well as communicating and trading with
the large city kingdoms that reside on the shores of Eilean Mor.

For many city dwellers that follow the three goddess' of the old ways, the
Dirgewood is a popular destination for a pilgrimage and spiritual
enlightenment. The people of the Dirgewood are known to be highly spiritual,
and to them the stories, spirit worship, chants and spells of Old Ways are
part of their daily lives. They welcome any humanoid species that also follows
the Old Ways, or at least one of the three mothers, and adheres to their rules
and customs.

The people of the Dirgewood are known as hardy survivors, expert hunters,
fierce raiders, and proud warriors, and above all else spiritually enlightened
people that take great care in preserving their aeon old traditions and
believes. They dress in plain clothes, but take great pride in elaborate
jewellery made out of bone, antlers, animal teeth, precious metals or
gemstones. To combat, a hunt, or ritual ceremonies they also wear detailed
white, grey or blue face and body paint. The body paint acts as camouflage,
but also to intimidate their enemies.

Common male names are: Aeron, Aled, Andras, Arwyn, Bryn, Cadell, Dylan,
Eirian, Gareth, Glyn, Eifan, Ivor, Maldwyn, Owain, Rhys, Tegid, Wyn

Common female names are: Aerona, Anwen, Bethan, Blodwen, Branwen, Cadi,
Catrin, Deryn, Eira, Elin, Gwendoline, Gwenith, Llinos, Mairwen, Nia,
Siana, Tegwen, Wynne
