\section{Slavery}
\label{sec:Slavery}

Slavery is a sad reality among many city kingdoms, nations, baronies and
tribes of Aror. Many nations and kingdoms allow their citizens to own
sentient monstrous races, or even other humanoids. Slaves often do
hard and menial labour, or even more dangerous tasks such as fighting
as front line troops, work in extreme conditions, or mine the poisonous
\hyperref[sec:Everblack]{everblack}.

To many of these city kingdoms slavery is an integral part to the economy,
and the economic pressure to keep maintaining such an inhumane system
is relatively high. For example \nameref{sec:El-Fayam} would not be able
to flourish, without the slaves toiling away beneath the city to mine
the dangerously toxic everblack.

In most of these systems the slave is nothing more than property. A slave
can be sold, purchased, stolen and property damage is to be paid should
someone hurt the slave of another. The detailed laws may vary based on
the specific kingdom and nation, but often recognise the right of citizens
of another nations to their slaves.

Almost all slaves are registered similar to citizens and there are often
official organisations involved in trading, capturing, and tracking down
slaves that have fled. One such organisation is the \nameref{sec:Hunters
  Guild} of \nameref{sec:Norbury}, or the \nameref{sec:Velvet Hand} of
\nameref{sec:Fes al-Bashir}. These organisations often hire slave hunters,
slavers, task masters and other personnel required to maintain the slave
system within a city. They also offer administrative services, such as looking
up slave registrations, transfers of ownerships and often operate slave
markets and auctions where new slaves are sold.

All slaves are marked as such but the methods vary greatly. Some slaves
are just branded as slaves, often behind the ear or at the back of neck.
Other kingdoms use more sophisticated measures, such as the
\nameref{sec:Slave Band} or \nameref{sec:Slave Mark}. These magical
chains are then often keyed to one or more \nameref{sec:Master Ring} which
are worn by the owners of the slave.

\subsection{Indentured Servitude}
\label{sec:Indentured Servitude}

The system of indentured servitude is practised in many city states, even
those that ban slavery. It is a punishment often placed upon criminals found
guilty of property crimes such as property damage, theft or defaulting on
debts. The criminal is convicted, and then forced to work off his damage in
service to the person or institution to whom the damage was done. These
punishments are often limited in time, but may result in actual prison
sentences if the convict attempts to flee from his duties.

\subsection{Regulated Slavery}
\label{sec:Regulated Slavery}

It also varies greatly between the nations and city kingdoms how slavery is
encoded by law. Most city kingdoms do have laws regarding the treatment of
slaves, such as requiring them to be fed and clothed, or inflicting punishment
to owners should they wilfully neglect or kill their own slaves. Some systems
even encode conditions how children of slaves are to be treated, and whether
they are free or not. And often also encode conditions on how a slave can free
themselves from their bondage. This form of slavery is often called
\emph{regulated slavery}. This system is practised in \nameref{sec:Helmarnock}
and \nameref{sec:Kesmar}.

\subsection{Unregulated Slavery}
\label{sec:Unregulated Slavery}

A rather rare form of slavery is \emph{unregulated slavery}. In this legal
system slaves have no rights whatsoever, and are the complete mercy of their
owner. They are still counted as property within the legal system, and thus
can be sold and bought. But the owner has no obligations towards the slave.
This system is practised in \nameref{sec:Norbury} as well as
\nameref{sec:Morkan}, and by the gnolls of \nameref{sec:Esmayar}.

\subsection{Vonir Accord}
\label{sec:Vonir Accord}

The \emph{Vonir Accord}, named after king Sigmund Vonirson of Norbury, is a
treaty signed by almost all major city kingdoms and most bigger baronies of
the north. In exchange for immunity to enslavement of the signing parties'
citizens, all Norbury slaves found on foreign soil must be returned to
Norbury. Further all signing parties must acknowledge the owner's claim over
his slave and not interfere with his rights to his property. This treaty
also allows members of the \nameref{sec:Hunters Guild} to search for, and
apprehend escaped slaves on the land of signing parties.
