\subsection{Pale}
\label{sec:Pale}

Toward the far north, past the \nameref{sec:Cnamh Mountains} lies the northern
pole, and down south, far beyond the deserts and steppes of \nameref{sec:Arania}
lies the southern pole. Both are vast desolate lands covered in snow and ice,
and are commonly just referred to as the \emph{great pale}.

Both poles are harsh climates, reaching temperatures below 50 degrees Celsius
during the night. Inhospitable to most life forms, only a few specially adapted
creatures such as ice bears, penguins, sabre tooth tigers and woolly mammoths.

\subsubsection{Tribal Snow Elves}
\label{sec:Tribal Snow Elves}

The pale is also the home to the tribal \hyperref[sec:Snow Elves]{snow elves},
often called \emph{elves of the pale}, or \emph{pale elves}. These names are
also used to differentiate the tribal snow elves from those that live in
cities and nations.

They live in small families, or together in small clans and best the bitter
cold better than any other sentient race. Their physique makes them naturally
resistant to the extreme cold, and they have produced some of the finest
archers and hunters that can hunt, track and kill even the most dangerous game
found in the pale. Pale elves have developed special methods to construct
their \hyperref[sec:Snow Elf Bow]{special hunting bows}, that are now sought
after all across the world.

Pale elves prefer to remain on the move, and roam the pale wastes as nomads.
They value their own community and family above all else, but are a strictly
patriarchal society, where only men may lead the tribe. Women may not assume
any dangerous activities within society, such as being a huntress or warrior,
and thus often become priestesses, gatherers or remain with the children. It
is also customary to gift younger females off to other tribes, often without
their saying or choice in the matter, in an attempt to bring the roaming
nomadic tribes closer together.

While the more civilised nations and states, as well as the snow elves living
within them, would laud and look down upon these tribal structures and
traditions, they are just that: Old and tried traditions that have ensured the
survival of the snow elves for centuries. Some tribes have broken with these
traditions, allowing women to pursue a more active role in society but these
are far and few between. Such change if often necessitated by a shortage of
able bodied men, good hunters, or a lack of a male heir to the tribal
chieftain, rather than actual pursuit for equality within the tribe.

Tribal pale elf tradition is communicated through stories of great ancestors,
and their noble and selfless deeds. And it is exactly those ancestors that the
snow elves revere and pray too. However those stories also contain
supernatural beings, especially an aspect of \nameref{sec:Forun} as the all
mother of life and patron mother of snow elves, as well as a trickster aspect
of \nameref{sec:Isamir} that serves as her foil. Isamir is not evil in these
stories, however he assumes the role of a trickster deity.

To tribal snow elves the community, family and their own tribe is everything, as
it ensures survival and in the frozen tundras. The harshest punishment within
these communities, reserved for major crimes, is exile which is often equal to
a death penalty. Since these exiles are then also shunned by other snow elven
tribes, they sometimes wander away from their homes and join city kingdoms or
baronies.

Furthermore many slavers attempt to move into the frozen waters of the pale
in the hopes of capturing and enslaving tribal snow elves. These expeditions
are extremely dangerous and many slavers have perished to the cold, dangerous
beasts or the expert archery of the snow elves.
