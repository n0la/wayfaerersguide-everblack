\chapter{Magic Items}
\label{sec:Magic Items}

Magic is flows through the world of Aror. Magical devices and items
are made possible by ingenious engineers and wizards that embed
\nameref{sec:Everblack} into every day items to allow them to harness
the power of magic. Magical items, trinkets and devices are sold and
purchased by the wealthy, and are a major source of economical income
for many large nations and city kingdoms.

These devices would not be possible without everblack, and all native
magical devices and items made on Aror contain at least trace amounts
of the black crystal. It depends of course on the item in question how
much everblack is required to create or power a magical artefact, but
the black crystal is always required in the creation process.

\begin{35e}{Everblack in Magical Items}
  Everblack is needed in the creation of all magical items and artefacts.
  Take the cost in gold required during the item's creation and convert
  it into platinum. Then divide that number by two and you know how much
  \hyperref[sec:Shard]{shards} worth of everblack crystals are required to
  create the item. The other half are material costs as normal, and may be
  paid with gold.
\end{35e}

\section{Arcane Marks}
\label{sec:Arcane Marks}

\subsection{Citizen Mark}
\label{sec:Citizen Mark}

A more expensive way to mark citizens is the arcane \emph{citizen mark}. It is
based upon \hyperref[sec:Deepkin]{deepkin} blood magic, and infuses itself
with the citizens body. At will, the citizen may then produce or hide an
arcane tattoo anywhere on their own body, which contains all the necessary
information to identify him as a citizen of a city kingdom or nation. These
usually cost up to 500 shins to inscribe, but cannot be stolen, misplaced or
so easily forged.

\begin{35e}{Citizen Mark}
  \srditem{Citizen Mark}{These magical runes are inscribed onto citizens of
    a specific city kingdom or nation, and are used to identify them as
    rightful citizens. They show themselves as tattoos upon the wearers skin,
    and often show the kingdom's or nations banner along with a few words,
    not more than seven, to help identify the person. Common inscriptions are
    name, address, race or the persons status within the nation or kingdom.
    These tattoos can be shown and hidden at will by the wearer.}
  \srditem{Crafting}{Caster Level: 1rd, Prerequisites:
    \emph{Craft Wondrous Item}, \emph{Arcane mark}, Price: \emph{500 shins}}
\end{35e}

\subsection{Nobility Mark}
\label{sec:Nobility Mark}

Nobility marks work just the same as citizen marks, except that they are
customised toward the house of nobility that issues them. Instead of
identifying that person as a citizen of a kingdom, these marks identify the
person to be of noble blood line. These marks are scribed usually at birth of
a new member of a noble family, and act as a proof of nobility across the
known world of Aror.

\begin{35e}{Nobility Mark}
  \srditem{Nobility Marks}{These marks work just the same as a
    \nameref{sec:Citizen Mark}, would, except are hardened against magical
    tampering and forgery, and cost more to inscribe.}
  \srditem{Crafting}{Caster Level: 7rd, Prerequisites:
    \emph{Craft Wondrous Item}, \emph{Arcane mark}, Price: \emph{100 shard}}
\end{35e}

\subsection{Slave Mark}
\label{sec:Slave Mark}

Slave marks function similar to \nameref{sec:Citizen Mark}. They are arcane
marks that are permanently crafted onto the slaves body as a tattoo. Often
behind the ears, or onto the slaves neck. But unless with the citizen marks,
they cannot be made invisible by the wearer. These marks often encode the
slave's name registration number, owner, and the city or nation of the
owner. Slave marks are bonded to \nameref{sec:Master Ring} just like slave
collars are, and can thus be used as a target for certain magical spells and
powers that will then target the slave. The mark contains healing magic, and
will always heal any attempt to carve out, burn or otherwise destroy the skin
the tattoo is placed upon.

\begin{35e}{Slave Mark}
  \srditem{Slave Mark}{These magical arcane marks are made to permanently
    identify a person as a slave. They attach permanently to a person's body,
    and can encode simple messages, not more than seven words or numbers. A
    slave mark is keyed to one or more \hyperref[sec:Master Ring]{Master
      Rings}. Once inscribed the slave mark cannot be removed except by the
    owner of the Master Ring. A bearer of slave mark is subject to the damage,
    and any divination spell (with a -4 penalty to saves) from the wearer of
    the keyed master ring and can exchange messages with the wearer of the
    keyed master ring three times per day as if using the \emph{Sending}
    spell. The wearer of the slave mark cannot directly touch or use the keyed
    master ring, and suffers 3d6 points of damage in such an attempt.}
  \srditem{Crafting}{Caster Level: 11th, Prerequisites:
    \emph{Craft Wondrous Item}, \emph{Crushing Despair},
    Price: \emph{1000 shards}.}
\end{35e}

\section{Jewellery}
\label{sec:Jewellery}

\subsection{Master Ring}
\label{sec:Master Ring}

This iron ring, often emblazoned with a set of keys or an iron chain, is used
to exert control over slaves. The ring can be keyed \nameref{sec:Slave Band}
or \nameref{sec:Slave Mark}, and then allow the owner of the master ring to
control, spy or even damage the wearers of the slave marks or bands. Someone
who wears a keyed slave band or mark cannot touch or use the master ring
directly without suffering horrible, flesh eating damage.

\begin{35e}{Master Ring}
  \srditem{Master Ring}{An iron ring, emblazoned with either a set of keys or
    chains can deal 3d6 points of either normal or non-lethal damage per round
    as a free action to anyone wearing a slave mark or band keyed to
    it. Furthermore the owner of the master ring can, at will and as a free
    action, exchange messages with anyone wearing a keyed slave band or mark
    as if using the \emph{Sending} spell. The master ring can also be used as
    a focus for divination spells, which then target a wearer of a keyed slave
    band or mark as decided by the caster. The target of this divination spell
    has a -4 penalty on saving throws against any divination spell cast through
    the master ring.}
  \srditem{Crafting}{Caster Level: 11th, Prerequisites: \emph{Forge Ring},
    \emph{Dominate Person}, Price: \emph{500 shards}}
\end{35e}

\subsection{Slave Band}
\label{sec:Slave Band}

A slave band is either an iron collar, ring, or shackle for the wrists or
ankles. Once put on it cannot be easily removed except by the wearer of the
keyed \nameref{sec:Master Ring}. However wears the slave band is subject to
the damage and any divination spells the wearer of the keyed master ring.
Unlike slave marks, slave bands can be removed but require a combination of
mechanical and arcane skill to be safely unlocked.

In the last centuries various counter measures to the slave collars have been
developed, allowing slaves to bet set free. In turn the craftsmen have changed
to permanent slave tattoos instead, which cannot be so easily removed. Slowly
the arcane tattoos (called \hyperref[sec:Slave Mark]{slave runes}) are
replacing the arcane collars.

\begin{35e}{Slave Band}
  \srditem{Slave Band}{An iron ring or collar, once put on cannot be easily
    removed except by the wearer of the keyed master ring. The wearer of the
    slave band is subject to the damage from the keyed master ring. The wearer
    of a slave band can also exchange messages with the wearer of the keyed
    master ring three times per day as if using the \emph{Sending} spell. The
    wearer of the slave band cannot directly touch or use the keyed master
    ring, and suffers 3d6 points of damage during such an attempt.}
  \srditem{Crafting}{Caster Level: 5th, Prerequisites:
    \emph{Craft Wondrous Item}, \emph{Crushing Despair},
    Price: \emph{100 shards}.}
\end{35e}

\section{Wondrous Items}
\label{sec:Wondrous Items}

\subsection{Null Stone}
\label{sec:Null Stone}

A \emph{null stone} is a specially prepared and treated everblack crystal that
slowly siphons power to it from its surroundings. Such stones are expensive to
make but are invaluable as they can negate magical artefacts and wielders of
magic that they come in contact with. These null stones are often embedded in
shackles and binds to keep witches, warlocks and other spell wielders from
using their powers. Although such null stones can be made to explode by
overcharging them as any everblack crystal, this outcome is rarely in the best
interest of the person who was made to wear null stone.

Null stones do not permanently destroy the magical ability of the person or
the artefact but merely suppress them. Once the null stone has been removed
from the vicinity of a magical object or magically gifted person, the powers
regenerate within twenty-four hours. Also null stones do not work immediately,
but require to be near the object or caster for an hour until they have
drained all the residual magic power.

These null stones are often embedded into a \nameref{sec:Slave Band} to
suppress a slaves' or prisoners' magical abilities. Sometimes they are also
crafted directly beneath a slaves' or prisoner's skin to permanently dampen
magical abilities.

\begin{35e}{Null Stone, Lesser}
  \srditem{Null Stone, Lesser}{This lesser null stone is capable of suppressing
    any magical effect of spell level 3 or lower. It needs to be placed at the
    source of the magical effect, or near the origin of the spell (like a
    caster or magical item) and takes effect after one hour. It's suppressive
    effects last as long as the null stone remains intact and is near the
    source of the magic. Once the null stone has been removed any magical
    effects or sources (such as casters or magical items) regain their power
    after twenty four hours.}
  \srditem{Crafting}{Prerequisites: \emph{Craft Wondrous Item}, 5000 shards
    worth of everblack crystals.}
\end{35e}

\begin{35e}{Null Stone}
  \srditem{Null Stone}{This null stone is capable of suppressing any magical
    effect of spell level 6 or lower. It needs to be placed at the source of
    the magical effect, or near the origin of the spell (like a caster or
    magical item) and takes effect after one hour.  It's suppressive effects
    last as long as the null stone remains intact and is near the source of
    the magic. Once the null stone has been removed any magical effects or
    sources (such as casters or magical items) regain their power after twenty
    four hours.}
  \srditem{Crafting}{Prerequisites: \emph{Craft Wondrous Item}, 10000 shards
    worth of everblack crystals.}
\end{35e}

\begin{35e}{Null Stone, Greater}
  \srditem{Null Stone, Greater}{This greater null stone is capable of
    suppressing any magical effect of spell level 9 or lower. It needs to be
    placed at the source of the magical effect, or near the origin of the
    spell (like a caster or magical item) and takes effect after one
    hour. It's suppressive effects last as long as the null stone remains
    intact and is near the source of the magic. Once the null stone has been
    removed any magical effects or sources (such as casters or magical items)
    regain their power after twenty four hours.}
  \srditem{Crafting}{Prerequisites: \emph{Craft Wondrous Item}, 20000 shards
    worth of everblack crystals.}
\end{35e}
