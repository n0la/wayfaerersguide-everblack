\twocolumn
\section*{Races}

The people of Aror make up a multi flavoured stew of various races and
backgrounds. Apart from may races that you might already know from your
homeworld, such as humans, elves, dwarves or gnomes; Aror is also home to some
you might have never heard of before. Two of these are the
\emph{Deepkin} and the \emph{Umgeher}.

You will find that race matters little. An elf of \emph{Norbury} often shares
the warrior culture, and faith in the meritocratic societal ideas
of \emph{Norbury}. While a dwarf of \emph{Nen-Hilith} will be accustomed and
to the artistic culture and creative ways of the artisans and performers of
that city.

\subsection*{Humans}

Humans are one of the most ancient races of \emph{Aror}, and also the dominant
race of our planet. Human artifacts have been found dating back thousands of
years, far beyond the history of any species.

\subsubsection*{Human Lands}

Humans can be found everywhere on Aror. But history indicates that they
originated on the southern continent of \emph{Arania}, and migrated to all
other continents during the last ice age. Wherever humans settle they build
villages, cities, and large kingdoms and often become the dominant culture and
social structure.

\section*{Elves}

Elves are tall - often two and a half metres - tall slender race, with long
ears. Like humans elves come in with variety skin and hair colours.  As with
humans these differences are superficial only. They have their own kingdom
they have built together with the \emph{Halflings} called Nen-Hilith. Elves
are often curious, free thinking and often rather jovial even in spite
adversity.

\section*{Dwarves}

Dwarves are natural fighters, miners and smiths. Among all of the races they
are the most reclusive of all. They hardly join other cultures, and prefer to
continue to live like their ancestors did. Dwarves organise themselves into
huge clans or families and live in deep caverns or mountains in a strict caste
society.


%% TODO
