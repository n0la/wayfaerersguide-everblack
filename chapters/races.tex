\twocolumn
\section*{Races}

The people of Aror make up a multi flavoured stew of various races and
backgrounds. Apart from may races that you might already know from your
homeworld, such as humans, elves, dwarves or gnomes; Aror is also home to some
you might have never heard of before. Two of these are the
\emph{Deepkin} and the \emph{Umgeher}.

You will find that race matters little. An elf of \emph{Norbury} often shares
the warrior culture, and faith in the meritocratic societal ideas
of \emph{Norbury}. While a dwarf of \emph{Nen-Hilith} will be accustomed and
to the artistic culture and creative ways of the artisans and performers of
that city.

\subsection*{Humans}

Humans are one of the most ancient races of \emph{Aror}, and also the dominant
race of our planet. Human artifacts have been found dating back thousands of
years, far beyond the history of any species.

\subsubsection*{Human Lands}

Humans can be found everywhere on Aror. But history indicates that they
originated on the southern continent of \emph{Arania}, and migrated to all
other continents during the last ice age. Wherever humans settle they build
villages, cities, and large kingdoms and often become the dominant culture and
social structure. Human kingdoms have often endured for thousands of years,
and have bested many difficulties that had driven other societies to ruin.
The ingenuity of the humans, their stubborn attitude and their uncanny ability
to adapt to any difficulty makes them the dominant race of \emph{Aror}.

\section*{Elves}

Elves are tall - often two and a half metres - tall slender race, with long
ears. Like humans elves come in with variety skin and hair colours.  As with
humans these differences are superficial only. They have their own kingdom
they have built together with the \emph{Halflings} called Nen-Hilith. Elves
are curious, artistic, free thinking and often jovial even in spite
adversity.

There are five major elven societies.

\subsection*{Dark Elves}

The \emph{dark elves} live underground, have black to blue skin and white
hair. They are often live primitive lives, adapted to the harsh realities of
the depths below. They value community and family above all else. Many dark
elves live also on the surface, and much like their surface cousins they blend
in with human societies quite easily.

\begin{35e}
  \textbf{Dark Elf Traints (Ex)}: The following traits are in \emph{addition}
  to the high elf traits, except when noted.
  \begin{itemize}[noitemsep]
    \item +2 Intelligence, +2 Charisma
    \item Darkvision out to 120 feet.
    \item Automatic languages: Common, Elven, Undercommon.
    \item Favoured Class: Any. When determining whether a multiclass takes an
    experience point penalty, his or her highest-level class does not count.
    \item Level adjustment +1
  \end{itemize}
\end{35e}

\subsection*{Snow Elves}

Far to the north live the \emph{snow elves}, nomadic elves with white hair and
blue or white hair that have adapted to living in the harsh cold of the north.
They live in very small tribes or families, focusing mostly on hunting and are
renown for the arcane weapon smiths.

\begin{35e}
  \textbf{Snow Elf Traints (Ex)}: The following traits are in \emph{addition}
  to the high elf traits, except when noted.
  \begin{itemize}[noitemsep]
    \item +2 Wisdom, +2 Dexterity
    \item \textbf{Pale Wastes (Su)}: A pale elf can live comfortably in
    conditions of extreme cold, even with barely any clothing or external
    sources of warmth. This ability functions like a continuous \emph{Endure
    Elements} but for cold conditions only.
    \item Automatic languages: Common, Elven
    \item Favoured Class: Any. When determining whether a multiclass takes an
    experience point penalty, his or her highest-level class does not count.
    \item Level adjustment +1
  \end{itemize}
\end{35e}

\subsection*{Savage Elves}

In the jungles \emph{Yua'cata} live the \emph{savage elves}, a loose
collection of tribes of cannibalistic and demon worshipping elves. They rarely
wander beyond the confines of their jungle and are one of the rarest elves to
meet in the civilised areas of \emph{Aror}.

\begin{35e}
  \textbf{Savage Elf Traints (Ex)}: The following traits are in \emph{addition}
  to the high elf traits, except when noted.
  \begin{itemize}[noitemsep]
    \item +2 Strength, +2 Constitution
    \item Automatic languages: Elven
    \item Favoured Class: Any. When determining whether a multiclass takes an
    experience point penalty, his or her highest-level class does not count.
    \item Level adjustment +1
  \end{itemize}
\end{35e}

\subsection*{High Elves}

And then finally the elves of \emph{Nen-Hilith} (now \emph{Avenfjord}). They
have built a blooming society with the halflings and value artistic expression,
exploration and their study of the arcane arts. Their city is a sight to behold,
and truly shines as a beacon of civilisation, artistry and nobility among the
large city kingdoms.

\section*{Dwarves}

Dwarves are natural fighters, miners and smiths. Among all of the races they
are the most reclusive of all. They hardly join other cultures, and prefer to
continue to live like their ancestors did. Dwarves organise themselves into
huge clans or families and live in deep caverns or mountains in a strict caste
society. Those that do not fit into these strict frameworks of societies are
cast out, and then look for other realms to live. Ever since the deep have
become ever more dangerous many smaller dwarven clans have settled on the
surface near human or elven settlements, while others have seemingly integrated
into the large city kingdoms. Offering their expertise on trade and smithing
to the other races.

\section*{Deepkin}

\aren{I had the misfortune of seeing my people's decline with my own eyes.
Powerless, and unable to stop it. We are but a shadow of our former selves.}

\graham{But still you live. And once the time is right, you may venture forth
and reclaim what was once yours.}

The \emph{Deepkin} are the cavern dwelling cousins of humans. They are capable
of seeing in the dark, have white pale skin, and often red to brownish hair.
\emph{Deepkin} society is one of the oldest societies on \emph{Aror}, with a
long standing history in the arcane arts, building magnificent underground
cities, libraries and workshops. Ancient \emph{Deepkin} were master arcane
smiths, golem constructors, inventor of many magic based constructs and
technology still used today.

However the ancient deepkin had their dominance challenged by the other races
of the deep, namely the dwarves, dark elves, and a now extinct species called
the \emph{Ilians}. Thousand of years of conflicts lead to the decline of their
magnificent ancient civilisation. Instead of perishing however, they fled to
the surface, where they were warmly received by their surface cousins. Nowadays
\emph{Deepkin} culture is but a shadow of what it was, although they still try
to reclaim what was once theirs with expeditions into the deep.

\begin{35e}
  \begin{itemize}[noitemsep]
    \item Medium: as medium creatures, \emph{Deepkin} have no special bonuses or
    penalties due to their size.
    \item \emph{Deepkin} base land speed is 30 ft.
    \item For all manners regarding racial restrictions or classifications
    \emph{Deepkin} count as humans.
    \item Darkvision out to 120 feet.
    \item Bonus Feat: Just as their human cousins, \emph{Deepkin} can choose a
    bonus feat at first level.
    \item Automatic languages: Common, Undercommon.
    \item Favoured Class: Any. When determining whether a multiclass takes an
    experience point penalty, his or her highest-level class does not count.
  \end{itemize}
\end{35e}

\section*{Halflings}

Most of the halflings are nomads, who travel across the lands in search for
places they could explore. If you think you have found a secluded spot on
\emph{Aror}, where no one else had sat foot before, you can be sure it is
already named after an halfling explorer. They travel in small families, and
in exchange for money offer their services to any small town they come across
on their travels. Halflings are curious, adventurous often find themselves
exploring the depths and other inhospitable places of \emph{Aror}.

A huge group of halfling families once decided to start another adventure: of
their own kingdom. They settled down near an elven kingdom and named their new
kingdom \emph{Brèagha Hilith}. After decades of peaceful coexistence between
the two races, they finally decided to tear down the last barriers and simply
merge the two kingdoms into a new one: \emph{Nen-Hilith}. It became a shining
beacon of civilisation, artistry and stability under the known city kingdoms
of \emph{Aror}.

\aren{Until a few giants stepped on them...}

\section*{Umgeher}

\aren{I was once the midwife of an \emph{Umgeher}...}

\graham{The less shared about this experience the better.}

\emph{Umgeher} are half undead humanoids that retain their own individuality,
will and determination across the process that turned them into undead. They
were created by the ancient vampires that reign in the city kingdom of
\emph{Kilmarnock} centuries ago, and were given the freedom to reproduce of
their own volition. Soon they travelled and spread across the entire known
world. Although they are immortal, they are not invincible. More so, their
flesh and skin is in a constant state of decay, and must combat their never
ending dissolution with oils and ointment. They have no bodily hair, and often
wear wigs to blend into normal population.

\emph{Umgeher}, like any species, are free to determine their own fates and
shape their own thoughts, they do have to constantly combat the ignorance and
fear of the other races. Many religious institutions and city states have now
allowed \emph{Umgeher} to live there without a fear of being
persecuted. However this is a recent trend, and \emph{Umgeher} prefer to build
their own little communities, towns and even cities.

\begin{35e}
  \begin{itemize}[noitemsep]
    \item Medium: as medium creatures, \emph{Umgeher} have no special bonuses or
    penalties due to their size.
    \item \emph{Umgeher} base land speed is 30 ft.
    \item As undead creatures \emph{Umgeher} gain all undead traits.
    \item \emph{Umgeher} have a live long experience in disguising themselves as
    humans, and thus gain a \emph{+2 racial bonus} on \emph{Disguise} and
    \emph{Bluff}.
    \item Favoured Class: Any. When determining whether a multiclass takes an
    experience point penalty, his or her highest-level class does not count.
    \item Automatic languages: common. Bonus languages: draconic.
  \end{itemize}
\end{35e}

\section*{Diarim}

\graham{The expression \emph{explaining freedom to a Diarim} has become
ingrained in many cultures as a metaphor for a fruitless labour.}

The \emph{Diarim} are a race of humanoid creatures that were bread for
specific tasks by the dark sorcerers and witches of the fire giants living on
the continent of \emph{Farlar}. They are the youngest of the races
on \emph{Aror}, and are a mixture of various other humanoid races.

\emph{Diarim} come in in as many shapes and sizes as the giants had uses and
tasks for their engineered slave race. But most share a common set of features:
the light, fair skin of the \emph{Deepkin}, blue hair of the \emph{Snow elves},
and an ingrained sense of duty and loyalty to their overlords - the giants. Very
few \emph{Diarim} ever escape the slavery of their masters, and those few that
do find it hard to shake their submissiveness and eagerness to please and help
that the giants have ingrained into them. Almost all have blue tribal tattoos
all over their body, which identify the current and past owners of the specific
\emph{Diarim}.

The most common variant are labourers, small but stout breed, that was created
by introducing dwarven heritage into the \emph{Diarim}. They excel at physical
labour, such as mining and construction. Siegers were bread with monstrous
races, often even giants, and are used as front line soldiers, gladiators and
shock troops. They are larger than any other \emph{Diarim}, and often also
serve as slave overseers over the others. Exciters were bred and selected for
their beauty, and are priced possessions to be traded and gifted to other
giants. Their primary role is to entertain their overlords through song, dance
and company.

Most \emph{Diarim} have no names. They never refer to themselves with names,
and are only being given a names by the giants if they have distinguished
themselves, either through heroic deeds, or through crimes.

In the recent decades more and more Diarim have escaped the continent
of \emph{Farlar} and joined the other humanoid races. The giants had to learn
that you cannot suppress the curiosity, love for wandering and freedom for long
when you create a species based on humans, halflings and elves.

\begin{35e}
  \textbf{Diarim Traits (EX)}:
  \begin{itemize}[noitemsep]
    \item Medium: as medium creatures, \emph{Umgeher} have no special bonuses or
    penalties due to their size.
    \item A \emph{Diarim}'s base land speed is 30 ft.
    \item \textbf{Weak Will (EX)}: All \emph{Diarim} have a -2 penalty to will
    saves against charms and similar effects.
    \item Automatic languages: Giant, Common
  \end{itemize}

  \textbf{Sieger Traits (EX)}: the following traits are in \emph{addition} to
  the \emph{Diarim} traits, except when noted otherwise.
  \begin{itemize}[noitemsep]
    \item Large size. -1 penalty to Armor Class, -1 penalty on attack rolls,
    -4 penalty on Hide checks, +4 bonus on grapple checks, lifting and
    carrying limits double those of Medium characters.
    \item Space/Reach: 10 feet/5 feet
    \item +8 Strength, -2 Intelligence, -2 Charisma, -2 Wisdom
    \item Favoured Class: Barbarian
    \item Level Adjustment: +1
  \end{itemize}

  \textbf{Exciter Traits (EX)}: the following traits are in \emph{addition} to
  the \emph{Diarim} traits, except when noted otherwise.
  \begin{itemize}[noitemsep]
    \item +4 Strength, -2 Strength, -2 Constitution
    \item Favoured Class: Bard, Sorcerer
  \end{itemize}
\end{35e}
