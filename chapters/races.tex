\twocolumn
\section*{Races}

The people of Aror make up a multi flavoured stew of various races and
backgrounds. Apart from may races that you might already know from your
homeworld, such as humans, elves, dwarves or gnomes; Aror is also home to some
you might have never heard of before. Two of these are the \emph{deepkin}, the
\emph{diarim} and the \emph{umgeher}.

You will find that race matters little. An elf of \emph{Norbury} often shares
the warrior culture, and faith in the meritocratic societal ideas
of \emph{Norbury}. While a dwarf of \emph{Nen-Hilith} will be accustomed and
to the artistic culture and creative ways of the artisans and performers of
that city.

\subsection*{Humans}

\emph{Humans} are one of the most ancient races of \emph{Aror}, and also the
dominant race of the planet. Human artifacts have been found dating back
hundreds of thousands of years, far beyond the history of any species. There
are two separate ``races'' of humans: those inhabiting the southern part of
the hemisphere who usually have darker skin, and the ``northerners'' who
usually have fair skin. This distinction is superficial only. Apart from the
tone in skin colour, there is no other biological difference between the
various human tribes and civilisations. Humans live up to 80 years, and are
known for being statesmen, diplomats, farmers, adventurers, explorers and
scientists.

\subsubsection*{Language}

Humans speak \emph{Teranim}, either as their primary language, or as their
secondary language together with their local language. \emph{Teranim} has
become the de-facto language of \emph{Everblack} and is usually spoken almost
everywhere, even among the beast races. \emph{Teranim} is written in its own
alphabet of the same name. \emph{Old Teranim} exists, spoken by the ancient
humans, and is the root of most other languages and various local dialects.
Although \emph{old teranim} is no longer actively spoken, various books,
poems, stories and songs still exist in that language.

Humans living in the southern hemisphere often speak a language that is
stuck halfway between old and new \emph{Teranim}, called \emph{Kalest}. While
the people of \emph{Forsby} (and surrounding regions) have their own distinct
dialect, which can be so hard to understand that it has received its own
classification and name: \emph{Reatham}.

Albeit many local dialects exist, almost all humans, and the other races
living with them are capable of speaking \emph{Teranim}. It is, after all,
the official language of many governments, the language that is printed and
used in official capacity as well as in inter-kingdom cultural exchange and
barter.

\begin{35e}
The language of \emph{Teranim} is equal to \emph{Common} of D\&D.
\end{35e}

\begin{note}
To explain how all of these languages are related, let me give you an example:
\emph{Teranim} (or Common) is German, while \emph{Old Teranim} is High German;
\emph{Reatham} is the Bavarian dialect and \emph{Kalest} is Alemannic German.
Although they are all based off German, they are distinct from another. But a
Bavarian, Austrian or a Swiss-German are still capable of speaking standard
High German (even though we often claim otherwise).
\end{note}

\subsubsection*{Human Lands}

Humans can be found everywhere on Aror. But history indicates that they
originated on the southern continent of \emph{Arania}, and migrated to all
other continents during the last ice age. Wherever humans settle they build
villages, cities, and large kingdoms and often become the dominant culture and
social structure. Human kingdoms have often endured for thousands of years,
and have bested many difficulties that had driven other societies to ruin.
The ingenuity of the humans, their stubborn attitude and their uncanny ability
to adapt to any difficulty makes them the dominant race of \emph{Aror}.

\subsubsection*{Human Culture}

Like most races, humans have no inherent global culture, tradition or customs.
Instead their believes and customs are ever evolving, and specific to the
realm they live on. But there is one trait that the average human has: ingenuity
in the face of adversity. No other species has managed to settle every corner
of the world, endure, and build lasting civilisations out of the hostile
environment they found themselves in.

\subsection*{Elves}

Elves are tall - often between 1.90 and 2.30 metres - tall slender
race, with long and pointy ears. Their physique and build is slender
as well, with long skinny legs and arms. Elves live up to 800 years of
age, and thus often pick up professions that take longer to master, such
as wizardry, artistry or the sciences. Although elves live long, they
often shift focus in your goals. Apart from the humans, the elves are one
of the oldest races of \emph{Everblack}.

Unlike humans, elves do have distinct sub races that differ from each other
in various physical and biological aspects. The \emph{snow elves} for example
have a natural resistance to cold like no other species have, while the
\emph{dark elves} have adapted to see better in the dark caverns they
inhabit.

There are five major elven races that are recognised across the world of
\emph{Everblack}:

\subsubsection*{High Elves}

The most numerous are the high elves of \emph{Avenfjord}.  High elves have
fair, skin with a hint of yellow and gold. Their hair ranges from blond, fiery
red, black to pale white. High elves are the most adaptable and curious of the
elves, and often live within human city kingdoms, adapting and integrating
well with other cultures and societies. Even though they have their own
kingdom, the vast majority of high elves live outside the kingdom
of \emph{Avenfjord}.

High Elves speak \emph{Enro'ad}, a variant of \emph{Old Teranim}, but use the
halfling alphabet to write it.

\begin{35e}
  \emph{Enro'ad} is elvish, albeit the elven alphabet is now the \emph{halfling}
  script.

  \textbf{High Elf Traits (Ex)}:
  \begin{itemize}[noitemsep]
    \item +2 Dexterity, -2 Constitution
    \item Medium: As Medium creatures, elves have no special bonuses or
    penalties due to their size.
    \item Low-Light Vision: An elf see twice as far as a human in starlight,
    moonlight, torchlight, and similar conditions of poor illumination. She
    retains the ability to distinguish color and detail under these
    conditions.
    \item +2 racial bonus on Listen, Search, and Spot checks. An elf who
    merely passes within 5 feet of a secret or concealed door is entitled to a
    Search check to notice it as if she were actively looking for it.
    \item Automatic languages: Elven, Common
    \item Favoured Class: Any. When determining whether a multiclass takes an
    experience point penalty, his or her highest-level class does not count.
  \end{itemize}
\end{35e}

\subsubsection*{Dark Elves}

The \emph{dark elves} live underground, have black to blue skin, and their
hair ranges from a faint hint of blue, silver to snow white. They are the
smallest of all elven races, and range from 1.70 to 1.90 metres in height.

They are often live primitive lives, adapted to the harsh realities of the
depths below. They value community and family above all else. Many dark elves
also live on the surface, and much like their surface cousins, they blend in
well with human societies very easily.

\begin{35e}
  \textbf{Dark Elf Traints (Ex)}: The following traits are in \emph{addition}
  to the high elf traits, except when noted.
  \begin{itemize}[noitemsep]
    \item +2 Intelligence, +2 Charisma
    \item Darkvision out to 120 feet.
    \item Automatic languages: Common, Elven, Undercommon.
    \item Favoured Class: Any. When determining whether a multiclass takes an
    experience point penalty, his or her highest-level class does not count.
    \item Level adjustment +1
  \end{itemize}
\end{35e}

\subsubsection*{Snow Elves}

\aren{Snow Elves are the pinnacle of beauty...}

Far to the north live the \emph{snow elves}, nomadic elves with white to silver
blue skin, white or blue hair, and bright blue or green eyes. They have adapted
well to the colder climates, and are expert hunters and weapon smiths. Many
\emph{snow elves} live in smaller families and tribes, although many have moved
south and integrated with the northern human city kingdoms.

\begin{35e}
  \textbf{Snow Elf Traints (Ex)}: The following traits are in \emph{addition}
  to the high elf traits, except when noted.
  \begin{itemize}[noitemsep]
    \item +2 Wisdom, +2 Dexterity
    \item \textbf{Pale Wastes (Su)}: A pale elf can live comfortably in
    conditions of extreme cold, even with barely any clothing or external
    sources of warmth. This ability functions like a continuous \emph{Endure
    Elements} but for cold conditions only.
    \item Automatic languages: Common, Elven
    \item Favoured Class: Any. When determining whether a multiclass takes an
    experience point penalty, his or her highest-level class does not count.
    \item Level adjustment +1
  \end{itemize}
\end{35e}

\subsubsection*{Savage Elves}

In the jungles \emph{Yua'cata} live the \emph{savage elves}, a loose
collection of tribes of cannibalistic and demon worshipping elves. They rarely
wander beyond the confines of their jungle and are one of the rarest elves to
meet in the civilised areas of \emph{Aror}. Their skin is usually light to dark
brown, and their hair and eyes are often light to deep green.

\begin{35e}
  \textbf{Savage Elf Traints (Ex)}: The following traits are in \emph{addition}
  to the high elf traits, except when noted.
  \begin{itemize}[noitemsep]
    \item +2 Strength, +2 Dexterity, +2 Constitution (replaces high elf ability
    modifiers)
    \item Automatic languages: Elven
    \item Favoured Class: Any. When determining whether a multiclass takes an
    experience point penalty, his or her highest-level class does not count.
    \item Level adjustment +1
  \end{itemize}
\end{35e}

\subsection*{Dwarves}

\emph{Dwarves} are natural fighters, miners and smiths. Among all of the races
they are the most reclusive of all. Dwarves usually stand between 1.2 and 1.4
metres high, but are on average almost has heavy as humans. Their skin is
usually tan brown, their skin ranges from brown to black, and they often have
brown eyes to match. For most male dwarves the beard is a symbol of status,
and often stands as a symbol for that particular dwarves caste within the clan.
Female dwarves do the same, but with their head hair.

\subsubsection*{Dwarven Culture}

They hardly join other cultures, and prefer to continue to live like their
ancestors did. \emph{Dwarves} organise themselves into huge clans or families
and live in deep caverns or mountains in a strict caste society. Those that do
not fit into these strict frameworks of societies are cast out, and then look
for other realms to live. Ever since the deep have become ever more dangerous
many smaller dwarven clans have settled on the surface near human or elven
settlements, while others have seemingly integrated into the large city
kingdoms. Offering their expertise on trade and smithing to the other races.

Dwarves speak their own language called \emph{Rutari} with its own alphabet of
the same name.

\subsubsection*{Caste System}

Most dwarven clans strictly enforce their caste society to maintain order and
control within their societies. Anyone who does not fit within this system is
met with suspicion, contempt or exclusion and exile at the worst. The lowest
of the caste are slaves, which are drawn from the pool of criminals among the
dwarves, as well as from the other races that live within the dwarven society.
Other races, such as humans, elves, or even \emph{gnomes}, often have no chance
to advance out of the slave caste. Most dwarven clans however allow visitors
into their strongholds.

\begin{35e}
  \textbf{Dwarf Racial Traits (EX):} The following traits replace the standard
  dwarven traits given by the Player's Handbook.
  \begin{itemize}[noitemsep]
    \item Favoured Class: Any. When determining whether a multiclass takes an
    experience point penalty, his or her highest-level class does not count.
  \end{itemize}
\end{35e}

\subsection*{Deepkin}

\aren{I had the misfortune of seeing my people's decline with my own eyes.
I was powerless, and unable to stop it. We are but a shadow of our former
selves...}

\graham{But still you live. And once the time is right, you may reclaim what
once was.}

The \emph{deepkin} are the cavern dwelling cousins of humans. They are capable
of seeing in the dark, have white pale skin, and often red to brownish hair.
\emph{deepkin} society is one of the oldest societies on \emph{Aror}, with a
long standing history in the arcane arts, building magnificent underground
cities, libraries and workshops. Ancient \emph{deepkin} were master arcane
smiths, golem constructors, inventor of many magic based constructs and
technology still used today.

However the ancient deepkin had their dominance challenged by the other races
of the deep, namely the dwarves, dark elves, and a now extinct species called
the \emph{ilians}. Thousand of years of conflicts lead to the decline of their
magnificent ancient civilisation. Instead of perishing however, they fled to
the surface, where they were warmly received by their surface cousins. Nowadays
\emph{deepkin} culture is but a shadow of what it was, although they still try
to reclaim what was once theirs with expeditions into the deep.

\begin{35e}
  \begin{itemize}[noitemsep]
    \item Medium: as medium creatures, \emph{Deepkin} have no special bonuses or
    penalties due to their size.
    \item \emph{Deepkin} base land speed is 30 ft.
    \item For all manners regarding racial restrictions or classifications
    \emph{Deepkin} count as humans.
    \item Darkvision out to 120 feet.
    \item Bonus Feat: Just as their human cousins, \emph{Deepkin} can choose a
    bonus feat at first level.
    \item Automatic languages: Common, Undercommon.
    \item Favoured Class: Any. When determining whether a multiclass takes an
    experience point penalty, his or her highest-level class does not count.
  \end{itemize}
\end{35e}

\subsection*{Halflings}

Most of the halflings are nomads, who travel across the lands in search for
places they could explore. If you think you have found a secluded spot on
\emph{Aror}, where no one else had sat foot before, you can be sure it is
already named after an halfling explorer. They travel in small families, and
in exchange for money offer their services to any small town they come across
on their travels. Halflings are curious, adventurous often find themselves
exploring the depths and other inhospitable places of \emph{Aror}.

A huge group of halfling families once decided to start another adventure: of
their own kingdom. They settled down near an elven kingdom and named their new
kingdom \emph{Brèagha Hilith}. After decades of peaceful coexistence between
the two races, they finally decided to tear down the last barriers and simply
merge the two kingdoms into a new one: \emph{Nen-Hilith}. It became a shining
beacon of civilisation, artistry and stability under the known city kingdoms
of \emph{Aror}.

\aren{Until a few giants stepped on them...}

\subsection*{Umgeher}

\aren{I was once the midwife of an \emph{Umgeher}...}

\graham{The less shared about this experience the better.}

\emph{Umgeher} are half undead humanoids that retain their own individuality,
will and determination across the process that turned them into undead. They
were created by the ancient vampires that reign in the city kingdom of
\emph{Kilmarnock} centuries ago, and were given the freedom to reproduce of
their own volition. Soon they travelled and spread across the entire known
world. Although they are immortal, they are not invincible. More so, their
flesh and skin is in a constant state of decay, and must combat their never
ending dissolution with oils and ointment. They have no bodily hair, and often
wear wigs to blend into normal population.

\emph{Umgeher}, like any species, are free to determine their own fates and
shape their own thoughts, they do have to constantly combat the ignorance and
fear of the other races. Many religious institutions and city states have now
allowed \emph{umgeher} to live there without a fear of being
persecuted. However this is a recent trend, and \emph{umgeher} prefer to build
their own little communities, towns and even cities.

\begin{35e}
  \begin{itemize}[noitemsep]
    \item Medium: as medium creatures, \emph{Umgeher} have no special bonuses or
    penalties due to their size.
    \item \emph{Umgeher} base land speed is 30 ft.
    \item As undead creatures \emph{Umgeher} gain all undead traits.
    \item \emph{Umgeher} have a live long experience in disguising themselves as
    humans, and thus gain a \emph{+2 racial bonus} on \emph{Disguise} and
    \emph{Bluff}.
    \item Favoured Class: Any. When determining whether a multiclass takes an
    experience point penalty, his or her highest-level class does not count.
    \item Automatic languages: common. Bonus languages: draconic.
  \end{itemize}
\end{35e}

\subsection*{Diarim}

\graham{The expression \emph{explaining freedom to a Diarim} has become
ingrained in many cultures as a metaphor for a fruitless labour.}

The \emph{Diarim} are a race of humanoid creatures that were bread for
specific tasks by the dark sorcerers and witches of the fire giants living on
the continent of \emph{Farlar}. They are the youngest of the races
on \emph{Aror}, and are a mixture of various other humanoid races.

\emph{Diarim} come in in as many shapes and sizes as the giants had uses and
tasks for their engineered slave race. But most share a common set of features:
the light, fair skin of the \emph{Deepkin}, blue hair of the \emph{Snow elves},
and an ingrained sense of duty and loyalty to their overlords - the giants. Very
few \emph{Diarim} ever escape the slavery of their masters, and those few that
do find it hard to shake their submissiveness and eagerness to please and help
that the giants have ingrained into them. Almost all have blue tribal tattoos
all over their body, which identify the current and past owners of the specific
\emph{diarim}.

The most common variant are labourers, small but stout breed, that was created
by introducing dwarven heritage into the \emph{diarim}. They excel at physical
labour, such as mining and construction. Siegers were bread with monstrous
races, often even giants, and are used as front line soldiers, gladiators and
shock troops. They are larger than any other \emph{diarim}, and often also
serve as slave overseers over the others. Exciters were bred and selected for
their beauty, and are priced possessions to be traded and gifted to other
giants. Their primary role is to entertain their overlords through song, dance
and company.

Most \emph{diarim} have no names. They never refer to themselves with names,
and are only being given a names by the giants if they have distinguished
themselves, either through heroic deeds, or through crimes.

In the recent decades more and more \emph{diarim} have escaped the continent
of \emph{Farlar} and joined the other humanoid races. The giants had to learn
that you cannot suppress the curiosity, love for wandering and freedom for long
when you create a species based on humans, halflings and elves.

\begin{35e}
  \textbf{Diarim Traits (EX)}:
  \begin{itemize}[noitemsep]
    \item Medium: as medium creatures, \emph{Umgeher} have no special bonuses or
    penalties due to their size.
    \item A \emph{Diarim}'s base land speed is 30 ft.
    \item \textbf{Weak Will (EX)}: All \emph{diarim} have a -2 penalty to will
    saves against charms and similar effects.
    \item Automatic languages: Giant, Common
  \end{itemize}

  \textbf{Sieger Traits (EX)}: the following traits are in \emph{addition} to
  the \emph{diarim} traits, except when noted otherwise.
  \begin{itemize}[noitemsep]
    \item Large size. -1 penalty to Armor Class, -1 penalty on attack rolls,
    -4 penalty on Hide checks, +4 bonus on grapple checks, lifting and
    carrying limits double those of Medium characters.
    \item Space/Reach: 10 feet/5 feet
    \item +8 Strength, -2 Intelligence, -2 Charisma, -2 Wisdom
    \item Favoured Class: Barbarian
    \item Level Adjustment: +1
  \end{itemize}

  \textbf{Exciter Traits (EX)}: the following traits are in \emph{addition} to
  the \emph{diarim} traits, except when noted otherwise.
  \begin{itemize}[noitemsep]
    \item +4 Strength, -2 Strength, -2 Constitution
    \item Favoured Class: Bard, Sorcerer
  \end{itemize}
\end{35e}

\subsection*{Gnomes}

\emph{Gnomes} are the children of \emph{halflings} and \emph{dwarves}. Since
they have no identity of their own, no culture of their own they often try to
integrate either with their parents cultures. Albeit they rarely fit in with
either societies. Gnomes themselves are sterile, and cannot have any children,
however they inherit the longevity of their dwarven parent. All in
all \emph{gnomes} are among the rarest races on \emph{Aror}. They often
dedicate their lives to adventuring and other dangerous businesses, and rarely
settle down.

There are no \emph{gnome} settlements or even kingdoms, and most of them live
scattered across the world in the large city kingdoms. Offering their services
as spies, thieves and adventurers to anyone willing to pay. Most large dwarven
cities and halfling settlements also have a small \emph{gnome} population.

\subsection*{Half-Orc}

Less rare than \emph{Gnomes} are the offspring of the union of human or elf
with an orc. Much like other half breed races \emph{half-orcs} are sterile,
and thus have no inner need or desire to settle down or start families. They
often live amongst their human parents offering their brute strength and
enduring physique as heavy labourers or fighters. A human mother giving birth
to a \emph{half-orc} has a high chance of dying during child birth. And since
very few women take such a risk, many \emph{half-orc} children are not the
result of a voluntary union. This sad reality, combined with a short temper,
less than flattering appearance and mental deficits, often elicit condescending
or outright demeaning behaviour from the other races towards \emph{half-orcs}.
