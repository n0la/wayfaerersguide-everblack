\chapter{Magic in Everblack}
\label{sec:Magic}

Magic is ever present in the world of Everblack. It is used by scholars and
wizards, to improve the lives of everyone, or by dark and ruthless mages to
bring harm and war upon the world. This chapter discusses the foundations of
magic within the world of Everblack, and brings new rules for magical effects
such as \emph{rituals} and \emph{soul magic}.

\section{Three Sources of Magic}

There are three main sources of magic in the world of Everblack. Both divine
and arcane magic, which stem from the same source. The distinction is that
divine magic is granted already prepared in the form of spells to divine
casters, while arcane researchers and scholars build their own spells out of
that same raw magical energy that also fuels divine spells.

The second power are psionical powers, which manifest in the powerful minds of
psykers, and psionic creatures such as \nameref{sec:Ilians}. They do not draw
their power from the same pool as divine or arcane power, and come directly
from the mind of the powerful being that casts that spell. Compared to the
other two, psionic powers are the rarest on Aror.

The third magical source of power is the soul of living beings, which allows
one to cast \nameref{sec:Soul Magic}. This power is either drawn from one's
own soul, or from souls of those around them. Much like psionic power, it
stems directly from the trained or inherent power of the individual, and does
not rely on external sources, such as a raw energy web that seeps in from
other planes. While arcane or divine necromancy is the art of manipulating
and corrupting bodies, soul magic is the art and craft of using ones inherent
power to shape and fuel souls.

\subsection{Divine Magic}
\label{sec:Divine Magic}

Divine Magic comes directly from a deity, a deity representing a concept, or a
powerful individual, such as \nameref{sec:Daemons} or \nameref{sec:Devils}. In
the case of lesser deities, the deity in question grants that power, in the
understanding that the power is used to further the deities interests, and
goals. Once that power is granted, it cannot be revoked by the lesser deity
until it is used, or lost. True deities do not exist as living beings, being
concepts given power, and thus work differently. A priest that follows a true
deity draws strength and power from the concept said deity represents. For
example a priest of the \nameref{sec:Order} draws strength from an inner desire
to create order out of chaos, and from defeating the evil that may stem from
chaos. A priest of a true deity may lose their power should they stray too
far from their true deities concepts, ideas and ideology.

\subsubsection{Resurrection}

Resurrection magic works differently on Aror. Upon death all souls dilute into
the soul well, coming apart by the seams until they can no longer be recovered.
Much like you cannot recover the exact same water particles once you have
poured them into the ocean. The soul of a recently deceased can only survive
this dissolution of the self if another powerful soul of the well intervenes.
While some powerful souls may do so to further their own reasons, many will only
do so if they see a benefit for themselves. So those that wish to cheat death
will have to make a deal with a powerful daemon.

\begin{35e}{Resurrection}
  Any resurrection or reincarnation magic only works within 1d4 hours of
  death. After that resurrection magic only succeeds if a powerful daemon
  wishes for it to succeed.
\end{35e}

\subsection{Arcane Magic}
\label{sec:Arcane Magic}

Those deities that give power to their clerics, and priests cannot do so with
perfect efficiency. The process of granting, and using divine powers leaks
magical energy which remains trapped on Aror. This magical energy can be
harvested, shaped, and channelled into spells by those that study the craft of
\emph{arcane magic}. Due to its versatile nature, and its inherent
independence from a higher power, arcane magic is considered more powerful
that divine magic but also exceptionally difficult to study, hard to
understand, and dangerous to wield.

\begin{35e}{Arcane Magic}
  Arcane Magic of \emph{all} forms require years to learn, and wield even at
  the most basic levels. Any and all arcane wielders (even those that use
  ``inherent'' arcane magic like bards and sorcerers) are usually 10 to 20
  years older their divine or martial counterparts to make up for the years
  spent training, and learning. Unless of course they one of the rare
  \hyperref[sec:Graham Balance]{child prodigies}.
\end{35e}

\subsection{Summoning}
\label{sec:Summoning}

The \emph{summoning} and \emph{calling} sub schools of arcane magic are
related with bringing extra-planar creatures to or from Aror. Due to the
existence of a device called the \nameref{sec:Monolith} these sub schools of
magic are harder to perform and practice than other schools of magic. Since
summoning devils, or even \nameref{sec:Demons} is incredibly dangerous, many
cities ban these schools of magic in their entirety.

Any extra planar creature summoned to Aror will fade away to the soul well
just like any other living being residing on Aror. This has made Aror a very
unpopular destination for many more powerful extra-planar species, as they
will not be sent home upon defeat, but risk permanent death at the hands of
the soul well. Further details about how this affects summoning magic can
be found in the book's section about the \nameref{sec:Monolith}.

\subsection{Necromancy}
\label{sec:Necromancy}

Necromancy is the art and craft of manipulating both body and soul to achieve
a purpose. In many cases the craft destroys the soul, leaving only a soulless
husk behind, while in others it does exactly the reverse. The art of
necromancy was discovered when \nameref{sec:Morana} turned some of her
followers to vampires, and has since been excessively studied by scholars,
priests and wizards. \nameref{sec:Isamir} also gave the power of necromancy to
the \nameref{sec:Inua} who closely guard the secrets of their rituals, spells
and incantations.

Liches, and \nameref{sec:Vampires} are the epitome of applied necromancy, and
many scholars have spent millennia studying them to better understand the
craft. While many necromancers wish only to study the vampire to better help
them survive, much like a doctor would for the living, others use the powers to
do evil, creating vicious and horrid creatures to do their bidding. Necromancy
is thus outlawed in many regions, and cities, requires oversight, or a special
permit.

% Rituals
\section{Rituals}
\label{sec:Rituals}

Rituals, or incantations, are powerful \emph{soul magic} or \emph{divine}
rituals can be cast by anyone, even those that are not arcane or divine
casters. These rituals often require specific words to be chanted, ritual
places specifically crafted and arranged for the incantation, often take
several hours of preparation and then to perform, and often require more than
person to be successfully performed.

The most prominent source of rituals are the lore and history of the
\nameref{sec:Old Ways}, as well as the lore of druidic circles. While the old
ways use rituals to heal, the druids use them for nefarious purposes, such as
cursing enemies with \hyperref[sec:True Lycanthropes]{true lycanthropy}.

\begin{35e}{Rituals}
  See the Unearthed Arcana variant magic rules on \emph{Incantations} on how
  to perform soul magic rituals.
\end{35e}

% Bone to Bone and Flesh to Flesh
\subsection{Bone to Bone}
\label{sec:Bone to Bone}

\songquote{Merserburger Zaubersprüche}{
  sôse bênrenki, sôse bluotrenki, sôse lidirenki: \\
  bên zi bêna, bluot zi bluoda, \\
  lid zi geliden, sôse gelîmida sîn.
}

\emph{Bone to Bone} is an ancient healing ritual performed by the followers of
the \nameref{sec:Old Ways}. It is meant to cure someone of all wounds, as well
as restore broken or damaged limbs. It also restores one lost body part, such
as cut off limbs, lost eyes or missing ears. The ritual is cast by one shaman of
the old ways, with the help of six others.

First, a very shallow grave must be dug, in which the recipient of the healing
magic must be placed when both moons stand high up in the sky. All casters must
stand around the grave, chanting the healing words repeatedly, supported
musically by drums or a low rhythmic drone. The recipient of the ritual is fed
a specially brewed potion which puts them to sleep for twenty four hours. Once
sleep has set in, the grave is covered thinly with dirt. Not too much, so that
the patient my escape by himself, but just enough to hide him from the world.

Then, last but now least, a living creature, often livestock, captured wild
animal, is bound and shackled, and then sacrificed above the grave. The blood
of the sacrifice is then allowed to seep into the grave, giving its life to
the patient buried underground.

Once twenty four hours are up, and the ritual was completed successfully, the
live of the animal has been transferred to the patient, healing him and
restoring lost limbs. The recipient is freed from the grave, with his body
healed and any missing limb restored. If the ritual failed, the person wakes
up prematurely and unhealed, and must dig itself out or suffocate beneath the
dirt.

A variation of this ritual exists, called \emph{Blood to Blood}, in which
a sentient humanoid creature is sacrificed instead of an animal. This ritual
resurrects the dead buried in the shallow grave, but is often banned in many
tribes and societies. If this alternate version of the ritual fails, the
caster is killed along the sacrifice to allow the deceased to rise again.
However sometimes the soul of the target is broken in the process.

\begin{35e}{Bone to Bone}
  \srditem{Effective Level}{6th}
  \srditem{Skill Check}{Knowledge (soul magic) or Knowledge (old ways), DC20,
    3 successes \textbf{and} Perform (oratory), DC20, 3 successes}
  \srditem{Failure}{%
    Target awakens prematurely after 2d4 hours, and must succeed a DC13
    fortitude saving throw or suffocate to death in the shallow grave. No hit
    points or effects are healed.
  }
  \srditem{Components}{V, S, M, F}
  \srditem{Casting Time}{60 minutes}
  \srditem{Range}{Personal}
  \srditem{Target}{One target, buried in the grave}
  \srditem{Duration}{Instantaneous}
  \srditem{Saving Throw}{Will negates, harmless}
  \srditem{Spell Resistance}{Yes, harmless}
  \srditem{Focus}{Shallow grave with the patient}
  \srditem{Components}{One creature of type \emph{animal} as sacrifice. One
    potion of \emph{deep sleep} that costs 20 shards in materials to make.}
  \srditem{Description}{%
    If the ritual succeeds the patient awakens after 24 hours, and heals the
    \emph{animals number of HD x d12} of hit points, and curing all of the
    following status effects: ability damage, blinded, confused, dazed,
    dazzled, deafened, diseased, exhausted, fatigued, feebleminded, insanity,
    nauseated, sickened, stunned, and poisoned.

    The spell also regrows one lost limb.
  }
\end{35e}

\begin{35e}{Blood to Blood}
  \srditem{Effective Level}{7th}
  \srditem{Skill Check}{Knowledge (soul magic) or Knowledge (old ways), DC24,
    3 successes \textbf{and} Perform (oratory), DC20, 3 successes}
  \srditem{Failure}{%
    Target is resurrected, but the caster's live is taken (no saving throw)
    along with the sacrifice.

    There is a chance the target returns with a
    \hyperref[sec:Broken Soul]{broken soul}.
  }
  \srditem{Components}{V, S, M, F}
  \srditem{Casting Time}{60 minutes}
  \srditem{Range}{Personal}
  \srditem{Target}{One target, buried in the grave}
  \srditem{Duration}{Instantaneous}
  \srditem{Saving Throw}{Will negates, harmless}
  \srditem{Spell Resistance}{Yes, harmless}
  \srditem{Focus}{Shallow grave with the patient}
  \srditem{Components}{One creature of type \emph{humanoid} as sacrifice. One
    potion of \emph{eternal sleep} that costs 500 shards in materials to
    make.}
  \srditem{Description}{%
    Upon failure or success of the ritual, the target rises from the dead
    after 24 hours, as if the seventh level cleric spell \emph{Resurrection}
    had been cast upon it.

    Failure kills the caster (no saving throw), and might break the targets
    soul as if by \nameref{sec:Broken Soul} (will save DC 26).
  }
\end{35e}

% Primal Curse
\subsection{Primal Curse}
\label{sec:Primal Curse}

The \emph{primal curse} is an ancient druidic ritual, in which druids bestow
the curse of \hyperref[sec:True Lycanthropes]{true lycanthrophy} upon a
target.

First a one or two litres of water must be gathered that has rested in the
foot prints of the desired animal for a few minutes. So, if the target should
become a werewolf, the water must have been in the foot prints of a wolf for a
few minutes. This water must then be blessed by mixing it with a drop of blood
of all the druids involved in the ritual. Then half of the blood of the victim
must be drained, while he is simultaneously force fed the mixture of blood and
water.

If the ritual succeeds the target becomes a
\hyperref[sec:True Lycanthropes]{true lycanthrope}, and if the spell fails the
druids that have given their blood to bless the water, are forced into rabid
animal shapes of the intended were creature, and will prey on each other.

\begin{35e}{Primal Curse}
  \srditem{Effective Level}{6th}
  \srditem{Skill Check}{Knowledge (nature) DC26, 4 successes}
  \srditem{Failure}{%
    All druids are forced into a \emph{chaotic evil} animal shape corresponding
    to the animal of the intended were creature. They cannot determine friend
    from foe, and will thus attack each other.
  }
  \srditem{Components}{V, S, M, F}
  \srditem{Casting Time}{60 minutes}
  \srditem{Range}{Personal}
  \srditem{Target}{One target, bound and helpless}
  \srditem{Duration}{Instantaneous}
  \srditem{Saving Throw}{Will negates, DC: 16 + caster's \emph{Wis} modifier}
  \srditem{Spell Resistance}{Yes}
  \srditem{Focus}{The water gathered from the animal's foot prints, as well as
    the stone altar upon which the victim is bound.}
  \srditem{Components}{One or two litres of water gathered from an animal's
    foot prints. The animal from which this water is gathered determines the
    animal form of the lycanthrope if the ritual succeeds. As well as drops of
    blood from each druid involved in the ritual, costing each druid 1000 XP.}
  \srditem{Description}{%
    If the ritual succeeds the patient turns into a true lycanthrope.

    If the ritual fails each druid that has given blood are forced into a
    \emph{chaotic evil} animal shape corresponding to the animal of the
    intended were creature. They cannot determine friend from foe, and will
    thus attack each other.
  }
\end{35e}

% Summon Runemaster
\subsection{Summon Runemaster}
\label{sec:Summon Runemaster}

The ritual \emph{Summon Runemaster} is used to conjure the Runemaster and plead
for him to each one \nameref{sec:Rune Magic}. He never shows up himself, instead
sending his minions (erinyes) instead, due to security concerns.

A successful summoning requires a summoning circle in the shape of a
pentagram, adorned with candles in each corner, in which a living humanoid is
sacrificed in his honour. After the sacrifice has been killed, one must plead
his or her case on why one is worthy enough to receive the power that comes
with rune magic. The Runemaster then either honours this plea by sending a
minion (or show up directly) or deny this request by ignoring it.  If a devil
does appear, one must strike a deal with the devil which often includes aid in
learning rune magic.

Many captured rune carvers have reported that it took them several attempts to
gain the master's attention. While others have tried to impress with the
Runemaster by carving embellished runes into their summoning sacrifice's skin
with various degrees of success. Other rune carvers have offered powerful
magical artefacts, their servitude, or the souls and bodies of other living
humanoids in the hope of gaining favour with the devil.

\begin{35e}{Summon Runemaster}
  \srditem{Effective Level}{6th}
  \srditem{Skill Check}{Knowledge (planes) DC22, 1 success, Perform
    (Oratory) DC:22 3 successes
  }
  \srditem{Failure}{Nothing.}
  \srditem{Components}{V, S, M, F}
  \srditem{Casting Time}{60 minutes}
  \srditem{Range}{Personal}
  \srditem{Target}{None}
  \srditem{Duration}{Instantaneous}
  \srditem{Saving Throw}{None}
  \srditem{Spell Resistance}{No}
  \srditem{Focus}{A pentagram with five candles in each corner, and a humanoid
    sacrifice in the middle of the pentagram. As well as a dagger, short sword
    or knife that will be blessed by the Runemaster, or one of his minions,
    should they grant an audience.
  }
  \srditem{Components}{A masterwork dagger, short sword or knife. And one living
    humanoid creature as a sacrifice.
  }
  \srditem{Description}{The living humanoid sacrifice must be killed with the
    dagger, knife or short sword. After which the caster may start his plea
    on why he is worthy to receive rune magic. If the plea is heard and found
    worthy, the \nameref{sec:Runemaster} will send a minion to negotiate a deal
    which often encompasses aid in learning \nameref{sec:Rune Magic}.
  }
\end{35e}



% Runemagic
\section{Rune Magic}
\label{sec:Rune Magic}

Rune magic is a perverted form of arcane magic that is taught by the mysterious
\hyperref[sec:Devils]{devil} called the \nameref{sec:Runemaster}. He teaches it
to any mortal he deems worthy to wield that power - i.e. is evil enough to go
through with the ritual sacrifice required to create runes.

It draws upon the souls of the living to fuel arcane and divine runes carved
into the caster's skin. These runes are mostly passive in nature, providing a
constant beneficial protective effect to whoever wears them. The magical
benefit stops only once the rune is destroyed, and are thus highly sought
after by anyone seeking lasting and permanent arcane protection.

Rune magic is taught directly by the Runemaster, or his minions, or learned
from a book drafted by the Runemaster, called the \emph{runic lexicon}. The
rituals to craft these runes all require living sacrifice, and are thus
forbidden in almost all city stations, nations and baronies.

Runes are carved into the flesh of the wearer with a ritualistic knife, and
thus permanently scar and deform the wearer's skin and flesh. Some runes
become rather huge patterns of intricate forms, shapes and lines, limiting the
amount of runes that may be applied at any given time to a body. The
ritualistic knife must first be hallowed in the blood of a living humanoid
sacrifice, that is dedicated to the Runemaster himself. If he deems the
subject willing, he will bless the knife, and then teach rune magic.

\graham{Are you going to teach Runemagic in my book?}

\aren{Hell no. But I thought it wise to include just enough information to be
  useful in identifying Runemagic should our esteemed readers encounter it.}

The runes themselves are then carved into a living sacrifices skin, often in
delicate intricate patterns spanning the entire body and skin, accompanied by
secondary sacrifices, chanting and the recitation of abyssal incantations. A
smaller version of the rune is then carved into the casters skin, and then the
sacrifice is killed with the ritualistic dagger in the name of the
Runemaster. If all is done correctly, the soul power of the slain sacrifice is
then used to power the rune's magical effect on the wearer.

Rune magic is often used by evil arcane and divine casters, who are already
engaged in living sacrifices (for example for necromancy, or to appease other
evil creatures), or who cannot afford magical items or those who cannot cast
spells themselves. No arcane or divine knowledge or spell casting ability is
required to create and carve runes.

\begin{35e}{Runemagic}
  Any cleric or arcane spell level 4 or below that could be cast upon yourself
  as the wearer of the rune, can be used in a rune magic ritual. It requires a
  blessed dagger, with which the ritual \nameref{sec:Summon Runemaster} must
  be performed.

  Runemagic requires that a large special rune must be carved into the skin of
  a living humanoid creature, with HD equal or higher to \emph{2 x spell level
    - 1} which takes \emph{spell level} hours to complete. The wearer must
  complete a \emph{Craft (Rune)} check with DC \emph{10 + caster level of
    spell} every hour or fail with the crafting of the rune. Once failed, the
  caster has to start over with a new sacrifice. If successful then a smaller
  rune must be carved into the wearer, and the humanoid creature must be
  killed with the blessed dagger to convey the benefits of the spell to the
  rune. The killed sacrifice can no longer be resurrected unless with a
  \emph{Resurrection, Greater} spell.
\end{35e}


% Soul Magic
\section{Soul Magic}
\label{sec:Soul Magic}

Soul magic is one of three main pillars of magic on Aror, alongside divine
magic (and its closely related cousin arcane magic), and psionic
powers. It taps into the very soul of the caster, and thus, much like psionic
energy, is inherent to the caster.

All living creatures on Aror have a soul, which is fundamentally intertwined
with the body. Normally no creature is aware of its own soul, and must first
experience a traumatic event called a \emph{soul awakening}. During the
awakening the inherent bond between the body and the soul are broken, and the
soul is free to be experienced on its own.

Those aware of their own souls can attempt to manipulate it, and tap into
its vast energy reserves. These reserves, often called \emph{soul power} or
\emph{soul magic}, can then be shaped into useful spells and incantations.

\subsection{Soul Well}
\label{sec:Soul Well}

Every soul on Aror, from every bird, animal, tree, fish and humanoid walking
on its surface, emits a faint soul aura. The energy emitted feeds a vast sea
of soul energy that permeates the world. This omnipresent soul energy field is
called the \emph{soul well}. Its presence is incredibly faint in many places,
but it can vary in strength depending on the presence or absence of life in an
area.

Once a creature with a soul dies, its soul slowly withers and its power leaks
slowly into the soul well until it is dissolved. This is the reason why
many outsiders - whose souls would naturally return to their home plane - die
permanently on Aror. It is also the reason why lesser deities have trouble
retrieving souls of their followers from Aror. This in turn makes it incredibly
difficult to use magic to raise the dead.

The soul well can not only vary in strength, but can also be absent, or 
even corrupted in certain areas. Corruption can occur through necromancy, 
defilement, excessive use of divine magic, or through violent events. 
Fields of battle where thousands have perished, graveyards that have 
repeatedly been used for rituals of necromancy, or even areas repeatedly 
hallowed (or unhallowed) by divine magic can be places where the soul well 
is corrupted. Areas where the soul well is corrupted can lead increased 
activity of undead (as the souls cannot return to the well), as well as 
interruptions to life in general. Children and young might be born without 
souls, current souls might become broken and life as a whole might perish 
in a given area leaving behind a desolate patch of land.

It is possible for a soul to donate part of its power to such areas to restore
its connection to the soul well. This art of restoring such broken areas is
called ``preservation''. The reverse is also possible, and a soul may draw
upon the soul well to fuel his own powers, depriving the area of its connection
to the soul well, and thus the life that is connected to it. This power is
called ``defiling'', and is highly controversial.

\graham{``Highly controversial''? Not the choice of words I'd have used.}

\subsection{Soul Awakening}
\label{sec:Soul Awakening}

During a soul awakening the inherent connections that interweave both the 
body and the soul are disrupted, allowing each to live without the other.
Creatures that have awoken become aware of their own soul, and can thus use
it to create spells. Strong souls may even leave their bodies behind, and
some extra ordinarily powerful souls can construct temporary vessels for
themselves. 

There is a natural force that draws a soul to its original body, and a body 
to its original soul. If either is destroyed, the result is usually an 
undead creature that roams endlessly in search for something that no longer 
exists. Bodies that live without their souls are often corporeal undead, 
such as zombies, skeletons and \hyperref[sec:Umgeher]{umgeher}. The 
technical term for a soul without a body is a ``free soul`` (see below), as 
not all free souls are undead. However the vice versa holds true, as all 
incorporeal undead such as wraiths, ghosts, spectres, and spirits are free 
souls.

Awakenings come from traumatic experiences that either affect the body or the
soul. Near death experiences, loss of someone important, necromancy, or being
unintentionally soul broken by another spell caster are some of the more
common ways to awake.

\begin{35e}{Soul Awakened}
Soul Awakened is an acquired template that can be added to any living
intelligent creature that has a soul (referred to hereafter as base
creature). This template cannot be taken, only acquired.
\srditem{Requirements}{The base creature must have had a significant incident
  with souls, soul magic or a traumatic event in the past to be eligible for
  this template. Furthermore the base creature must have a soul to begin with.
}
\srditem{Size and Type}{The base creature’s type and size remains unchanged,
  and retains all of the base creature’s statistics and special abilities
  except as noted here.
}
\srditem{Skills}{An awakened creature immediately adds \emph{Knowledge (Soul
    Magic)} and \emph{Soulcraft} to its list of available class skills. See
  this article on soul magic for details.
}
\srditem{Soul Sight (Ex)}{At will, as a free action, any soul awakened
  creature can turn on and off their ability to perceive a shining aura
  surrounding souls and creatures with souls. This will show absence and
  presence of souls, and if the creature studies a soul aura for three rounds
  it can use a \emph{Soulcraft} check in place of the appropriate Knowledge
  check to deduce powers, abilities, weaknesses and capabilities of a specific
  creature. This works just like the Knowledge skill would when studying
  creatures. If the check against the DC fails, nothing is learned about the
  specific soul and the check cannot be made again on that specific soul until
  new ranks in Soulcraft are gained.
}
\end{35e}

\subsection{Soul Sight}
\label{sec:Soul Sight}

Perceiving the faint soul aura that emitted by living beings is the first 
power that a freshly soul awoken learns. This power is granted 
automatically, as one may now perceive the world through the eyes of the 
soul, much like one would through the eyes of the body. The strength and 
intensity of the light depends on the target soul's strength and prowess, 
and creatures without a soul have no aura.  Learning to suppress this soul 
sight is often the first thing a soul awakened practices, as seeing the 
massive amount of lights - either from animals and plants in a forest, or 
from the crowds within a city - can be overwhelming.

Advanced practitioners of soul magic can also use their soul sight to
carefully study other creature's soul. This allows them to deduce some aspects
of the specific person's character, powers and weaknesses. Through intensive
studies of another soul it may become familiar to the awoken, and such an
understanding of a another person's soul can be enhanced and augmented by a
deep emotional connection and understanding of the person. It is not uncommon
for two soul practitioners to recognise each other solely by their soul aura,
regardless of what sort of body their soul might inhabit at a time.

\aren{Staring into the soul of a dragon is akin to staring into the suns.}

\begin{35e}{Soul Sight}
  See the entry in the soul awakened template on how soul sight works in terms
  of 3.5e mechanics.
\end{35e}

\subsection{Free Soul}
\label{sec:Free Soul}

A free soul or spirit, is a soul whose body has already withered away, and for
one reason or the other, does not wish to possess a new body. Being a free
soul cuts you off from most sensory inputs, such as touch, smell, and tactile
senses. Furthermore any free spirit has to expend an enormous amount of mental
strength to remain intact while the soul well siphons their power. This lack
of stimulus and the constant mental stress drives many free spirits insane or
even evil.

There are countless words for free spirits that roam the forests, caves, ruins
or the dark corners of the city. A \emph{mavka}, for example, is a mad free
spirit of a young woman that is sometimes dangerous to children and young men.
Children that have died a horrible death, because they were abandoned by their
parents are called \emph{myling}, and usually cause havoc and destruction on
adults. Free spirits that possess other dead bodies (i.e. not their own) are
called \emph{wiedergaenger}. In many cultures free spirits and ghosts that try
to make themselves known by making noises and sounds are called
\emph{poltergeists}. But generally free souls that roam the living world are
called \emph{spirits}, \emph{Geister}, \emph{shades} or \emph{Cean Gŵla}. Free
spirits that have grown in power, and learned how to remain intact while
returning to the soul well are called \nameref{sec:Daemons}.

Many, but not all, spirits slowly turn mad and crazy. Free souls keep most of
their memories, mental capacity and thus also mostly keep their mentality,
attitude and character traits. Those that do remain sane often become
eccentric, but rarely become threats to the living.

Free spirits will retain their original appearance they had in live, but will
appear translucent and glimmer either in a soft blue or green light. Some
spirits retain ghost or spirit versions of priced possessions, usually
clothing, and jewellery, but also rarely more intricate belongings such as
armour or weapons.

Free souls may still be twisted to evil through necromancy, turning them into
\emph{wraiths}, \emph{spectres}, and \emph{shadows}.

\begin{35e}{Spirit}
  ``Spirit'' is an acquired template that can be added to any living creature
  that has separated its soul from its body and has succeeded its will saves
  (see above). This creature is referred to hereafter as the base creature.

  \srditem{Size and Type}{The creature’s type changes to Undead
    (Incorporeal). It retains any subtype and subtypes that indicate kind. It
    does not gain the augmented subtype. It uses all of the base creature’s
    statistics and special abilities except as noted here.
  }
  \srditem{Hit Dice}{All current and future Hit Dice become d12s}
  \srditem{Speed}{A spirit does not walk, and loses any base land
    speed. Instead it gains fly speed 30ft. - unless the base creature has
    higher fly speed - with perfect manoeuvrability.
  }
  \srditem{Abilities}{Same as the base creature, except that its Charisma is
    increased by +4.
  }
  \srditem{Special Qualities}{A spirit has the same special qualities as the
    base creature as well as those below.
  }
  \srditem{Soul Damage (Ex)}{Touch attacks of a spirit do \emph{soul damage}.
  }
  \srditem{Damaged Soul (Ex)}{Any negative levels, level drain, ability drain
    or ability damage the base creature might have suffered transfer to the
    spirit. They continue to work just the same as they would have on the
    base creature. If the spirit possesses a body these drains and damages
    move over to the new body.
  }
  \srditem{Telepathy (Ex)}{A spirit can speak and hear, and can also
    communicate with others through Telepathy (60 ft.).
  }
  \srditem{Possession (Ex)}{A spirit can make a special touch instead
    instead of a full round action. This attack provokes an attack of
    opportunity. If the touch attack succeeds the spirit may attempt to
    posses the target creature. See under ``Possession'' for further
    information on how this works. If the touch attack fails it may try again
    next round.
  }
  \srditem{Level Adjustment}{Same as the base creature +2.}
\end{35e}

\subsection{Possession}
\label{sec:Possession}

Any free soul may attempt to possess another living being's body. To do so 
it must touch the target, and then begins a struggle in which the soul and 
body of the target fight against the hostile takeover. If the free spirit 
wins, it destroys the soul of the target and takes over the body. After the 
takeover the soul must get used to the new body, which results in a period 
of sickness and weakness after the possession.

Very few spirits that have remained sane will attempt to possess another, since
it will result in the death of the target. Forcing them to face a conundrum:
Either being stuck in spirit form slowly draining away into the soul well,
or commit a hideous crime.

\graham{A crime you have always committed freely.}
\aren{If the choice is my life or theirs, I will always choose survival.}

\begin{35e}{Possession}
  A free spirit must make a touch attack against a target in an attempt to
  possess it. If the touch attack fails it can try again next round. Once
  touched the target can make a \emph{will} and \emph{fortitude} save to
  resist possession. If any of these saves succeed, the free spirit may not
  attempt to possess the same target again for 24 hours, and gains a stacking
  -2 penalty for any further attempt.

  The DC for the possession attempt is 10 + \sfrac{1}{2} spirit's HD +
  spirit's \emph{charisma modifier}.

  If the size of the spirit and the target differs, the spirit gains a
  \emph{-2 penalty} for each step of size difference. Furthermore if the
  HD difference between the target and the spirit is \emph{4} the spirit gains
  a further \emph{-4 penalty} to the DC. Any additional HD difference above
  4 adds further stacking \emph{-2 penalties} to the DC. If the spirit and
  the target are of different creature types another \emph{-2 penalty} is
  added to the DC.
\end{35e}

\subsection{Soul Power}
\label{sec:Soul Power}

The soul itself is made out of pure energy, called \emph{soul energy},
\emph{soul essence} or \emph{soul power}. It replenishes by itself during rest,
much like the body replenishes itself after physical exhaustion. People that
are aware of their own souls can draw that power, direct it, and put it to
work. How much soul power that is available is directly related to the
strength of the person itself. The soul is only as powerful as the combined
trinity of soul, body and personality.

Soul power grows, and stands in direct correlation to the overall power,
strength, and prowess of the being that wields it. Physical strength is
reflected in a persons ability to wield soul magic, just as much as his
character strength, intellect, presence and self confidence. Those ill, sick,
broken in spirit, and wounded mentally find it harder to conjure and tap into
their own souls.

The power that stems from the soul is linked with colour blue. Raw soul fire
is blue, and so are most the visual effects produced by soul power. Soul
witcher and witches often gain visually distinct blue effects - such as
glowing blue eyes, strongly luminescent blue veins or blue crackling, sparks
and flames emanating from their body when they tap into the raw power of
their souls.

Soul essence is the polar opposite of the power that fuels arcane and divine
magic. Thus casters that also wield arcane and divine magic often find it
difficult to channel soul power, and vice versa. Often arcane and divine
casters avoid wielding soul magic, while soul witches avoid wielding arcane or
divine magic to avoid devastating interference effects when those two sources
of power meet.

Individual soul spells that can be fuelled with soul power, are discussed
later in the book.

\begin{35e}{Soul Power}
  Each creature that has undergone \emph{soul awakening} (see above), 
  gains \emph{soul power points}. These soul power points stem from \emph{
  soul power hit dice} that is determined by the creature's class. 
  Charisma increases the soul power pool, and functions to soul power 
  points like constitution works for hit points, granting $ \emph{CHA 
  modifier} \cdot HD $ additional soul power points. You gain soul power 
  points retroactively for any HD or class levels you might have gained   
  before having been awoken to your soul potential.

  The soul power hit die starts at \emph{d6}, but there are several factors
  that may increase or decrease the soul power hit dice steps. The steps are:
  0 (none), d4, d6, d8, d10, d12. These factors stack with each other, and may
  result in a class being fundamentally unable to wield soul magic.

  A fourth level wizard with 14 charisma would gain 0+2 (two steps down due to
  arcane wielder, and half BAB) power points on level up.

  A fifth level fighter with 10 charisma would gain 1d6+0 (one step up due to
  full BAB, one step down due to one strong save) power points on level up.
\end{35e}

\begin{table}[!htb]
  \captionsetup{labelformat=empty,font={large,bf},position=top}
  \caption{Soul Power Hit Die}
  \rowcolors{1}{white}{light-grey}
  \begin{tabular}{p{5cm} l}
    \textbf{Condition}            & \textbf{Steps from d6} \\
    Arcane wielder                & -2 steps \\
    Divine Wielder                & -2 steps \\
    One strong save               & -1 step \\
    Two strong saves              &  0 step \\
    Three strong saves            & +1 steps \\
    Half BAB                      & -1 steps \\
    Three-quarter BAB             & +0 steps \\
    Full BAB                      & +1 step \\
    2 base skill points per level & -1 steps \\
    4 base skill points per level &  0 step \\
    6 base skill points per level & +1 steps \\
    8 base skill points per level & +2 steps \\
    d4 HD                         & -2 steps \\
    d6 HD                         & -1 step \\
    d8 HD                         & 0 steps \\
    d10 HD                        & +1 step \\
    d12 HD                        & +2 steps
  \end{tabular}
\end{table}

\begin{table}[!htb]
  \captionsetup{labelformat=empty,font={large,bf},position=top}
  \caption{Soul Power HD for base classes}
  \rowcolors{1}{white}{light-grey}
  \begin{tabular}{p{5cm} l}
    \textbf{Class} & \textbf{Soul Power HD} \\
    Barbarian      & d10         \\
    Bard           & -           \\
    Cleric         & -           \\
    Druid          & -           \\
    % TODO: Give fighter more skill points
    Fighter        & d6          \\
    Monk           & d8          \\
    Paladin        & -           \\
    Ranger         & d6          \\
    Rogue          & d6          \\
    Sorcerer       & -           \\
    Wizard         & -           \\
  \end{tabular}
\end{table}

\subsection{Soul Fire}
\label{sec:Soul Fire}

Soul fire is uncontrolled soul power that manifests itself in the world as a
cold unfeeling blue fire. Like actual fire it can disintegrate and destroy
everything it touches, but it is fed solely by soul power. If it doesn't burn
anything with a soul (or a soul) it slowly withers and sizzles. Soul fire does
damage to living creatures and souls, and does electricity damage to objects.

\begin{35e}{Soul Fire}
  Soul Fire burns like regular fire, but does \nameref{sec:Soul Damage} to any
  creature that has a soul, and \emph{electricity} damage to objects.
\end{35e}

\subsection{Broken Soul}
\label{sec:Broken Soul}

\aren{The most common way to cure a broken soul, is to journey to the
  \nameref{sec:Walburga} witches, and plead or trade for a cure.}

A \emph{broken soul} is the soul equivalent of a terminal disease. Through
horrible failure in wielding soul powers, or through necromancy the soul is
broken and slowly leaks soul power until it is spent. And once the soul dies
the creature usually dies along with it. A broken soul can be cured through
appropriate soul powers, or through some divine spells.

\begin{35e}{Broken Soul}
  Every day that a creature spends with a broken soul it must make a DC: 15
  fortitude save or suffer 1 point of charisma and constitution drain. If
  the check succeeds the creature takes 3d6 points of soul damage instead.
\end{35e}

\subsection{Soul Damage}
\label{sec:Soul Damage}

Soul damage is a special kind of damage that cuts right through the body and
attacks the soul. Creatures that are damaged by this special damage, for
example through \nameref{sec:Soul Fire}, have not only their body wounded,
but also their soul.

\begin{35e}{Soul Damage}
  Soul Damage is a new type of damage, that is dealt by some soul powers,
  weapons and by soul fire. It only functions against creature that have a
  soul, or are souls. Soul damage does no damage against objects, or soulless
  creatures such as skeletons, constructs or zombies.
\end{35e}

\subsection{Soul Spells}
\label{sec:Soul Spells}

Raw soul magic can be shaped into useful spells often called \emph{soul spells}.
These may cause damage and harm, start soul fires and cause massive destruction,
but may also heal, cure and aid those in need if used for good. Many soul spells
were forged and created to be used in battle, and to destroy wayward and corrupt
souls.

Soul spells are usually not taught in classes or academy, and cannot be learned
directly from scrolls like arcane magic. However accomplished soul casters have
written books that teach fundamentals, but most soul spells are manifest by
experimentation and hard work on the part of the individual practitioner.

\begin{35e}{Soul Spells}
  Soul spells are listed later in the section about \nameref{sec:Heroic
    Characteristics}.

  Although common soul spells are listed in that section, dungeon masters and
  players are highly encouraged to create new soul spells specifically for
  individual characters. They should express that characters unique talents,
  quirks, weaknesses and strengths.
\end{35e}

\subsection{Soul Aura}
\label{sec:Soul Aura}

Soul auras is soul magic which emanates from certain powerful souls. These
auras often represent the person's strengths or weaknesses. Some people
emanate their natural charisma and leadership, making it easier for others to
follow them into battle, while others radiate away their fierce power and
brutally scaring weaker creatures into submission. While others project their
shadowed existence, making them easily overlooked, dismissed or even
completely ignored. The power of such auras are felt subconsciously by those
who have not awoken, while those that have, will see the aura with their soul
sight.

It is possible for some people, especially natural leaders, strong fighters
and people of great power to radiate a soul aura without them knowing about it.
They must not be awoken to their soul potential to be able to produce a strong
soul aura.

\begin{35e}{Soul Aura}
  Soul Auras are special soul powers that can be activated, and as long as
  they remain active they reduce the maximum soul power pool of a character. The
  effects of the aura may transfer to other creatures surrounding the caster,
  depending on the specific aura. Even creatures that have not awoken my radiate
  one soul aura, without them knowing, in which case the aura is always active.
\end{35e}

\subsection{Soul Magic in the World}
\label{sec:Soul Magic in the World}

Soul magic by itself is the rarest form of magic among humanoids. Most humanoid
creatures focus on arcane magic, which can be learned by everyone dedicated
enough, or divine magic, which is granted to everyone pious enough. Soul magic
itself is something personal, individualistic in nature, and different from
every witch and witcher. Practitioners of soul magic are rare, but generally
viewed favourably by most humanoid tribes and settlements.

However a few religious institutions see soul magic as an affront to the power
of the gods, and thus seek to actively suppress or even eradicate soul magic.
\nameref{sec:Lor} is known as a fervent enemy of soul magic, and teaches the
destruction of all free roaming souls to return to the ``natural order'' of
the world.

