\chapter{Magic in Everblack}
\label{sec:Magic}

Magic is ever present in the world of Everblack. It is used by scholars and
wizards, to improve the lives of everyone, or by dark and ruthless mages to
bring harm and war upon the world. This chapter discusses the foundations of
magic within the world of Everblack, and brings new rules for magical effects
such as \emph{rituals} and \emph{soul magic}.

\section{Three Sources of Magic}

There are three main sources of magic in the world of Everblack. Both divine
and arcane magic, which stem from the same source. The distinction is that
divine magic is granted already prepared in the form of spells to divine
casters, while arcane researchers and scholars build their own spells out of
that same raw magical energy that also fuels divine spells.

The second power are psionical powers, which manifest in the powerful minds of
psykers, and psionic creatures such as \nameref{sec:Ilians}. They do not draw
their power from the same pool as divine or arcane power, and come directly
from the mind of the powerful being that casts that spell. Compared to the
other two, psionic powers are the rarest on Aror.

The third magical source of power is the soul of living beings, which allows
one to cast \nameref{sec:Soul Magic}. This power is either drawn from one's
own soul, or from souls of those around them. Much like psionic power, it
stems directly from the trained or inherent power of the individual, and does
not rely on external sources, such as a raw energy web that seeps in from
other planes. While arcane or divine necromancy is the art of manipulating
and corrupting bodies, soul magic is the art and craft of using ones inherent
power to shape and fuel souls.

\subsection{Divine Magic}
\label{sec:Divine Magic}

Divine Magic comes directly from a deity, a deity representing a concept, or a
powerful individual, such as \nameref{sec:Daemons} or \nameref{sec:Devils}. In
the case of lesser deities, the deity in question grants that power, in the
understanding that the power is used to further the deities interests, and
goals. Once that power is granted, it cannot be revoked by the lesser deity
until it is used, or lost. True deities do not exist as living beings, being
concepts given power, and thus work differently. A priest that follows a true
deity draws strength and power from the concept said deity represents. For
example a priest of the \nameref{sec:Order} draws strength from an inner desire
to create order out of chaos, and from defeating the evil that may stem from
chaos. A priest of a true deity may lose their power should they stray too
far from their true deities concepts, ideas and ideology.

\subsubsection{Resurrection}

Resurrection magic works differently on Aror. Upon death all souls dilute into
the soul well, coming apart by the seams until they can no longer be recovered.
Much like you cannot recover the exact same water particles once you have
poured them into the ocean. The soul of a recently deceased can only survive
this dissolution of the self if another powerful soul of the well intervenes.
While some powerful souls may do so to further their own reasons, many will only
do so if they see a benefit for themselves. So those that wish to cheat death
will have to make a deal with a powerful daemon.

\begin{35e}{Resurrection}
  Any resurrection or reincarnation magic only works within 1d4 hours of
  death. After that resurrection magic only succeeds if a powerful daemon
  wishes for it to succeed.
\end{35e}

\subsection{Arcane Magic}
\label{sec:Arcane Magic}

Those deities that give power to their clerics, and priests cannot do so with
perfect efficiency. The process of granting, and using divine powers leaks
magical energy which remains trapped on Aror. This magical energy can be
harvested, shaped, and channelled into spells by those that study the craft of
\emph{arcane magic}. Due to its versatile nature, and its inherent
independence from a higher power, arcane magic is considered more powerful
that divine magic but also exceptionally difficult to study, hard to
understand, and dangerous to wield.

\begin{35e}{Arcane Magic}
  Arcane Magic of \emph{all} forms require years to learn, and wield even at
  the most basic levels. Any and all arcane wielders (even those that use
  ``inherent'' arcane magic like bards and sorcerers) are usually 10 to 20
  years older their divine or martial counterparts to make up for the years
  spent training, and learning. Unless of course they one of the rare
  \hyperref[sec:Graham Balance]{child prodigies}.
\end{35e}

\subsection{Necromancy}
\label{sec:Necromancy}

Necromancy is the art and craft of manipulating both body and soul to achieve
a purpose. In many cases the craft destroys the soul, leaving only a soulless
husk behind, while in others it does exactly the reverse. The art of
necromancy was discovered when \nameref{sec:Morana} turned some of her
followers to vampires, and has since been excessively studied by scholars,
priests and wizards. \nameref{sec:Isamir} also gave the power of necromancy to
the \nameref{sec:Inua} who closely guard the secrets of their rituals, spells
and incantations.

Liches, and \nameref{sec:Vampires} are the epitome of applied necromancy, and
many scholars have spent millennia studying them to better understand the
craft. While many necromancers wish only to study the vampire to better help
them survive, much like a doctor would for the living, others use the powers to
do evil, creating vicious and horrid creatures to do their bidding. Necromancy
is thus outlawed in many regions, and cities, requires oversight, or a special
permit.

% Rituals
\section{Rituals}
\label{sec:Rituals}

Rituals, or incantations, are powerful \emph{soul magic} or \emph{divine}
rituals can be cast by anyone, even those that are not arcane or divine
casters. These rituals often require specific words to be chanted, ritual
places specifically crafted and arranged for the incantation, often take
several hours of preparation and then to perform, and often require more than
person to be successfully performed.

The most prominent source of rituals are the lore and history of the
\nameref{sec:Old Ways}, as well as the lore of druidic circles. While the old
ways use rituals to heal, the druids use them for nefarious purposes, such as
cursing enemies with \hyperref[sec:True Lycanthropes]{true lycanthropy}.

\begin{35e}{Rituals}
  See the Unearthed Arcana variant magic rules on \emph{Incantations} on how
  to perform soul magic rituals.
\end{35e}

% Bone to Bone and Flesh to Flesh
\subsubsection{Bone to Bone}
\label{sec:Bone to Bone}

\songquote{Merserburger Zaubersprüche}{
  sôse bênrenki, sôse bluotrenki, sôse lidirenki: \\
  bên zi bêna, bluot zi bluoda, \\
  lid zi geliden, sôse gelîmida sîn.
}

\emph{Bone to Bone} is an ancient healing ritual performed by the followers of
the \nameref{sec:Old Ways}. It is meant to cure someone of all wounds, as well
as restore broken or damaged limbs. It also restores one lost body part, such
as cut off limbs, lost eyes or missing ears. The ritual is cast by one shaman of
the old ways, with the help of six others.

First, a very shallow grave must be dug, in which the recipient of the healing
magic must be placed when both moons stand high up in the sky. All casters must
stand around the grave, chanting the healing words repeatedly, supported
musically by drums or a low rhythmic drone. The recipient of the ritual is fed
a specially brewed potion which puts them to sleep for twenty four hours. Once
sleep has set in, the grave is covered thinly with dirt. Not too much, so that
the patient my escape by himself, but just enough to hide him from the world.

Then, last but now least, a living creature, often livestock, captured wild
animal, is bound and shackled, and then sacrificed above the grave. The blood
of the sacrifice is then allowed to seep into the grave, giving its life to
the patient buried underground.

Once twenty four hours are up, and the ritual was completed successfully, the
live of the animal has been transferred to the patient, healing him and
restoring lost limbs. The recipient is freed from the grave, with his body
healed and any missing limb restored. If the ritual failed, the person wakes
up prematurely and unhealed, and must dig itself out or suffocate beneath the
dirt.

A variation of this ritual exists, called \emph{Blood to Blood}, in which
a sentient humanoid creature is sacrificed instead of an animal. This ritual
resurrects the dead buried in the shallow grave, but is often banned in many
tribes and societies. If this alternate version of the ritual fails, the
caster is killed along the sacrifice to allow the deceased to rise again.
However sometimes the soul of the target is broken in the process.

\begin{35e}{Bone to Bone}
  \srditem{Effective Level}{6th}
  \srditem{Skill Check}{Knowledge (soul magic) or Knowledge (old ways), DC20,
    3 successes \textbf{and} Perform (oratory), DC20, 3 successes}
  \srditem{Failure}{%
    Target awakens prematurely after 2d4 hours, and must succeed a DC13
    fortitude saving throw or suffocate to death in the shallow grave. No hit
    points or effects are healed.
  }
  \srditem{Components}{V, S, M, F}
  \srditem{Casting Time}{60 minutes}
  \srditem{Range}{Personal}
  \srditem{Target}{One target, buried in the grave}
  \srditem{Duration}{Instantaneous}
  \srditem{Saving Throw}{Will negates, harmless}
  \srditem{Spell Resistance}{Yes, harmless}
  \srditem{Focus}{Shallow grave with the patient}
  \srditem{Components}{One creature of type \emph{animal} as sacrifice. One
    potion of \emph{deep sleep} that costs 20 shards in materials to make.}
  \srditem{Description}{%
    If the ritual succeeds the patient awakens after 24 hours, and heals the
    \emph{animals number of HD x d12} of hit points, and curing all of the
    following status effects: ability damage, blinded, confused, dazed,
    dazzled, deafened, diseased, exhausted, fatigued, feebleminded, insanity,
    nauseated, sickened, stunned, and poisoned.

    The spell also regrows one lost limb.
  }
\end{35e}

\begin{35e}{Blood to Blood}
  \srditem{Effective Level}{7th}
  \srditem{Skill Check}{Knowledge (soul magic) or Knowledge (old ways), DC24,
    3 successes \textbf{and} Perform (oratory), DC20, 3 successes}
  \srditem{Failure}{%
    Target is resurrected, but the caster's live is taken (no saving throw)
    along with the sacrifice.

    There is a chance the target returns with a
    \hyperref[sec:Broken Soul]{broken soul}.
  }
  \srditem{Components}{V, S, M, F}
  \srditem{Casting Time}{60 minutes}
  \srditem{Range}{Personal}
  \srditem{Target}{One target, buried in the grave}
  \srditem{Duration}{Instantaneous}
  \srditem{Saving Throw}{Will negates, harmless}
  \srditem{Spell Resistance}{Yes, harmless}
  \srditem{Focus}{Shallow grave with the patient}
  \srditem{Components}{One creature of type \emph{humanoid} as sacrifice. One
    potion of \emph{eternal sleep} that costs 500 shards in materials to
    make.}
  \srditem{Description}{%
    Upon failure or success of the ritual, the target rises from the dead
    after 24 hours, as if the seventh level cleric spell \emph{Resurrection}
    had been cast upon it.

    Failure kills the caster (no saving throw), and might break the targets
    soul as if by \nameref{sec:Broken Soul} (will save DC 26).
  }
\end{35e}

% Primal Curse
\subsection{Primal Curse}
\label{sec:Primal Curse}

The \emph{primal curse} is an ancient druidic ritual, in which druids bestow
the curse of \hyperref[sec:True Lycanthropes]{true lycanthrophy} upon a
target.

First a one or two litres of water must be gathered that has rested in the
foot prints of the desired animal for a few minutes. So, if the target should
become a werewolf, the water must have been in the foot prints of a wolf for a
few minutes. This water must then be blessed by mixing it with a drop of blood
of all the druids involved in the ritual. Then half of the blood of the victim
must be drained, while he is simultaneously force fed the mixture of blood and
water.

If the ritual succeeds the target becomes a
\hyperref[sec:True Lycanthropes]{true lycanthrope}, and if the spell fails the
druids that have given their blood to bless the water, are forced into rabid
animal shapes of the intended were creature, and will prey on each other.

\begin{35e}{Primal Curse}
  \srditem{Effective Level}{6th}
  \srditem{Skill Check}{Knowledge (nature) DC26, 4 successes}
  \srditem{Failure}{%
    All druids are forced into a \emph{chaotic evil} animal shape corresponding
    to the animal of the intended were creature. They cannot determine friend
    from foe, and will thus attack each other.
  }
  \srditem{Components}{V, S, M, F}
  \srditem{Casting Time}{60 minutes}
  \srditem{Range}{Personal}
  \srditem{Target}{One target, bound and helpless}
  \srditem{Duration}{Instantaneous}
  \srditem{Saving Throw}{Will negates, DC: 16 + caster's \emph{Wis} modifier}
  \srditem{Spell Resistance}{Yes}
  \srditem{Focus}{The water gathered from the animal's foot prints, as well as
    the stone altar upon which the victim is bound.}
  \srditem{Components}{One or two litres of water gathered from an animal's
    foot prints. The animal from which this water is gathered determines the
    animal form of the lycanthrope if the ritual succeeds. As well as drops of
    blood from each druid involved in the ritual, costing each druid 1000 XP.}
  \srditem{Description}{%
    If the ritual succeeds the patient turns into a true lycanthrope.

    If the ritual fails each druid that has given blood are forced into a
    \emph{chaotic evil} animal shape corresponding to the animal of the
    intended were creature. They cannot determine friend from foe, and will
    thus attack each other.
  }
\end{35e}

% Summon Runemaster
\subsection{Summon Runemaster}
\label{sec:Summon Runemaster}

The ritual \emph{Summon Runemaster} is used to conjure the Runemaster and plead
for him to each one \nameref{sec:Rune Magic}. He never shows up himself, instead
sending his minions (erinyes) instead, due to security concerns.

A successful summoning requires a summoning circle in the shape of a
pentagram, adorned with candles in each corner, in which a living humanoid is
sacrificed in his honour. After the sacrifice has been killed, one must plead
his or her case on why one is worthy enough to receive the power that comes
with rune magic. The Runemaster then either honours this plea by sending a
minion (or show up directly) or deny this request by ignoring it.  If a devil
does appear, one must strike a deal with the devil which often includes aid in
learning rune magic.

Many captured rune carvers have reported that it took them several attempts to
gain the master's attention. While others have tried to impress with the
Runemaster by carving embellished runes into their summoning sacrifice's skin
with various degrees of success. Other rune carvers have offered powerful
magical artefacts, their servitude, or the souls and bodies of other living
humanoids in the hope of gaining favour with the devil.

\begin{35e}{Summon Runemaster}
  \srditem{Effective Level}{6th}
  \srditem{Skill Check}{Knowledge (planes) DC22, 1 success, Perform
    (Oratory) DC:22 3 successes
  }
  \srditem{Failure}{Nothing.}
  \srditem{Components}{V, S, M, F}
  \srditem{Casting Time}{60 minutes}
  \srditem{Range}{Personal}
  \srditem{Target}{None}
  \srditem{Duration}{Instantaneous}
  \srditem{Saving Throw}{None}
  \srditem{Spell Resistance}{No}
  \srditem{Focus}{A pentagram with five candles in each corner, and a humanoid
    sacrifice in the middle of the pentagram. As well as a dagger, short sword
    or knife that will be blessed by the Runemaster, or one of his minions,
    should they grant an audience.
  }
  \srditem{Components}{A masterwork dagger, short sword or knife. And one living
    humanoid creature as a sacrifice.
  }
  \srditem{Description}{The living humanoid sacrifice must be killed with the
    dagger, knife or short sword. After which the caster may start his plea
    on why he is worthy to receive rune magic. If the plea is heard and found
    worthy, the \nameref{sec:Runemaster} will send a minion to negotiate a deal
    which often encompasses aid in learning \nameref{sec:Rune Magic}.
  }
\end{35e}



% Runemagic
\section{Rune Magic}
\label{sec:Rune Magic}

Rune magic is a perverted form of arcane magic that is taught by the mysterious
\hyperref[sec:Devils]{devil} called the \nameref{sec:Runemaster}. He teaches it
to any mortal he deems worthy to wield that power - i.e. is evil enough to go
through with the ritual sacrifice required to create runes.

It draws upon the souls of the living to fuel arcane and divine runes carved
into the caster's skin. These runes are mostly passive in nature, providing a
constant beneficial protective effect to whoever wears them. The magical
benefit stops only once the rune is destroyed, and are thus highly sought
after by anyone seeking lasting and permanent arcane protection.

Rune magic is taught directly by the Runemaster, or his minions, or learned
from a book drafted by the Runemaster, called the \emph{runic lexicon}. The
rituals to craft these runes all require living sacrifice, and are thus
forbidden in almost all city stations, nations and baronies.

Runes are carved into the flesh of the wearer with a ritualistic knife, and
thus permanently scar and deform the wearer's skin and flesh. Some runes
become rather huge patterns of intricate forms, shapes and lines, limiting the
amount of runes that may be applied at any given time to a body. The
ritualistic knife must first be hallowed in the blood of a living humanoid
sacrifice, that is dedicated to the Runemaster himself. If he deems the
subject willing, he will bless the knife, and then teach rune magic.

\graham{Are you going to teach Runemagic in my book?}

\aren{Hell no. But I thought it wise to include just enough information to be
  useful in identifying Runemagic should our esteemed readers encounter it.}

The runes themselves are then carved into a living sacrifices skin, often in
delicate intricate patterns spanning the entire body and skin, accompanied by
secondary sacrifices, chanting and the recitation of abyssal incantations. A
smaller version of the rune is then carved into the casters skin, and then the
sacrifice is killed with the ritualistic dagger in the name of the
Runemaster. If all is done correctly, the soul power of the slain sacrifice is
then used to power the rune's magical effect on the wearer.

Rune magic is often used by evil arcane and divine casters, who are already
engaged in living sacrifices (for example for necromancy, or to appease other
evil creatures), those who cannot afford magical items, or those who cannot
cast spells themselves. No arcane or divine knowledge or spell casting ability
is required to create and carve runes.

Runes of rune magic are permanent, and cannot be healed through divine magic.
They can be destroyed however by using the ritualistic knife to destroy the
pattern. Normal wounds that happen to destroy the skin or the runes (i.e. from
sustaining a cut in battle), do not destroy rune magic.

\begin{35e}{Runemagic}
  Any cleric or arcane spell level 4 or below that could be cast upon yourself
  as the wearer of the rune, can be used in a rune magic ritual. It requires a
  blessed dagger, with which the ritual \nameref{sec:Summon Runemaster} must
  be performed.

  Runemagic requires that a large special rune must be carved into the skin of
  a living humanoid creature, with HD equal or higher to \emph{2 x spell level
    - 1} which takes \emph{spell level} hours to complete. The wearer must
  complete a \emph{Craft (Rune)} check with DC \emph{10 + caster level of
    spell} every hour or fail with the crafting of the rune. Once failed, the
  caster has to start over with a new sacrifice. If successful then a smaller
  rune must be carved into the wearer, and the humanoid creature must be
  killed with the blessed dagger to convey the benefits of the spell to the
  rune. The killed sacrifice can no longer be resurrected unless with a
  \emph{Resurrection, Greater} spell.

  A medium creature is limited to 3d4 runes on his body, a small creature to
  2d4, and a large creature to 4d4 respectively. Roll this value at the first
  rune to see how many more may fit onto the body.
\end{35e}


% Soul Magic
\section{Soul Magic}
\label{sec:Soul Magic}

Soul magic is one of three main pillars of magic on Aror, alongside divine
magic (and its closely related cousin arcane magic) and psionic
powers. It taps into the very soul of the caster, and thus, much like psionic
energy, is inherent to caster and does not come from external sources like
divine or arcane energy.

All living creatures on Aror have a soul, that is fundamentally intertwined
with the body. Normally no creature is aware of its own soul, and must first
experience a traumatic event called a \emph{soul awakening}. During the
awakening the inherent bond between the body and the soul are broken, and the
soul is free to be experienced on its own.

\subsection{Soul Awakening}
\label{sec:Soul Awakening}

A \emph{soil awakening} is by nature a traumatic event for both body and soul.
During the awakening the inherent connections that interweave both the body and
the soil are disrupted, allowing each to live without the other. Bodies that
live without their soils are often corporeal undead, such as zombies,
skeletons and \hyperref[sec:Umgeher]{umgeher}. Souls without a body are also
often undead, such as wraiths, ghosts and spirits.

Those that experience soul awakening have their soul broken, and must first
heal before they can tap their souls for power. There are a few ways to heal
a broken soul, such as rest, intense mental and bodily training, or specific
soul spells.

Awakenings come from traumatic experiences that either affect the body or the
soul. Near death experiences, loss of someone important, necromancy, or being
intentionally soul broken by another spell caster are some of the more common
ways to awake.

\begin{35e}{Soul Awakened}
Soul Awakened is an acquired template that can be added to any living
intelligent creature that has a soul (referred to hereafter as base
creature). This template cannot be taken, only acquired.
\srditem{Requirements}{The base creature must have had a significant incident
  with souls, soul magic or a traumatic event in the past to be allegible for
  this template.
}
\srditem{Size and Type}{The base creature’s type and size remains unchanged,
  and retains all of the base creature’s statistics and special abilities
  except as noted here.
}
\srditem{Skills}{An awakened creature immediately adds Knowledge (Soul Magic)
  and Soulcraft to its list of available class skills. See this article on
  soul magic for details.
}
\srditem{Awaken (Ex)}{A soul awakened creature can attempt, once per day, to
  awaken a non-awakend being with a soul. The target must be be willing, and
  if it is not, the attempt simply fails using up the attempt for the day. The
  awakened creature must touch his target, upon which the target must succeed
  a will save (DC: 15 + HD of target) and a fortitude save (DC: 15 + HD of
  target). If the will save fails then the target’s soul becomes broken. If
  the fortitude save fails then the body rejects the soul, and both take 2d6
  points of damage (for 4d6 points in total).
}
\end{35e}

