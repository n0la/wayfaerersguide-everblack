\chapter{Adventuring Equipment}
\label{sec:Adventuring Equipment}

%% Armour
\section{Armour}
\label{sec:Armour}

Aror has some specific armours that you might find here, but not anywhere
else.  Typically armour on Aror is crafted to be lightweight, protective, but
also cost effective. Armour comes in many shapes and sizes, depending on the
culture where it was made, the skill of the armourer as well as the depth of
the owners pockets. Higher grade armour, such as breast plates or full sets of
plate armours are also status symbols, and often lavishly embellished and
engraved.

\begin{table*}[!htb]
  \small
  \captionsetup{labelformat=empty,font={large,bf},position=top}
  \caption{Overview of Armours}
  \rowcolors{1}{white}{light-grey}
  \begin{tabular}{p{2cm} l p{1cm} l l l p{1cm} p{2.5cm} l}
    \textbf{Armour} & \textbf{Cost} & \textbf{Armour Bonus} & \textbf{MDB} & \textbf{ACP} & \textbf{ASF} & \textbf{Speed Penalty} & \textbf{DR} & \textbf{Weight} \\
    \multicolumn{9}{l}{\textbf{Light Armour}} \\
    Gambeson        &   2 shards  & 2 & 8 &  0 &  5\% & -                  & 2/bludgeoning                  &  2kg  (5 lbs.) \\
    Jack of Plates  &   5 shards  & 3 & 7 & -1 & 10\% & -                  & 2/bludgeoning, and 2/piercing  &  4kg  (9 lbs.) \\
    Mail Shirt      &  10 shards  & 4 & 5 & -2 & 15\% & -                  & 5/slashing                     &  8kg (17 lbs.) \\
    \multicolumn{9}{l}{\textbf{Medium Armour}} \\
    Hide Armour     &   1 shard   & 3 & 4 & -3 & 20\% & -                  & -                              & 12kg (25 lbs.) \\
    Coat of Plates  &  10 shards  & 5 & 3 & -4 & 25\% & -                  & 4/bludgeoning, and 2/piercing  & 15kg (33 lbs.) \\
    Brigandine      &  30 shards  & 6 & 4 & -3 & 30\% & -                  & 4/bludgeoning, and 4/piercing  & 14kg (30 lbs.) \\
    \multicolumn{9}{l}{\textbf{Heavy Armour}} \\
    Breastplate     &  60 shards  & 7 & 3 & -4 & 25\% & -10 ft. / -5 ft.   & 5/slashing, and 5/piercing     & 20kg (44 lbs.) \\
    Plate Armour    & 150 shards  & 9 & 1 & -6 & 35\% & -10 ft. / -5 ft.   & 10/slashing, and 5/piercing    & 25kg (55 lbs.) \\
  \end{tabular}
\end{table*}

A \emph{gambeson} (aketon, padded jacket or arming doublet) is heavily padded
defensive jacket, which was either worn on its own, or in combination with
mail or plate armour. It serves as a padding to soften incoming blows and
offers some resistance against cuts. While meant as a padding beneath actual
armour, it is also often used a stand alone armour especially with those that
cannot afford better armour.

\emph{Jack of Plates} is a lightweight padded vest (or a converted gambeson)
into which thin metal plates have been sewed. It is often fashioned from a
gambeson to give it better resistance against thrusts and cuts. Since a jack
of plates is rarely made by a professional armourer, they are of lesser quality
and thus provide less protection. However is extremely popular with civilians,
militias and bandits due to its low cost.

A \emph{Mail Shirt} (also chain mail, chain shirt, or hauberk) is a shirt,
vest or jacket made out of small interlocking metal rings. It provides
excellent protection against cuts, but is also more expensive to
manufacture. They do come in a wide variety, as some feature protection for
the groin, upper arms, or sometimes even have long sleeves, while others only
protect the central torso. It is often worn over a gambeson.

\emph{Hide Armour} is a catch-all term for all sorts of home made armours
that are used by various brigands, barbarians, raiders, and bandits. It is often
fashioned out of leather, fur, linen and hide. Since barbarian armours often
feature the hide and fur of trophy animals (such as bears or wolves) these
armours are often simply referred to as ``hide armour''.

\emph{Coat of Plates} is a professionally crafted jack of plates, whose steel
plates are thicker, and cover all of the torso and the groin. While jack of
plates tend to have gaps in their plating due to limited craftsmanship, a coat
of plate was professionally crafted to carry several large and heavy steel
plates to protect the torso, and also sometimes the groin and the upper legs. A
coat of plates can also be worn over a gambeson.

A \emph{Brigandine} is a refined version of the coat of plates, which instead of
a few large steel plates uses many small oblong steel plates riveted into the
fabric for enhanced manoeuvrability. It is in all aspects superior to a coat of
plates, but also more expensive to manufacture. A brigandine can also be worn
over a gambeson.

\emph{Breastplate} (chest plate or cuirass) is a heavy armour worn at the
torso. It is a part of a full steel plate armour set, covering the front and
back of the upper body. Many also feature tassets that protect the groin and
the upper legs. The breast plate acts as additional defence of the torso,
in combination with mail which is worn beneath the plate.

\emph{Plate Armour} is a full suit of armour entirely encasing the wearer. It
features a breast plate for the torso, and several smaller pieces protecting
the legs, arms, hands and feet. Special round discs called rondel were affixed
to vulnerable spots in the armour (such as joints) to ward off thrusts and
piercing attacks. It is often worn over simple garments, and clothing. It is
by far the most expensive armour, and is often only reserved for nobility,
knights, as well as kings and queens. Although the protection from a full suite
of plate is superior to any other armour, it does have weak spots (e.g. at
joints) and is susceptible to bludgeon attacks.

\begin{note}
  If an armour states that it can be worn over another, that is exactly what
  the players can do. So a player can buy a gambeson, and then wear a hauberk
  over it. The effects do not stack, but the player can magically enhance each
  of these armours with different enhancements and enchantments.

  Normal restrictions regarding medium armour, as well as restrictions regarding
  plate armour (which is equivalent to a ``Full plate'') apply as normal.

  Damage reductions that affect the same damage source (e.g. slashing) do not
  stack, only the highest applies. Damage reductions for different damage types
  do indeed stack.
\end{note}


\section{Clothing}
\label{sec:Clothing}

The kind of clothing worn throughout the city kingdoms depends both on the
status of the individual, their personal taste as well as local trends in
fashion.  Tailors make a wide variety of garments, from commoner to artisan
clothing to royal garbs. In many parts of Aror the clothes do make the
person. People are expected to dress themselves according to their wealth, and
will infer your social status from the clothes you wear. So do not be puzzled
if people in \nameref{sec:Norbury} mistreat you if you appearance is unkempt,
and you wear linen rags meant for slaves.

\begin{35e}{Clothing}
  The clothing, as described in the ``Player's Handbook'' all exist, and
  are readily available. The setting puts a bigger focus on dressing properly,
  and a player's social status will be inferred from the clothes he or she
  wears.
\end{35e}

% Demon Bone
\section{Demon Bone}
\label{sec:Demon Bone}

Demon bone (or ``scourge bone'') is the organic, yet sturdy and strong
material the \nameref{sec:Scourge} produces to add structural integrity to its
structures and growth patterns. The bones of the demons spawned by the scourge
also contain this material, hence the name ``demon bone''. Demons often shape
weapons out of such bones, which they scrounge from the corpses of other
demons. The material of the demon bone is very hard to shape and brittle, but
can easily be sharpened and always has a rough and jagged edge, that causes
horrific wounds and cuts, that are difficult to heal and treat.

Demon bone cannot be made, created (even by spells), or mined, and must thus
be harvested either from dead demons, or by deconstructing the growths of the
scourge.

\begin{35e}{Demon Bone}
  Weapons can be fashioned out of demon bone. The material is sadly too brittle
  to make armours and shields out of it.

  Slashing weapons can have their blade made out of demon bone, and are
  especially effective. Their damage dice is then doubled. For example a
  \emph{Longsword} made out of demon bone does \emph{2d8} damage instead of
  \emph{1d8}. On a critical hit these weapons cause a lasting wound that does
  2 points of damage each round until the wound is treated with a healing
  spell, or with a DC 15 \emph{Heal} check. Making a demon bone weapon adds
  2,000 gp to the crafting cost, and the weapon must be of masterwork quality.
\end{35e}


% Documents
\section{Documents}
\label{sec:Documents}

\subsection{Citizen Papers}
\label{sec:Citizen Papers}

The cheapest of all identification methods are the citizen papers. Often a
single scroll, elaborately designed and embellished that holds the citizen's
name, race, gender and age; along with a seal of the kingdom or nation to
which they are a citizen off. These papers are important travel documents for
most citizens, and usually cost between five or ten \hyperref[sec:Shin]{shins}
to be issued. Families often just have one citizen paper for the entire family.

\subsection{Business Licence}
\label{sec:Business Licence}

Business licences are important documents for anyone that intent to form their
own businesses, or conduct trade across the world of Aror. The paper is usually
a beautifully embellished scroll, that contains basic information about the
business, the owner, and the seat of the kingdom or barony in which the business
has its head quarters. Many larger businesses ask for copies or the original
paperwork before conducting trade with another business, and many agents of
businesses are issued a copy of the licence as a document of identification.

Most city kingdoms issue business licences, while some city kingdoms and
baronies also offer business licences for areas of commerce that may be
outlawed in another area. For example \nameref{sec:Helmarnock} issues licences
for necromancers, while the all slaving nations issue business licences for
slavers, which might be outlawed in nations that ban slavery.


% Drugs
\section{Drugs}
\label{sec:Drugs}

The trade of often illicit, dangerous, addictive yet stimulating or arousing
substances is another sad reality of Aror. Often entire businesses have
evolved around manufacturing, smuggling and selling these drugs to those
addicted, and the global drug trade makes untold numbers of shards per year.

The more harmless drugs, such as tobacco, alcohol and sarelis are allowed in
most states and nations, while the more dangerous drugs are illegal to produce
and possess. Nevertheless an underground network of local thieves guilds,
smugglers and well hidden farmers and alchemists, produce, ship and supply all
corners of the world with their toxic creations.

\begin{35e}{Drugs}
  The rules for drugs, their usage, as well as any rules on how to make them
  are laid out in the 3.0 book called \emph{Book of Vile Darkness}.
\end{35e}

\begin{table*}[!htb]
  \captionsetup{labelformat=empty,font={large,bf},position=top}
  \caption{Overview of Drugs}
  \rowcolors{1}{white}{light-grey}
  \begin{tabular}{l l l l l}
    \textbf{Name} & \textbf{Type}  & \textbf{Price} & \textbf{Alchemy DC} & \textbf{Addiction} \\
    Atropa        & Ingested DC 17 & 5 shins/g      & 12                  & High \\
    Karthas Paste & Ingested DC 20 & 50 shins/g     & 20                  & Extreme \\
    Sarelis       & Inhaled DC 12  & 10 shins/g     & 27                  & Low \\
    Synemium      & Ingested DC 18 & 50 shards/ml   & 25                  & Extreme \\
    Xoridina      & Ingested DC 15 & 20 shins/g     & 15                  & Medium \\
  \end{tabular}
\end{table*}

\subsection{Atropa}
\label{sec:Atropa}

Atropa is a small flower that grows in abundance on most northern continents
of Aror. It has white pedals, and small green stalk. It has a soft, nutty and
sweet smell. It is mildly poisonous but it would take a large amount of the
flower to actually harm a human being.

Petals from the plant are dried in the sun, and then ground into a fine
grained white powder. This powder is then mixed with other herbs and
ingredients. The mixture is then inhaled through the nose, or mixed with water
and ingested. It is then capable of suppressing the need for shape shifters
(especially were creatures) to transform into their animal form. The powder
mixture is highly addictive however. It is highly valued by were creatures
that live in society, and is also often forcefully injected into captured and
imprisoned were creatures and druids.

\begin{35e}{Atropa}
  \srditem{Addiction Rating}{High}
  \srditem{Satiation Time}{3 days}
  \srditem{Damage}{It deals normal damage to shape shifters (were creatures,
    druids, and other creatures that can change form), and half damage to
    non-shifters.
  }
  \srditem{Initial Effect}{1d4+1 points of wisdom damage to non-shifters,
    1d8+1 damage to shifters.}
  \srditem{Secondary Effect}{Shape shifting ability is suppressed for the time
    of the satiation. The character loses all ability to shape shift, and
    cannot be forced to shape shift through external means, such as a full
    moon.
  }
  \srditem{Overdose}{Non-shape shifters who take the drug more than once
    within 24 hours take 2d6 points of damage. Shape shifters who take the
    drug more than once within 12 hours must make a separate save (Fort DC 36)
    or die in terrible pain.
  }
\end{35e}

\subsection{Karthas Paste}
\label{sec:Karthas Paste}

Karthas paste is a mixture of several herbs and mushrooms, all of which can be
found in damp and cold places (such as forests or caves). The ingredients are
dried, boiled, and then condensed to a thick, brownish paste that tastes sweet.
The ingredients are rather common, but the process of making the paste potent
enough requires a skilled alchemist.

It is highly addictive, wards off hunger, thirst and pain. It's main effect
clouds the users judgement, making him more susceptible to commands and
orders, often impairing the user's judgement so far as to make him believe
that potential dangerous orders and tasks are a good idea. It is thus often
mixed with \hyperref[sec:Food]{bird butter} and fed to slaves, and those that
have to do dangerous work such as mine \nameref{sec:Everblack}, or are sent
into battle against their will.

\begin{35e}{Karthas Paste}
  \srditem{Addiction Rating}{Extreme}
  \srditem{Satiation Time}{2 days}
  \srditem{Initial Effect}{Hunger is suppressed for two days.}
  \srditem{Secondary Effect}{User gains a -4 penalty to will saves, and
    gains an additional -8 penalty to skill checks to resist suggestions,
    bluffs or intimidations through the appropriate skills.
  }
  \srditem{Overdose}{If more than one dose is taken in a 12 hour period, the
    user takes 2d6 points of non-lethal damage. Using it more than three times
    within 24 hours causes 2d6 points of damage and paralyses the user for 2d4
    hours.
  }
\end{35e}

\subsection{Sarelis}
\label{sec:Sarelis}

Sarelis is a small weed or herb that is grown in tropical environments, such as
the vast jungles north of \hyperref[sec:South Goltir]{Goban mountain}, the
jungles of the \nameref{sec:Silver Isles} or \nameref{sec:Yuacata}. The weed
is harvested, dried and then mixed with tobacco and smoked.

Among all the drugs available on Aror it is the least potent, but nonetheless
addictive. It calms the nerves, makes one physically sluggish and causes mild
auditory and visual hallucinations. However it also heightens all senses, and
generally calms even the most aggressive people down allowing them to remain
calm and collected. It is quite popular, but never smoked pure but often mixed
with normal, harmless weeds.

\begin{35e}{Sarelis}
  \srditem{Addiction Rating}{Low}
  \srditem{Satiation Time}{10 days}
  \srditem{Initial Effect}{Harmless visual and auditory hallucinations}
  \srditem{Secondary Effect}{2 alchemical bonus to wisdom, as well as +5
    alchemical bonus to \emph{Diplomacy}.}
  \srditem{Side Effect}{None.}
  \srditem{Overdose}{Taking a second dose before the first has worn off causes
  the user to be nauseated for 1d4 x 10 minutes.}
\end{35e}

\subsection{Synemium}
\label{sec:Synemium}

Synemium, often simply shortened to \emph{``syn''} or the \emph{``blue gold''},
is the refined, blue, shimmering and thickish fluid that is made out of the
resin of the tree of the same name. The tree grows only in the jungles of
\nameref{sec:Yuacata}, and its resin is thus very hard to extract. The resin
itself must be refined and distilled before it can be used as a drug.

It is highly toxic in larger quantities (30 millilitres), and is thus only taken
in small drops. These drops are often absorbed through blood, or ingested
through the mucous membranes of the nose. It heightens and sharpens the
intellect, as well as allowing the user to stay awake and sharp for several
days without the need for sleep or rest.

It is highly priced and valued among those that do mental labour, such as
wizards, clerics, tacticians or researchers, but its astronomical price
makes it a luxury drug.

\begin{35e}{Synemium}
  \srditem{Addiction Rating}{High}
  \srditem{Satiation Time}{variable}
  \srditem{Initial Effect}{1d4+1 strength damage}
  \srditem{Secondary Effect}{1d4+1 alchemical bonus to intelligence and wisdom}
  \srditem{Side Effects}{Once taken, it automatically heightens (as the
    Heighten Spell Feat) the next 2d4 spells the character casts. Once all
    heightened spells have been cast satiation ends, and withdrawal of the
    drug kicks in. If the user does not cast spells, or his spells cannot be
    heightened, then withdrawal kicks in within 1d4+1 days.
  }
  \srditem{Overdose}{Those who take the drug more than once within 24 hours
    must make a separate save (Fort DC 28 negates) or die in terrible pain.
  }
\end{35e}

\subsection{Xoridina}
\label{sec:Xoridina}

Xoridina, often simply shortened to \emph{``dina''} or \emph{``devil's nut''},
is a family of large gigantic trees that grow in the southern realms of
Aror. They produce a small, pebble-sized nut, that becomes highly addictive
once the nuts have ripened. The wood of the tree is prized, as it is hard and
sturdy and thus often used to build ships.

The ripe nuts are often crushed to a fine powder, and then added to drinks and
foods. It is highly addictive, and has a calming effect on those who consume
it, and it makes them lethargic, relaxed and laid back. It numbs the senses,
as well as any pain and is thus often used as a battle field pain relief
medicine.

It is one of the most commonly available, as well as one of the cheapest drugs
available, as the tree itself grows in large numbers and each tree contains
hundreds, if not thousands, of nuts.

\begin{35e}{Xoridina}
  \srditem{Addiction rating}{Medium}
  \srditem{Satiation Time}{5 days}
  \srditem{Initial Effect}{2 dexterity damage}
  \srditem{Secondary Effect}{The user gains DR 5/- for 1d2 hours}
  \srditem{Overdose}{Those that take this drug more than once in 24 hours must
    make a separate save (DC 20) or fall asleep for 8 hours.}
\end{35e}


% Everblack has its own file, due to size
\subsection{Everblack}
\label{sec:Everblack}

There is one thing that is unique to the world of Aror: \emph{Everblack}. It is
a pitch black crystal, resilient as adamantine but harder to mine and work than
the metal.

It is found in small quantities all over the world of \hyperref[sec:Aror]{Aror},
but especially in deep soil and embedded in the stones of the mountains. Some
places are richer in everblack than others, and entire economies are built
around mining the metal. For example the dwarves \nameref{sec:Kesmar} mine the
\nameref{sec:Cnamh Mountains} for everblack, and then sell most of it to the
other city nations. Another large quantity has been found beneath the city of
\nameref{sec:El-Fayam}.

\subsubsection{Mining}

It can be mined easily, as untreated and unheated it is rather brittle. However
the everblack dust that is whirled up during the process is highly toxic when
inhaled. This makes mining the brittle crystal rather dangerous for all miners
and workers involved. Early symptoms include coughing, temporary blindness,
dizziness and diarrhoea. Prolonged exposure can lead to a bloody cough,
permanent loss of the ability to perceive colours, perceived symptoms of
hypothermia, such as being cold, shivering and blue limbs and lips and an
increased risk of heart failure. Very few miners risk working these mines
voluntarily, and thus either enslaved labour or only work these mines for very
high pay voluntarily.

\subsubsection{Arcane Battery}

Everblack is capable of holding and storing magical power, and is also able to
release it in a controlled matter. This makes everblack invaluable in arcane
and divine research, as well as making arcane machinery and artefacts. all
artefacts, wands and even scrolls made on Aror have trace amounts of
everblack, that holds the arcane or divine energy required to make these
magical devices work and function. Everblack is capable of holding arcane,
divine, psionic and even soul energy making it highly sought after all around
the world.

Charged chunks and pieces of the crystal are embedded in magical weapons and
armour, as well as wands and other divine and arcane artefact.

A charged everblack crystal or charged composite everblack is warm to the touch.
It gets warmer and warmer the more power is stored within it. If it is charged
to its capacity it will begin to glow in a low, and orange light. Once charged
beyond its ability to safely store power it will begin a low droning hum and
vibrate. Everblack that is overcharged will explode, shattering the crystal to
dust and damaging everything and everyone caught in the explosion. However if
an overcharged crystal is left alone it will release excess magical power in
the form of light and warmth until it returns to maximum capacity.

The excess storage capacity of everblack is very high, and even small
everblack shards require almost three times the power that would make them
full to cause an explosion. The explosion of a small shard is barely enough to
damage a normal sized humanoid. Although highly expensive, everblack is
sometimes fashioned into bigger explosive devices with devastating results.

\aren{Making everblack explosives is like making catapult ammunition out of
  platinum.}

\subsubsection{Everblack Ink}

Everblack Ink is made by crushing the crystals, mixing them with water and
boiling the resulted mixture down. It is used in the inscription of magical
scrolls, as well as runes and seals. Everblack ink is poisonous if consumed
directly, and one must be careful to avoid prolonged exposure of everblack
ink as it can be absorbed through the skin.

\subsubsection{Power Dampening}

An area filled with natural or artificial everblack crystal acts as a magical
dampening field. Such areas impose a natural and environmental potential for
spell failures upon everyone who seeks to cast spells within them. The
crystals redirect the magic and absorb it, often nullifying the power.
Devices and places are often fashioned deliberately out of everblack, so that
powerful witches and wielders of the arcane arts can be robbed of their power.
A direct application of this power dampening power of the everblack crystals
are \hyperref[sec:Null Stone]{null stones}.

\subsubsection{Composite Everblack}
\label{sec:Composite Everblack}

The crystal itself can also be hardened to incredible strengths by melting
everblack in a blast furnace to remove impurities, and then adding trace
amounts of carbon and iron. This everblack alloy, known as \emph{composite
everblack}, is then harder and denser than adamantine. This alloy does not lose
the ability to store magic, and can be used to build larger everblack crystals
and structures, as well as golems and everblack weapons.

Working with already smelted composite everblack bears no risk of poisoning the
smith. However the smelting process releases gases that are toxic if inhaled
or absorbed through the skin.

\begin{35e}{Composite Everblack as a Material}
  Weapons, armour, and shields can be fashioned out of composite everblack.
  Weapons, armour and shields made out of the composite have half more hit
  points than normal, and 50 hit points per inch of thickness as well as
  hardness 30. Composite everblack materials are always costly enough that all
  weapons, armour and shields are always made of masterwork quality.  Only
  weapons, armour and shields normally made of metal can be fashioned from
  composite everblack.

  Light armour made out of composite everblack grant a spell resistance of 14
  while worn, but costs 10.000 gp more, medium armour grant a spell resistance
  of 16 while worn but costs 15.000 gp more, while heavy armour made out of
  composite everblack, grant spell resistance 18 while worn but costs 20,000
  GP more.

  Any shield made out of composite everblack can be called upon to store
  spells that would normally target the wearer of the shield. An composite
  everblack light shield can store up to two spell levels of spells but cost
  2000 gp more, a heavy shield can store up to four levels of spells but costs
  4000 gp more, and a composite everblack tower shield can store up to 6 spell
  levels but costs up to 6000 gp more.

  Bludgeoning weapons which have a heavy steel head (such as maces), can have
  their head made out of \emph{composite everblack}, and are then especially
  effective. Their damage dice they make are then doubled. For example a
  \emph{Heavy Composite Everblack Mace} does \emph{2d8} damage instead of
  \emph{1d8}.

  Any weapon made out of composite everblack can store one spell of spell
  level three or lower within itself. Upon the next successful attack with
  that weapon the spell is released upon the target, as if it were cast on
  the target of the attack.
\end{35e}

\subsubsection{Everblack Golem}
\label{sec:Everblack Golem}

Even though everblack consumes all arcane power, it is possible to construct
golems out of composite everblack. The technique of their construction is a
closely guarded secret, known only to a few highly skilled golem engineers of
\nameref{sec:Stenheim}. They are marvels of engineering and arcane wonders but
are expensive to make. They absorb all magical energy directed at them, and
can then release it when they strike their attackers.


\section{Food, Drink, And Lodging}
\label{sec:Food}

Food, drink and lodging is offered in most towns and even small hamlets by
the local inn. One can expect to find inns and taverns almost everywhere on
Aror, where most of them cater to a specific demographic. It is not uncommon
to find slave only taverns, as well as taverns where slaves or even monstrous
races are forbidden from entering.

Most inns serve what is called a never-ending stew, a broth that has been
cooking for several weeks, if not months and is replenished in the morning
with fresh ingredients. It is served with bread and watered ale. Most inns
also serve a cold platter of dried meat, bread and cheese. While the southern
regions and \nameref{sec:Tredegar} have a long tradition in making and
enjoying tea, while the northern regions prefer coffee. Coffee culture, with
lots of small coffee shops that also serve food and snacks are popular
especially in \nameref{sec:Forsby}.

Slaves that are not regularly feed by their owners often make their own
secret taverns and inns. These then serve food scraps, water and other low
quality food for free to other slaves. Although these inns and taverns are
often illegal, they are more often than not tolerated and sometimes even
receive food from priests and temples that seek to aid those in need.

A common low quality slave food is a white thick paste made out of nuts,
roots, fat and law quality meat such as rat or pigeon. It is eaten with
stale or old bread, and has many different names such as ``bird butter'',
or ``dead man's shoe''.

\nameref{sec:Ramesk} is bought in a water skin or vial, and lasts a vampire
for five meals.

\begin{table*}
  \captionsetup{labelformat=empty,font={large,bf},position=top}
  \caption{Food and Drink} \label{tbl:Food and Drink}
  \rowcolors{1}{white}{light-grey}
  \begin{tabular}{p{10cm} l}
    Dead Man's Shoe             &  1 shin \\
    Ramesk                      &  6 shin \\
  \end{tabular}
\end{table*}

\section{Slaves}
\label{sec:Slave Prices}

The prices for slaves varies largely per region. They are most expensive in
regions that do not actively engage in slavery on a massive scale, and are
generally cheaper in the slaving nations. Basic prices for slaves start at
around 50 shards, and then increase or decreased based on economy and
condition of the slave.

Most slaves are bought for menial labour, so healthy and capable workers are
the most common type of slave being sold. Slaves that are wounded, sick, or
otherwise usually cost less, while exotic and specially trained slaves cost
more.

\begin{table*}
  \captionsetup{labelformat=empty,font={large,bf},position=top}
  \caption{Slave Prices} \label{tbl:Slave Prices}
  \rowcolors{1}{white}{light-grey}
  \begin{tabular}{p{10cm} l}
    Humanoid male slave             &  50 shards \\
    Slavery is common in the region & -10 shards \\
    Slavery is rare in the region   & +20 shards \\
    Slave's race is exotic          & +20 shards \\
    Sick or otherwise impaired      & -20 shards \\
    Expertly skilled                & +10 shards per CL or HD \\
    Physically Attractive           & +50 shards
  \end{tabular}
\end{table*}

\section{Services and Spellcasting}
\label{sec:Services}

The various institutions on Aror offer certain services to anyone who has the
shards to pay for them. Most institutions, churches and orders have divisions
in most major city kingdoms, as well as smaller outposts in smaller baronies
or towns.

\textbf{Church of \nameref{sec:Forun}} usually offers shelter, housing and
lodging for the downtrodden, as well as regular medicine and healing for those
that cannot afford to pay a cleric to treat wounds.

\textbf{First Order} often runs churches devoted to the \nameref{sec:Order},
and their priests offer services as judges to settle disputes or act as
mediators in diplomatic meetings. Their churches and holy sites are often
used as neutral ground by warring factions that seek to reconcile through
negotiation and diplomacy. The \textbf{Second Order} offers their libraries,
scholars, and researchers for anyone that seeks knowledge. While the
\textbf{Third Order} are sent if criminals have to be caught, justice has to
be served, or vile and evil creatures have to be captured or destroyed.

Most large city kingdoms have a dedicated wizards guild, or arcane academy
that sells scrolls, wands, and magical artefacts. It is one of their main
sources of income, and they often hire dedicated arcane smiths that produce
artefacts and magical items specifically to be sold through their stores.
Wizards of these arcane schools also offer their spell casting abilities to
anyone who can afford them. Most of these academies of the arcane arts also
house well stocked libraries, reading rooms and places to study and conduct
arcane and historical research.

\section{Transport}
\label{sec:Transport}

There are a wide variety of different modes of transportation available on
Aror. All large city kingdoms are situated near the sea, and thus have large
ports and shipping enterprises that ferry people and goods all over the
world. Transportation over land is largely done through horses and carriages.

Ever since \hyperref[sec:Dragon Teleporter]{dragon teleporters} were installed
to connect most major city kingdoms with each other, travelling became cheaper
and easier. A ticket for a dragon teleporter costs between \textbf{15 and 20
  shins}.

%% Weapons
\section{Weapons}
\label{sec:Weapons}

The people of Aror are exceptional sword makers, and thus there is a wide
variety of new weapons that you will find on offer at the smithies. Although
spears and halberds are the most common weapons used in warfare, swords are
still with both professional soldiers and civilians alike.

\begin{table*}[!htb]
  \captionsetup{labelformat=empty,font={large,bf},position=top}
  \caption{Overview of Weapons}
  \rowcolors{1}{white}{light-grey}
  \begin{tabular}{p{2.5cm}     l               p{1cm}             p{1cm}             p{1.5cm}            p{2cm}          p{3.0cm}}
    \textbf{Martial Weapons} & \textbf{Cost} & \textbf{Dmg (S)} & \textbf{Dmg (M)} & \textbf{Critical} & \textbf{Weight} & \textbf{Type} \\
    \multicolumn{7}{l}{\large{\textbf{Light Weapons}}} \\
    Basler       & 1 shard (10 gp) & 1d4  & 1d6  & 19-20x2 & ~1kg (2lbs)   & finesse, piercing or slashing \\
    Rondel       & 3 shard (30 gp) & 1d6  & 1d8  &    20x3 & ~0.8kg (1lbs) & finesse, piercing \\
    \multicolumn{7}{l}{\large{\textbf{One-Handed Weapons}}} \\
    Arming Sword & 2 shard (20 gp) & 1d6  & 1d8  & 19-20x2 & ~1.1kg (2lbs) & slashing, or half-swording, or mordhau \\
    Long Sword   & 4 shard (40 gp) & 1d8  & 1d10 & 19-20x2 & ~1.4kg (3lbs) & two handed, slashing, or half-swording, or mordhau \\
    Side Sword   & 3 shard (30 gp) & 1d4  & 1d6  & 18-20x2 & ~1.1kg (2lbs) & slashing, or piercing \\
    \multicolumn{7}{l}{\large{\textbf{Two-Handed Weapons}}} \\
    Estoc        & 5 shard (50 gp) & 1d10 & 2d6  & 18-20x2 & ~2.0kg (4lbs) & finesse, one handed, piercing \\
    Man Catcher  & 2 shard (20 gp) & 1d4  & 1d6  & 20x2    & ~1.8kg (3lbs) & non-lethal, piercing, grapple \\
    Zweihänder   & 5 shard (50 gp) & 1d10 & 2d6  & 19-20x2 & ~2.5kg (5lbs) & slashing, or half-swording, or mordhau \\
  \end{tabular}
\end{table*}

A \emph{basler} (also baselard, Katzbalrger or short sword) is sword with
shorter blade, between 40 or 60 centimetres. It is often carried as a side arm
by soldiers who would use a pole arm or a ranged weapon as their main
armament.

A \emph{Rondel} dagger (also misericorde, or stiletto) is a long knife or
dagger with a sharp and hardened tip. It is mainly used against knights wearing
full plate armour, either in the heat of a grapple or as a mercy kill to a
seriously wounded knight. The weapon is commonly used as a side weapon for
knights and soldiers, but also favoured by many assassins.

An \emph{arming sword} is a one handed sword with a blade length between 70
and 80 centimetres. It is the main armament of many knights (and thus often
called ``knightly sword''). It can be used to do half swording, or a mordhau.

The \emph{long sword} (or bastard sword) is a sword that can either be wielded
one handed, or two handed. It usually has a blade length of 80 to 110
centimetres and is the favoured weapon of many advanced swordsmen and -women,
as well as veteran soldiers and mercenaries. Much like the arming sword it
can be used for half-swording or performing a mordhau.

A \emph{side sword} is a transitional sword between an arming sword and a
rapier. It has both a wider blade and an edge for cutting, and a hardened
point for thrusting. This kind of sword is very popular in cultures with
extensive focus on fencing, such as \nameref{sec:Avenfjord}.

An \emph{Estoc} is a long two handed sword that is used for thrusting. Many
estocs do not have an edge, and its primary role is to defeat full plate
armour. It is often used with two hands, although it can also be wielded with
one.

The \emph{Man Catcher} is a long pole weapon with two metal prongs at the end.
Some have a spring-based contraption that allowed the limbs or necks to enter
the prongs, but not leave it, while others have a rope based contraption that
allows the wielder to tighten a noose. The main purpose of the weapon is to
catch humanoids or humanoid-like monsters without seriously harming them. Most
man catchers also feature sharp edges or spikes to allow the wielder to coerce
the victim into moving. They are mostly used in a law enforcement role to
catch violent criminals, transfer prisoners or subdue feral animals. Man
Catchers also see use in castle defence where they are especially useful
against attackers that scale castle walls. Slavers also favour this weapon to
catch, subdue and move their victims.

\begin{35e}{Man Catcher}
  A man catcher can deal non-lethal damage without the wielder having
  incurring a -4 penalty to his or her attack roll. After a successful hit
  with a man catcher, the attacker can opt to grapple his opponent (with a +4
  bonus) as a free action, without needing to make a touch attack or provoking
  an attack of opportunity from the defender.
\end{35e}

A \emph{Zweihänder} (also bihander, great sword or claymore) is a huge sword,
between 1.4 and 2 metres, that can only be wielded with two hands. It is a
dedicated weapon of war, and is often used by front line troops to break
through shield walls defended by pole weapons. Much like its smaller cousins
it can still be used with half-sword techniques, and to perform a mordhau.

\emph{Half-Swording} refers to a technique in which the off hand grips the
central blade (or just beyond the cross guard) to allow for greater force and
accuracy while thrusting. The main purpose of this technique is to defeat
armoured opponents who would otherwise be well protected against cuts.

The \emph{Mordhau} (literally ``death hit'') is a technique where sword is
held inverted with bonds on the blade, and striking the opponent with either
the pommel or the cross guard. This technique allows the wielder to use the
sword to deal blunt damage to an opponent. It is especially effective against
opponents wearing strong armour (such as thick natural armour, plate) that
would otherwise be well protected against cuts.

\begin{note}
  Half-swording attacks deals piercing damage, and a mordhau attack deals
  bludgeoning damage. You can switch between normal stance, half-swording, and
  mordhau stance as part of a move action. If you have the \emph{Quick Draw}
  feat, you can move between these stances as a free action.

  Mordhau deals less damage, as if the sword was one size category
  smaller. For example a medium sized long sword would deal 1d8 points of
  bludgeoning damage, compared to 1d10 points of damage when cutting.
\end{note}

