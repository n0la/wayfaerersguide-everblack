\subsection{Eigyr}
\label{sec:Eigyr}

Eigyr Waylin (born circa GT:-550, and died circa GT:-320) was a high elven
warlord, politician, shamaness of the \nameref{sec:Old Ways} that rose to
prominence in her campaign against the monstrous races on \nameref{sec:Eilean
  Mor}.

Her history, deeds, accomplishments but also most of her defeats and failings
are well documented in the stories, songs and traditions of the Old Ways. Within
humanoid traditions and culture of both the Eilean Mor, and the northern part of
\nameref{sec:Goltir} she is universally celebrated as a heroine, champion and
liberator. Her fame also stretches across the continents, and her tactics and
teachings have also been extensively studied by the scholars of Arania and
Avenfjord. Eigyr's fame turns into notoriety among the monstrous races, who see
her as conqueror and defiler of their lands.

In her early years she was an accomplished huntress, archer, and wielder of
two blades in each hand, and fought the monstrous races by leading small war
bands, and employing mostly hit-and-run tactics. Through many early victories,
and skilled diplomacy she rallied more and more villages and tribes behind
her. Soon she had a formidable army, and began to assault, and besiege larger
monstrous towns. Eigyr lost many of these earlier large scale battles, as she
failed to adapt tactics, and logistics from the small skirmishes she knew well,
to tactics and logistics required of a large and vast armies. Her pride and
bullheadedness are often attributed to those seemingly needless losses.  After
appointing new advisers, and generals to lead her ever growing armies the winds
turned again in her favour. She is remembered as a woman who fought with
honour and courage in battle, never taking lives of the innocent (be they
beast or men) and actively punishing her men and women if they committed war
crimes against the innocent.

Eigyr is described by many stories as a wise, natural, honour-bound and
charismatic leader, but also as a spiritual purist. She detested the worship
of lesser deities, especially the \nameref{sec:Three Kings} but also those of
\nameref{sec:Lor}, and \nameref{sec:Griannar}. The stories tell, that she often
refused to aid those that worshipped ``false gods'' or those that she felt were
without honour.

Her campaigns ended around GT:-390, after she had defeated all major monstrous
cities on Eilean Mor, and driven most monstrous tribes across the northern sea
to \nameref{sec:Iafandir}. Eigyr ruled wisely and justly over her newly formed
empire in the centre of Eilean Mor until her death. In her final years she
also often travelled her empire, telling her own story to the people. Eigyr,
becoming wiser in her years, never failed to mentioned her failings and lost
battles, urging people to learn from them. She especially regretted the
needless losses in the early sieges, brought on by her inability see that she
could not lead and oversee the grand army all by herself.

Many towns, villages, cities and even \hyperref[sec:Wayfaerers
  Guild]{organisations} are named in her honour, although her empire fell
apart soon after she departed. Most large humanoid city kingdoms and some
baronies of Eilean Mor owe their founding to her legacy. Also many of the
newer traditions of the \nameref{sec:Old Ways} are directly based on her
teachings, and retelling of her own story in her later years.
