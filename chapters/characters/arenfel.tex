\subsection{Aren Fel}
\label{sec:Aren Fel}

Aren Fel of \nameref{sec:Nicte}, most commonly known as the ``Undying Witch''
(born sometime around GT:24) is a \hyperref[sec:Deepkin]{Deepkin}
\hyperref[sec:Soul Magic]{soul witch}, priestess of the \nameref{sec:Silent
  Queen}, thief, diplomat, author, historian and scholar.

Her family tree, blood line, birth and association with House Nicte have been
thoroughly documented and dated to GT:24. She was born to small deepkin
commune, which now lies within the kingdom of \nameref{sec:Stenheim}. And even
though she possess new female deepkin bodies to stay alive over the aeons, her
continued longevity as been independently confirmed by several other
long-lived species, including vampires, dragons and even giants, who were able
to recognise her unique soul across the centuries.  This is also why her
appearance is not described, as it would be a fruitless labour.

Initially born into the Nicte family clan, she was trained preliminary as a
diplomat, thief, spy and trader. As the battles of the aeon of strife came
closer to her family, she was also hastily trained in combat, and joined the
skirmishes as an archer and scout. She suffered a major trauma and wound during
those skirmishes, which also \hyperref[sec:Soul Awakening]{broke her soul}.
After her clan realised her condition, Aren was sent off to be healed by the
\nameref{sec:Walburga} witches. Aren took the opportunity to learn as much as
she could about soul magick from them, including how to possess other people
or bodies. Although she was inducted into the coven as a witch, she was
ultimately asked to leave due to her allegiance to a lesser deity. Aren still
uses this power by possessing newer, younger bodies to prolong her life.
Although the races she inhabits now vary, she always remains female.

During her unnaturally long life-span she used her extensive
training from her Deepkin clan to forge alliances, gather artefacts,
manoeuvre political developments, and to manipulate the rich and powerful.
She consolidated many smaller baronies, bolstered their political and military
power, and directly steered many of them to join or start wars that would see
an end to the strife on Goltir. Aren achieved this goal over the course of
several centuries by holding high ranking offices, such as adviser to the baron
or baroness. She was thus instrumental in the early history of the
north-western part of \nameref{sec:Goltir}, by making sure the core humanoid
settlements remained allied with each other. Aren was instrumental in paving
the way for both Stenheim and \nameref{sec:Forsby} to rise to be global
powers.

She was also heavily involved in the \nameref{sec:Holy Crusade}. There she
managed to lessen the extend and cruelty of the crusade by actively opposing
the persecution of low-ranking followers of Griannar, including the general
congregation and low-ranking acolytes. Aren directly stood against her own
high-priestess \nameref{sec:Aria}, causing a schism within the church of the
\nameref{sec:Silent Queen}. Between the warring faction following Aria on one
side, and the more neutral, and pacifist congregation that followed Aren, on
the other. Aren further aided \nameref{sec:Hraglund} during the plague, and
assisted the kingdom in finding a cure by securing the help of important
arcane institutions including the \nameref{sec:Hall of Knowledge} and
\nameref{sec:Magistrata Arcanum}. The last recorded appearance of Aren was
during the siege of Forsby, where she smuggled \nameref{sec:Everblack Golem}
made by Stenheim into the city to help with the defence against the devils.

Although her achievements are well recorded, so are the means by which she
attained them. Aren is notorious for her Machiavellian scheming, and often
pursuing careless and reckless plans, in which she prioritises a quick and
satisfactory end of the crisis at hand above anything. Her philosophy that the
ends justify the means, has lead to her to commit blackmail, theft, slavery,
and even murder. She is often accused of showing a blatant disregard for the
lives of others, for example by killing the souls of future hosts simply to
extend her own lifespan, or by deliberately starting wars during the
strife. Her many, and well documented crimes, has made her an enemy of many
judicial organisations, such as the church of \nameref{sec:Lor}, or the church
\nameref{sec:Order}. She is furthermore a persona non grata in four city
kingdoms: Fes al-Bashir, Forsby, Hraglund, and Tredegår.

\label{sec:Witch Hunt}
The power she wields, both literally as a powerful soul witch, and
figuratively through her web of alliances, caused a lot of animosity and open
hostility from other powerful organisations. In MI:20, after the Holy Crusade,
the \nameref{sec:Knight Order of Tavos} attempted to catch, trial and execute
Aren. A literal witch hunt began, that lasted from MI:20 to MI:28 that caused
many innocent Deepkin women harm, caused them to lose their freedom and even
their lives in an attempt to bring Aren to justice. This is period of time is
now known as the ``Witch Hunt'', and is considered the sole fault of the Knight
Order of Tavos.

Nevertheless her knowledge of the world, its inhabitants, the creatures and
threats of the planes, arcane, and soul magic is extensive, as is her web work
of power, alliances, and ability to influence even the most powerful
organisations. As a member of \hyperref[sec:Two Courts]{Court of the Suns},
she is very often asked for aid, when a global crisis afflicts Aror.

\begin{note}
  Aren can serve many purposes in your campaign: she can be the secondary
  villain, prime villain, or simply be an aid to the party, preparing them
  to face the actual villain. She will \emph{never} work towards the doom of
  Aror, or its citizens on purpose, but will use \emph{whatever means
    necessary} to achieve goals. Even though these goals may overlap with
  that of a good PC party (i.e. saving the world), her methodology and
  philosophy make her an evil character in terms of D\&D alignment.
\end{note}
