\clearpage
\incgraph[
  overlay={\node[black] at ([xshift=-10cm,yshift=2cm] page.south east)
    (main)[text width=0.9\paperwidth]{
      \large \centering
      \textbf{``Two know-it-alls arguing about what should be in this book, and what should be left out.''}
    };
    \node[black, below=of main,yshift=1cm]{
      Hall of Wisdom, \nameref{sec:Fes al-Bashir} circa GT:2103
    };
  }
]{media/authors-a4.png}
\clearpage

\subsection{Graham Balance}
\label{sec:Graham Balance}

Graham Balance, born GT:2084 somewhere in the Dirgewood, died GT:2139 in
\nameref{sec:Fes al-Bashir}, was one of the great polymaths of early Arorian
history. He was an accomplished writer, musician, historian, philosopher,
arcane wielder, politician and grand magus of the hall of knowledge.

As a young adult Graham travelled around the world, from his home in the
Dirgewood across the sea to Goltir, all the way down south, across Farlar, and
then met the dragons of Draigynus, before finally settling Arania. On his
pilgrimage across the world of Aror he penned the most popular, and most
widely printed and copied book on all of Aror: the ``Wayfaerer's Guide to
Everblack''. He also penned most of his other popular works, specifically songs
and poetry, both the originals, and those he collected from various cultures
around the world, on his year long journey. His songs are still sung today,
and his rhymes and words are still recited to the next generation so they are
not forgotten.

After arriving in \nameref{sec:Fes al-Bashir} he joined the \nameref{sec:Hall
  of Knowledge}, were his broad knowledge of the world earned him a teaching
position. As a professor for cultural history, ancient societies and languages
he taught several student generations, before deepening his understanding of
arcane magic, philosophy and science. It was at the academy where he wrote his
greatest work: ``The History of Divine Form'', which is a detailed treaty and
look on the old religions, true deities and their teachings. It also put forth
a hypothesis: that if enough people believe in a true deity, one might be
``willed'' into existence. It is in this book were he founded a new religion
and dogma, with another true deity at its centre: the \nameref{sec:Sea
  Priestess}.

After about fifteen years of professorship he rose to be the Grand Magus of
the Hall of Knowledge, and thus, in turn, became the most powerful political
figure in Fes al-Bashir. He ruled the city and the Hall of Knowledge until his
death in GT:2139. Although his reign was short, he moved both the Hall of
Knowledge away from worship of lesser deities, and towards the sciences. And
although internal power struggles within the cities prevented him from
enacting lasting change in the city, he is still remembered as a wise,
tempered, and just ruler of the city.

Graham balance is mostly remembered when he was Grand Magus. As a man in his
advanced years, short hair, a beard that reaches down to his chest and covers
most of his face. He had sharp blue eyes, a long elongated face, and was
decently handsome. His charisma stemmed from his sharp and eloquent wit, as
well as his commandeering presence.
