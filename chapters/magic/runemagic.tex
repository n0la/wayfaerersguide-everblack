\section{Rune Magic}
\label{sec:Rune Magic}

Rune magic is a perverted form of arcane magic that is taught by the mysterious
\hyperref[sec:Devils]{devil} called the \nameref{sec:Runemaster}. He teaches it
to any mortal he deems worthy to wield that power - i.e. is evil enough to go
through with the ritual sacrifice required to create runes.

It draws upon the souls of the living to fuel arcane and divine runes carved
into the caster's skin. These runes are mostly passive in nature, providing a
constant beneficial protective effect to whoever wears them. The magical
benefit stops only once the rune is destroyed, and are thus highly sought
after by anyone seeking lasting and permanent arcane protection.

Rune magic is taught directly by the Runemaster, or his minions, or learned
from a book drafted by the Runemaster, called the \emph{runic lexicon}. The
rituals to craft these runes all require living sacrifice, and are thus
forbidden in almost all city stations, nations and baronies.

Runes are carved into the flesh of the wearer with a ritualistic knife, and
thus permanently scar and deform the wearer's skin and flesh. Some runes
become rather huge patterns of intricate forms, shapes and lines, limiting the
amount of runes that may be applied at any given time to a body. The
ritualistic knife must first be hallowed in the blood of a living humanoid
sacrifice, that is dedicated to the Runemaster himself. If he deems the
subject willing, he will bless the knife, and then teach rune magic.

\graham{Are you going to teach Runemagic in my book?}

\aren{Hell no. But I thought it wise to include just enough information to be
  useful in identifying Runemagic should our esteemed readers encounter it.}

The runes themselves are then carved into a living sacrifices skin, often in
delicate intricate patterns spanning the entire body and skin, accompanied by
secondary sacrifices, chanting and the recitation of abyssal incantations. A
smaller version of the rune is then carved into the casters skin, and then the
sacrifice is killed with the ritualistic dagger in the name of the
Runemaster. If all is done correctly, the soul power of the slain sacrifice is
then used to power the rune's magical effect on the wearer.

Rune magic is often used by evil arcane and divine casters, who are already
engaged in living sacrifices (for example for necromancy, or to appease other
evil creatures), those who cannot afford magical items, or those who cannot
cast spells themselves. No arcane or divine knowledge or spell casting ability
is required to create and carve runes.

Runes of rune magic are permanent, and cannot be healed through divine magic.
They can be destroyed however by using the ritualistic knife to destroy the
pattern. Normal wounds that happen to destroy the skin or the runes (i.e. from
sustaining a cut in battle), do not destroy rune magic.

\begin{35e}{Runemagic}
  Any cleric or arcane spell level 4 or below that could be cast upon yourself
  as the wearer of the rune, can be used in a rune magic ritual. It requires a
  blessed dagger, with which the ritual \nameref{sec:Summon Runemaster} must
  be performed.

  Runemagic requires that a large special rune must be carved into the skin of
  a living humanoid creature, with HD equal or higher to \emph{2 x spell level
    - 1} which takes \emph{spell level} hours to complete. The wearer must
  complete a \emph{Craft (Rune)} check with DC \emph{10 + caster level of
    spell} every hour or fail with the crafting of the rune. Once failed, the
  caster has to start over with a new sacrifice. If successful then a smaller
  rune must be carved into the wearer, and the humanoid creature must be
  killed with the blessed dagger to convey the benefits of the spell to the
  rune. The killed sacrifice can no longer be resurrected unless with a
  \emph{Resurrection, Greater} spell.

  A medium creature is limited to 3d4 runes on his body, a small creature to
  2d4, and a large creature to 4d4 respectively. Roll this value at the first
  rune to see how many more may fit onto the body.
\end{35e}
