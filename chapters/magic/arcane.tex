\subsection{Arcane Magic}
\label{sec:Arcane Magic}

Those deities that give power to their clerics, and priests cannot do so with
perfect efficiency. The process of granting, and using divine powers leaks
magical energy which remains trapped on Aror. This magical energy can be
harvested, shaped, and channelled into spells by those that study the craft of
\emph{arcane magic}. Due to its versatile nature, and its inherent
independence from a higher power, arcane magic is considered more powerful
that divine magic but also exceptionally difficult to study, hard to
understand, and dangerous to wield.

Arcane Magic of \emph{all} forms require years to learn, and wield even at 
the most basic levels. Any and all arcane wielders (even those that use``
inherent'' arcane magic like bards and sorcerers) are usually 10 to 20 
years older their divine or martial counterparts to make up for the years 
spent training, and learning. Unless of course they one of the rare 
\hyperref[sec:Graham Balance]{child prodigies}.

Arcana is associated with the colour red, and there is no ``evil`` or ``good``
arcane magic, as all arcane magic is neutral.

\subsection{Summoning}
\label{sec:Summoning}

The \emph{summoning} and \emph{calling} sub schools of arcane magic are
related with bringing extra-planar creatures to or from Aror. Due to the
existence of a device called the \nameref{sec:Monolith} these sub schools of
magic are harder to perform and practice than other schools of magic. Since
summoning devils, or even \nameref{sec:Demons} is incredibly dangerous, many
cities ban these schools of magic in their entirety.

Any extra planar creature summoned to Aror will fade away to the soul well
just like any other living being residing on Aror. This has made Aror a very
unpopular destination for many more powerful extra-planar species, as they
will not be sent home upon defeat, but risk permanent death at the hands of
the soul well. Further details about how this affects summoning magic can
be found in the book's section about the \nameref{sec:Monolith}.

\subsection{Necromancy}
\label{sec:Necromancy}

Necromancy is the art and craft of manipulating both body and soul to achieve
a purpose. In many cases the craft destroys the soul, leaving only a soulless
husk behind, while in others it does exactly the reverse. The art of
necromancy was discovered when \nameref{sec:Morana} turned some of her
followers to vampires, and has since been excessively studied by scholars,
priests and wizards. \nameref{sec:Isamir} also gave the power of necromancy to
the \nameref{sec:Inua} who closely guard the secrets of their rituals, spells
and incantations.

Liches, and \nameref{sec:Vampires} are the epitome of applied necromancy, and
many scholars have spent millennia studying them to better understand the
craft. While many necromancers wish only to study the vampire to better help
them survive, much like a doctor would for the living, others use the powers to
do evil, creating vicious and horrid creatures to do their bidding. Necromancy
is thus outlawed in many regions, and cities, requires oversight, or a special
permit.