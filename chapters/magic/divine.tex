\subsection{Divine Magic}
\label{sec:Divine Magic}

Divine Magic comes directly from a deity, a deity representing a concept, or a
powerful individual, such as \nameref{sec:Daemons} or \nameref{sec:Devils}. In
the case of lesser deities, the deity in question grants that power, in the
understanding that the power is used to further the deities interests, and
goals. Once that power is granted, it cannot be revoked by the lesser deity
until it is used, or lost. True deities do not exist as living beings, being
concepts given power, and thus work differently. A priest that follows a true
deity draws strength and power from the concept said deity represents. For
example a priest of the \nameref{sec:Order} draws strength from an inner desire
to create order out of chaos, and from defeating the evil that may stem from
chaos. A priest of a true deity may lose their power should they stray too
far from their true deities concepts, ideas and ideology.

Divine magic is heavily associated with the colours white (good and neutral
divine magic), and black (evil divine magic). Evil divine magic is not
corrupted by the wielder, instead it is already a potent force of corruption.

\subsubsection{Resurrection}

Resurrection magic works differently on Aror. Upon death all souls dilute into
the soul well, coming apart by the seams until they can no longer be recovered.
Much like you cannot recover the exact same water particles once you have
poured them into the ocean. The soul of a recently deceased can only survive
this dissolution of the self if another powerful soul of the well intervenes.
While some powerful souls may do so to further their own reasons, many will only
do so if they see a benefit for themselves. So those that wish to cheat death
will have to make a deal with a powerful daemon.

\begin{35e}{Resurrection}
  Any resurrection or reincarnation magic only works within 1d4 hours of
  death. After that resurrection magic only succeeds if a powerful daemon
  wishes for it to succeed.
\end{35e}