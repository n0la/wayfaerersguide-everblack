\section{Soul Magic}
\label{sec:Soul Magic}

Soul magic is one of three main pillars of magic on Aror, alongside divine
magic (and its closely related cousin arcane magic) and psionic
powers. It taps into the very soul of the caster, and thus, much like psionic
energy, is inherent to the caster, and does not come from external sources
like divine or arcane energy.

All living creatures on Aror have a soul, that is fundamentally intertwined
with the body. Normally no creature is aware of its own soul, and must first
experience a traumatic event called a \emph{soul awakening}. During the
awakening the inherent bond between the body and the soul are broken, and the
soul is free to be experienced on its own.

Those aware of their own souls, can attempt to manipulate it, and tap into
its vast energy reserves. These reserves, often called \emph{soul power} or
\emph{soul magic}, can then be shaped into useful spells and incantations.

\subsection{Soul Well}
\label{sec:Soul Well}

Every soul on Aror, from every bird, animal, tree, fish and humanoid walking
on its surface, emits a faint soul aura. The energy emitted feeds a vast sea
of soul energy that permeates the world. This omnipresent soul energy field is
called the \emph{soul well}. Its presence is incredibly faint in many places,
but it can vary in strength depending on the presence or absence of life in an
area.

Once a creature with a soul dies, its soul slowly withers and its power leaks
slowly into the soul well until it is gone for good. This is the reason why
many outsiders - whose souls would naturally return to their home plane - die
permanently on Aror. It is also the reason why lesser deities have trouble
retrieving souls of their followers from Aror. This in turn makes it incredibly
difficult to use magic to raise the dead.

The soul well can not only vary in strength, but can also be absent, or even
corrupted. Corruption can occur through necromancy, defilement, excessive use
of divine magic, or through violent events in a particular area. Fields of
battle where thousands have perished, graveyards that have repeatedly been
used for rituals of necromancy, or even areas repeatedly hallowed (or
unhallowed) by divine magic can be places where the soul well is corrupted.
Areas where the soul well is corrupted can lead increased activity of undead
(as the souls cannot return to the well), as well as interruptions to life in
general. Children and young might be born without souls, current souls might
become broken and life as a whole might perish in a given area leaving behind
a desolate patch of land.

It is possible for a soul to donate part of its power to such areas to restore
its connection to the soul well. This art of restoring such broken areas is
called ``preservation''. The reverse is also possible, and a soul may draw
upon the soul well to fuel his own powers, depriving the area of its connection
to the soul well, and thus the life that is connected to it. This power is
called ``defiling'', and is highly controversial.

\graham{``Highly controversial''? Not the choice of words I'd have used.}

\subsection{Soul Awakening}
\label{sec:Soul Awakening}

A \emph{soul awakening} is by nature a traumatic event for both body and soul.
During the awakening the inherent connections that interweave both the body and
the soul are disrupted, allowing each to live without the other. Bodies that
live without their souls are often corporeal undead, such as zombies,
skeletons and \hyperref[sec:Umgeher]{umgeher}. Souls without a body are also
often undead, such as wraiths, ghosts and spirits.

Those that experience soul awakening have their soul broken, and must first
heal before they can tap their souls for power. There are a few ways to heal
a broken soul, such as prolonged rest, intense mental and bodily training, or
specific soul spells.

Awakenings come from traumatic experiences that either affect the body or the
soul. Near death experiences, loss of someone important, necromancy, or being
unintentionally soul broken by another spell caster are some of the more
common ways to awake.

\begin{35e}{Soul Awakened}
Soul Awakened is an acquired template that can be added to any living
intelligent creature that has a soul (referred to hereafter as base
creature). This template cannot be taken, only acquired.
\srditem{Requirements}{The base creature must have had a significant incident
  with souls, soul magic or a traumatic event in the past to be allegible for
  this template. Furthermore the base creature must have a soul to begin with.
}
\srditem{Size and Type}{The base creature’s type and size remains unchanged,
  and retains all of the base creature’s statistics and special abilities
  except as noted here.
}
\srditem{Skills}{An awakened creature immediately adds \emph{Knowledge (Soul
    Magic)} and \emph{Soulcraft} to its list of available class skills. See
  this article on soul magic for details.
}
\srditem{Soul Sight (Ex)}{At will, as a free action, any soul awakened
  creature can turn on and off their ability to perceive an shining aura
  surrounding souls and creatures with souls. This will show absence and
  presence of souls, and if the creature studies a soul aura for three rounds
  it can use a \emph{Soulcraft} check in place of the appropriate Knowledge
  check to deduce powers, abilities, weaknesses and capabilities of a specific
  creature. This works just like the Knowledge skill would when studying
  creatures. If the check against the DC fails, nothing is learned about the
  specific soul and the check cannot be made again on that specific soul until
  new ranks in Soulcraft are gained.
}
\end{35e}

\subsection{Soul Sight}
\label{sec:Soul Sight}

Once awoken to the potential of their souls the first sign of their newfound
powers is the ability to perceive a blue or white aura surrounding souls and
creatures that have souls. Likewise creatures without a soul will have no aura
surrounding them. The strength and intensity of the light depends on the
target soul's strength and prowess.  Learning to suppress this soul sight is
often the first thing a soul awakened learns, as seeing the massive amount of
lights - either from animals and plants in a forest, or from the crowds within
a city - can be overwhelming.

Advanced practitioners of soul magic can also use their soul sight to
carefully study other people's soul. This allows them to deduce some aspects
of the specific person's character, powers and weaknesses. Through intensive
studies of another soul it may become familiar to the awoken, and such an
understanding of a another person's soul can be enhanced and augmented by a
deep emotional connection and understanding of the person. It is not uncommon
for two soul practitioners to recognise each other solely by their soul aura,
regardless of what sort of body their soul might inhabit at a time.

\aren{Staring into the soul of a dragon is akin to staring into the suns.}

\begin{35e}{Soul Sight}
  See the entry in the soul awakened template on how soul sight works in terms
  of 3.5e mechanics.
\end{35e}

\subsection{Free Soul}
\label{sec:Free Soul}

A free soul or spirit, is a soul whose body has already withered away, and for
one reason or the other, does not wish to possess a new body. Being a free
soul cuts you off from most sensory inputs, such as touch, smell, and tactile
senses. Furthermore any free spirit has to expend an enormous amount of mental
strength to remain intact while the soul well siphons their power. This lack
of stimulus and the constant mental stress drives many free spirits insane or
even evil.

There are countless words for free spirits that roam the forests, caves, ruins
or the dark corners of the city. A \emph{mavka}, for example, is a mad free
spirit of a young woman that is sometimes dangerous to children and young men.
Children that have died a horrible death, because they were abandoned by their
parents are called \emph{myling}, and usually cause havoc and destruction on
adults. Free spirits that possess other dead bodies (i.e. not their own) are
called \emph{wiedergaenger}. In many cultures free spirits and ghosts that try
to make themselves known by making noises and sounds are called
\emph{poltergeists}. But generally free souls that roam the living world are
called \emph{spirits}, \emph{Geister}, \emph{shades} or \emph{Cean Gŵla}. Free
spirits that have grown in power, and learned how to remain intact while
returning to the soul well are called \nameref{sec:Daemons}.

Many, but not all, spirits slowly turn mad and crazy. Free souls keep most of
their memories, mental capacity and thus also mostly keep their mentality,
attitude and character traits. Those that do remain sane often become
eccentric, but rarely become threats to the living.

Free spirits will retain their original appearance they had in live, but will
appear translucent and glimmer either in a soft blue or green light. Some
spirits retain ghost or spirit versions of priced possessions, usually
clothing, and jewellery, but also rarely more intricate belongings such as
armour or weapons.

Free souls may still be twisted to evil through necromancy, turning them into
\emph{wraiths}, \emph{spectres}, and \emph{shadows}.

\begin{35e}{Spirit}
  ``Spirit'' is an acquired template that can be added to any living creature
  that has separated its soul from its body and has succeeded its will saves
  (see above). This creature is referred to hereafter as the base creature.

  \srditem{Size and Type}{The creature’s type changes to Undead
    (Incorporeal). It retains any subtype and subtypes that indicate kind. It
    does not gain the augmented subtype. It uses all of the base creature’s
    statistics and special abilities except as noted here.
  }
  \srditem{Hit Dice}{All current and future Hit Dice become d12s}
  \srditem{Speed}{A spirit does not walk, and loses any base land
    speed. Instead it gains fly speed 30ft. - unless the base creature has
    higher fly speed - with perfect manoeuvrability.
  }
  \srditem{Abilities}{Same as the base creature, except that its Charisma is
    increased by +4.
  }
  \srditem{Special Qualities}{A spirit has the same special qualities as the
    base creature as well as those below.
  }
  \srditem{Soul Damage (Ex)}{Touch attacks of a spirit do \emph{soul damage}.
  }
  \srditem{Damaged Soul (Ex)}{Any negative levels, level drain, ability drain
    or ability damage the base creature might have suffered transfer to the
    spirit. They continue to work just the same as they would have on the
    base creature. If the spirit possesses a body these drains and damages
    move over to the new body.
  }
  \srditem{Telepathy (Ex)}{A spirit can speak and hear, and can also
    communicate with others through Telepathy (60 ft.).
  }
  \srditem{Possession (Ex)}{A spirit can make a special touch instead
    instead of a full round action. This attack provokes an attack of
    opportunity. If the touch attack succeeds the spirit may attempt to
    posses the target creature. See under ``Possession'' for further
    information on how this works. If the touch attack fails it may try again
    next round.
  }
  \srditem{Level Adjustment}{Same as the base creature +2.}
\end{35e}

\subsection{Possession}
\label{sec:Possession}

Any free soul that roams around may attempt to possess another living being's
body. To do so it must touch the target, and then begins a struggle in which
the soul and body of the target fight against the hostile takeover. If the
free spirit wins, it destroys the soul of the target and takes over the body.
After the takeover the soul must get used to the new body, which results in
a period of sickness and weakness after the possession.

Very few spirits that have remained sane will attempt to possess another, since
it will result in the death of the target. Forcing them to face a conundrum:
Either being stuck in spirit form slowly draining away into the soul well,
or commit a hideous crime.

\graham{A crime you have always committed freely.}
\aren{If the choice is my life or theirs, I will always choose survival.}

\begin{35e}{Possession}
  A free spirit must make a touch attack against a target in an attempt to
  possess it. If the touch attack fails it can try again next round. Once
  touched the target can make a \emph{will} and \emph{fortitude} save to
  resist possession. If any of these saves succeed, the free spirit may not
  attempt to possess the same target again for 24 hours, and gains a -2
  penalty for any further attempt.

  The DC for the possession attempt is 10 + \sfrac{1}{2} spirit's HD +
  spirit's \emph{charisma modifier}.

  If the size of the spirit and the target differs, the spirit gains a
  \emph{-2 penalty} for each step of size difference. Furthermore if the
  HD difference between the target and the spirit is \emph{4} the spirit gains
  a further \emph{-4 penalty} to the DC. Any additional HD difference above
  4 adds further stacking \emph{-2 penalties} to the DC. If the spirit and
  the target are of different creature types another \emph{-2 penalty} is
  added to the DC.
\end{35e}

\subsection{Soul Power}
\label{sec:Soul Power}

The soul itself is made out of pure energy, called \emph{soul energy},
\emph{soul essence} or \emph{soul power}. It replenishes by itself if used,
much like the body replenishes itself after physical exhaustion. People that
are aware of their own souls can draw that power, direct it, and put it to
work. How much soul power that is available is directly related to the
strength of the person itself. The soul is only as powerful as the combined
trinity of soul, body and personality.

Soul power grows, and stands in direct correlation to the overall power,
strength, and prowess of the being that wields it. Physical strength is
reflected in a persons ability to wield soul magic, just as much as his
character strength, intellect, presence and self confidence. Those ill, sick,
broken in spirit, and wounded mentally find it harder to conjure and tap into
their own souls.

The power that stems from the soul is linked with colour blue. Raw soul fire
is blue, and so are most the visual effects produced by soul power. Soul
witcher and witches often gain visually distinct blue effects - such as
glowing blue eyes, strongly luminescent blue veins or blue crackling, sparks
and flames emanating from their body when they tap into the raw power of
their souls. Soul power colour can also vary by creature, as soul fire cast
by dragons is often white in appearance.

Soul essence is the polar opposite of the power that fuels arcane and divine
magic. Thus casters that also wield arcane and divine magic often find it
difficult to channel soul power, and vice versa. Often arcane and divine
casters avoid wielding soul magic, while soul witches avoid wielding arcane or
divine magic to avoid devastating interference effects when those two sources
of power meet.

Individual soul spells that can be fuelled with soul power, are discussed
later in the book.

\begin{35e}{Soul Power}
  Each creature that has undergone \emph{soul awakening} (see above), gains
  \emph{soul power points}. Once awoken each creature gains \emph{soul power
    hit die} soul power points for every new level (or HD) he or she gains after
  being awoken. Charisma increases the soul power pool, and functions to soul
  power points like constitution works for hit points, granting
  $ \emph{CHA modifier} \cdot HD $ additional soul power points. You gain soul
  power points retroactively for any HD or class levels you might have gained
  before having been awoken to your soul potential.

  The soul power hit die starts at \emph{d6}, but there are several factors
  that may increase or decrease the soul power hit dice steps. The steps are:
  0 (none), d4, d6, d8, d10, d12. These factors stack with each other, and may
  result in a class being fundamentally unable to wield soul magic.

  A fourth level wizard with 14 charisma would gain 1+2 (two steps down due to
  arcane wielder, and half BAB) power points on level up.

  A fifth level fighter with 10 charisma would gain 1d6+0 (one step up due to
  full BAB, one step down due to one strong save) power points on level up.
\end{35e}

\begin{table}[!htb]
  \captionsetup{labelformat=empty,font={large,bf},position=top}
  \caption{Soul Power Hit Die}
  \rowcolors{1}{white}{light-grey}
  \begin{tabular}{p{5cm} l}
    \textbf{Condition}            & \textbf{Steps from d6} \\
    Arcane wielder                & -2 steps \\
    Divine Wielder                & -2 steps \\
    One strong save               & -1 step \\
    Two strong saves              &  0 step \\
    Three strong saves            & +1 steps \\
    Half BAB                      & -1 steps \\
    Three-quarter BAB             & +0 steps \\
    Full BAB                      & +1 step \\
    2 base skill points per level & -1 steps \\
    4 base skill points per level &  0 step \\
    6 base skill points per level & +1 steps \\
    8 base skill points per level & +2 steps \\
    d4 HD                         & -2 steps \\
    d6 HD                         & -1 step \\
    d8 HD                         & 0 steps \\
    d10 HD                        & +1 step \\
    d12 HD                        & +2 steps
  \end{tabular}
\end{table}

\begin{table}[!htb]
  \captionsetup{labelformat=empty,font={large,bf},position=top}
  \caption{Soul Power HD for base classes}
  \rowcolors{1}{white}{light-grey}
  \begin{tabular}{p{5cm} l}
    \textbf{Class} & \textbf{Soul Power HD} \\
    Barbarian      & d10         \\
    Bard           & -           \\
    Cleric         & -           \\
    Druid          & -           \\
    % TODO: Give fighter more skill points
    Fighter        & d6          \\
    Monk           & d8          \\
    Paladin        & -           \\
    Ranger         & d6          \\
    Rogue          & d6          \\
    Sorcerer       & -           \\
    Wizard         & -           \\
  \end{tabular}
\end{table}

\subsection{Soul Fire}
\label{sec:Soul Fire}

Soul fire is uncontrolled soul power that manifests itself in the world as a
cold unfeeling blue fire. Like actual fire it can disintegrate and destroy
everything it touches, but it is fed solely by soul power. If it doesn't burn
anything with a soul (or a soul) it slowly withers and sizzles. Soul fire does
damage to living creatures and souls, and does electricity damage to objects.

\begin{35e}{Soul Fire}
  Soul Fire burns like regular fire, but does \nameref{sec:Soul Damage} to any
  creature that has a soul, and \emph{electricity} damage to objects.
\end{35e}

\subsection{Broken Soul}
\label{sec:Broken Soul}

\aren{The most common way to cure a broken soul, is to journey to the
  \nameref{sec:Walburga} witches, and plead or trade for a cure.}

A \emph{broken soul} is the soul equivalent of a terminal disease. Through
horrible failure in weilding soul powers, or through necromancy the soul is
broken apart and slowly leaks soul power until it dies. And once the soul dies
the creature usually dies along with it. A broken soul can be cured through
appropriate soul powers, or through some divine spells.

\begin{35e}{Broken Soul}
  Every day that a creature spends with a broken soul it must make a DC: 15
  fortitude save or suffer 1 point of charisma and constitution drain. Even if
  the check succeeds the creature takes 3d6 points of soul damage.
\end{35e}

\subsection{Soul Damage}
\label{sec:Soul Damage}

Soul damage is a special kind of damage that cuts right through the body and
attacks the soul. Creatures that are damaged by this special damage, for
example through \nameref{sec:Soul Fire}, have not only their body wounded,
but also their soul.

\begin{35e}{Soul Damage}
  Soul Damage is a new type of damage, that is dealt by some soul powers,
  weapons and by soul fire. It only functions against creature that have a
  soul, or are souls. Soul damage does no damage against objects, or soulless
  creatures such as skeletons, constructs or zombies.
\end{35e}

\subsection{Soul Spells}
\label{sec:Soul Spells}

Raw soul magic can be shaped into useful spells often called \emph{soul spells}.
These may cause damage and harm, start soul fires and cause massive destruction,
but may also heal, cure and aid those in need if used for good. Many soul spells
were forged and created to be used in battle, and to destroy wayward and corrupt
souls.

Soul spells are usually not taught in classes or academy, and cannot be learned
directly from scrolls like arcane magic. However accomplished soul casters have
written books that teach fundamentals, but most soul spells are manifest by
experimentation and hard work on the part of the individual practitioner.

\begin{35e}{Soul Spells}
  Soul spells are listed later in the section about \nameref{sec:Heroic
    Characteristics}.

  Although common soul spells are listed in that section, dungeon masters and
  players are highly encouraged to create new soul spells specifically for
  individual characters. They should express that characters unique talents,
  quirks, weaknesses and strengths.
\end{35e}

\subsection{Soul Aura}
\label{sec:Soul Aura}

Soul auras is soul magic which emanates from certain powerful souls. These
auras often represent the person's strengths or weaknesses. Some people
emanate their natural charisma and leadership, making it easier for others to
follow them into battle, while others radiate away their fierce power and
brutally scaring weaker creatures into submission. While others project their
shadowed existence, making them easily overlooked, dismissed or even
completely ignored. The power of such auras are felt subconsciously by those
who have not awoken, while those that have, will see the aura with their soul
sight.

It is possible for some people, especially natural leaders, strong fighters
and people of great power to radiate a soul aura without them knowing about it.
They must not be awoken to their soul potential to be able to produce a strong
soul aura.

\begin{35e}{Soul Aura}
  Soul Auras are special soul powers that can be activated, and as long as
  they remain active they reduce the maximum soul power pool of a character. The
  effects of the aura may transfer to other creatures surrounding the caster,
  depending on the specific aura. Even creatures that have not awoken my radiate
  one soul aura, without them knowing, in which case the aura is always active.
\end{35e}

\subsection{Soul Magic in the World}
\label{sec:Soul Magic in the World}

Soul magic by itself is the rarest form of magic among humanoids. Most humanoid
creatures focus on arcane magic, which can be learned by everyone dedicated
enough, or divine magic, which is granted to everyone pious enough. Soul magic
itself is something personal, individualistic in nature, and different from
every witch and witcher. Practitioners of soul magic are rare, but generally
viewed favourably by most humanoid tribes and settlements.

However a few religious institutions see soul magic as an affront to the power
of the gods, and thus seek to actively suppress or even eradicate soul magic.
\nameref{sec:Lor} is known as a fervent enemy of soul magic, and teaches the
destruction of all free roaming souls to return to the ``natural order'' of
the world.
