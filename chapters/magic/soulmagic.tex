\section{Soul Magic}
\label{sec:Soul Magic}

Soul magic is one of three main pillars of magic on Aror, alongside divine
magic (and its closely related cousin arcane magic) and psionic
powers. It taps into the very soul of the caster, and thus, much like psionic
energy, is inherent to caster and does not come from external sources like
divine or arcane energy.

All living creatures on Aror have a soul, that is fundamentally intertwined
with the body. Normally no creature is aware of its own soul, and must first
experience a traumatic event called a \emph{soul awakening}. During the
awakening the inherent bond between the body and the soul are broken, and the
soul is free to be experienced on its own.

\subsection{Soul Awakening}
\label{sec:Soul Awakening}

A \emph{soil awakening} is by nature a traumatic event for both body and soul.
During the awakening the inherent connections that interweave both the body and
the soil are disrupted, allowing each to live without the other. Bodies that
live without their soils are often corporeal undead, such as zombies,
skeletons and \hyperref[sec:Umgeher]{umgeher}. Souls without a body are also
often undead, such as wraiths, ghosts and spirits.

Those that experience soul awakening have their soul broken, and must first
heal before they can tap their souls for power. There are a few ways to heal
a broken soul, such as rest, intense mental and bodily training, or specific
soul spells.

Awakenings come from traumatic experiences that either affect the body or the
soul. Near death experiences, loss of someone important, necromancy, or being
intentionally soul broken by another spell caster are some of the more common
ways to awake.

\begin{35e}{Soul Awakened}
Soul Awakened is an acquired template that can be added to any living
intelligent creature that has a soul (referred to hereafter as base
creature). This template cannot be taken, only acquired.
\srditem{Requirements}{The base creature must have had a significant incident
  with souls, soul magic or a traumatic event in the past to be allegible for
  this template.
}
\srditem{Size and Type}{The base creature’s type and size remains unchanged,
  and retains all of the base creature’s statistics and special abilities
  except as noted here.
}
\srditem{Skills}{An awakened creature immediately adds Knowledge (Soul Magic)
  and Soulcraft to its list of available class skills. See this article on
  soul magic for details.
}
\srditem{Awaken (Ex)}{A soul awakened creature can attempt, once per day, to
  awaken a non-awakend being with a soul. The target must be be willing, and
  if it is not, the attempt simply fails using up the attempt for the day. The
  awakened creature must touch his target, upon which the target must succeed
  a will save (DC: 15 + HD of target) and a fortitude save (DC: 15 + HD of
  target). If the will save fails then the target’s soul becomes broken. If
  the fortitude save fails then the body rejects the soul, and both take 2d6
  points of damage (for 4d6 points in total).
}
\end{35e}

\subsection{Free Soul}
\label{sec:Free Soul}

A free soul, one that is not bound by any body, may linger in the living world
while the body slowly withers away. Being a free soul cuts you off from most
sensory inputs, such as touch, smell, and tactile senses. Furthermore any free
spirit has to expend an enormous amount of mental strength to be able to
materialise in the world to interact with objects or people. This lack of
stimulus and the constant mental stress drives many free spirits insane or
even evil.

There are countless words for free spirits that roam the forests, caves, ruins
or the dark corners of the city. A \emph{mavka}, for example, is a mad free
spirit of a young woman that is sometimes dangerous to children and young men.
Children that die because they were abandoned by their parents are called
\emph{myling}. While free spirits that possess the dead (or their own dead
corpse) are called \emph{wiedergaenger}. While in other cultures free spirits
that try to make themselves known by making noises and sounds are called
\emph{poltergeists}. But generally free souls that roam the living world are
called \emph{spirits}, \emph{Geister}, \emph{shades} or \emph{Cean Gŵla}.

While many spirits slowly turn mad and crazy, that does not apply to all. Free
souls keep most of their memories, mental capacity and thus also mostly keep
their mentality, attitude and character traits. Those that do remain sane
often become eccentric, but rarely are threats to the living.

Free souls may still be twisted to evil through necromancy, turning them into
\emph{wraiths}, \emph{spectres}, and \emph{shadows}.

\begin{35e}{Spirit}
  ``Spirit'' is an acquired template that can be added to any living creature
  that has separated its soul from its body and has succeeded its will saves
  (see above). This creature is referred to hereafter as the base creature.

  \srditem{Size and Type}{The creature’s type changes to Undead
    (Incorporeal). It retains any subtype and subtypes that indicate kind. It
    does not gain the augmented subtype. It uses all of the base creature’s
    statistics and special abilities except as noted here.
  }
  \srditem{Hit Dice}{All current and future Hit Dice become d12s}
  \srditem{Speed}{A spirit does not walk, and loses any base land
    speed. Instead it gains fly speed 30ft. - unless the base creature has
    higher fly speed - with perfect manoeuvrability.
  }
  \srditem{Abilities}{Same as the base creature, except that its Charisma is
    increased by +4.
  }
  \srditem{Special Qualities}{A spirit has the same special qualities as the
    base creature as well as those below.
  }
  \srditem{Soul Damage (Ex)}{Any negative levels, level drain, ability drain
    or ability damage the base creature might have suffered transfer to the
    spirit. They continue to work just the same as they would have on the
    base creature. If the spirit possesses a body these drains and damages
    move over to the new body.
  }
  \srditem{Telepathy (Ex)}{A spirit can speak and hear, and can also
    communicate with others through Telepathy (60 ft.).
  }
  \srditem{Possession (Ex)}{A Cean Gŵla can make a special touch instead
    instead of a full round action. This attack provokes an attack of
    opportunity. If the touch attack succeeds the spirit may attempt to
    posses the target creature. See here under the rules section for further
    information on how this works. If the touch attack fails it may try again
    next round.
  }
  \srditem{Level Adjustment}{Same as the base creature +2.}
\end{35e}

\subsection{Possession}
\label{sec:Possession}

Any free soul that roams around may attempt to possess another humanoid's
body. To do so it must touch the target, and then begins a struggle in which
the soul and body of the target fight against the hostile takeover. If the
free spirit wins, it destroys the soul of the target and takes the body over.
After the takeover the soul must get used to the new body, which results in
a period of sickness and weakness after the possession.

\begin{35e}{Possession}
  A free spirit must make a touch attack against a target in an attempt to
  possess it. If the touch attack fails it can try again next round. Once
  touched the target can make a \emph{will} and \emph{fortitude} save to
  resist possession. If any of these saves succeed, the free spirit may not
  attempt to possess the same target again for 24 hours, and gains a -2
  penalty for any further attempt

  The DC for the possession attempt is 10 + spirits HD + spirits CHA modifier.

  If the size of the spirit and the target differs, the spirit gains a
  \emph{-2 penalty} for each step of size difference. Furthermore if the
  HD difference between the target and the spirit is \emph{4} the spirit gains
  a further \emph{-4 penalty} to the DC. Any additional HD difference above
  4 adds further stacking \emph{-2 penalties} to the DC.
\end{35e}

\subsection{Soul Power}
\label{sec:Soul Power}

The soul itself is made out of pure energy, called \emph{soul energy} or
\emph{soul power}. It replenishes by itself if used, much like the body
replenishes itself after a few hours from physical exhaustion. People that
are aware of their own souls, can draw that power, direct it, and put it to
work. How much soul power that is available is directly related to the strength
of the person itself. The soul is only as powerful as the combined trinity of
soul, body and personality.

\begin{35e}{Soul Power}
  Each creature that has undergone \emph{soul awakening} (see above), gains
  \emph{soul power points} just like HP: hit dice are rolled retro actively
  and on future level ups and added to the \emph{soul power point} pool, and
  add any positive charisma modifier to the pool each time hit dice are rolled.
  If in doubt, these rules work just like the rules for HP, but on a separate
  pool with charisma instead of constitution.

  Soul power regenerates every time for the same amount, whenever HP are
  restored. It also decrease every time maximum HP would normally be reduced
  (e.g. through negative levels, or loss of charisma modifier).
\end{35e}

\subsection{Soul Fire}
\label{sec:Soul Fire}

Soul fire is uncontrolled soul power that manifests itself in the world as a
cold unfeeling blue fire. Like true fire it can disintegrate and destroy
everything it touches, but it is feed only soul power. If it doesn't burn
anything with a soul (or a soul) it slowly withers and sizzles. Soul fire does
its own damage to living creatures and souls, and does electricity damage to
objects.

\begin{35e}{Soul Fire}
  Soul Fire burns like regular fire, but does \nameref{sec:Soul Damage} to any
  creature that has a soul, and \emph{electricity} damage to objects.
\end{35e}

\subsection{Broken Soul}
\label{sec:Broken Soul}

A \emph{broken soul} is the soul equivalent of a terminal disease. Through
horrible failure with soul powers, or through necromancy the soul is broken
apart and slowly leaks soul power until it dies. And once the soul dies the
creature usually dies along with it. A broken soul can be cured through
appropriate soul powers, or through some divine spells.

\begin{35e}{Broken Soul}
  Every day that a creature spends with a broken soul it must make a DC: 15
  fortitude save or suffer 1 point of charisma drain. Even if the check
  succeeds the creature takes 3d6 points of soul damage.
\end{35e}

\subsection{Soul Damage}
\label{sec:Soul Damage}

Soul damage is a special kind of damage that cuts right through the body and
attacks the soul. Creatures that are damaged by this special damage, for
example through \nameref{sec:Soul Fire}, have not only their body wounded,
but also their soul.

\begin{35e}{Soul Damage}
  Soul damage not only damages HP, it also damages \nameref{sec:Soul Power} if
  the target creature has undergone soul awakening. If the creature has not
  undergone a soul awakening the creature takes double damage instead.
\end{35e}
