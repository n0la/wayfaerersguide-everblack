\subsection{Old Ways}
\label{sec:Old Ways}

The \emph{Old Ways} are not a god or deity, but instead a set of believes,
traditions, and stories that represent how the ancient humanoids worshipped
the three major deities of Aror. A few minor deities also appear in this
religion, such as \nameref{sec:Morana}.

\subsubsection{Three Mothers}
\label{sec:Three Mothers}

The core of the religion are three female deities that are worshipped as the
\emph{three mothers} of all humanoid races. Each of them represents one stage
of a woman's development. \nameref{sec:Forun} represents the young, hopeful,
and beautiful woman that raises her children with warmth, compassion and
love. She represents youth, beauty, fire and fertility. \nameref{sec:Marwaid}
represents the ageing mother, that has lost children, and sacrificed
everything that she has for her young. She is seen as the hardened, stern and
often embittered woman that raises her children to withstand the harsh
realities of life. \nameref{sec:Marwaid} also represents the strong fighter
within each person, that would fight to the bitter end to protect their
children and family. \nameref{sec:Nyddwr} represents the old woman, the crone,
that offers her immense wisdom to aid her already fully grown children and
family. She represents the tempered, wise, yet hardy old woman that survived
against all odds, and now shares her wisdom with others.

For a while there was a fourth mother, \nameref{sec:Morana}, who stood for the
dead mother, that still protect their children from beyond the afterlife.
After it was revealed that Morana had gained her status among the true deities
through lies and deception, she was cast from her role as one of the main
mothers. She is now vilified in the stories of the Old Ways, as the
\emph{great betrayer}.

\subsubsection{Stories}

Another important part of the Old Ways are how the knowledge, values propagate
and continue to live on: stories. A collection of mystic stories, parables,
and myths that are passed orally from one generation to the other. They often
have several purposes, but most stories are told to children to explain to
them the dangers, beauties, challenges, intricacies, but also the horrors of
the world. The collection of all these stories and myths is called the
\emph{old prose}.

The stories also contain heroic accounts of heroes of old. Bold tales about
people that ventured forth to slay beasts, save the innocent, become kings, or
return a powerful magical artefact back to save the village from certain doom.
But the stories also contain horror stories that end badly, such as the new
mother that goes to the woods only to be eaten by a werewolf.

The most important part of these stories however is to teach the core values
of the ``old ways''. The most important being dedication, loyalty and honour
to your own clan or tribe. Each member of a clan should give what he can, and
take as little as he requires. This not only extends to love, life, family but
also to nature, with which every follower of the Old Ways must strive to
respect.

Some of these stories also tell of failings, crimes, and their appropriate
punishment. While most of these stories tell of compassion for minor offences,
they lay out brutal punishments for serious crimes and even include the death
penalty or slavery for severe crimes such as murder or rape. The same stories
also tell of just rulers, their heroic behaviour and their favourable
personality that made them beloved by their followers.

However these old stories also lay down a strict social construct that often
portraits women as responsible for the household and family, while the man is
supposed to hunt, fight and protect. These structures are then reinforced by
heroic hero stories - who are more often than not - male, and by the stories
of the three mothers whose motherly attributes serve as role models for women.
While heroines, and thus precedent for a balanced and fair society exist - such
as the great huntress \nameref{sec:Eigyr} that slew more beasts than she could
eat, or the powerful witch \nameref{sec:Gweneth} that saved her village from a
terrible plague - many tribes that follow the old ways still see the woman's
place with the family.

\subsubsection{Life Bond}
\label{sec:Life Bond}

The Old Ways contain an old tradition that survives in many modern city
kingdoms, and cultures until this day: Life bonds. A life bond, a form of
ultimate servitude, is always offered from the party seeking to serve towards
the master. These parties could be individuals, or even complete villages
offering their serfdom to another village.

Within the Old Ways this life bond is an act of courage and honour. To many
looking at this practice from the outside, it appears simply as an act of the
defeated party offering their servitude as a slave towards the victor. While
many modern cultures that have adopted this practice from the Old Ways reduce
it to such a simple practice of slavery, even going as far making life bonded
serfs equal to slaves by law, the tradition within the Old Ways serves a
different purpose.

The act of offering a life bond to another, means that the defeated party
acknowledges their own shortcomings, weakness, lack of skill, courage and
honour compared to the victor. It further honours the victor by implying that
their victory was a result of superior courage, tactic, skills, combat prowess
or honour; compared to a victory gained through underhanded or cowardly
practices. Further, if a life bond is accepted, the victor is meant to teach
and tutor the defeated party, and release them as soon as the desired
strength, knowledge, or skill has been imparted onto the defeated party. The
practice has the defeated party (or a representative of the defeated party)
kneel before the victor, head low, and both hands forming a cup above the head
- as if offering both hands in servitude, and expecting knowledge to be poured
into it. The other may then accept, or dismiss the gesture. Dismissal of an
offer of a life bond is seen as an insult, as it alludes to the defeated
parties inability to learn, grow and regain honour and strength. It is also
allowed to accept, but immediately release an opponent from servitude, by
alluding that they their defeat was by a small margin, and in other
circumstances the roles might have been reversed.

Stories of the old ways tell of many who have offered their life in servitude
to others. These stories further tell, that offering a sincere life bond to a
victor is one of the ultimate sacrifices in the eyes of Marwaid. Furthermore
they convey that benevolent victors used these life bond to strengthen
and enriching those they conquered, releasing them stronger than they were
before the defeat. Other stories also warn against abusing the servitude of
another, and tell that the sole purpose of the life bond is impart knowledge,
skill and strength to those that are weak.

\subsubsection{Incantations}

The Old Ways do not have priests, but witches and shamans as spiritual
leaders. Also the ``old prose'' contains incantations that may be cast by
anyone versed well enough in the old stories and ancient Teranim. These
incantations are often soul or divine magic, and quite powerful. However they
often require hours of preparations, several people performing the incantation
together, complicated chants and songs in ancient Teranim, and sometimes even
live sacrifices to the three mothers to succeed. Practising witches and
witchers of the old ways are often the spiritual centre of a clan or
community, and tasked with gathering, teaching and performing these old
incantations when required.
