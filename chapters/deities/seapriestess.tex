\subsection{Sea Priestess}
\label{sec:Sea Priestess}

The \emph{Sea Priestess} is a proposed true deity that was perceived by
\nameref{sec:Graham Balance}, to be a new goddess of death and decay. She is
often portrayed as a pale, white haired woman with blue lips, inhabiting
bodies of water.

After studying the ancient texts, stories, and lore about the \nameref{sec:Old
  Ways} Graham concluded that the position of goddess of death was usurped by
Morana, and there had always been a fourth mother since the ancient times.
Graham did not learn the name of the ancient goddess, but instead proposed a
new one: ``The Sea Priestess''.

The Sea Priestess mythological woman that dwells deep beneath the
\hyperref[sec:Well of Souls]{sea of souls}, where she shepherds the souls to
return to the endless sea as fresh water. She also speaks mystical warnings,
reminding the living about their own mortality, and how careless acts may
jeopardise others. According to Graham's treatise, she accepts the dead that
are properly buried, either in the soil, on water, or through fire. She
opposes most soulless undead such as skeletons, zombies or ghouls, but accepts
intelligent undead and soul magic but warns caution in those areas.

After Graham had published the treatise on his new proposed religion, very
few people took it seriously. In the early days, and during the rest of his
lifetime worship of the sea priestess was limited to him, and his closest
friends. He continued to publish books, songs, and poetry about the sea
priestess, expanding her lore by adding much of his personal philosophy into
her teachings. For much of the late decades of GT, and early decades of MI
after Graham's death, the followers of the ``sea priestess'' steadily grew.
Many saw her as a ``goddess for disbelievers'', while some flocked to the old
ways after the tragedy surrounding the lesser deity \nameref{sec:Griannar}.
In MI:210, five hundred years after Graham's death, the first priest following
his practices in worship of the Sea Priestess received holy power through
divine magick.

Ever since it has been unclear whether the Sea Priestess truly is a true deity
of Aror, or whether yet another lesser deity saw its chance to impersonate
one. So far all indications point to her being a true deity, while many remain
sceptical, especially since \nameref{sec:Morana}'s great betrayal.

\begin{35e}{Sea Priestess}
  The Sea Priestess is considered neutral good, and her favourite weapon is
  the long bow. She accepts soul casters, and intelligent undead as followers.
\end{35e}
