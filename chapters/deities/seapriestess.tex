\subsection{Sea Priestess}
\label{sec:Sea Priestess}

\songquote{COIL}{here nature is naked \\
  her acrobats bathed in blood \\
  there is a beast of prey \\
  on the threshold of pleasure \\
  and a giantess sea priestess beckons the passers-by:\\
  ``Do not lose sight of the sea.''
}

The \emph{Sea Priestess} is a true deity representing nature, hope, eternity,
and the inevitable cycle of death and rebirth. In most interpretations she
is the living incarnation, guardian, and overseer of the \nameref{sec:Soul
  Well}.

After studying the ancient texts, stories, and lore about the \nameref{sec:Old
  Ways} \nameref{sec:Graham Balance} concluded that the position of goddess of
death was usurped by Morana, and there had always been a fourth mother since
the ancient times of the \nameref{sec:Tynrikke}. Graham did not learn the name of
the ancient goddess, but instead proposed a new one: the ``Sea Priestess''.

The Sea Priestess is a mythological woman that dwells deep beneath the
\hyperref[sec:Soul Well]{sea of souls}, where she shepherds the dead to
return to the endless sea. According to Graham she also reminds the living
of their own mortality, and how careless acts may jeopardise others. The Sea
Priestess accepts the dead that are properly buried, either in the soil, on
water, or through fire. She opposes most soulless undead such as skeletons,
zombies or ghouls, but accepts intelligent undead, druids, and soul
magic. Graham describes her as a mother that has lost everything she ever
loved to the inevitable, and has lost her sanity to grief and sadness. However
she is also described of having found her way back out of grief, but
having conjured true misery for herself in the journey. In her last attempt to
birth good from her misery, she is now said to preach hope to combat despair,
and misery.

After Graham had published the treatise very few people took it seriously. In
the early days, and during the rest of his lifetime, worship of the sea
priestess was limited to him, and his closest friends. He continued to publish
books, songs, and poetry about the sea priestess, expanding her lore by
studying the ancient stories of the \nameref{sec:Tynrikke}. In the early
decades of \emph{MI} the followers of the ``sea priestess'' steadily grew.
Many saw her as a beacon of hope after \emph{Morana}'s betrayal, and many
returned to the old ways after the tragedy surrounding the lesser deity
\nameref{sec:Griannar} In MI:210, five hundred years after Graham's death, the
first priest following his practices in worship of the Sea Priestess received
holy power through divine magick.

\aren{How did my good friend always put it: ``If it goes any faster, there
  would be an astral disaster.''}

Ever since it has been unclear whether the Sea Priestess truly is a true deity
of Aror, or whether yet another lesser deity saw its chance to impersonate
one. So far all indications point to her being a true deity, while many remain
sceptical, especially after \nameref{sec:Morana}'s great betrayal.

\begin{35e}{Sea Priestess}
  The Sea Priestess is considered neutral good, and she is seen as the patron
  of the soul well, nature, and the endless cycle of death and rebirth. Her
  favourite weapon is the long bow, and her domains are death, plants,
  animals, earth, air, water, and fire.
\end{35e}
