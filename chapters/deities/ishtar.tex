\subsection{Ishtar}
\label{sec:Ishtar}

Ishtar (or ``Iana'' in some regions) is female succubus, and a lesser deity of
healing, doctors, nurses and biological sciences. Ishtar's symbol is a snake
coiling around thin dagger without a cross guard.

Her role in the layers of hell is to free \nameref{sec:Demons} from the
scourge, and then aid them in their recovery process and integration into
devil society. Ishtar surgically corrects the disfigurements of the scourge,
surgically implements psionic blockers to avoid mind control from the scourge,
heals those that were wounded, but also performs physical enhancements through
surgery in the layers of hell. While she is a devil, many see her role within
the layer of abyss as one of a healer, freer of the enslaved and patron of
studies of medicine and biology. Ishtar teaches that conventional healing is
both art and science, and must be practised with great care and great
responsibility.  Ishtar also represents vanity and beauty, as she performs
surgery on the disfigured spawns of the \nameref{sec:Scourge}. Just like her,
her followers also practice plastic surgery on both the deformed, and those
whose beauty has faded due to old age.

Generally she accepts both the good natured healer that attempts to aid and
heal those that are sick, wounded or disfigured by illness and accident, as
well as the vain surgeon that performs plastic surgery on the rich nobility
for large amounts of money. And even though her religion is mostly practised
by a small minority of expert surgeons, healers and doctors, they are well
respected all across Aror. Some elements of her faith are questionable, as she
is vague on topics on whether healers and doctors require prior consent and
authorisation for treatments or experiments.

Ishtar can be summoned, in which case she will send a succubus, incubus or
tiefling minion to make deals with mortals. She is interested in granting
patronage to those that research medicine (especially surgery and triage) and
biology. She will often ask for research results, as well as gained knowledge
in exchange for favours, knowledge and artefacts.

Her followers are mostly well liked and well received, especially by those who
cannot afford divine healing magic to cure illnesses and treat wounds. Some of
Ishtar's more morally grey followers will run afoul with local authorities, or
members of the \nameref{sec:Order} if they conduct treatments and experiments
without consent. \nameref{sec:Lor} bans devil worship outright, and so also
prosecutes those that follow Ishtar.

\begin{35e}{Ishtar}
  ``Ishtar'' is considered \emph{neutral good} and her favoured weapon is the
  khukri. Ishtar herself is not powerful enough to grant divine spells or
  favour, but very few of her followers actually need it anyway. Ishtar
  teaches science and medicine, and not divine dogma and worship.
\end{35e}
