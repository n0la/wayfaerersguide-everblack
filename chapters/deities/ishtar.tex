\subsubsection{Ishtar}
\label{sec:Ishtar}

Ishtar (or \emph{Inanna} in some regions) is a lesser deity. She is in truth a
female succubus that is often revered as the patron of doctors, freed slaves,
conventional healers, surgeons and those that study medicine or biology.
Ishtar's symbol is a snake coiling around thin dagger without a cross guard.

Her role in the layers of hell are to free \nameref{sec:Demons} from the
scourge, and then aid them in their recovery process and integration into
devil society. She is aided in this role by her master and teacher
\nameref{sec:Asmoday}. While she is a devil, many see her role within the
layer of abyss as one of a healer, freer of the enslaved and patron of studies
of medicine and biology. Ishtar teaches that conventional healing is both art
and science, and must be practised with great care and great
responsibility. Ishtar also represents vanity and beauty as she performs
surgery on the disfigured spawns of the \nameref{sec:Scourge}. Her followers
also practice plastic surgery on both the deformed, and those whose beauty is
fading due to old age.

Generally she accepts both the good natured healer that attempts to aid and
heal those that are sick, wounded or disfigured by illness and accident, as
well as the vain surgeon that performs plastic surgery on the rich nobility
for large amounts of money. And even though her religion is mostly practised
by a small minority of expert surgeons, healers and doctors, they are well
respected all across Aror. Some elements of her faith are questionable, as she
is vague on topics on whether healers and doctors require prior consent and
authorisation for treatments or experiments.

Ishtar can be summoned, in which case she will send a succubus or incubus
minion to make deals with mortals. She is interested in granting patronage to
those that research medicine (especially surgery) and biology. She will often
ask for research results, as well as gained knowledge in exchange for favours,
knowledge and artefacts.

Her followers are mostly well liked and well received, especially by those who
cannot afford divine healing magic to cure illnesses and treat wounds. Some of
Ishtar's more morally grey followers will run afoul with local authorities, or
members of the \nameref{sec:Order} if they conduct treatments and experiments
without consent. \nameref{sec:Lor} bans devil worship outright, and so also
prosecutes those that follow Ishtar.

\begin{35e}{Ishtar}
  All in all \emph{Ishtar} is considered \emph{neutral} (with a slight tilt
  towards \emph{neutral good} when it comes to treat her own fellow devils),
  and her favoured weapon is the kukri.
\end{35e}
