\subsection{Morana}
\label{sec:Morana}

\emph{Morana}, often called \emph{lady death} or the \emph{great betrayer}, is
the lesser deity of death and the undead. In the old manuscripts she is often
depicted as a black veiled female humanoid with pale skin. In newer works of
art and literature, made after the her deceit was discovered, she is often
shown as a blue, translucent female humanoid ghost, stealing or shepherding
souls.

\subsubsection{Great Betrayer}
\label{sec:Great Betrayer}

Originally she was believed to be a greater deity of death, and thus was often
seen as the fourth sister to \nameref{sec:Forun}, \nameref{sec:Marwaid} and
\nameref{sec:Nyddwr}. She was seen as the last stage of motherhood: the old
woman that died, but still holds her protective hand over her children from
the afterlife. Welcoming, and beckoning her children to her side once their
time had come. She was thus a major part of the \nameref{sec:Old Ways} once,
before she was almost unilaterally rejected, and now holds the role of a
villain in the religion.

During the \nameref{sec:Aeon of Strife} some of her followers prayed to her to
give them strength against the monsters and monstrous races that threatened
the humanoid races. As an answer she revealed to the early humanoids the
knowledge on how to create \nameref{sec:Vampires}. She did so secretly and
indirectly, as to not arouse suspicion that she was not in fact a greater
deity.

Her followers were promised great power, strength and eternal life, but Morana
did not reveal the many drawbacks and sacrifices that came with the undead
life. Many of her followers accepted her gift. Upon realising that their new
form was savage, evil and animalistic in nature, and a danger to the very
humanoids they sought to protect, the majority of her followers turned away
from her. This angered Morana greatly, and in retaliation she openly
threatened the vampires and high priests with death should they abandon her
faith.

Since higher deities do not speak to their followers, as they are abstract
concepts and not extra planar entities that live and breathe, her deception
was brought to the light. This deception and betrayal angered almost all of
her followers who in turn abandoned her. Morana however made good on her
threat, and smote and killed most of her undead followers and arch priests.

This betrayal was never forgotten and she's now simply known as the
\emph{great betrayer} among most of the humanoid species. Her name is never
spoken directly, as it is considered too much honour for a being so petty,
deceitful and evil. Her remaining followers call her either \emph{lady Death}
or \emph{Morana}.

\subsubsection{Followers}

Most baronies and city kingdoms ban her worship. The duties of burying the
dead have been taken over by the church of \nameref{sec:Forun} or the church
of \nameref{sec:Lor}. Although she has followers among the \nameref{sec:Inua},
as well as less civilised undead, such as feral vampires or liches, her worship
is rejected among the civilised undead such as \nameref{sec:Vampires} and
\nameref{sec:Umgeher}.

\subsubsection{Teachings}

Morana's modern teachings openly encourages the creation and spreading of
evil undead, and she often helps powerful necromancers to achieve lichdom.
She also welcomes anyone who wishes to practise necromancy, and sees all
undead as her children, even if they reject her.

\subsubsection{Relations}

She now holds poor relations with most other deities and their followers,
especially the true deities that are still worshipped in the
\nameref{sec:Old Ways}. Her direct enemy is \nameref{sec:Lor}, who openly
opposes her followers for creating and summoning undead.

Morana and \nameref{sec:Isamir} appear to be on good terms, as they are both
worshipped together among the \nameref{sec:Inua} of the \nameref{sec:Silver
  Isles}.

\begin{35e}{Morana}
  She is considered \emph{neutral evil}, and her favoured weapon is the
  morning star. Her domains are evil, death, destruction and knowledge.
\end{35e}
