\subsection{Order Above Chaos}
\label{sec:Order}

The \emph{Order Above Chaos} or simply the \emph{Order} is a true deity, that
represents order and justice, regardless of prevailing law. It is seen as the
\emph{higher order of all things} intrinsic and true to all societies. The
fundamental laws and rules that govern all species, from which no one can
escape, and that which thrones above everyone and everything, even kings.

\subsubsection{Personification}

The Order Above Chaos is often depicted as some form of being hovering over
everything that lives below in the earthly domain. More often than not the
Order is not personified at all, and instead represented by a dagger pointing
downward to form a cross. This dagger pointing downward is also worn by all
members and believers as a tattoo, and has become a widely recognised symbol
of the religion.

\subsubsection{Prevailing Dogma}

The \emph{Order Above Chaos} has two major churches and dogmas: The
\nameref{sec:Five Holy Orders}, and the \nameref{sec:Holy Church of Aleaste}.
Both follow the same basic tenets of an order above chaos, but differ
substantially in the amount they are allowed to interfere in local affairs.

The dogma states that all things must submit to a higher state of order. All
social structures created by intelligent creatures (such as humanoids or
intelligent monstrous races) will only work in the long term if they orient
themselves towards that higher order which regulates a peaceful and productive
together. Corrupt, heavily fascistic, collectivised or socially unfair social
structures stray from that ideal and will thus inevitably fail and fall
apart. All members believe that only a fair and open system of government, which
is lead by a fair, competent and just ruler that listens to concerns of his
subjects, will ultimately succeed in bringing long term stability by keeping
chaos at bay. This higher order is above all things, such as kings, emperors
or even lesser deities.

This higher order not only applies to social systems as a whole, but also to
individuals. The followers believe that every individual being should strife
towards that higher order in their personal life, and will thus, inevitably
also contribute to the higher order of the social structure they are a part
of by aligning themselves towards its virtues. Individual virtues valued by
the Order Above Chaos are conscientiousness, honesty, modesty, strength in
both mind and body and forgiveness.

Most churches and interpretations vary greatly by the tools available to those
that wish to seek out and destroy corruption. Some argue that only the virtuous
themselves are allowed to counter chaos, while others argue that some amount of
chaos is required to fight chaos itself. Discord among the believers is so
great in this regard that it lead to a schism, which split off the Holy Church
of Aleaste from the Four Holy Orders.

\subsubsection{Rivalry with Lor}

The followers of the Order are in a dialectic conflict with the followers of
the lesser deity of \nameref{sec:Lor}. Their argument is that the entity known
as Lor represents chaos disguised as a just crusade, and does nothing to
maintain the order of things in the long term. In return the followers of Lor
accuse the followers of the Order to indulge in vengeance, and vigilante
justice instead of seeking true and lasting order. Other followers of Lor
claim that he is the very embodiment of the ``just ruler'' that the Order
Above Chaos predicts, while followers of the Order counter that ``true order''
stands even above powerful entities such as \nameref{sec:Lor}.

\begin{35e}{Order}
  The \emph{Order Above Chaos} is considered Lawful Neutral, and their favoured
  weapon is the dagger.
\end{35e}

\FloatBarrier

\ifimages
\clearpage
\incgraph[
  overlay={\node[black] at ([xshift=0cm,yshift=-1cm] page.north)
    (main)[text width=0.9\paperwidth]{
      \large \centering
      \textbf{``The concept of the \emph{Order Above Chaos} as illustrated in
        a mural in the cathedral of \emph{Stenheim}.''}
    };
    \node[black, below=of main,yshift=1cm,xshift=-6cm]{
      \nameref{sec:Stenheim} circa GT:2101
    };
  }
]{media/order-colour.\imagesuffix}
\clearpage
\fi
