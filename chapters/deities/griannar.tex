\subsection{Griannar}
\label{sec:Griannar}

\emph{Griannar} was once the lesser deity of light, piety, forgiveness and
repentance, until he was killed by the \nameref{sec:Silent Queen} in
\emph{MI:0}. He was often portrayed as a sunbeam, or as the two suns rising on
the horizon.

Worship of Griannar stretched back thousands of years, and before his death,
the \nameref{sec:Church of Light} was one the dominant religion in many
human city kingdoms. His church was among the most powerful institution on
\hyperref[sec:Aror]{Aror}, and at its peak, counted millions of followers.
The church held vast and unparalleled influence, and political power. His
church was not only an institution to further his worship, but also included a
knight order as a military wing, that could rival many armies in terms of
manpower, training and equipment.

\subsubsection{Holy Crusade}
\label{sec:Holy Crusade}

Griannar was always an open rival of the \nameref{sec:Silent Queen}. This
rivalry existed over centuries, and lead to the occasional skirmishes, violent
disputes, and armed clashes between the two religions and their
followers. Over the years the power of the Holy Church grew, and began to
entrench itself deeply in the political landscape of the major city
kingdoms. Since it openly tried to enforce a policy of zero tolerance against
corruption, impurity, debauchery, evil and the undead (regardless on whether
they were evil or not), many noble houses began to secretly support the
followers of the Silent Queen.

The Queen's followers, who where often rich thieves, smugglers, corrupt barons
and nobles, began to fund mercenaries and assassins, to drive the followers of
Griannar of their land, or to assassinate powerful priests and bishops of the
Holy Church. The church retaliated by sending her knights to defend churches,
pilgrims and protect the bishops. As the open engagements between the Queen's
mercenaries and the knights became frequent, more and more noble houses, who
saw the Church as a threat to their power, began funding the Queen's war
campaign.

In \emph{GT:3391} the holy church openly called for a holy crusade against the
evil that is the Silent Queen and her followers. The declaration was met by a
rival declaration by the \nameref{sec:House deVar}, who openly supported the
Silent Queen in the crusade. This plunged \nameref{sec:Helmarnock} into war
against other city nations that openly supported Griannar, including
\nameref{sec:Hraglund} and \nameref{sec:Forsby}.

The Holy Crusade lasted for almost twenty years, and reached its conclusion
at the decisive siege of \nameref{sec:Hraglund} in \emph{GT:3408}. The
siege lasted three years, and finally ended when the archbishop of Griannar
tried to flee the city in secret. He was betrayed by the nobles of the city,
and delivered to be executed by the high priestess \nameref{sec:Aria} of the
Silent Queen.

\subsubsection{Death}

Scholars still debate Griannar's death to this day, but in \emph{MI:0}, all
priests of Griannar lost their divine power, and their prayers went
unanswered. Alongside him, his rival the \nameref{sec:Silent Queen} disappeared
(or died) as well, giving rise to a new religion surrounding the high priestess
\nameref{sec:Aria} a few decades later.

\subsubsection{Legacy}

Over the course of many decades the Holy Church of His Divine Light slowly
lost influence and power, until it slid into obscurity. Ruined temples and
churches of the church can still be found all over the world, while the major
sites of worships within the city kingdoms were either demolished or have been
taken over by other faiths.

The death of such a major deity was a major event, and the scholars of
\nameref{sec:Fes al-Bashir} tried for a long time to understand the
intricacies of such a world shattering event. The sad occasion was
immortalised in the dawn of the new aeon of \emph{Midaris} that began with
year zero in the year of Griannar's death.

\begin{35e}{Griannar}
  Griannar was considered lawful-good, his favoured weapon was the arming sword.
  Griannar is considered dead, and his followers and priests no longer receive
  divine power.
\end{35e}
