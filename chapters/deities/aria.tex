\subsection{Aria}
\label{sec:Aria}

\emph{Aria} is the lesser deity of secrets, intrigues, knowledge and hidden
things. She is often depicted as a woman clad in black robes, hiding her face
from onlookers.

Aria was once high priestess of the \nameref{sec:Silent Queen}, before she
challenged the queen's reign after fighting against \nameref{sec:Griannar}
in the \nameref{sec:Holy Crusade}. Aria is directly, or indirectly,
responsible for killing the Silent Queen and usurping her domain, and
power. It is unclear on how exactly she achieved this feat, and scholars
debate the details of the coup to this day.

Not only does she encourage the gathering of knowledge and information, she
also openly encourages her followers to use said knowledge for personal gain.
Many of her followers are thus often those that seek to control society from
the shadows, while amassing wealth and power in secrecy.

Aria is in direct conflict with the \nameref{sec:Runemaster}, as he gifts
powerful magic and teachings to mortals. She also openly opposes anyone who
seeks to investigate unethical practices such as necromancy. The
uncompromising methods of her followers, such as theft or outright
assassination, brings her and her followers often in direct conflict with the
\nameref{sec:Order} or the knights of \nameref{sec:Lor}.

\begin{35e}{Aria}
  Aria is considered \emph{lawful evil} and her domains are knowledge, travel,
  magic and trickery. She is the patron of thieves, wizards, librarians,
  archivists, and researchers. Her favoured weapon is the short sword.

  There are two feats associated with Aria: \featref{Adept of Aria} and
  \featref{Well of Truth Agent}.
\end{35e}
