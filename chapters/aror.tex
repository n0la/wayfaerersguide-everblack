\chapter{The World of Aror}
\label{sec:Aror}

The world of \emph{Everblack} is situated on a small planet that is
called \emph{Aror}. \emph{Aror} rotates around a binary star system we
call \emph{Carinae}. Both suns can be seen on the sky, with one being
slightly smaller than the other.

The planet also has two moons, one of which is named \emph{Lilest} and
the other \emph{Storst}, in honour of the two magi that first
identified most of the celestial bodies in our solar sytem. They have
many different names in different cultures and languages, but the
names \emph{Lilest} and \emph{Storst} are always understood to refer
to the two moons in the night sky.

\emph{Aror} takes roughly twenty-four (24) hours to rotate around its
own axis, while \emph{Lilest} goes around the planet in thirty-two
(32) days, the bigger moon \emph{Storst} takes forty-eight (48) days
to make one trip around \emph{Aror}. \emph{Aror} itself takes 385 days
to travel around the binary star constellation.

\emph{Aror} is not alone in the system. It is in fact the third planet
from the suns. The first, we call \emph{Forun}, and is a molten rock
of magma, insane temperatures and poisonous gases. \emph{Parin} is
seated between \emph{Forun} and \emph{Aror}, and can be seen with the
naked eye on a clear night sky as a blue shimmering light. Then follow
\emph{Ivir} and \emph{Novar} which are believed to be huge planets made
of various gases. Far beyond the reaches of \emph{Novar} is
\emph{Piad} a smaller, rocky planet on the edge of our binary star
system.

% Geography of Aror
\section{Geography of Aror}
\label{sec:Geography}

The land of Aror encompasses eight great continents, of which all but one are
inhabited by sentient races. The planet has two polar ice caps, as well as lush
rainforests around the equator. Both the northern and southern hemispheres
experience moderate to cold climate depending on the latitude. Latitude 0 runs
through what has been determined to be central island in the middle of the
archipelago known as the \emph{Silver Isles}.

\subsection{Arania}
\label{sec:Arania}

To the west of Karnak, and in the south western corner of Aror lies the
continent of \emph{Arania}. It is considered the birth place of the ancestors
of all sentient humanoid races, such as the elves, halflings, dwarves and
humans. The continent is a huge and vast steppe, dotted with the occasional
mountain ranges, deserts and fertile oasis. Arania is home to many native and
nomadic tribes, wild animals such as elephants, lions as well as three of the
largest and oldest city kingdoms: \nameref{sec:Fes al-Bashir} to the north
east, and \nameref{sec:Esmayar} and \nameref{sec:El-Fayam}.

In the centre of the continent lies a vast sand desert called the
\nameref{sec:Suam Desert}, which is home to many nomadic tribes of humanoids,
gnolls, hobgoblins, naga, and medusa. South of the desert lies a huge lake,
which feeds many steppes, oasis and other fertile regions surrounding it.

\subsection{Draigynus}
\label{sec:Draigynus}

\emph{Draigynus} is an elongated continent that encircles Farlar to the south
in a half-moon shape. It is close to the southern pale, and thus features a
temperate continental climate to the north, and a subarctic climate to the
south. Very few humanoid races live there, however a small colony of
\hyperref[sec:Dragons]{dragons} has made this continent their home. It is
mostly inhabited by a few sturdy tribes of \hyperref[sec:Snow Elves]{snow
  elves}, and various subarctic beast races.

\subsection{Eilean Mor}
\label{sec:Eilean Mor}

Far to the west, and north of \emph{Arania} lies the continent of \emph{Eilean
  Mor}. It is split in half by a large mountain range called the
\nameref{sec:Great Divide} that runs from all the from the north east down to
the south west. South of the Great Divide lies a vast boreal forest called the
\nameref{sec:Dirgewood}. The continent features subarctic climate in the
north, and temperate climate in the middle, and warmer climate in the
south. Home to many sentient humanoid races, but most predominantly elves,
humans, halflings, and dwarves, it houses four large city kingdoms:
\nameref{sec:Norbury} off the coast in the far north, \nameref{sec:Hraglund}
to the east, \nameref{sec:Helmarnock} to the south east, and
\nameref{sec:Tredegar} to the south west. Eilean Mor is the first continent
that was settled by the humanoid races after the great ice age, and often
referred to as the \emph{old world}.

\subsubsection{Eafiadir}
\label{sec:Eafiadir}

North of Eilean Mor lies a massive dormant volcano called \emph{Eafiadir}. It
is the source of many rivers, and is surrounded by many poisonous sulphur pits
and lakes. Although the area around the volcano is more sparsely populated
there are many smaller baronies and kingdoms on the shores of the north and
north west of the mountain.

\subsection{Farlar}
\label{sec:Farlar}

South of Goltir lies the smaller continent of \emph{Farlar}. Known for its
diverse biome and hilly country side in the north, and cold tundra in the
south. It is now home to mostly \hyperref[sec:Giants]{giants}, giant races
such as ogres and trolls, and also \hyperref[sec:Diarim]{diarim}. The north is
covered with thick and lush forests, which are watered by several large
streams and lakes. These lakes and streams are fed by the fresh rainfall that
falls against the mountain range called \emph{Lias'wa}. The mountain range
splits the continent in half, and allows for a colder climate in the southern
part of the continent. Among all the continents Farlar is by far the smallest,
and once housed the sprawling city kingdom of \emph{Nen-Hilith} before the
giants forcefully took the land as their own.

\subsection{Goltir}
\label{sec:Goltir}

The largest and westernmost continent of Aror, named \emph{Goltir} stretches
from the northern polar ice caps beyond the equator. It is separated in two
large landmasses by a mountain range called \emph{Torainn}. Although the
entire continent is called Goltir, it is often divided into north and south
Goltir, split east to west by the Torainn mountain range.

North Goltir is the home of the city kingdoms of \nameref{sec:Forsby},
\nameref{sec:Kesmar} and \nameref{sec:Stenheim}. North Goltir also known as
the ``land of lakes'' as its entire central landmass is dotted with millions
of smaller lakes, rivers and springs. The central area between Kesmar and
Forsby is called the \emph{Toralian} highlands, and is ruled by various smaller
baronies, earldoms and individual tribes.

A mountain range called the \emph{Alparan} mountains stretches from the centre
to north east, reaching into the polar ice cap of the northern pale. In the
west, the continent stretches out a land tongue towards the Silver Isles,
which was responsible for allowing the early humanoids to spread to the
archipelago. To the south the continent features a vast sand desert, called
the \emph{Walusian Desert} that reaches the foot of the mountain chain
\emph{Torainn}.

North Goltir is home to many sentient humanoid as well as beast races. Humans,
halflings, elves and dwarves live in small baronies in the moderate climate of
the central regions. They share their land with tribes of orcs, bugbears,
hobgoblin, goblins, ogres, trolls, and gnolls who prefer the desert region of
the south.

\subsubsection{Cnámh Mountains}
\label{sec:Cnamh Mountains}

The Cnámh mountain range resides on the eastern shores of North Goltir. They
run from the central regions of the Torlian highlands all the way east towards
the great sea. The mountain range is famous for housing the
\hyperref[sec:Kesmar]{Blackhammer Clan} the biggest and oldest dwarven clan on
all of Aror. South of the mountain range, along a river fed by the mountains,
lies the city of \nameref{sec:Kesmar}.

\subsubsection{South Goltir}
\label{sec:South Goltir}

South of the mountain range of \emph{Torainn} lies the continent \emph{South
  Goltir}. From the mountain range southward it is mostly steppe and grassland
but turns into a thick rainforest around the equator. The most prominent
feature is the almost 9000 metre high \emph{Goban} mountain that peeks forth
from the impenetrable rainforest. The mountain is also the source of the
\emph{Al'ahri} river that runs all the way south through the rainforest, the
southern desert into the sea. The desert and jungle are mostly inhabited by
native and nomadic tribes of elves, monstrous races, as well as certain beast
races such as harpies and gorgons. Gnolls and nomadic humanoid tribes roam the
desert, while the far south hosts the city kingdom of \emph{Avenfjord}.

\subsubsection{Goban}
\label{sec:Goban}

Goban is a large mountain peak that towers above a smaller mountain range in
the centre of southern Goltir. With a size of roughly 9000 metres, it is the
largest mountain on all of Aror. It is the origin of many rivers that worm
their way through southern Goltir, making the land surrounding the mountain
fertile and humid.

Just north of the mountain, and south of Torainn, is also a vast grassland and
jungle called \emph{Goban Jungle}. It is fuelled by the many rivers and their
side arms that spring forth from both the Torainn and Goban mountain range.
The jungle is home to many exotic animals, various tribes of both humanoid and
monstrous races, as well as many elusive and exotic beast races.

South of the mountain is a vast sand desert, called the \emph{Goban desert}.
It features a few fertile oasis and grass lands, but is mostly an inhospitable
sea of sand and heat. Many nomadic humanoid tribes (especially halflings), as
well as many nomadic monstrous tribes (such as gnolls, hobgoblins and ogres)
call this desert their home.

\subsection{Iâfandir}
\label{sec:Iafandir}

Far to the north west lies the continent of \emph{Iâfandir}. It is often
called the \emph{savage lands}, because of its harsh subarctic climate, cold
winters, and the many beasts and beast races that call it home. Most of the
civilisations and tribes there live on the shores along the southern bank, as
the land becomes a frozen tundra further north. It is home of many
\hyperref[sec:Snow Elves]{snow elves} and the \nameref{sec:Tynrikke}, as well as
various monstrous races such as ogres, trolls and hobgoblins, it became
synonymous with a cold climate, harsh conditions and a primal, more savage way
of living.

Although only populated by smaller tribes, villages and perhaps the occasional
snow elven or hobgoblin city, the continent was the birth place of the youngest
city kingdom on Aror: \emph{Morkan}.

\subsection{Isles of Conflagration}
\label{sec:Isles of Conflagration}

The \emph{Isles of Conflagration} are a group of islands off the eastern
shores of Goltir. Often called the \emph{archipelago of the dead}, because it
is an inhospitable place where volcanic eruptions, geological activity and a
rocky and infertile surface denies most life a chance to prosper. Many attempts
have been made to settle the islands, but most have been thwarted by the extreme
conditions found in the archipelago. More recently however, various races have
settled there that can better brave the hot and poisonous climate, including a
tribe of \nameref{sec:Umgeher}, as well as a small colony of
\nameref{sec:Devils}.

\subsection{Karnak}
\label{sec:Karnak}

West of the land of the dragons is the massive continent of \emph{Karnak}.The
continent itself features a massive lake in its centre, called \emph{Mu'ut}.
The lake is encased in huge mountain ranges, and mostly inaccessible by
land. The lake itself is considered holy to many of the native tribes of the
continent, and is also the home of an ancient evil god called
\nameref{sec:Xir}.

The northern, eastern and western parts of the continent are covered in a vast
and untouched rain forest called \nameref{sec:Yuacata}. The continent is
mostly inhabited by a vast amount of small tribes of wood elves, beast races
such as gorgons, harpies and many different and often dangerous animals such
as apes, primates, tigers and jaguars.

It has a vast chain of large islands off the northern coast, which are mostly
covered in rain forests, called the \emph{Kanaria Archipelago}. It houses many
native tribes but is also home to many smaller outposts of various trading
guilds, mining corporations, slavers and pirates, that exploit the archipelago
for its resources.

\subsection{Silver Isles}

In the centre of the vast sea that separates all the continent, stretching in
from the land tongue of \emph{North Goltir}, lies the archipelago called the
\nameref{sec:Silver Isles}. A huge network of thousands of smaller islands and
inlets, which are mostly covered with rain forests and jungles. It is the
native home of many beasts, native tribes as well as newly settled villages
and expeditions that seek to find what the islands are named after: mineral
riches.

% Regions of Aror
\section{Regions}
\label{sec:Regions}

Although civilisation can mostly be found in the large city kingdoms, it does
not mean that the more rural areas are filled with savages. Many big regions
share similar cultures or backgrounds, and it is not uncommon to find people
that have moved out of their city kingdoms into these areas. Either to start a
simpler live as farmers, or because a business opportunity or the love of
their life has drawn them there.  Likewise many people move from the country
side to the big cities in the hopes of finding better opportunities, such as
jobs and education there.

The following chapter of this book should introduce you to several regions of
Aror. Their culture, their believes, as well as their struggles and victories.
Tread carefully in these regions, as you might find friend and foe alike.

%% Dirgewood
\subsection{Dirgewood}
\label{sec:Dirgewood}

The \emph{Dirgewood} is a vast forest south east of the \nameref{sec:Great
  Divide}, a large mountain chain that splits the continent of
\nameref{sec:Eilean Mor} in half north by south. It runs along most of the
southern side of the Great Divide, reaching up the mountains until the tree
border, but does not extend to the shores. It is a vast temperate and boreal
forest and marshland, that is split by a few major rivers that have their
origins in the Great Divide. The thick woods, harsh and unwelcoming marches,
and hilly terrain has made it almost impossible to build large castles in the
Dirgewood. However it is dotted with thousands of smaller towns and villages.

The Dirgewood is mostly settled by humans, wood elves, snow elves, and
halflings. The hilly regions of the Dirgewood are also home to deepkin, and
dark elves. Many monstrous tribes remain, especially hobgoblins, bugbears,
kobolds and goblins as well as many tribes and packs of lycanthropes. The
Dirgewood is also known for still housing many faeries and fey. The Dirgewood
is one of the few areas in which the \nameref{sec:Aeon of Strife} is still
actively fought between humanoid and beast races.

Albeit the humanoid villages are as diverse as any city they have two things
in common: Most villages know of each other, and even know people from other
villages. They trade with each other often, intermarry, and also come to each
other's aid in case of an attack. Most of the time they fight monstrous races
and beasts, but have also killed bandits that settled in their land or halted
the expansionist dreams of baronies that board the Dirgewood. Skirmishes,
feuds and even wars between the villages are known to happen, but are far and
few in between.

The other thin in common is a staunch belief and adherence to the tradition of
the \nameref{sec:Old Ways}. Almost all villages are lead by a village elder, a
shaman, and a war chief. Most villages and small hamlets celebrate the rituals,
incantations, spells and sacrifice demanded by them by the three (or four)
mothers, and view outsiders that do not believe in the old ways as
suspect. They harbour an open animosity against anyone that would follow a
lesser deity. This animosity is rarely violent, except to those that would
come to the Dirgewood as missionaries.

Although the old traditions are part of their religion and culture, there
exist a great variety in how they are followed among the various tribes. Some
are more strict than others, and while others only revere three mothers, other
revere four and in turn those are split on who the fourth mother truly is. But
most tribes have forgone with the oldest, most brutal traditions. For example
very few tribes still practice humanoid sacrifices to \emph{Marwaid}, or do so
extremely rarely. The Old Ways, and their interpretation within the Dirgewood
and Golian Heights is the birth place for modern equality among men and
women. During the aeons of conflict with the beast races everyone is
encouraged to follow his or her inner calling (or ``true will of the mothers''
as it is often referred to), and thus contribute to the family, village or
town to the best of their ability.

The many rivers that flow from the mountains to the south-eastern shore lines of
\emph{Eilean Mor} are used to transport people and goods in and out of the
Dirgewood. The villagers use the river network to trade with each other, raid
monstrous encampments and tribes, as well as communicating and trading with
the large city kingdoms that reside on the shores of Eilean Mor.

For many city dwellers that follow the three goddess' of the old ways, the
Dirgewood is a popular destination for a pilgrimage and spiritual
enlightenment. The people of the Dirgewood are known to be highly spiritual,
and to them the stories, spirit worship, chants and spells of Old Ways are
part of their daily lives. They welcome any humanoid species that also follows
the Old Ways, or at least one of the three mothers, and adheres to their rules
and customs.

The people of the Dirgewood are known as hardy survivors, expert hunters,
fierce raiders, and proud warriors, and above all else spiritually enlightened
people that take great care in preserving their aeon old traditions and
believes. They dress in plain clothes, but take great pride in elaborate
jewellery made out of bone, antlers, animal teeth, precious metals or
gemstones. To combat, a hunt, or ritual ceremonies they also wear detailed
white, grey or blue face and body paint. The body paint acts as camouflage,
but also to intimidate their enemies.

Common male names are: Aeron, Aled, Andras, Arwyn, Bryn, Cadell, Dylan,
Eirian, Gareth, Glyn, Eifan, Ivor, Maldwyn, Owain, Rhys, Tegid, Wyn

Common female names are: Aerona, Anwen, Bethan, Blodwen, Branwen, Cadi,
Catrin, Deryn, Eira, Elin, Gwendoline, Gwenith, Llinos, Mairwen, Nia,
Siana, Tegwen, Wynne


%% The Great Divide
\subsection{Great Divide}
\label{sec:Great Divide}

The continent of \nameref{sec:Eilean Mor} is split in half, from the north west,
to the south east, by a large mountain chain called the \emph{Great Divide}.

Many dwarven, deepkin and especially dark elven clans call the mountains their
home, and have built small underground cities and settlement in the vast caves
of the mountain chains. The mountains are also known for being home to many
beast races, such as tribes of orcs and goblinoids.

Many rivers spring forth from the massive glaciers that top the great divide,
feeding hundreds of lakes as well as making the land fertile both to the north
and the south. South of the great divide lies a vast boreal forest called the
\nameref{sec:Dirgewood}.

The mountain chain has not only divided the land in half, but has also been
a major divider on the continent in the culture of the people. Since the north
of Eilean Mor has no major city kingdoms except \nameref{sec:Tredegar}, it has
always been viewed as more uncivilised than the southern region. This of
course holds no longer true, but that sentiment is still stuck with the people
of the south. So much so that the phrase \emph{``north of the divide''} has
become a moniker to denote someone who lacks manners.

Still the mountain range is a huge geographical obstacle, and only a few
dangerous mountain paths lead safely through the mountains. Since these paths
are often waylaid by monstrous bandits, rarely anyone ever crosses these
mountains, unless they absolutely have to. Most of the baronies and small
cities north of the great divide instead trade with the south, and the rest
of the world, through the sea.

The northern tip of the great divide is currently claimed by
\nameref{sec:Norbury}, who owns and operates a large castle in the mountain
range called castle \emph{Rothorn}. The castle oversees several mining
operations based in the mountain range producing ore such as copper and silver
for the kingdom.

The southern end is owned by \nameref{sec:Tredegar}, and is heavily mined
by various dwarven clans for the city kingdom. The iron and adamantine ore
from the mountains is heavily used by the kingdom to make excellent arms and
armour.


%% Silver Isles
\subsection{Silver Isles}
\label{sec:Silver Isles}

In the centre of the vast sea that separates all continents lies a vast
archipelago called the \emph{Silver Isles}. A large collection of small
islands that stretch out into all directions, and are loosely connected to the
continent of \nameref{sec:Goltir} by a land bridge. Almost all of the islands
either lie on, or very close to the equator of \hyperref[sec:Aror]{Aror}. Most
are thus tropical paradises, with white beaches, palm trees, clear water and
covered with thick and lush rain forests.

While not technically a city kingdom, the region itself holds vast power,
both politically and commercially. The islands are rich with all manner of
natural resources, including silver, iron, copper, gold and Everblack. The
people of the isles (often simply called ``islanders'') have organised
themselves by establishing many large cities with vast trading empires,
as well as smaller ports, harbours and settlements all around the archipelago.

\subsection{Masia}
\label{sec:Masia}

These settlements are numerous, but one large harbour port sits in the centre
of it all: Masia. With about 400,000 citizens it is the largest, most
prosperous of all the harbour towns of the isle. It is not recognised as
independent as it is owned by the \nameref{sec:Ror-Aram Trading Corporation},
with \nameref{sec:House Ranian} holding a large stake in the settlement. The
city is thus ruled by a steward, that oversees that the city turns a profit
for the trading corporation. The city is known as haven for smugglers, pirates,
slave trade, but also for unfathomable riches, in which all sorts of luxury
goods - such as tobacco, cigars, coffee and chocolate, are sold.

Common male names are: Agustin, Alfonso, Alsen, Alvaro, Benito, Damian, Diego,
Enrique, Felipe, Gabriel, Garcia, Leon, Manuel, Martin, Ramon, Salvador, Tello,
Valeriano, Ydalla

Common female names are: Ana, Angel, Clara, Elena, Elvira, Fatima, Felipa,
Francisca, Gracia, Juana, Luzia, Manuela, Olalla, Serena, Ynes, Ysabel

\subsection{Fleeting Alliances}

Most smaller harbours, and port towns are usually independent ventures that
live off their agricultural products, as well as mining the mountains of
the islands for their riches. Their allegiances, political, cultural and
commercial significance vary greatly over time, and many attempt to sabotage
their competitors or even try to expand their operations by taking over other
islands. Some do it through the vague veil of threats, sabotage and political
intrigue, while others do it through threats, violence and outright war.

The major players involved in the region are Ror-Aram Trading Guild, House
Ranian, the \nameref{sec:Velvet Hand} as well as the mysterious
\nameref{sec:Silver Hand}. Thus some ports are nothing more than slaver
holdouts, while others are smuggler ports or straight out pirate havens.

Even though the area is often in disarray due to political chaos, struggle our
outright conflict between varying factions, the people of the isles have often
proven that they can work together should the need arise. As was the case during
the Devil siege of \nameref{sec:Forsby}, when the islands offered their highly
secured bays and harbour facilities as a staging ground for a counter strike.
Many of the native cities and ports, including Masia, even joined the counter
attack and provided valuable resources, weapons, experienced sailors and ships.

Keeping the smaller ports and harbour towns separated and in constant conflict,
unable to rally around Masia to form a kingdom, is the main goal of the larger
city kingdoms as well as \nameref{sec:House Ranian} and the Trading Guild. The
divide and conquer tactics have been successful in keeping the people of the
isles from becoming a major power on Aror for several centuries.


%% Yua'cata
\subsection{Yua'cata}
\label{sec:Yuacata}

Yua'cata is a vast rain forest on the main land of \nameref{sec:Karnak}. It is a
wild and savage place, untouched by civilisation. The rain forest surrounds the
vast fresh water lake of \emph{Mu'ut}.

Yua'cata is home to thousands of different species of animals, such as birds,
insects, reptilians as well as predatory felines and snakes. The warm climate
and the abundance of water and resources lead to a diverse and large biome.

\subsubsection{Mu'ut}
\label{sec:Muut}

\emph{Mu'ut} is a large fresh water lake in the centre of the continent of
Karnak. It is geologically active, and its warm water fuels thousands of
smaller and bigger rivers that flow in all directions towards the sea. Mu'ut
can be considered the heart of the Yua'cata rain forest, as its warm waters
fuel the tropical climate, rain and thus the vast rain forest of Yua'cata.

\subsubsection{Tribes}

The jungle is also inhabited by thousands of tribes of sentient monstrous and
humanoid races, including humans, \hyperref[sec:Wood Elves]{wood elves},
\hyperref[sec:Medusa]{medusa}, as well as lizardfolk and sahuagin. Most of
these are hunters and gathers, and often differ greatly in traditions, customs
and believes. There is one cultural element that many tribes share however: an
almost holy reverence of the lake of Mu'ut.

\subsubsection{Savage Elves}
\label{sec:Savage Elves}

The most notorious of the jungle tribes are the wood elven tribes of the
Xanarian river. Among other races these wood elves are often referred to as
\emph{savage elves}. They worship an ancient being called \nameref{sec:Xir}
who they believe sleeps beneath the surface of the holy rivers of Mu'ut. Xir
teaches that all unbelievers are impure, and can only be redeemed by feeding
them to him by drowning them in the river. These wood elves raid, pillage,
murder and sacrifice other sentient races they come across without mercy, as
in their eyes there is only one way for them to be redeemed in the eyes of
their lord.

The Xanarian tribes also practice cannibalism. They believe that eating their
own kind after death makes their deceased continue to live within them. They
rarely eat outsiders, as they believe their flesh to be corrupt and impure.



\section{Time Keeping}

A year on \emph{Aror} takes roughly 385 days. Years are prefixed with an
\emph{Aeon}, which are only changed in case of a world-shattering event
or shift, or in case the year numbers become unwieldy. As of the last
edit of this book, the current Aeon prefix is \emph{MI} which stands for
the Aeon of Midaris. In official records you will also find prefixes for
which calendar is used. For example \emph{S/20/05 MI:2002} denotes the
twentieth day of the fifth month in the year 2002 in the Aeon of
Midaris, as described by the \emph{Storst} calendar.

The last \emph{aeon} is now called ``gamla tiden'' (or the ``old age'' in old
teranim), and it began almost three and half thousand years ago (3411 to be
precise). Its prefix is \emph{GT}.

\subsection{Old Calendar}
\label{sec:Old Calendar}

If you use the orbit of the bigger moon \emph{Lilest} to create a
calendar, you get twelve (12) months, with thirty-two (32) days
each. A month is then often broken down to four (4) weeks with eight
(8) days each. Although the \emph{Lilest} based calendar was used
preliminary in the norther hemisphere, it has fallen out of favour and
has been replaced with the \emph{Storst} calendar by MI:1000. Many
cultures still use the old calendar, but most cultures of relevance
have made the switch to the new calendar.

\subsection{New Calendar}
\label{sec:New Calendar}

A calendar utilising the rotation of the moon \emph{Storst} is favoured
in the southern hemisphere. Since most arcane and scientific studies are
done in the southern hemisphere, it has been the de facto calendar of all
of Aror since a few decades.

It separates the year into eight (8) months, with forty-eight (48) days
each, which is then again subdivided into four months with twelve (12)
days each.

% Put timeline on a new page
\newpage
\begin{landscape}

  \hspace{-7cm}
  \subsection{Time Line}

  \subsubsection{GT}

  \startchronology[startyear=0,stopyear=3411,
    align=center,width=\paperwidth,arrow=false,dates=false
  ]
  % GT:0 or MI:-3411 - Fes al-Bashir
  \chronoevent[markdepth=40px]{0}{\nameref{sec:Fes al-Bashir}}

  % GT:339 or MI:-3072
  \chronoevent{339}{\nameref{sec:Hraglund}}

  % GT:890 or MI:-2521
  \chronoevent[markdepth=40px]{890}{\nameref{sec:Esmayar}}

  % GT:984 or MI:-2427
  \chronoevent{984}{\nameref{sec:Helmarnock}}

  % GT:1148 or MI:-2263
  \chronoevent[markdepth=80px]{1148}{\nameref{sec:Forsby}}

  % GT:1310 or MI:-2101
  \chronoevent[markdepth=120px]{1310}{\nameref{sec:Third Order}}

  % GT:1551 or MI:-1860
  \chronoevent[markdepth=40px]{1551}{\nameref{sec:Stenheim}}

  % GT:1849
  \chronoevent[markdepth=80px]{1849}{\nameref{sec:Norbury}}

  % GT:2084 or MI:-1363
  \chronoevent{2084}{Birth of Graham Balance}

  % GT:2992 or MI:-419
  \chronoevent[markdepth=40px]{2492}{\nameref{sec:Kesmar}}

  % GT:3391 or MI:-20
  \chronoevent{3391}{\nameref{sec:Holy Crusade} begins}

  \endchronology

  \subsubsection{MI}

  \startchronology[startyear=0,stopyear=2100,
    align=center,width=\paperwidth,dates=false
  ]

  % In inverse order so markers do not overlap
  % MI:263
  \chronoevent[markdepth=50px]{263}{\nameref{sec:Umgeher} are created}
  % GT:3421 or MI:0 - Death of Griannar
  \chronoevent{0}{
    Holy Crusade ends, death of \nameref{sec:Griannar}
  }

  % In inverse order so markers do not overlap
  % MI:890
  \chronoevent[markdepth=80px]{890}{\nameref{sec:Holy Order of Aleaste}}
  % MI:782
  \chronoevent{782}{%
    \nameref{sec:Dragon Teleporter} discovered
  }

  % In inverse order so markers do not overlap
  % MI:1260 Foundation of El-Fayam
  \chronoevent[markdepth=40px]{1260}{\nameref{sec:El-Fayam}}
  % MI:1210 Fall of Esmayar
  \chronoevent{1213}{Fall of \nameref{sec:Esmayar}}

  % In inverse order so markers do not overlap
  % MI:1782
  \chronoevent[markdepth=114px]{1682}{Hraglund - Terevar war}
  % MI:1780
  \chronoevent[markdepth=76px]{1680}{\nameref{sec:Plague of Hraglund} of Hraglund}

  % In inverse order so markers do not overlap
  % MI:2017
  \chronoevent[markdepth=152px]{2017}{\nameref{sec:Devil Siege} of Forsby}
  % MI:1938 Morkan
  \chronoevent[markdepth=44px]{1938}{\nameref{sec:Morkan}}
  % MI:1920 Fall of Nen-Hilith
  \chronoevent{1920}{Fall of Nen-Hilith, \nameref{sec:Avenfjord}}

  \endchronology

\end{landscape}

\newpage

% And history
\section{History of the World}
\label{sec:History}

Proper recording of history started shortly before the foundation of the
first city kingdom of \nameref{sec:Fes al-Bashir}. The earliest scholars
were able to accurately reconstruct the earliest history with the help of
the Aurelis and the Dragons. These two races were advanced enough in the
early aeons of the humanoids of Aror, to accurately chronicle the history
of the core humanoid races.

The first race to settle Aror were the dragons, who found it uninhabited
of intelligent life, yet found that great and mighty beasts roamed the
planet. They opted to stay on the barren continent of \nameref{sec:Draigynus},
as to not disrupt the natural evolution of life on the planet. Their peaceful
life came to end, when the giants invaded Aror to wage war against the dragons,
and created intelligent monstrous creatures out of the beasts that roamed the
world. The giants engineered these monstrous races to be intelligent, and
aggressive in an attempt to create the perfect soldiers for their campaign
against the dragons.

According to the dragons, the \nameref{sec:Tynrikke} where elevated to
intelligence by the giants yet revolted against them. The Týn instead joined
the dragons in their battle for survival, and would be instrumental in
driving the giants from Aror.

While the war on the surface raged between the dragons and the giants,
the \nameref{sec:Ilians} became ever more intelligent and organised, resulting
in the creation of the great Ilian civilisation. Unlike the other intelligent
races at the time, who either came from other planes or were engineered, the
Ilians were the first intelligent life that was native to Aror.

\clearpage
\onecolumn
\begin{longtable}{@{\,}r <{\hskip 6pt} !{
  \color{LightSteelBlue3}
  \makebox[0pt]{\textbullet}
  \hskip -2.6pt
  \vrule width 1pt
  \hspace{\labelsep}
  }
  >{\raggedright\arraybackslash}p{11cm}
}
    %% Actual entries
    ca. GT:-800.000  & Dragons settle Aror. \\
    ca. GT:-600.000  & Giants first land on Aror to fight the dragons, and
                       populate the planet with engineered monstrous races \\
    ca. GT:-500.000  & The \nameref{sec:Tynrikke} build a great civilisation,
                       which lasts for roughly four thousand years, before it
                       falls for unknown reasons. Their temples, hallowed
                       shrines, and teleportation magic between Goltir and
                       Eilean Mor can still be found to this day. \\
    ca. GT:-480.000  & The \nameref{sec:Ilians} become intelligent and rapidly
                       develop, building a great civilisation beneath the
                       surface of Aror. \\
    ca. GT:-200.000  & The anatomically modern core humanoid races develop \\
    ca. GT:-60.000   & \nameref{sec:Second Ice Age} covers Aror, humanoid races
                       settle all continents \\
    ca. GT:-9.000    & Population boom on Aror, core humanoid races and
                       monstrous races struggle resources. The so called
                       \nameref{sec:Aeon of Strife} begins. \\
    ca. GT:-8.300    & The great \nameref{sec:Schism} occurs, where a sizeable
                       number of core humanoid races flee beneath the surface,
                       to avoid the battles of the strife. \\
    ca. GT:-8.000    & \nameref{sec:Vampires} are created, leading to an event
                       called the ``great betrayal'' in which the deity
                       \nameref{sec:Morana} falls from grace. \\
    GT:-7.633        & The great \nameref{sec:War in Heaven} begins between the
                       \nameref{sec:Aurelis} and the \nameref{sec:Scourge}. \\
    GT:-7.600        & Aurelis succeed in freeing some demons from the Scourge.
                       \nameref{sec:Devils} are born, and settle a new planet
                       called the ``layers of hell''. \\
    ca. GT:-7.000    & The great war in heaven is won by the Aurelis, leaving
                       most of their people dead, their culture destroyed, and
                       their planet in ruins. \\
    GT:-1400         & The deepkin invent the \nameref{sec:Everblack Golem},
                       creating the ultimate weapon against the Ilians. \\
    GT:-550          & \nameref{sec:Eigyr} is born on the continent of
                       \nameref{sec:Eilean Mor}. \\
    GT:-390          & Eigyr conquers all major monstrous cities, razing them,
                       and establishing her empire. This year marks the end of
                       the aeon of strife on \nameref{sec:Eilean Mor}. \\
    GT:0             & \nameref{sec:Fes al-Bashir} officially declares itself a
                       kingdom, marking the age of the city kingdoms. \\
    GT:24            & \nameref{sec:Aren Fel} is born. \\
    GT:339           & \nameref{sec:Hraglund} is recognised as a city kingdom
                       out of the remains of Eigyr's empire. \\
    GT:551           & \nameref{sec:Nen-Hilith} is recognised as a city state. \\
    GT:984           & The baronies of \nameref{sec:Helmarnock} found a kingdom
                       under the banner of the noble lineage of Helm. \\
    GT:1148          & \nameref{sec:Forsby} wins a major battle against the
                       monstrous races, driving them almost completely from
                       their land. They become recognised as a city kingdom. \\
    GT:1200          & The \nameref{sec:Great Exodus} begins, as many dwarves,
                       elves, deepkin, and halflings return to the surface
                       after failing to secure peace in the depth against
                       the \nameref{sec:Ilians}. \\
    GT:1310          & The largest cross-kingdom knight order called the
                       \nameref{sec:Third Order} is founded. \\
    GT:1551          & \nameref{sec:Stenheim}, after fighting countless battles
                       against competing large dwarven clans, is recognised as a
                       city kingdom. \\
    GT:1849          & \nameref{sec:Norbury} fights for its independence against
                       Hraglund and wins, becoming a city kingdom. \\
    GT:2084          & \nameref{sec:Graham Balance} is born. \\
    GT:2992          & \nameref{sec:Kesmar} is recognised as a city kingdom. \\
    GT:3391          & \nameref{sec:Holy Crusade} begins, seeing fighting
                       between the forces of \nameref{sec:Silent Queen} and
                       \nameref{sec:Griannar}. \\
    GT:3408          & After countless dead on both sides, the holy crusade ends
                       at the siege of Hraglund. Church of Griannar is but
                       extinct on Aror. \\
    GT:3411 MI:0     & Griannar disappears, all his clergy lose their power. \\
    MI:4             & Silent Queen disappears, her power and throne as a deity
                       absorbed by her high-priestess \nameref{sec:Aria}. \\
    MI:263           & Vampires of Helmarnock create \nameref{sec:Umgeher}. \\
    MI:782           & The first \nameref{sec:Dragon Teleporter} is discovered
                       ushering in an unprecedented age of travel, trade and
                       cultural unity. \\
    MI:890           & \nameref{sec:Holy Church of Aleaste} splits away from the
                       main church of the \nameref{sec:Order}. \\
    MI:1213          & \nameref{sec:Esmayar} is sacked by the gnolls. \\
    MI:1260          & \nameref{sec:El-Fayam} is recognised as a city state. \\
    MI:1680          & \nameref{sec:Plague of Hraglund} almost wipes out
                       Hraglund, followed by devastating wars against the
                       weakened city kingdom two years later. \\
    MI:1918          & Giants return to continue their war against the dragons.
                       Elves and halflings join the war against the giants
                       invaders. \\
    MI:1920          & \nameref{sec:Nen-Hilith} falls to a siege by the giants.
                       The elves and halflings flee north to Goltir and found
                       \nameref{sec:Avenfjord}. \\
    MI:1938          & \nameref{sec:Morkan} is recognised as a new kingdom. \\
    MI:1952          & Giants of Farlar create the \nameref{sec:Diarim}. \\
    MI:2017          & Devils invade North Goltir, laying siege to the city
                       of \nameref{sec:Forsby}. \\
  %%\rowcolors{1}{white}{light-grey}
  \captionsetup{labelformat=empty,font={large,bf}}
  \caption{Historic Events of Aror}
\end{longtable}
\twocolumn
\clearpage

\subsection{Second Ice Age}
\label{sec:Second Ice Age}

Roughly sixty thousand years ago an ice age covered most of the world
of \emph{Aror} in vast sheets of ice and snow. Continents that were separated
by sea, where now suddenly accessible through thick sheets of ice. This
allowed the \textbf{four core humanoid races}, the
\hyperref[sec:Humans]{humans}, \hyperref[sec:Elves]{elves},
\hyperref[sec:Dwarves]{dwarves} and \hyperref[sec:Halflings]{halflings}
to migrate away from \nameref{sec:Arania} and settle the entire known
world. Although the humanoid races were successful in adapting to the new
lands and challenges, they had to face fierce opposition.

\subsection{Aeon of Strife}
\label{sec:Aeon of Strife}

When the humanoid races arrived in the other continents around a thousands of
years before the aeon of \emph{GT}, they found that they were already
inhabited by sentient races. \textbf{Monstrous races} such as ogres,
minotaurs, orcs, trolls, hobgoblins, gnolls, bugbears already called these
continents their home, alongside non-sentient monstrous species such as
hydras, manticores and wyverns. Through ancient stories of the dwarves, early
writings, ancient stone tablets, tales from the \hyperref[sec:Dragons]{dragons},
and even cave paintings, it was revealed that the humanoid ancestors were in a
constant state of conflict with these monstrous races. Wars, skirmishes, and
often the destruction of entire early villages and even cities was a common
occurrence in early history. This early time of constant fighting, turmoil and
wars against the sentient beast races is now referred to as the \emph{aeon of
  strife}.

The aeon saw the birth of many new races, such as
\hyperref[sec:Vampires]{vampires}, \hyperref[sec:Fey]{fey} and
\hyperref[sec:Lycanthropes]{lycanthropes}. It also saw a vast destruction of
nature, as both sides of the war cut down forests, dried swamps, and forever
corrupted landscapes and areas with dark and foul magic. The aeon of strife
lasted well into the aeon of \emph{GT}.

Even though the aeon of strife is now hundreds of years in the past, it is
still vividly remembered in both humanoid and monstrous cultures through
stories, song, and tradition.

\subsection{Schism}
\label{sec:Schism}

The ultimately successful survival strategy to deal with such a hostility
during the strife ingrained itself as the cultural core of the humanoid
races. Many dwarves went underground and organised themselves in strict
hierarchical clans and cities, that allowed them to optimise their economies to
the harsh realities and lack of resources of the depths. Some humans and elves
followed the dwarves underground but ultimately failed to replicate their
success. With a few notable exceptions, such as \nameref{sec:Stenheim}.
Although the \nameref{sec:Deepkin}, the underground dwelling cousins of the
humans, built large civilisations underground they were ultimately defeated by
the \nameref{sec:Ilians} of the depths and driven to the surface. The dark elves
instead relied on small clans, families and by being constantly on the move to
ensure the survival of their species.

Meanwhile on the surface elves and halflings sought to settle as far away from
the monstrous races as possible, leading them to the continent of \emph{South
Goltir} and \emph{Farlar}. Other elves settled in the frozen north and south,
becoming known as the \nameref{sec:Snow Elves}.

Humans on the other hand used their ingenuity and skill to build great
civilisations and cities that could potentially withstand the skirmishes and
sieges of the sentient monstrous races. They were not alone, as many elves
(now known as \nameref{sec:Wood Elves}), and some halflings and dwarves stayed
with them. Through many iterations over the course of thousands of years,
which resulted in countless destroyed and ransacked cities and fallen
civilisations, these humanoids have now achieved the unthinkable: dethrone the
monstrous races as the predominant species across all of Aror.

\subsection{Great Exodus}
\label{sec:Great Exodus}

Now many elves and halflings have joined the human effort of building large
centres of civilisation, enriching the predominantly human city kingdoms that
dot the world of \emph{Aror}. Their struggle against the sentient monstrous
races is far from over, especially in the central regions of the continents,
or from the still predominantly monstrous continent of \emph{Iâfandir}. The
majority of dwarves have remained underground, continuing their strict ways of
life as it has served them for aeons. A success that was not shared by the
dark elves and deepkin, who have mostly abandoned the deep and returned to the
surface

\subsection{War in Heaven}
\label{sec:War in Heaven}

The ``War in Heaven'' (also known as the ``great war in heaven'') started in
GT:-7633 (almost 12,000 years ago), when the \nameref{sec:Scourge} successfully
gained a foothold on the \nameref{sec:Aurelis} home world of Celestis. This war
raged for over six hundred years, killing countless of Aurelis, destroying their
culture, society and leaving half of their planet uninhabitable. During the war,
the Aurelis attempted to free some of the demons from the psionic slavery of the
\nameref{sec:Abyss}. These newly freed demons refused to aid the Aurelis, and
instead left to find themselves a new homeworld. These freed demons became known
as the ``pacru\'i'' or \nameref{sec:Devils}. During the war, the Aurelis' king
\nameref{sec:Lor} made himself known to the people of Aror, in an effort to
recruit external help for his war.

\subsection{First City Kingdoms}

The tide of the aeon long war against the monstrous humanoids turned in favour
for the humanoid races when the first large city kingdoms were founded. These
huge and vast cities provided defence, security and above else a place where
society, culture, and means of production could grow and flourish. The first
city kingdom to be founded was \nameref{sec:Fes al-Bashir}, and its success
in the battles during the aeon of strife, was soon copied across the world.
The early years of what is now the the aeon of \emph{GT}, saw the formation
of many powerful nations, such as \nameref{sec:Hraglund}, \nameref{sec:Esmayar}
and \nameref{sec:Helmarnock}, which helped cement the humanoid victory and
conquest with brick and stone.


\section{A Global World}

The world of \hyperref[sec:Aror]{Aror} has a few major city kingdoms that are
vastly populated centres of society and civilisation. These immense cities are
the major power players in the world, and often have power over the fates of
many million people. Not only in their own realm, but also in the lands,
baronies and smaller kingdoms onto which they project their power.

Knowledge about the other city kingdoms is wide spread, and almost all people
have a rough understanding on how big the world is, and what culture lives
where. Although travel is only feasible for the middle and upper classes, word
about foreign lands and the intricate politics of the city kingdoms spreads
easily to every corner of the land.

These large city kingdoms are also connected with each other with large
\hyperref[sec:Dragon Teleporter]{teleporters}, that were originally invented
by the dragons to inter connect places of interest on the planet to their home
continent of \nameref{sec:Draigynus}. This lead to the large city kingdoms to
become more and more interconnected. It helped spread the clean, high version
of \emph{Teranim} to be spoken all over the world as a lingua franca. Now
they are often seated centrally within the kingdom, and everyone can purchase
tickets to be transported to another city kingdom of their choosing. Prices
for tickets often range from five to twenty \hyperref[sec:Shin]{shins}. For
almost all city kingdoms the teleporters are a net loss commercially, but they
are aware that the trade, and commerce (and thus taxes) they help facilitate is
invaluable to the economy at large.

The dragon teleporters were designed to transport living creatures (dragons)
across great distances, and are almost incapable of transporting material
goods. One teleport can transport one person plus their light equipment, or
perhaps ten kilograms of innate material (such as armour, weapons and
clothes) at a time. This design choice was deliberate by the dragons, as they
do not wear, or own that much innate objects; and thus have designed their
teleporters to transport themselves safely and efficiently across vast
distances.

Many have feared that the trade by sea or land would become obsolete since
teleporters were introduced. But the dragons did not share their blueprints,
and the knowledge about the dragon teleporters is still actively researched
and reverse engineered to this day. Their design limitation and relatively
high power consumption has made them a bad economical alternative for
transporting goods.

\subsection{Global Trade}
\label{sec:Trade}

All city kingdoms and the majority of baronies mint their own coins, but they
are often only used locally. Although their value is held by precious metals
they contain, they have a strong variance in terms of metal purity, weight and
thus actual value. This made conversion of one local coin or currency to
another difficult. To make global trade and commerce easier, two new
currencies were introduces: the \emph{shin} and the \emph{shard} which are
made out of \hyperref[sec:Everblack]{everblack}.

Many banks offer a service to weigh, assess and convert local coins and
gemstones to shin and shard. And almost all baronies and kingdoms accept shins
and shards as payment method. Large purchases, such as properties or ships, are
mostly done in shards, or perhaps in bars of pure gold. Although crystal
everblack is more easily shattered and destroyed than gold, it is lighter
to carry and it cannot be as easily diluted or forged. Manipulating everblack
takes a highly skilled arcane wielder or scientific scholar, and thus cannot
be forged or manipulated like other ore or metals.

Thus many traders might simply refuse to take foreign coin, or even local
coins, preferring to deal with the more save shins and shards of everblack.
Assessing the purity of large amounts of gold or silver coins can be costly
and time consuming, and thus traders that deal with expensive items might
prefer the safety and convenience of the everblack based currency.

\subsubsection{Shin}
\label{sec:Shin}

Small chips of \hyperref[sec:Everblack]{everblack}, roughly a centimetre in
diameter - also referred to as \emph{shins} - are a common currency accepted
everywhere on the world of Aror. These are often useless for magical
applications due to their small size, but still hold material value. They are
roughly equivalent to one copper coin and are used to make everyday purchases.

\subsubsection{Shard}
\label{sec:Shard}

Bigger sticks of everblack, roughly five centimetres long, one centimetre thick
and perhaps one centimetre wide are called \emph{shards}. They are used in the
global economy of Aror for expensive purchases such as building projects,
property and artefacts. Shards are roughly equivalent to ten coins of gold.

\section{Slavery}
\label{sec:Slavery}

Slavery is a sad reality among many city kingdoms, nations, baronies and
tribes of Aror. Many nations and kingdoms allow their citizens to own
sentient monstrous races, or even other humanoids. Slaves often do
hard and menial labour, or even more dangerous tasks such as fighting
as front line troops, work in extreme conditions, or mine the poisonous
\hyperref[sec:Everblack]{everblack}.

To many of these city kingdoms slavery is an integral part to the economy,
and the economic pressure to keep maintaining such an inhumane system
is relatively high. For example \nameref{sec:El-Fayam} would not be able
to flourish, without the slaves toiling away beneath the city to mine
the dangerously toxic everblack.

In most of these systems the slave is nothing more than property. A slave
can be sold, purchased, stolen and property damage is to be paid should
someone hurt the slave of another. The detailed laws may vary based on
the specific kingdom and nation, but often recognise the right of citizens
of another nations to their slaves.

Almost all slaves are registered similar to citizens and there are often
official organisations involved in trading, capturing, and tracking down
slaves that have fled. One such organisation is the \nameref{sec:Hunters
  Guild} of \nameref{sec:Norbury}, or the \nameref{sec:Velvet Hand} of
\nameref{sec:Fes al-Bashir}. These organisations often hire slave hunters,
slavers, task masters and other personnel required to maintain the slave
system within a city. They also offer administrative services, such as looking
up slave registrations, transfers of ownerships and often operate slave
markets and auctions where new slaves are sold.

All slaves are marked as such but the methods vary greatly. Some slaves
are just branded as slaves, often behind the ear or at the back of neck.
Other kingdoms use more sophisticated measures, such as the
\nameref{sec:Slave Band} or \nameref{sec:Slave Mark}. These magical
chains are then often keyed to one or more \nameref{sec:Master Ring} which
are worn by the owners of the slave.

\subsection{Indentured Servitude}
\label{sec:Indentured Servitude}

The system of indentured servitude is practised in many city states, even
those that ban slavery. It is a punishment often placed upon criminals found
guilty of property crimes such as property damage, theft or defaulting on
debts. The criminal is convicted, and then forced to work off his damage in
service to the person or institution to whom the damage was done. These
punishments are often limited in time, but may result in actual prison
sentences if the convict attempts to flee from his duties.

\subsection{Regulated Slavery}
\label{sec:Regulated Slavery}

It also varies greatly between the nations and city kingdoms how slavery is
encoded by law. Most city kingdoms do have laws regarding the treatment of
slaves, such as requiring them to be fed and clothed, or inflicting punishment
to owners should they wilfully neglect or kill their own slaves. Some systems
even encode conditions how children of slaves are to be treated, and whether
they are free or not. And often also encode conditions on how a slave can free
themselves from their bondage. This form of slavery is often called
\emph{regulated slavery}. This system is practised in \nameref{sec:Helmarnock}
and \nameref{sec:Kesmar}.

\subsection{Unregulated Slavery}
\label{sec:Unregulated Slavery}

A rather rare form of slavery is \emph{unregulated slavery}. In this legal
system slaves have no rights whatsoever, and are the complete mercy of their
owner. They are still counted as property within the legal system, and thus
can be sold and bought. But the owner has no obligations towards the slave.
This system is practised in \nameref{sec:Norbury} as well as
\nameref{sec:Morkan}, and by the gnolls of \nameref{sec:Esmayar}.

\subsection{Vonir Accord}
\label{sec:Vonir Accord}

The \emph{Vonir Accord}, named after king Sigmund Vonirson of Norbury, is a
treaty signed by almost all major city kingdoms and most bigger baronies of
the north. In exchange for immunity to enslavement of the signing parties'
citizens, all Norbury slaves found on foreign soil must be returned to
Norbury. Further all signing parties must acknowledge the owner's claim over
his slave and not interfere with his rights to his property. This treaty
also allows members of the \nameref{sec:Hunters Guild} to search for, and
apprehend escaped slaves on the land of signing parties.


% City Kingdoms
\section*{City Kingdoms}

The world of \emph{Aror} is covered in a few huge and ancient human kingdoms
that exert their influence and power from vast cities. These cities often house
millions of humanoid creatures, and often the kingdom's influence and power
extended outward to surrounding areas.

All of these city kingdoms are connected together with \emph{dragon
  teleporters}, which allows travel and trade between the kingdoms. They are
often a beacon of stability and security. Surrounded by wild nature, unstable
baronies, and small and greedy earldoms that vie for power and wealth.

Many of these city kingdoms have their own unique culture, local language
dialects, their own laws and customs, and often also unique governments. However
most of these kingdoms are ruled by humans, where other races often take a
secondary role as minorities within the general populace. Still the mixture
of races, backgrounds and various cultures often add a unique touch to the
whole.

In many cases the town or city kingdom you came from defines you more than
your race. A \emph{snow elf} born in \emph{Avenfjord} will have come to grow
up with that cities culture, most likely favouring the arts over the hunting
bow.

All of these city kingdoms, as well as many bigger baronies, have a system
where they register citizens and also issue official documents to their
citizens. These are often copper rings for common citizens, or even official
documents of identification in case of officials or noble houses. Being a
citizen of one city kingdom not only grants considerable benefits in your
home, but often also grants you special status or privileges in allied city
kingdoms.

\subsection*{Baronies and Earldoms}

In between most of these city kingdoms stretches out a vast network of smaller
baronies and earldoms. Most of these are too insignificant to mention, and the
unstable nature of small rulings makes tracking them through time a far too
tedious task to be worthwhile for a book of this scope.

\aren{Perhaps in a separate book. With three thousand pages...}

\begin{note}
  The region between city kingdoms is left blank intentionally so that dungeon
  masters and players have leeway to design their own baronies and villages.
\end{note}

%% City Kingdom of Avenfjord
\subsection*{Avenfjord}

The rebuilt kingdom of the high elves and halflings sits on the southern shore
of the continent of \emph{Goltir}. It settled in the vast and fertile river
delta of the \emph{Al'ahri} river. Avenfjord is one of the youngest of all the
city kingdoms, and also the smallest.

\subsubsection*{Banner}

The banner of Avenfjord depicts two large crowned towers built together by
a bridge, with a backdrop of blue (for the sea) and green (for the fertile
farm land).

\subsubsection*{History}

It was founded in MI:1920 when the giants, an extra planar race of towering
humanoids, destroyed the previous city of the halflings and elves called
\emph{Nen-Hilith}. They had invaded the continent of \emph{Farlar} thirty
years prior, to wage war against the dragons who also call that continent
their home. \emph{Nen-Hilith} was situated on the northern shores of
\emph{Farlar} (just across the sea from where Avenfjord stands now).

For many years the elves and halflings were untouched and neutral in the
war between the dragons and the giants. But as the giants had seized most of
the rivers and lakes south of \emph{Nen-Hilith} and the fighting had crept
north towards the outlying villages and towns the elves and halflings joined
the war on the side of the dragons in MI:1912. After several failed campaigns
to regain control over the city's water supply, the giants began a devastating
siege against the city in MI:1918. After enduring the siege for more than two
years, with the military aid of various allies, the starved and weakened elves
and halflings were forced to abandon \emph{Nen-Hilith} and flee across the sea
to the north. The giants then razed the city to the ground.

\subsubsection*{Formian War}

In MI:1936 a war broke out between a hive of formians who objected to the
elves settling in their lands. The kingdom was still in the early stages of
rebuilding, and the new war threatened the very existence of the nation. The
then ruler of Avenfjord, King Ishmael the II., allowed a travelling wizard
named \emph{Taras} to combat the formians by adapting the \emph{black blight}
to infect and weaken them. Going far beyond anyone's expectation, the modified
plague infected and killed the vast majority of formians living in the Goban
desert north of \emph{Avenfjord}. After realising that he had ordered genocide
upon an entire sentient species \emph{King Ishmael II} exiled \emph{Taras} and
committed suicide. His son \emph{King Ishmael III} took the throne soon after.

\aren{If he just had enough resolve left to take Taras with him...}

\subsubsection*{Culture}

Although one might suspect that the near destruction would shake the culture
of the elves and halflings to the core, you might be wrong. Instead the
formian war, and the destruction of their previous city hardened the nation's
stance of non-interference. The people of Avenfjord prefer not to meddle in
other people's issues and troubles, and prefer diplomatic solutions over
conflict.

Avenfjord are known for their generous patronage for the arts and sciences and
house many theatres, libraries, galleries and workshops. They fund one of the
largest arcane academies on \emph{Aror} and generally hold good relations with
most other city kingdoms. The elves and halflings of Avenfjord are often
described as jovial, carefree but creative and expert diplomats. They are an
open society, and especially welcome other artists and craftsmen into their
cities.

\subsubsection*{Population}

The city of Avenfjord is now home to roughly two million people, yet the city
of \emph{Nen-Hilith} was home to almost twenty million at its peak. Most of
the citizens are elves (48\%) and halflings (39\%), with dwarves following
third (8\%) and all other races making up the remaining 5\%.

\subsubsection*{Rule}

Avenfjord is a dual monarchy, where both the reigning king or queen of the
elves, and the reigning monarch of the halflings rule together. Neither of
the monarchs has absolute reign alone. Although Avenfjord holds houses of
nobility, their power and influence is minimal compared to the houses of
other city kingdoms. Slavery and serfdom is outlawed, and all citizens of
Avenfjord are free people. The city also has an independent court and police,
that enact the laws the monarch sign into law fairly.

\subsubsection*{Relations}

Avenfjord has not signed the \emph{Vonir Accord} with Norbury. Although this
puts citizens of Avenfjord at risk of enslavement should they travel to
Norbury, such cases are extremely rare. Neither Norbury nor Avenfjord wish
to raise diplomatic issues between the two city kingdoms.

Though traditionally heavily aligned with the city kingdom of Forsby, the
failure to send military aid that was promised during that cities siege
soured relations with most city kingdoms, but with Forsby specifically.

%% City Kingdom of Esmajar
\cleardoubleevenemptypage

%% TODO: Artwork

\begin{infobox}{City Kingdom of El-Fayam}
  %% TODO: Crest
  \begin{multicols}{2}
    \begin{itemize}[label={},noitemsep,leftmargin=0.0cm,topsep=0pt]
      \infoboxitem{Location}{North western shores of \nameref{sec:Arania}
      }
      \infoboxitem{Languages}{Gnoll, Giant}
      \infoboxitem{Government}{Absolute Monarchy}
      \infoboxitem{Major Religions}{\nameref{sec:Three Kings},
        \nameref{sec:Isamir}
      }
      \infoboxitem{Area}{est. 120,000 $km^2$}
      \infoboxitem{Population}{unknown}
      \infoboxitem{Non Grata}{any non-monstrous races without a proper permit
      }
      \infoboxitem{Magic}{all magic banned for non-citizens
      }
      \infoboxitem{Slavery}{yes, all forms}
      \infoboxitem{Special Laws}{unknown}
      \infoboxitem{Notable Organisations}{unknown}
      \infoboxitem{POI}{slave actions, gladiatorial arena, shrine to the
        \nameref{sec:Three Kings}
      }
    \end{itemize}
  \end{multicols}
\end{infobox}

\clearpage

\subsection{Esmayar}
\label{sec:Esmayar}

Esmayar (``white jewel'' in Kalest, also known as ``Arcania'' in local
dialect) was one of the older city kingdoms of Aror, with a rife history but
has since fallen to the gnoll raiders in \emph{MI:1213}. It is located on the
north western shores of \nameref{sec:Arania}.

The city's old humanoid banner featured a yellow double headed eagle, upon a
light red background, embedded in a shield upon which rests a crown. Yellow
and red were the city nation's primary colours.

The new banner shows a dark red head of a gnoll upon white background, beneath
a golden crown. Red, gold and white are the primary colours of the monstrous
kingdom of Esmayar.

\subsubsection{History}

It was founded in \emph{GT:592} in the delta, and along the banks of the
\emph{Balran} river on the north western shores of \nameref{sec:Arania}. Much
like Fes al-Bashir, with which the city shared a long historical and cultural
friendship, it got rich by selling agricultural products, such as exotic
fruits, tobacco, chocolate and coffee to the other city kingdoms and
baronies. This wealth attracted raiders, bandits and the gnoll tribes of the
desert who besieged the city constantly over the course of thousands of
years. Although it was always capable of defending against these attacks, the
constant ransacking and pillaging of the outlying farms and plantations
drained the kingdom's resources to the point of bankruptcy. With the military
aid of Fes al-Bashir, and northern city kingdoms it held on to the power within
the city but had long lost the outlying villages. They fragmented into smaller
nomadic tribes and proofed difficult to reintegrate into the waning
empire. Unable to reunite the land, the kingdom fell apart leaving only the
city of around four million inhabitants under the control of the ruling
monarch.

In \emph{MI:1213} the final and last siege of the kingdom began as an army of
gnolls marched upon the walled city. The siege, with the aid of Fes al-Bashir
and the ``devils of the north'' (soldiers and mercenaries from
\nameref{sec:Norbury}) were able to hold off the besieging army for several
months before the city fell in the late months of that year. Many citizens
were evacuated by sea, moving further down south to ultimately found
\nameref{sec:El-Fayam}.

\subsubsection{Culture before the Fall}

The ``white jewel'', as it was known, got its name from the beautiful, and
elaborate buildings its architects build within the city. Most of its
buildings were painted white - hence the name - giving the city an almost
angelic appearance. Its main attraction was the Colosseum, a huge arena
that was used for both musical, theatrical performances, as well as for
sport events. The Colosseum is now used mainly for fights between enslaved
gladiators, price fighters, and monsters by the gnolls.

The kingdom itself was always deemed imperialistic, and expanded its borders
through war, conflict and subjugation. Its army was well trained, equipped,
and matched in ferocity only by the \hyperref[sec:Norbury]{wolves of the
  north}. Its imperialistic nature was also the root causes of its down fall,
as it often conquered more land than it could hold, and defend.

The average Esmayan was proud, family oriented, and followed the
\nameref{sec:Order} for guidance. They had a strong legal system, and were one
of the few city kingdoms that only used slavery for indentured servitude of
criminals. Although they did not sign the \nameref{sec:Vonir Accord}, their
tradition, and culture which focused on warriors, pride, and honour made them
staunch allies of \nameref{sec:Norbury}.

\subsubsection{Gnoll Kingdom}

The gnolls ransacked the city and the surrounding lands, killing all of the
remaining defenders through ritualistic immolation. Many civilians fled
towards Fes al-Bashir by ship with the aid of the retreating army of the
Norbury and Fes al-Bashir. The gnolls further enslaved all civilians that were
unable to flee. Although many kingdoms ignored the self proclaimed gnoll
kingdom, dismissing it as a short-lived and temporary kingdom that would fall
apart on its own due to the gnoll's limited knowledge on how to effectively
run such a vast and big empire.

The retreating humanoid races destroyed the dragon teleporter within the city,
and all other humanoid city kingdoms avoided contact with the new gnoll
empire, isolating it both diplomatically and economically. Most organisations
of power believed that this was sufficient to starve out the gnolls, and make
their kingdom fall within two decades. But to much surprise, the gnollish
kingdom has now lasted for over eight hundred years. They withstood famine,
plague and many attempts of the humanoid races to retake the city. At first
they took and plundered everything their kingdom required, but are now, after
using their slaves to rebuild and tend to farms and plantations, self
sufficient.

At first their laws against humanoids was harsh and unforgiving, enslaving any
and every humanoid they found. But over the course of centuries their society
opened up, and even allowed humanoid species to visit and trade within the
city. The gnolls have shifted the focus of the empire's economy towards
precious stones, silver, and gold, while engaging heavily in the slave trade
of humanoid species. They are not a signer of the \nameref{sec:Vonir Accord},
and thus enslave anyone that misbehaves or commits crimes (real or merely
accused) within their city.

The gnoll pack leader claims the king's throne and is the supreme ruler of the
city, although their state as a city kingdom is not recognised by the other
major humanoid city kingdom. Although they are now one the weakest and
smallest of the city kingdoms, no one dares to besiege or reclaim it, due to
the fear of either a military defeat, or heavy losses such a campaign would
bring. It is widely known that the gnolls have made pacts and alliances with
other monstrous baronies, especially those of \nameref{sec:Eilean Mor} and
\nameref{sec:Iafandir} to bolster their power and influence.

\subsubsection{Relations}

\aren{It is fine if you enslave your own races, but mothers' forbid if
  the monstrous races do it.
}

None of the city kingdoms accept or recognise the gnoll's reign over Esmayar,
and thus the city kingdom has no official allies. In truth however the various
agencies of the other slaver nations (such as the \nameref{sec:Velvet Hand}
and the \nameref{sec:Hunters Guild}) have been known to trade with the gnoll
empire. A tactic that is both lucrative and highly controversial.

The city kingdom of \nameref{sec:Morkan} has often used the kingdom as an
example of what happens when the humanoids show lenience and weakness against
monstrous invaders. Taras has sworn on many occasions to turn his gaze towards
the gnolls of the city once the continent of \nameref{sec:Iafandir} has been
cleansed of the monstrous races.

Esmayar is allied with many smaller monstrous kingdoms, such as hobgoblin
kingdoms and larger ogre tribes or bugbear baronies. It openly seeks to
establish new relations with monstrous villages and tribes, especially with
the other nomadic gnoll tribes that still wander the continent of
\nameref{sec:Arania}.

%% City Kingdom of Fes al-Bahsir
\cleardoubleevenemptypage

%% TODO: Artwork

\begin{infobox}{City Kingdom of Fes al-Bashir}
  %% TODO: Crest
  \begin{multicols}{2}
    \begin{itemize}[label={},noitemsep,leftmargin=0.0cm,topsep=0pt]
      \infoboxitem{Location}{North eastern shores of \nameref{sec:Arania}
      }
      \infoboxitem{Languages}{Kalest, Teranim}
      \infoboxitem{Government}{Oligarchy}
      \infoboxitem{Major Religions}{\nameref{sec:Forun}, atheism
      }
      \infoboxitem{Area}{est. 330,000 $km^2$}
      \infoboxitem{Population}{est. 33 million in total}
      \infoboxitem{Non Grata}{monstrous races (except slaves), devils, druids
      }
      \infoboxitem{Magic}{all magic must be permitted by the \nameref{sec:Hall
          of Knowledge}
      }
      \infoboxitem{Slavery}{yes, all forms, overseen by the
        \nameref{sec:Velvet Hand}, used as punishment for criminals, signer
        for the \nameref{sec:Vonir Accord}
      }
      \infoboxitem{Special Laws}{-}
      \infoboxitem{Notable Organisations}{\nameref{sec:House Ranian},
        headquarters of \nameref{sec:Velvet Hand}, headquarters of the
        \nameref{sec:Ror-Aram Trading Corporation}
      }
      \infoboxitem{POI}{\nameref{sec:Hall of Knowledge}, including the biggest
        library on all of Aror, grand bazaar, many oasis which have been
        converted into recreational parks, the sea of towers
      }
    \end{itemize}
  \end{multicols}
\end{infobox}

\clearpage

\subsection{Fes al-Bashir}
\label{sec:Fes al-Bashir}

The city kingdom of \emph{Fes al-Bashir} is the oldest city kingdom, and also
the oldest civilisation on Aror. It was founded in \emph{GT:0}. This is not a
coincidence. When the calendars were consolidated by the scholars of the city,
they realigned the years to use the foundation of the city as point zero for
the old calendar.

The banner of Fes al-Bashir is a light yellow shield, featuring a dark red two
headed lion beneath two dark red half moons which point downward. Light yellow
(or the yellow of the sands), and red are the kingdom's main colours.

\subsubsection{History}

The city was first founded when many nomadic tribes began to permanently
settle down in the vast delta of the river \emph{Alis} on the eastern shores
of \nameref{sec:Arania}. Throughout its history it was always in conflict with
other nomadic tribes, and gnolls that sought to claim the vast and fertile
oasis for themselves. Although the city was sacked, besieged and even
conquered by gnolls several times throughout its history, the nomadic tribes
always managed to reconquer and retake the city. It had been continuously
inhabited by humanoid species for several thousand years, but had to endure
many sieges and attacks in its vast and long standing history.

The city has a vast network of old and ruined watchtowers, walls and military
camps just outside the main walls. Although these are in various states of
disrepair they are still used as way points, and watchtowers should enemies
lay siege to the city. This area has been nicknamed the \emph{sea of towers},
as the old watchtowers can still be seen dotting the landscape from the city
walls.

\subsubsection{Culture}

The people of \emph{Fes al-Bashir} have learned that the art of war, is as
important as the sciences or the arcane studies. They are known as
imperialistic expanders, and seek to prevent and extinguish threats before
they materialise against the city itself. Although the outlying small towns of
the river banks are sparsely manned by the army, each and every citizen is
encouraged to train in combat or even in the arcane for self defense.

Most of the city's population is atheistic, and the only religion that still
manages to keep a hold onto believers within Fes al-Bashir is the main church
of \nameref{sec:Forun} (known there as Nuit). Although proud, family oriented
and staunch believers in their heritage, they do not see the gods and deities
as the answer to their problems. The city kingdom has the oldest university of
all of Aror, and the belief of the population is that scientific and arcane
research can solve any problem, and does so better than adherence to belief,
or old superstition.

Due to harrowing temperatures in the summer months (around 40 to 50 degrees)
the people of Fes al-Bashir tend to wear long, light clothes with
turbans. Their dark skin - which they share with most of Arania's inhabitants
- also aids them against the unbearable heat of the desert they inhabit.

Within the society the traditional roles of men are women still run strong, as
women are encouraged to seek safer employment, instead of becoming soldiers or
labourers. Although women are not bared from seeking employment in these
fields by the law, the cultural pressure for them not to do so remains high.

People hailing from Fes al-Bashir are often described as arrogant or snobby,
as they see their oldest city as the birth place of modern civilisation, and
often find the other city kingdoms often lacking in culture, or outright
primitive. A trait that is often down played as a jest in good spirit when
brought up by foreigners.

The cuisine of Fes al-Bashir relies heavily on the spices they grow around the
oasis, and is thus fiery hot and spicy. The people of the nation also heavily
drink tea and coffee, and both drinks have become a national and traditional
staple in every day life.

\subsubsection{Architecture}

The city itself has one central area called the grand bazaar, that contains the
main campus and the Hall of Wisdom. The grand bazaar also housed the palace of
the monarchy, which has since been converted into a garrison for the armed
forces of Fes al-Bashir. The former royal palace also serves a meeting hall
for the city's council. While the central plaza has many larger buildings, and
estates, the surrounding city's architecture is vastly different. Families live
small apartments, often just a single room, where each house providing several
apartments on several floors. Each floor is accessible separately from the
outside through wooden staircases, and many houses also interconnect with each
other with wooden bridges. This levelled and modest architecture allows the
city to house a large population of several millions in a small space in the
desert. Most of these houses are built out of sandstone and painted white, while
the estates, palaces, and the House of Wisdom are built out of marble.

The city itself spreads itself across several smaller oases, and thus the city
is interspersed with many green parks that surround small ponds of fresh
water. These small parks are mostly used for fresh water supply, and for
recreation, albeit polluting these ponds is punished harshly. The city itself
has many smaller bazaars spread across the vast network of streets and smaller
settlements, that sell spices, food, luxury goods and slaves.

\subsubsection{Population}

The city itself and its surrounding areas is one of the largest civilisations
on Aror, housing over 33 million people. Most of these are equally spread out
among the four major humanoid races, with humans leading by a small margin
(26\%), followed by elves (25\%) and halflings (22\%) and dwarves (20\%) with
various half races and undead make up the rest (7\%).

Common male names are: Ali, Adam, Amir, Bakar, Fahim, Farouk, Hamid, Gamal,
Hasan, Jaffar, Karim, Musa, Nadeem, Nassim, Raheem, Tarek, Roshua

Common female names are: Amina, Adila, Aida, Amira, Dana, Hagir, Hala, Hannah,
Inas, Jamila, Leyla, Lina, Nada, Raisa, Sarah, Thamina, Yasmin

\subsubsection{Rule and Laws}

Although called a city-kingdom, it is no longer a monarchy. The previous ruling
monarchy - called Malek (king) or Malekha (queen) - has been abolished in
favour of a ruling council comprised of important and powerful artisans,
scientists, mages and scholars from the Hall of Wisdom. By tradition, the high
magus of the Hall of Wisdom oversees the council and acts as its mouthpiece,
and as an external representative.

The city kingdom still practises slavery, which are drawn from the criminals,
the poor, captured enemy warriors and those that the kingdom has bought from
other slavering nations. The city is a signer of the \nameref{sec:Vonir
  Accord}, and is also known for being one of the major slave trading hubs in
the region. The city operates its own slaver's guild called the
\nameref{sec:Velvet Hand}.

Intelligent undead are not banned from the city, and may become citizens of
the city. The city, and its surrounding areas thus has a sizeable community of
\nameref{sec:Umgeher} and \nameref{sec:Vampires} within its borders.

\subsubsection{Hall of Knowledge}
\label{sec:Hall of Knowledge}

The \emph{Hall of Knowledge} is the state run university of Fes al-Bashir. It
is the oldest, biggest and most respected university of the sciences and
arcane arts. Many foreign scientists and wizards come here to study and
learn, and the university has a track record for educating some of the most
prestigious wizards and scientists. Many of the biggest scientific
advancements and discoveries have been made within the Hall of Knowledge: the
true nature of the solar system, advancements in physics, engineering,
mathematics and medicine. Its most accomplished student and leader was none
other than \nameref{sec:Graham Balance}.

As an institution the Hall of Knowledge holds vast political power within
Fes Al-Bashir, and many of its leaders are also powerful, charismatic and highly
intelligent politicians and leaders within the city itself. While the Hall of
Knowledge holds no direct power, it has very close ties to the ruling elite of
the city, and provides them with a means and a place to network amongst each
other.

\subsubsection{Relations}

The city kingdom is known for its scientific prowess and vast riches in
agricultural produce, such as herbs, spices, coffee and chocolate. It is more
than willing to trade these goods to others through the \nameref{sec:Ror-Aram
  Trading Corporation}. This, along with a propensity to lend military and
scientific aid to its allies, has lead to a good diplomatic standing with
almost all other city kingdoms. However the fall of \nameref{sec:Esmayar} and
the kingdom's subsequent defeat there has dampened the power and influence of
the nation on the continent of Aror.

%% City Kingdom of Forsby
\cleardoubleevenemptypage

\begin{figure*}[ht!]
  \centering
  \vspace{-4.6cm}
  \centerline{
    \includegraphics[width=\paperwidth,keepaspectratio]{media/forsby-a3-sm.png}
  }
  \par
  View from the Midway tavern, GT:2104
\end{figure*}

\vspace{-0.5cm}

\begin{infobox}{City Kingdom of Forsby}
  \begin{subfigure}[t]{\textwidth}
    \centering
    \includegraphics[width=0.20\linewidth]{media/forsby_banner_sm.png}
  \end{subfigure}
  \vspace{-1.0cm}
  \begin{multicols}{2}
    \begin{itemize}[label={},noitemsep,leftmargin=0.0cm,topsep=0pt]
      \infoboxitem{Location}{Western shores of North \nameref{sec:Goltir}}
      \infoboxitem{Languages}{Reatham (dialect), Teranim, Doresh}
      \infoboxitem{Government}{Absolute Monarchy}
      \infoboxitem{Major Religions}{\nameref{sec:Lor}, \nameref{sec:Old Ways}
        \nameref{sec:Order}
      }
      \infoboxitem{Demographics}{est. 150,000 $km^2$, pop. est. 12 million
      }
      \infoboxitem{Non Grata}{devils, druids, monstrous races, undead without
        any citizenship
      }
      \infoboxitem{Magic}{soul magic, necromancy, summoning, and calling
        outlawed; divine, psionic and arcane magic require permits
      }
      \infoboxitem{Slavery}{outlawed, indentured servitude as punishment for
        criminals, signer of the \nameref{sec:Vonir Accord}
      }
      \infoboxitem{Special Laws}{\nameref{sec:Gorgons} must suppress their
        powers while in public
      }
      \infoboxitem{Notable Organisations}{headquarters of the
        \nameref{sec:Magistrata Arcanum}, \nameref{sec:House Ranian},
        \nameref{sec:Ror-Aram Trading Corporation}, headquarters of the
        \nameref{sec:Holy Church of Aleaste}
      }
      \infoboxitem{POI}{Castle Forchtenberg, Midway Point, the
        \nameref{sec:Pit}, cathedral of \nameref{sec:Lor}, public library,
        university of the Magistrata Arcanum with dragon teleporter, Forchtenberg
        park, recreational area alongside the river Hafon, knight's temple of Aleaste
      }
    \end{itemize}
  \end{multicols}
\end{infobox}

\FloatBarrier
\clearpage

\subsection{Forsby}
\label{sec:Forsby}

\aren{You can tell someone hails from Forsby from the first word that comes
  out of their mouth.}

The city kingdom of \emph{Forsby} (``city of streams''), also known as
\emph{Forchtenberg} in local dialect, was founded in \emph{GT:1148}, and
resides on the western shores of the continent of \hyperref[sec:Goltir]{North
Goltir}. Up until the devil siege, it was one of the biggest and wealthiest
of the city kingdoms.

Forsby's banner depicts a yellow hippogriff within a shield, upon a red
backdrop. A crown with three tips rests upon the hippogriff's head.

\subsubsection{The City}

Forsby castle was built the edge of a narrow fjord, upon white cliffs, giving
it a grant view of the sea. The noble district, common area, as well as the
main square lie in a semi circle around the main castle up on the cliff
itself. The kingdom's lands extend beyond the wall of the city, as it owns
large areas of fertile farmlands in the Toralian Highlands. The largest
bulk of the kingdoms populations lives on top of the cliffs, either in the
common area around the castle or as farmers in the highlands. The northern
lands are fed by the river \emph{Hafon}, and the kingdom uses the river both
for its fertile farmland further east, as well as for fresh drinking water
within the city. Any waste from the city is also carried down river into the
sea. The Hafon river crashes into the bay at the southern side of the cliff.

At the bottom of the white cliffs lies the city kingdom's harbour district. The
harbour district is completely built on wooden and stone stilts and pillars,
and fills out most of the cliffs basin. The surrounding cliffs that encircle
the harbour act like a natural defensive barrier, forcing all incoming ships
to pass a narrow passage into the city's harbour. The cliff itself is adorned
with many grand statues holding fire cauldrons, light houses, temples, as well
as fortifications and make the city a truly magnificent view when approaching
it from the sea. Many people, especially workers, traders, and sailors also
live in the extensive harbour district.

Several elevators, both for people and goods, connect the upper cliff side
with the harbour down below. There are also many stairs, passages, and small
caves that were hewn into the cliff side that allow people to travel between
the upper city and the lower harbour. Some people also live in smaller houses
hewn into the cliff side. A well known, and old tavern called ``Midway Point''
rests halfway between the harbour and the castle.

Lying in the northern parts of Goltir, Forsby experiences harsh and cold
winters, and mild summers. While the soil is fertile, only the strongest crops
grow in this harsher climate. Forsby is surrounded by lush and ancient pine
forests, as well as ample farmland.

\subsubsection{Population}

Forsby, and its surrounding outlying villages, is home to roughly 12 million
people, of which most (58\%) are human. Closely followed by deepkin (16\%)
and the elven races (12\%), dwarves (8\%) and various half races (4\%) and
others (2\%) such as Diarim and Umgeher.

Common male names are: Alois, Anton, Bogdan, Boris, Dejan, Emil, Ferdinand,
Gal, Gustaf, Janos (Jan), Johannes (Johan, Hannes), Robert, Miklos, Nikola
(Nikolaus, Nikolai), Radek, Soma, Simon, Viktor, Walter

Common female names are: Anna, Abigel, Bianka, Brana, Elisabeth (Elsa, Lisa),
Elena, Emma, Emelie (Amalie), Helene, Johanna, Magda, Melinda, Nada (Nadia,
Nadezhda), Pauline (Line, Lina), Rosalie (Rosa), Sobena, Vera, Vesna

\subsubsection{Pit}
\label{sec:Pit}

But Forsby also has a darker and secretive side. Beneath the main castle and
noble district, deep with the cliff side, lies the aqueduct that transports
the waste down into the sea. It is simply referred to as the \emph{pit}.
Although a vast and dangerous sewer complex, many people and creatures call
its many levels their home. Existence of the pit is an open secret, and many
entrances to it can be found across the cliff side and in the harbour for
Forsby. Strategies on how to handle the pit, and its denizens, vary greatly
from monarch to monarch. But most follow the ``out-of-sight'' strategy: As
long as no criminal activity bleeds to the surface the pit is free to do as
it pleases.

The upper level houses a vast bazaar and active criminal underworld, where
everything and everyone can be bought for the right price. It also houses a
large tavern and inn, called the ``Golden Mug'' that serves as a neutral
ground for the pit's many criminal gangs. Citizens do sometimes wander the
bazaar, and apart from pickpockets, the bazaar is a save place to do both
illegal and legal business. Most criminal gangs now that scarring away
customers by openly fighting and feuding at the bazaar is counter to their
business' wealth and prosperity.

The lower levels are a different beast however. They are mainly uncharted, and
often house creatures that would not be welcome on the surface. Sightings of
feral vampires, sentient flesh golem butchers, devils, and hags have been
reported in the lower depths of the pit.

\subsubsection{Culture}

Although Forsby has a long standing tradition of chivalry and nobility, the
general population is well aware of the hypocrisy by not moving against the
criminal underworld of the \emph{pit}. The average citizen is nevertheless
proud of the ancient history of the city, its wealth and that a long line of
just kings and queens have lead the kingdom to become one the most powerful on
the planet. Forsby is world-renowned for its arcane academy - called the
\nameref{sec:Magistrata Arcanum} - which has produced many fine and
outstanding wizards. As well as for its public schools that teach any children
within the city and kingdom, and give them a basic education for free.

The people of Forsby are known as stubborn, patriotic, determined but hard
working to achieve their goals. They are well recognised by their rather
cryptic dialect of the common language called \emph{Reatham}. The city itself
used to be open to every civilised race, but after knights of Lor took hold,
they outlawed monstrous races, and devils from entering the city.

The most worshipped religions of Forsby \nameref{sec:Forun},
\nameref{sec:Lor}, but also \nameref{sec:Aria} with the less noble elements
that work and live in the \emph{pit}. Many holidays of Forun are also national
holidays and celebrations in Forsby, including the harvest festival and the
festival of day of candles.

\subsubsection{Nobility}

The city kingdom of Forsby has a long standing tradition of noble
families that have been in charge of the cities affairs for several hundred
years. Their names carry weight even outside the borders of the city, and
they are well established in other city kingdoms across the globe.

The three major families are \emph{Hohenhaus}, \emph{Burgau} and \emph{Csaky}
who have, in turn, ruled the city since its foundation. Hohenhaus is embedded
in the church of \nameref{sec:Lor}, and many of its members are knights,
scribes and priests within the holy church of Lor. Burgau runs the kingdoms
financial sector, and owns many of the artisans, smithies and crafting
guilds. While Csaky is deeply embedded in the city's \nameref{sec:Magistrata
  Arcanum}, and many of its patrons and leaders are well respected wizards,
arcane scholars and arcane craftsmen. These three houses have been known to
feud and fight amongst each other, yet still they share one common goal: the
survival and prosperity of the kingdom.

While these three vie for the throne, there are several smaller houses of
nobility that control smaller aspects of the city: House \emph{Forholm} runs
the cities trading company and shipping docks, and have been known to do
business with the shadier aspects of the \emph{pit}. The House of
\emph{Lemberg} operates most of the kingdom's farms outside of the city, and
are vital to the city's self sustainability in terms of food and agricultural
products.

\subsubsection{Slavery}

The city kingdom of Forsby outlawed slavery, but remains the right to force
criminals into indentured servitude. However this servitude is not extended
to private individuals, and criminals work off their debts and crimes in public
service instead. However the city is a signer of the \nameref{sec:Vonir Accord},
and thus does not free slaves of other signer states, and accepts other slave
owners right to the property.

\subsubsection{Devil Siege}
\label{sec:Devil Siege}

\aren{It is hard to find someone that was not a part or at least affected by
  the war against the devils.}

In MI:2017 the city came under siege by a large force of devils who besieged
the city by land and by sea. After the initial wave of attackers were driven
back by the local knights of the \nameref{sec:Order}, \nameref{sec:Lor}, city
guards and mages of the tower; a second, even bigger wave began laying siege
on the city. During the initial weeks of the siege many of the surrounding
villages of the kingdom were destroyed and its inhabitants were either killed,
or had to flee into the city. The kingdom was both besieged from the sea by
infernal ships made of steel and iron, as well as from several devil legions
by land.

As the siege dragged on for weeks, many more allies of Forsby - such as
\nameref{sec:Hraglund}, \nameref{sec:Norbury}, \nameref{sec:Helmarnock} or
\nameref{sec:Tredegar} sent reinforcements. After the elves and halflings of
Avenfjord joined the coalition, the defenders created a military base on the
Silver Isles to jointly attempt to break the sea embargo that cut off
Forsby. In the last moments before the assault, even \nameref{sec:Morkan} joined
with a small fleet war ships. Although the steel ships of the devils were
superior, the vast amount of ships and men raised by the coalition broke
through the sea embargo. The coalition lost many ships and men, and only a
handful of vessels made it to the harbour of Forsby. Even though the elves and
halflings of \nameref{sec:Avenfjord} promised ships, none of their ships or
troops made it to the rally point to aid Forsby.

During the time of the siege a group of heroes recruited by the coalition
secured the help of an elder white dragon named \emph{Northwind}. Northwind,
and his army of monstrous creatures proved invaluable to defeat the armies of
the devils. It is unknown what happened to the leader of the devils, though
the prevailing theory is that he abandoned the battle after the dragon and its
army came to aid the city.

The siege lasted for almost 7 months, and was formally declared won in the
second month of MI:2018. The reigning monarch \emph{Johann Paul of Hohenhaus}
died during the final battle of the siege, and his son \emph{Johann II of
  Hohenhaus} became king. The dragon remained nearby, and - although most of
his minions perished - it still holds a vast political influence in Forsby.

The siege left all affected kingdoms in a weakened state. Many men and women
perished during the siege, and towns and hamlets were left empty or without
a significant amount of their population. Many ships were lost and the
struggle to rebuild became an arms race among the city kingdoms and larger
baronies. But the siege proved that all kingdoms could work together when
faced with a common enemy. The outcome of the war soured relationship between
Forsby and Avenfjord, who failed to provide promised resources and ships. It
also lead to hostility toward anyone perceived to be a devil sympathiser or
spy, causing massive extra-judicial persecutions and killings of perceived
devil worshippers and summoners, as well as \nameref{sec:Tieflings}.

It is unknown why the devils attempted to invade Forsby, but many, including
the church of Lor, suspected the mage's guild. The guild however denies any
responsibility or knowledge of the attack prior to the siege. Still the
nobility, and the church of Lor, interfered into the guild's business, banning
several practises, such as necromancy and heavily restricting others such as
conjuration, and especially summoning.

\subsubsection{Relations}

Forsby holds good relations with most city kingdoms, even with
\nameref{sec:Morkan}. The kingdom of Forsby is a signer of the
\nameref{sec:Vonir Accord}. The city holds a large church of \nameref{sec:Lor}
and its nobility is tightly interwoven with the church.

%% City Kingdom of Helmarnock
\subsection{Helmarnock}
\label{sec:Helmarnock}

The city kingdom of \emph{Helmarnock} (ancient Teranim for ``land of the
people called Helm'') lies on the south-eastern edge of \emph{Eilean Mor}. It
is unique as the kingdom is resides on four islands as well as the eastern
shore, which were once ruled by separate five baronies that joined together
into one kingdom.

\subsubsection*{Geography}

Helmarnock lies spread out on four different islands off the south eastern
shore of \emph{Eilean Mor}. The islands are joined together by four huge
stone bridges, that converge in a crossroad in the middle of the sea between
the islands. The kingdom also owns land on the shores of \emph{Eilean Mor}.
The northern island is called \emph{Temen}, the eastern island \emph{Thorm},
the southern island \emph{Eledis} and the western island is called
\emph{Saremen}.

All of the islands are covered with city, and the islands are surrounded by
huge harbours and ports. The land on the shore is mainly used as farmland as
the huge stream \emph{Dunast} makes the land fertile for crops.

\subsubsection*{History}

The city kingdom was founded in \emph{GT:984} when the four baronies that
ruled the islands and the shore joined together to form one big city kingdom.
The rulers of these baronies all intermarried and created a new noble family
called house \emph{Helm}. The islands were joined with bridges and a small
public forum was built in the centre of this island from which the king could
rule. However in \emph{GT:1588} the noble family ended up without an heir and
the throne became vacant.

For two hundred years the five noble families fought between each over the
throne. Although the conflict never escalated into full civil war, the houses
raided each other, abducted and assassinated important members of other
houses, as well as sabotaged companies and production facilities. These two
hundred years are now known as the \emph{bloody years}, and it deeply
entrenched itself within the population, splitting them into five major
factions. The deep trenches that divided the people of these islands then can
still be felt after thousands of years since the conflict.

Finally in \emph{GT:1601} the dispute was settled. The noble houses decided
to not elect a new ruler, but instead rule the kingdom as a council of family
leaders. The resulting government is called the \emph{high council} and rules
the city absolute. But stability and security returned slowly to the city
kingdom over the course of many decades.

In \emph{GT:3401} the city kingdom was dragged into the
\nameref{sec:Holy Crusade} by the vampire noble house \emph{de\'Var} who openly
rallied against the Holy Church of \nameref{sec:Griannar}. The other noble
houses followed the war declaration as the united \emph{kingdom of
  Helmarnock}. Although it cost the kingdom many lives, it was ultimately a
victory and a source of national pride, that united the divided spirits under
the banner of one greater kingdom.

\subsubsection*{Culture}

\graham{Their reputation is undeserved. For the most part.}

The citizen of Helmarnock are known to still harbour resentment for each other,
and are mostly patriotic towards their own noble family. Years of underhanded
and illegal dealings, corruption within the city guard and armies have given
the people of Helmarnock a reputation for cutting corners, being sleazy,
lacking in morals or being outright criminals. Although this does not hold
true for the vast amount of the working population; the climate of harsh
competition between the islands tends to promote ruthless and underhanded
business tactics amongst the nobility and upper class citizens.

Even though most people still divide themselves into five distinct cultures
aligned with the five noble houses, the high council is working towards
uniting the kingdom not only in government, but also in culture. The people
are encouraged to move between the islands, are encouraged to see themselves
as citizens of \emph{Helmarnock} instead of the old baronies, and often organise
events such as festivities, plays, concerts and trade festivals in the central
forum in an attempt to unite the kingdom.

\subsubsection*{House Eseriel of Tenem}
\label{sec:House Eseriel}

The northern island of Tenem is ruled by the dark elven noble House of
\emph{Eseriel}. Their banner is a red dragon upon a white background. White
and red are the main colours of the house. They operate a large shipping
enterprise and are mostly responsible for imports and exports of the
kingdom. Being natural traders and business men, House \emph{Eseriel} is the
richest among the noble houses. Members of the house are known as excellent
and reliable business partners, and often fund expensive undertakings such
as expeditions into unknown realms, large construction projects or arcane
or scientific research projects.

\subsubsection*{House Onariel of Thorm}
\label{sec:House Onariel}

To the east reigns the high elven noble house of \emph{Onariel}. Their main
business is information and espionage, as well as offering counter espionage
to the entire kingdom. Their are infamous and notorious for their criminal
activities, such as smuggling, drug trafficking, espionage and oppression
of enemies of the kingdom. The bad reputation the kingdom and its citizen
have, mostly stems from various scandals that have been uncovered in which
House Onariel was involved. The house is known world wide as one of the most
effective spy organisation, that has agents and informants spread all across
the world.

Their banner is a side view of dark green owlbear in a white shield, atop of
which rests a silver crown. White and green are main colours of the house.

\subsubsection*{House Kelemor of Eledis}
\label{sec:House Kelemor}

To the south rule the dwarven noble family of \emph{Kelemor}. They once
hailed from the great divide, but made their riches by building most of
the city's buildings and bridges. They operate most of the smithies and
artisan guilds of the kingdom, and often work together with House
\emph{Eseriel} to export their manufactured goods to other kingdoms.
They also provide the equipment to the joint army of Helmarnock. Kelemor
are often the first to call out other houses for breaking the peace or
for attempting to sabotage the finely tuned political machinery that
is the high council. Although Kelemor is a voice for order and piece within
the kingdom, they are also the fiercest proponent for continuing slavery
within the kingdom.

The house's banner is an upside down golden hammer resting on a white
background. White and gold are the main colours of their house, and their
emblem can be found on my armours, weapons they have produced and buildings
they have built throughout the kingdom.

\subsubsection*{House de\'Var of Saremen}
\label{sec:House deVar}

The western island is ruled by the House of \emph{de\'Var}, whose mainstay
members are vampires made up from various races of \emph{Aror}. Even though
vampires generally pose a threat to the general populace, House \emph{de\'Var}
had to sign a treaty during the foundation of the kingdom to not harm any of
the kingdom's citizen, or citizens of their allied city kingdoms. Although
the idea of being ruled by vampires is scary to outsiders, House \emph{de\'Var}
are fair and just rulers. Their members abide by the law, and do not feast on
humanoids, instead preferring to live off \emph{Ramesk}, a mineral rich fluid
invented by the Ilians as a blood substitute. The liquid also allows the
vampires of House \emph{de'Var} to endure the sunlight for a limited amount of
time, albeit they prefer to wander the streets in daylight with protective
clothing instead.

The house also provides the city guard and justice system for the entire
kingdom, and is paid for this service by the other houses. The island of
Saremen also houses the kingdom's library and arcane institution. Despite
having an obvious reason to be for slavery, house \emph{de\'Var} is openly
against it. They do not require slaves for nourishment, and even if they were
to feast upon them, they'd lose both respect and renown within the very
population they try to rule. The vampires of the house see themselves as
effective and just rulers, since they do not tire or become ill and clouded in
judgement, and wish to proof that very fact to their subjects.

House \emph{de\'Var} was also responsible for creating the race \emph{Umgeher}
in \emph{MI:263}. In their hubris, they attempted to create the perfect
``undead citizen''. Those were supposed to be used as soldiers, guards,
workers and labourers. However the \emph{Umgeher} were poorly received by the
population. Mostly because deceased citizens were used in the process, and
many found their deceased loved ones among the new workforce. Many workers
feared that the undead would replace them, stripping them of their lively hood
and ability to feed their own families. The people of Saremen demanded that
the \emph{Umgeher} should be exiled from their kingdom. Out of fear of
upsetting their population any further, house \emph{de\'Var} gave the
\emph{Umgeher} the gift of reproduction and exiled them from the city. The
exile was lifted in \emph{MI:1100}.

The house's banner shows a blue raven, flying downward upon a white
background.  Blue and white are the main colours of the house.

\subsubsection*{House Liares}
\label{sec:House Liares}

The human house \emph{Liares} rules the territories on the shores and main
land of \emph{Eilean Mor}. Although they are the smallest house of nobility in
Helmarnock, they are one of the most important. They own the vast majority of
farms on the mainland and are vitally important both to provide and produce
the food for the islands, as well as for growing the roots and herbs required
to brew \nameref{sec:Ramesk}, the blood substitution drink for the
vampires. House \emph{Liares} are also responsible for maintaining good
relations with the other city kingdoms, and thus run embassies and provide
ambassadors.

House Liares' banner is a black tree upon white background. White and black
are their main colours.

\subsubsection*{Banner}

The banner of Helmarnock is a shield upon white background, divided into five
section. Each section bears the banner of one the five houses. Top left the
red dragon of Eseriel, top right the owlbear of Onariel, middle right the
golden hammer of Kelemor, at the bottom the tree of Liares, and middle left
the raven of de\'Var.

\subsubsection*{Population}

The kingdom is a sprawling metropolis, housing roughly 18 million people on
all four islands and the mainland. Most numerous are the various kinds of
elves (38\%) followed by humans (31\%) and dwarves (28\%) with the half
races, Umgeher and Diarim making up a small minority (3\%).

\subsubsection*{Society}

Although the high council rules supreme, each distinct old barony (so each
island the main land) has a local government whose main task is to administer
the parts of the kingdom, as well as enforcing laws enacted by the high council.
Slavery is allowed within the city, however slaves do have certain rights. For
example a slave owner must take care, feed and clothe a slave, and a slave
owner who intentionally lets a slave starve is charged with murder.
\emph{Indentured servitude} is quite common, mostly as a way to compensate
victims of crimes by forcing the perpetrator to work off his debts or fine if
they are unable to pay them. The houses of Eseriel, Kelemor and Liares are main
proponents of slavery as they profit most from the free labour. House de\'Var
is a fierce opponent of slavery, but find themselves alone in that position.

\subsubsection*{Academy of Arcane Arts}

Helmarnock is also known around the world for being a generous patron of the
arcane arts and sciences. The academy resides on the island of \emph{Saremen}
and is mainly funded by house de\'Var and house Eseriel. It is both famous,
and infamous, for teaching and researching necromancy. The academy not only
studies and teaches the arcane arts but also most major sciences such as
medicine, physics and astronomy.

\subsubsection*{Expeditionary Force}

The mainstay armies of Helmarnock are called the Helmarnock Expeditionary
Forces (often simply abbreviated ``the forces''). It recruits both from
within the kingdom, but also from people from outside. Foreigners who join the
forces for at least five years receive special consideration when they apply for
citizenship within the kingdom. Many foreigners join, as the expeditionary
forces have a reputation for being a well trained, equipped and capable, who
also pay fair soldier's wages and compensation to families in case of death.

\subsubsection*{Relations}

Due to the widespread acceptance of law-abiding undead throughout the city, the
kingdom has no, or bad relations with most religions and churches that see
undead as an affront to the natural order. Among these are the
\emph{Inquisition of the Third Order}, the \emph{Holy Church of Aleaste} (even
though Aleaste was of House Eseriel) and the various churches and knight
orders of \emph{Lor} are direct enemies of Helmarnock. This rivalry has often
cumulated in skirmishes between the city kingdom and the knight orders, but
neither side risks an all out war.

Helmarnock has signed the \emph{Vonir Accord} with Norbury, and is also one
of Norbury's closest allies.

%% City Kingdom of Hraglund
\subsection{Hraglund}
\label{sec:Hraglund}

\emph{Hraglund} is a humanoid city kingdom on the eastern shores of
\nameref{sec:Eilean Mor}. It is one of the oldest city kingdoms on Aror,
having been founded in \emph{GT:339}.

\subsubsection*{Banner}

The kingdom flies a green banner with a white shield that contains a five
headed hydra. This is also the banner which had been used by the
\emph{Wayfaerer's Guild} since the foundations of the first hunting camp.

\subsubsection*{History}

\emph{Hraglund} grew out of smaller villages that settled around a fortified
hunting camp called \emph{Wayfaerer's Guild}. The hunters proofed to be
effective in deterring the attacks and raids from the monstrous races that
lived in the surrounding area of the guild, and thus attracted more and more
settlers, farmers and workers. Soon the hunting camp grew, first into a small
military outpost and over the span of many centuries into a full kingdom.

Once \emph{Hraglund} was the biggest city on \emph{Eilean Mor}, with over
31 million people living within the kingdom and in the vast outlying lands
that included many smaller and bigger cities, such as the river trading city
of \emph{Braemer}.

\subsubsection*{Siege}

In \emph{MI:-2} the city was under siege by vast armies that had declared war
against \emph{Griannar} and his followers, in the name of the
\emph{Silent Queen}. This siege was part of a decade long war known as the
\nameref{sec:Holy Crusade}. Most of these armies that laid siege to the city,
were mercenaries that had been paid by the priests and priestesses of the
Silent Queen. Hraglund had been a major centre for the faith of Griannar, run
by the cardinal of the Holy Church. At first the city defended itself against
the siege for several months, but the majority of the population grew unruly
and attempted to oust the Church of Griannar from the city. After a civil
unrest and even skirmishes had broken out between the city guard, believers
and the part of the population that wanted to exile the church from the
city. They believed that if the church were ousted from the kingdom, the
armies in front of the gates would abort their siege. During that civil unrest
the cardinal fled the city, but was ultimately betrayed, captured and executed
by the followers of the Silent Queen. Just as the rebellious elements
predicted, the besieging army left soon after.

\subsubsection{Plague}
\label{sec:Plague of Hraglund}

In \emph{MI:1680} the city was struck with a major outbreak of the
\emph{black blight}, in which roughly 12 million people lost their lives. The
city was crippled, weakened, and lost most of its military power and influence
in the resulting chaos. The city not only lost many of its inhabitants, but
also they lost their king and the last heir of their noble house of
\emph{Altrizzi} that had ruled the city for many centuries. Now the city is
ruled by a council of high ranking officials and dukes that have been trying
to restore order to the kingdom. It was later revealed that a mad wizard and
follower of the \nameref{sec:Aria} had allowed a plague bearer into the city
in an attempt to take revenge against the kingdom. The wizard claimed he had a
personal vengeance against the kingdom, his brother was executed by the
kingdom for necromancy. He was sentenced and publicly hanged for high treason.

\subsubsection*{War with Terevar}
\label{sec:Terevar}

Many of the surrounding baronies sensed the weakness in Hraglund after the
plague had been defeated in \emph{MI:1682}. After consolidating into a large
alliance called Terevar the baronies declared war on Hraglund in an attempt to
take control of the city. Although Terevar was inferior in most ways to
Hraglund the war dragged on for several years. Hraglund had lost most of its
fighting forces but its overwhelming wealth made it possible for the kingdom
to hire mercenaries to fight the war for them. The war cost both sides
thousands of lives, displaced even more civilians but was ultimately lost by
Terevar, when Tredegår joined the war as allies on the side of
Hraglund. Terevar fell apart again into various smaller baronies in the chaos
and aftermath of their defeat. As war reparations the kingdom annexed most of
the border baronies and integrated them into the kingdom. Afterwards the
council of Hraglund, supported by their allies from Tredegår, claimed on
multiple occasions that Norbury had secretly aided the alliance of Nerever to
weakened Hraglund's power in the region. Something that Norbury vehemently
denied.

\subsubsection{Royal Guard}

During the war with Terevar the city hired many mercenaries with its immense
wealth, including a sizeable group of \hyperref[sec:Hobgoblins]{hobgoblins}.
These hobgoblin soldiers impressed the then ruling queen, \emph{Ambria the
  Third}, with their fighting prowess, unwavering loyalty and fierceness in
battle that she hired them as her royal guard. These hobgoblins, were allowed
to settle in the city, and their descendants live there to this day,
fulfilling their ancestor's pledge to protect the ruling monarch of the
Hraglund.

\subsubsection{Population}

After the plague the city had roughly 19 million people, of which most were
humans (43\%), elves (25\%), halflings (21\%) and dwarves (9\%) and other
various half races (2\%).

\subsubsection{Culture}

After both the religious siege and later on the plague, many citizens of
Hraglund lost faith in the new religions of the lesser deities. Most of
the citizens within the city turned towards atheism, science and arcane
studies. While the population of the country region returned to the old
ways. This had caused a trench between the city and country folk that split
the cultures of the kingdom in two. The city folk tend to be educated in
the sciences, arcane arts, enjoy art and are welcoming. While the country folk
returned to the old ways and traditions, including a strict adherence to the
shamanic ways of the three mothers. Although the country folk are known to be
blunt and perhaps a bit abrasive, they are still welcoming, good hosts and see
themselves as citizens of the kingdom first.

\subsubsection{Society}

The council of dukes and government officials now rules the kingdom in the
absence of a ruling monarch. The city kingdom is known for its stability,
fair laws and for having fought and defeated many problems and enemies in
its past. They have a strong culture, and are proud of their nation, which
their society and culture embodies. Unlike most of its neighbours the kingdom
does not practice slavery, yet still practices indentured servitude as an
alternative to incarceration.

\subsubsection{Relations}

Hraglund always had an uneasy relationship with neighbouring baronies
and realms, as they are known to aggressively extend their influence by any
means necessary. This often put them in direct conflict with both
Helmarnock as well as Norbury. However the kingdom holds good
relations with both Forsby and Tredegår.

%% City Kingdom of Morkan
\cleardoubleevenemptypage

%% TODO: Artwork

\begin{infobox}{City Kingdom of Morkan}
  %% TODO: Crest
  \begin{multicols}{2}
    \begin{itemize}[label={},noitemsep,leftmargin=0.0cm,topsep=0pt]
      \infoboxitem{Location}{Eastern shores of north \nameref{sec:Iafandir}
      }
      \infoboxitem{Languages}{Teranim, Danvark, Old Teranim}
      \infoboxitem{Government}{Tyranny}
      \infoboxitem{Major Religions}{\nameref{sec:Church of Morkan}}
      \infoboxitem{Area}{est. 40,000 $km^2$}
      \infoboxitem{Population}{est. 490 thousand registered citizens, est.
        2 to 3 million non-citizens and slaves
      }
      \infoboxitem{Non Grata}{monstrous races, undead except vampires, Gorgons,
        devils, druids
      }
      \infoboxitem{Magic}{Magic use outlawed for anyone, except sanctioned
        government agents
      }
      \infoboxitem{Slavery}{yes, all forms, signer of the \nameref{sec:Vonir
          Accord}
      }
      \infoboxitem{Special Laws}{all religions except the Church of Morkan are
        outlawed, monstrous races (including Gorgons) are enslaved or culled
      }
      \infoboxitem{Notable Organisations}{\nameref{sec:Church of Morkan},
        \nameref{sec:Kaltar}
      }
      \infoboxitem{POI}{Central Spire of \nameref{sec:Taras}, slave auction
        houses, as well as a blood arena, and Colosseum, cathedral to
        \nameref{sec:Taras}
      }
    \end{itemize}
  \end{multicols}
\end{infobox}

\clearpage

\subsection{Morkan}
\label{sec:Morkan}

\aren{Oh Ishmael, thy weakness caused so much tragedy and death.}

Morkan (old Teranim for ``beacon'') is a city kingdom on the south eastern
shores of \nameref{sec:Iafandir}. It is often referred to as the ``northern''
or simply the ``tyrant kingdom''.

The kingdom of Morkan has a red, and white banner with a large golden grown
nested in the middle of a shield. Red and white are the main colours of
Morkan.

\subsubsection{History}

The youngest city kingdom of them all, it was founded in MI:1938 by Xian, a
human warrior and barbarian native to Iâfandir. He gathered support for a
campaign towards Iâfandir in \nameref{sec:Hraglund} and \nameref{sec:Norbury}
to establish a human settlement and military outpost in the rather
inhospitable lands of the beast races. Especially in Norbury his plan of
another military outpost against the invasion was well received, and liked
among both nobility and the general populace. He sailed from the shores of
Norbury with his friends and allies Arissa, and Taras in MI:1930. It took Xian
eight years to build a small settlement, as well as bringing most of the local
humanoid tribes of his new land under his reign. But in MI:1938 Norbury
officially supported and endorsed the foundation of a new monarchy, ruling
over the city kingdom of Morkan.

Xias ruled just, and preferred inviting and integrating tribes, rather than
conquering them. Most of the local tribes were humanoid, and many of his
followers from Norbury supported that decision. But when Xias peacefully
integrated non humanoid races, namely a tribe of trolls and ogres, his
popularity waned within his own kingdom. Taras openly opposed him, claiming
that beast races had no place within city kingdoms. In MI:1942 an
altercation between a handful of dwarves and ogres escalated into a short
civil conflict, costing the lives of almost 400 citizens. Xian, still insisted
on finding a peaceful solution to the problem, lost the majority of his power
in a coup to Taras who proclaimed himself ``steward''. Taras continued to
consolidate his power in the royal court, and also gained the support of the
humanoid population by forcing the ogres into exile.

The ogres were forcefully and violently driven from the town in the winter of
MI:1942, and many of the tribe - especially weaker ogres such as the
elderly, women and children - starved or froze to death during the exodus. Even
though the trolls were innocent in this altercation they supported the ogres,
leaving the town along with them. Xian, with his position of power now
severely weakened, found himself on the throne without any real power. His
fate as a ruler was sealed, when in the spring of MI:1943 the tribe of
ogres and troll returned to lay siege upon the city.

The ogres and trolls laid siege the small kingdom of Morkan for several
months. And although the kingdom withstood, the small nation was brought to
the brink of collapse. Taras was sent forth to secure aid from Norbury, while
Arissa and Xian tried to break the siege. Unbeknownst to Taras, they attempted
to break the siege through diplomacy. During the summer of MI:1943 Xian
had secured peace with the trolls and ogres through diplomatic means, and
called Taras home. However Taras returned with an army of volunteers from
Norbury, and rejected the peace treaty. Taras still had the support of a large
portion of the populace, and thus broke the siege forcefully killing most of the
ogres and trolls, while enslaving the rest. His popularity soared after the
victory, and he had Xian and Arissa arrested for high treason.

\subsubsection{A new ruler}

Taras outlawed all beast races from the city kingdom, instituted slavery,
and turned his gaze outward. He proposed a new plan for Iâfandir: make it free
of beast races that would otherwise go forth and raid Eilean Mor and terrorise
peaceful and innocent humanoids in the process. And in turn he'd give the newly
conquered land to those humanoid races to farm and settle upon. Although many
moderates disagreed with him and returned to Norbury, their word of mouth
attracted the radical hard-liners that found Norbury's policy of defence
insufficient or even a weakness.

Taras continued to reign absolute in the small northern kingdom. In MI:1968 it
was revealed that Taras had achieved lich-dom, potentially sealing his reign
of the kingdom for eternity.

\subsubsection{Beast Wars}
\label{sec:Beast Wars}

Taras formed two new armies: the red legion and the iron hand, and set them
loose upon the continent to enact his plan. Morkan is engaged in a never
ending genocidal campaign against any non-humanoid beast races (such as ogres,
trolls, harpies) while offering sanctuary to humanoid species (such as pale
elves or the northern human barbarian tribes). The campaign of massacres,
genocides and massive enslavement of the tribes and species of the continent
became known as the beast wars. The armies of Morkan move further west with
each passing year, forcing all other tribes to flee westward. Some tribes have
tried to flee across the ocean toward Eilean Mor, but are intercepted - and
often slaughtered - on the high sea by Norbury. As of MI:2022 this campaign
has been going on for almost eighty years, without any end in sight.

\subsubsection{Growth and Prosperity}

Over the course of decades Taras shaped the city kingdom of Morkan into a
prosperous, huge city kingdom. Although the continent was harsh and lacking
fertile farmland, he built the wealth of the kingdom by exporting precious
minerals and slaves. By MI:2000 it had roughly 490000 registered citizens, and
uncountable number of slaves. Most of the citizens were humans (33\%) and
elves (31\%), with dwarves (20\%) and halflings (16\%) being in the
minority. Slaves are not counted as citizens, but it is estimated that Morkan
has around 150.000 to 200.000 slaves.

Common male names are: Aki, Asvald, Björn, Einar, Folke, Gunar, Halfdan, Ivar,
Ketill, Olvir, Sigurd, Thorbjörn, Thorgnyr, Toki, Ulfrid, Walther, Yngvar

Common female names are: Alfhild, Asa, Brynja, Eydis, Gyda, Idunn, Katla,
Miriam, Nele, Siglind, Yngvild

\subsubsection{Culture}

Morkan is a tyrannical, fascistic regime that actively represses any dissident
within its own population. Youth are brainwashed at an early age to support
the kingdom's goals of the beast wars. Citizens of Morkan see themselves, and
the pure humanoid races, as the true rulers of Aror and have great disdain for
all beast races, and even half humanoid, half beast races (which they call
``mongrel races'').

The average Morkan is patriotic, militaristic and heavily aligned with the
ideals of the kingdom. Art, song and creative enterprises have no place within
Morkan, unless it furthers the kingdom's narrative of racial supremacy. However
Taras promotes a culture and idolised ``citizen of Morkan'': A person who is
orderly, hard working, conscientious always ready to pick up a sword to defend
king and country, and to whom family, camaraderie is as important as the
``common goal'' of the empire.

Foreigners are often greeted with mistrust, as they are often seen as spies
from foreign powers, trying to undermine the righteous goals of Morkan.
Visitors must expect to be under constant surveillance while staying in the
kingdom, as well as having any business dealings put under intense scrutiny.

\subsubsection{Laws}

Taras rules supreme, and not even his aspects (generals, advisers and
representatives) dare to speak against his word. Slavery is encouraged, as well
as enforced upon beast races and those deemed of ``impure'' blood, criminals
and citizens. Slaves have no rights, no freedoms, and are often sold for
profit to other city kingdoms.

A secret police called the Kaltar silences dissident movements and resistance
cells both outside and within Morkan. They are known to torture and abduct
people, and spy on the general populace to obtain information on dissidents.
The Kaltar are only responsible for anti-government crimes, while the
Inquisition handles criminal and civil law cases. Citizens of Morkan can
expect a reasonably fair trial, unless their crimes include treason or
anti-government activities.

\subsubsection{Aspects}

Taras appointed various high ranking officials with to be his aspects. He
proclaimed that these aspects are a mere extension of his will, and thus
entrusted them with a specific task within the kingdom.

Arissa, the Aspect of Shadow. The elven rogue, and diplomat Arissa returned
from her imprisonment in MI:1978 openly denouncing the actions of Xian, and
joining the kingdom at Taras' side as an aspect of his rule. She oversees the
Kaltar, the secret police and spy organisation of Morkan.

Torem, the Aspect of War, a human barbarian leads the Red Legion, the kingdoms
main fighting force. A brutal, and savage slaughterer of thousands, but also a
brilliant tactician and general.

Eilane, the Aspect of Law, a snow elven diplomat, and wizard. She oversees the
Inquisitors, a small but effective army of judges and executioners. She is
responsible for enforcing Taras' civil law both within Morkan and in the newly
conquered regions.

Karim, the Aspect of Purity is a dwarven cleric of the church of Morkan. He's
mainly responsible for leading the kingdom in spiritual matters, as well as
spreading the church of Morkan among the newly conquered territories.

Telena, the Aspect of Truth, is a human sorceress. She is responsible for the
internal propaganda of the kingdom. She is also directly responsible for the
diplomatic relations with other city kingdoms and nations.

\subsubsection{Church of Morkan}
\label{sec:Church of Morkan}

One by one Taras outlawed the other religions in Morkan. He replaced them with
a Church of Morkan, a religious institution headed by Karim the aspect of
purity. In the canon Taras himself is praised as the divine incarnation of the
will of his people to bring peace and purity upon the land. Membership in the
church is not mandatory for citizens, but required if one wishes to have a
prosperous career within Morkan. The church holds daily masses at midday, but
also teaches prayers and rituals to be held should one not be able to attend
church. Scholars do not know how or why, but priests and paladins of the
church of Morkan are granted divine spells and power. It is still an unsolved
mystery how Taras accomplishes this feat, as he is the sole deity worshipped
by the Church of Morkan.

\subsubsection{Iron Hand}
\label{sec:Iron Hand}

The Iron Hand are the elite soldiers of Morkan. Heavily armed, armoured and
well equipped with magical equipment, they partake in the conflicts of the
beast wars as necessary, but are also the personal guard of Taras
himself. They are picked from the best of the best of the Red Legion and also
trained in special operations. For many, there is no greater honour than to
serve as personal guards to Taras himself.

\subsubsection{Red Legion}
\label{sec:Red Legion}

The Red Legion is the main stay army of Morkan. It is known for its
fierceness and brutality in battle, as well as for its cruelty it inflicts
upon captured towns and villages. All subjects of Morkan must complete a
three year service within the red legion to attain citizenship status. Humanoid
slaves must complete a ten year service within the legion before they are
granted citizenship status.

The legion takes able bodied humanoid men and women from towns they
capture, while killing the rest. Those that show promise as fighters are
recruited, while the rest are enslaved. They are also tasked with enacting the
decree of purity, the genocide of beast races, upon Iâfandir.

A sizeable amount of legionaires were also sent to fight with the coalition
against the \hyperref[sec:Devil Siege]{devil invaders of Forsby}. Although
less organised and disciplined than its counter part armies, it proved itself
as a strong and ruthless fighting force in the face of a superior enemy.

\subsubsection{Kaltar}
\label{sec:Kaltar}

The Kaltar (old Teranim for ``owl'') is the secret police of Morkan. Lead by
the Arissa the aspect of shadow, it is tasked with finding and eradicating
dissident movements, traitors and enemies of the state.  It achieves this goal
with a vast network of civil informants, highly trained spies and field
operatives, and various outposts all over the world that help it enact Tara's
will outside of the city gates. The Kaltar are feared all over the kingdom for
abducting, torturing and murdering everyone they suspect of undermining the
rightful rule of Taras.

\subsubsection{Relations}

Morkan has only one true ally: Norbury. All other city kingdoms paused their
diplomatic relations with the kingdom after the beast wars began. The kingdom
of Morkan is a signer of the \nameref{sec:Vonir Accord}.

It is an open secret that Norbury trades heavily with Morkan. In exchange for
slaves, iron and copper, Norbury sends massive amounts of food and medicine to
its northern neighbour. The rapid and aggressive expansion, combined with a
scorched earth policy, and a heavy governmental focus on warfare has made
Morkan highly dependant on the aid of Norbury to be able to feed its own
population.

%% City Kingdom of Norbury
\subsection*{Norbury}

\graham{Surely the most vile city kingdom Aror has to offer...}
\aren{Bless thy innocent heart, for you have not lived long enough to see the
  rise of Morkan.}

The second youngest city kingdom, \emph{Norbury} resides on a large island off
the cost of northern coast of the continent \emph{Eilean Mor}.

\subsubsection*{History}

It was founded around \emph{GT:15500} as a joint military outpost of
\emph{Hraglund} and other northern baronies of Eilean Mor. It was originally
founded as first line of defence against the many raids of the beast races
that came from the northern most continent of \emph{Iâfandir}.

%% City Kingdom of Tredegår
\subsection{Tredegår}
\label{sec:Tredegar}

The city kingdom of Tredegår} (old-high Teranim for ``three courtyards'')
lies on the south western shores of \nameref{sec:Eilean Mor} west of the
\hyperref[sec:Great Divide]{great divide}. It is encased by two large rivers:
The Morre river to the north and west, and the Moy river to the
east and south. It's close proximity to the mineral wealth of the great
divide, the easy access to two rivers and the sea made Tredegår one of the
richest city kingdoms of all of Aror.

\subsubsection{An'Rath}
\label{sec:AnRath}

In the centre of the city is the vast palace of An'Rath, which houses roughly
a thousand people. Workers, nobles, clergymen, bureaucrats, judges, advisers,
as well as the ruling family live within the palace. The palace by itself is
an impressive architectural feat, and is constantly being extended with
additional towers and buildings. The palace has its own army, and its own
inner stone wall for protection, and even inner draw bridges and gates to
separate individual parts from one another in case of a siege. The outer ring
of the castle is open to the public, and houses a library, a school that
offers various courses, its own brewery and tavern the ``The Drunken Trader''.

\subsubsection{Banner}

The banner of Tredegår features a white tree with three thick branches resting
within a shield. A golden crown rests above the shield. Gold, white and blue -
representing the gold of the great divide, the white snow up on their peaks,
and the blue rivers and the sea - are the colours of Tredegår.

\subsubsection{Population}

Among all the city kingdoms Tredegår it is of the biggest. It houses roughly
32 million inhabitants, when all outlying smaller cities and villages are
included. Most of these are humans (55\%), with dwarves that once lived in the
great divide following close second (21\%) followed by elves (10\%), half
races (12\%) and various others (2\%).

Common male names are: Åke, Albert, Alex, Alfred, Birger, Bjarne, Björn, Danne,
Einar, Felix, Gunnar, Halvar, Hjalmar, Holger, Janne, Kalle, Karl, Magnus,
Mikael, Nils, Ola, Per, Roland, Sigfrid, Sven, Varg, Yngve

Common female names are: Agata, Aina, Alicia, Alva, Amanda, Anja, Åsa, Birgit,
Britta, Carolina, Dina, Eira, Elina, Emma, Hannah, Irene, Janna, Johanna,
Julia, Laila, Linnea, Margareta, Malin, Pia, Runa, Sara, Sofia, Tea, Thora,
Vera, Veronika, Vivian

\subsubsection{Culture}

The immense wealth of the city has trickled down to most of the lower and middle
class citizens. Thus the people of Tredegår live in an unparalleled state of
social security that is only known in that kingdom. Most people of the city
work far less than their contemporaries in other kingdoms. Still they make
enough money to life comfortably. This gives the people of Tredegår time and
opportunity to enhance their education and other skills. The people of Tredegår
are often described as well educated, easy going and relaxed; with strong focus
on self improvement and family. Some have described them as arrogant and snobby,
although those attitudes can often be exaggerated by envy. A typical Tredegår
craftsman for example works slow, meticulously and prefers to deliver quality
over quantity. This culture of ``do it right; or do not do it at all'' has
earned the city a good reputation as a reliable trading partner that delivers
excellent product.

This culture is also present within the kingdom were houses, streets, bridges
are seldom derelict or run down. Instead they are always lavishly decorated with
flowers, and the city maintains a network of arcane street illumination housed
in lamp posts. Public workers keep the streets clean and make sure that the
city is always in good shape. Although the kingdom has no slum or worker
district, poverty does still exist within the city, especially among the sick
and crippled.

\subsubsection{Steel and Smithing}

The Tredegår Steel and Gold Corporation is one a large company that
runs many forges and smithies across the city. It works off the raw ore mined
either in the great divide, or ore that has been brought in from the
\nameref{sec:Silver Isles}. The corporation is known for its excellent
craftsmanship in weapons and armours, and their products are sold all across
the major kingdoms and are renowned for their quality and steep price. It also
has the Tredegår tree as a logo which can be found on many of their armour
and weapons. This logo has become a sign and indication for top quality in
crafted goods, steel, weapons and armour.

\subsubsection{Society}

Due to the immense wealth of the city, slavery has been abolished several
centuries ago. The city has been ruled by the same noble family Gylleborg
for thousands of years. Both female and male heirs may rule. The city kingdom
has a long history of peace and prosperity, and the justice system is known
all around the world as one the best of its kind. Independent judges and
prosecutors work in tandem with private defence attorneys to ensure the
justice system remains fair and independent.

\subsubsection{Relations}

The city and its nobility is closely related to \nameref{sec:House Ranian},
and the house has the main headquarters near the main square of the city.

Due to their status as rich trading partner, the city kingdom of Tredegår is
in good standing with almost all other city kingdoms, except
\nameref{sec:Morkan}.  Their longest standing trading partner and ally is the
city kingdom of \nameref{sec:Fes al-Bashir} to the south. The city kingdom
strengthened their bond with the other city states by providing logistical
advisers, money and their best smiths during the rebuilding of
\nameref{sec:Forsby}. The kingdom is a signer of the \nameref{sec:Vonir
  Accord}.


