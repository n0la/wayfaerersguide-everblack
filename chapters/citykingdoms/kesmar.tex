\cleardoubleevenemptypage

\ifimages
\begin{figure*}[ht!]
  \centering
  \vspace{-4.6cm}
  \centerline{
    \includegraphics[width=\paperwidth,keepaspectratio]{%
      media/kesmar.\imagesuffix
    }
  }
  \par
  Kesmar has some technological wonders found nowhere else on Aror.
\end{figure*}
\fi

\begin{infobox}{City Kingdom of Kesmar}
  \ifimages
  \begin{subfigure}[t]{\textwidth}
    \centering
    \includegraphics[width=0.20\linewidth]{media/kesmar-banner.\imagesuffix}
  \end{subfigure}
  \vspace{-1.0cm}
  \fi
  \begin{multicols}{2}
    \begin{itemize}[label={},noitemsep,leftmargin=0.0cm,topsep=0pt]
      \infoboxitem{Location}{Eastern shores of north \nameref{sec:Goltir}
      }
      \infoboxitem{Languages}{Rutari, Teranim}
      \infoboxitem{Government}{Absolute Monarchy}
      \infoboxitem{Major Religions}{\nameref{sec:Order}}
      \infoboxitem{Area}{est. 60,000 $km^2$}
      \infoboxitem{Population}{est. 1.2 million dwarves, est. 6 to 8 million in
        total}
      \infoboxitem{Non Grata}{monstrous races, undead, Gorgons, devils, druids
      }
      \infoboxitem{Magic}{Magic use outlawed for non-citizens
      }
      \infoboxitem{Slavery}{yes, all forms, signer of the \nameref{sec:Vonir
          Accord}
      }
      \infoboxitem{Special Laws}{only dwarves may become citizens, non-dwarves
        are banned from entering the mountain halls without invitation
      }
      \infoboxitem{Notable Organisations}{\nameref{sec:House Ranian}}
      \infoboxitem{POI}{mountain halls of the Blackhammer clan, Colosseum for
        sports and gladiatorial championships, river delta of the Dranoa river
        used as recreational area
      }
    \end{itemize}
  \end{multicols}
\end{infobox}

\clearpage

\subsection{Kesmar}
\label{sec:Kesmar}

Kesmar is a large city kingdom on the eastern shores of
\hyperref[sec:Goltir]{North Goltir}, first recognised as a city kingdom is
GT:2992. The city's banner features a silver mountain over a wavy sea,
representing the city's mineral wealth, as well as the sea and trade route
that made the kingdom rich.

\subsubsection{History}

It was founded in \emph{GT:2492}, by a dwarven clan that sought to built a
trading post at the shores of the sea. A large river called the \emph{Dranoa}
connects the city with the northern mountain range called the
\nameref{sec:Cnamh Mountains} from which the dwarves originally hail. Ships
and boats are used to transport the wealth the dwarves mined from their
mountain down to their trading post. From there the various ores, gemstones,
and coal are sold to other city kingdoms.

It grew to this size by invading other smaller dwarven clan, and then
incorporating them into the Blackhammer clan. Any other deep folk that lived
in close proximity to the blackhammer clan, such as various deepkin, dark
elven and Ilians, were either displaced or killed by the dwarves to cement
their sole rule over that area of the Cnámh mountains.

The dwarves of the \emph{Blackhammer Clan} did not wish to ``soil'' their
culture and civilisation with other humanoid races or even beast
races. However they saw the need for a trading post to acquire some of the
luxury goods the other kingdoms had to offer. Although the original city
kingdom of the Blackhammer Clan came first, the city of Kesmar has now
overtaken the clan's mountain kingdom in sheer size and population. The
dichotomy of the two cities allowed the clan to keep their main mountain
kingdom pure, while allowing foreigners to mingle with the dwarves in their
city outpost.

\subsubsection{Population}

Officially the city holds 1.2 million dwarves. The size of the mountain
kingdom is unknown. It is uncertain how many other humanoid races live in
the city or the mountain kingdom as these are not registered citizens. But
estimates wary from 6 to 8 million humanoids.

Common male dwarven names are: Baermar, Ebdohr, Garmor, Grammus, Hjalthor,
Kardohr, Thorrik, Tornus, Vonmund

Common female dwarven names are: Arma, Belvia, Bryllyn, Mistnis, Nisnura, Tazvia,
Tisra, Sara, Tyleen, Vinara

\subsubsection{Blackhammer Clan}
\label{sec:Blackhammer Clan}

The Blackhammer Clan still lives deeply embedded in their old traditions in
their mountain kingdom buried deep within the Cnámh mountains. Highly
xenophobic, and unwelcoming of outsiders, only highly valued ambassadors,
foreign monarchs or wealthy and powerful traders are invited to the reclusive
dwarven clan. The dwarven clan leader also rules the city of Kesmar as a
monarch and king, albeit never sets foot into the city. The city is ruled per
proxy by a dwarven steward.

The clan mines the mountains for all sorts of ore (such as copper, iron and
silver), as well as precious gemstones, gold and coal. The wealth accumulated
from trading these minerals away to other city kingdoms is used to fund a huge
army, as well as importing agricultural products such as food, hops, coffee or
chocolate from other city kingdoms.

Furthermore the Blackhammer Clan's mountain kingdom has a reputation for being
nigh impossible to take militarily. The steep mountains, harsh climate of the
northern regions of Goltir, as well as the fierce fighting spirit of the dwarves
make a siege a costly endeavour.

\subsubsection{Kesmar}

The strict caste system is extended by one caste in the city, which is below
the soldier but above the slaves: \emph{foreigners}. The city grants this
caste to any and all foreigners that come to their city, thus allowing them to
live, trade and work within the city. Those that are given temporary visas as
foreigners hold, in theory, the same privileges as any dwarf within the city
as long as the visa is valid. However the dwarves of the city see themselves
as the rightful masters of the city, and treat all others with suspicion, or
even contempt. Although they know that these foreigners bring wealth and skill,
they often treat them as lesser than a regular dwarven citizen of the clan.
Much like any dwarf, foreigners run the risk of being permanently enslaved
should they break the city's rules and laws.

This climate and culture has created a high turn over rate for other humanoid
species, as many feel alienated and excluded within the city. Many other
humanoids stay as long as their business or endeavour requires them to stay,
and then leave straight away once that endeavour has concluded. Obtaining a
proper citizenship as a non-dwarf is all but impossible, but many humanoids
have been permanently seized as criminals by the dwarves and enslaved. These
are then forced to work the mines, smelters, factories and the shipping docks.

\subsubsection{Engineering}

The city is known for its vast production facilities, which create commodities
in large quantities for a fraction of the usual cost. Furthermore the city is
well known for its large metal works, forges, and smithies, and their looming
furnaces dot the skyline of the city, pumping both steam and black smoke into
the air. In stark contrast to \nameref{sec:Stenheim} which relies on the power
extracted from charged Everblack, most machines of the dwarves of Kesmar rely
on water, and steam.

Kesmar is further remarkable that all of its roads are cobbled, and almost
all of its houses are built from shaped stones, and their vast aqueducts and
sewers are equally a marvel of civil engineering. This wealth was made possibly
by both the precious resources exported by the kingdom, as well as the many
factories that could produce needed resources cheaply, and in large quantities.

The city's productivity also comes with a cost, as the smoke from excessive
burning of coal, and wood often darkens the skies, creates smog, and leave
walls, houses, and streets blackened by soot. This mountain keep is not affected
by the same pollution, as the mountain dwarves have ceased mass production of
goods millennia ago. The air pollution in Kesmar is sometimes so bad, that a
permanent cough (as experienced by miners) is called a ``Kesmar cold''.

\subsubsection{Rule}

The clan values old traditions above all else, and rules their mountain
kingdom as well as their city with a strict caste system. Of which the highest
caste are the nobles, followed by soldiers, and then the smiths, artisans and
traders, the foreigners, and then the slaves and criminals. Above the castes
sits the clan elder and monarch. Albeit the clan elder is also the king or queen
of the city, they instead defer rule of the city to a steward which directly
reports to the monarch.

The city is a signer of the \nameref{sec:Vonir Accord}, and openly trades away
their slaves to other slaving nations of Aror. Dwarven slaves from Kesmar and
the Blackhammer clan are highly regarded, as they make excellent miners and
heavy labourers.

\subsubsection{Relations}

The city kingdom holds rather poor relations with most other humanoid city
kingdoms, as the dwarven clan believes that contact with them should be limited
to the minimum necessary to facilitate trade. The Blackhammer Clan has almost no
diplomatic corps, and rarely joins or hosts diplomatic festivities. The
dwarven clan wishes to limit their diplomatic entanglement with other city
kingdoms, as they believe that their vast material wealth will allow them to
project their force and power without having to find mainstay allies among the
other city kingdoms.
