\cleardoubleevenemptypage

\ifimages
\begin{figure*}[ht!]
  \centering
  \vspace{-3.6cm}
  \centerline{
    \includegraphics[width=\paperwidth,keepaspectratio]{%
      media/avenfjord.\imagesuffix
    }
  }
  \par
  Avenfjord being rebuilt, circa MI:1922
\end{figure*}
\fi

\begin{infobox}{City Kingdom of Avenfjord}
  \ifimages
  \begin{subfigure}[t]{\textwidth}
    \centering
    \includegraphics[width=0.22\linewidth]{%
      media/avenfjord-bannersm.\imagesuffix
    }
  \end{subfigure}
  \fi
 \begin{multicols}{2}
    \begin{itemize}[label={},noitemsep,leftmargin=0pt,topsep=0pt]
      \infoboxitem{Location}{Southern coast of \nameref{sec:Goltir}}
      \infoboxitem{Languages}{Enro'ad, Taavid, Teranim}
      \infoboxitem{Government}{Absolute dual monarchy}
      \infoboxitem{Major Religions}{\nameref{sec:Order}, \nameref{sec:Lor},
        \nameref{sec:Forun}
      }
      \infoboxitem{Area}{est. 30,000 $km^2$}
      \infoboxitem{Population}{est. 2 million, mostly high-elves (48\%), and
        halflings (39\%)
      }
      \infoboxitem{Non Grata}{monstrous races, druids, devils}
      \infoboxitem{Magic}{all magic allowed, except \emph{summoning},
        \emph{calling}, and \emph{necromancy} which require a permit
      }
      \infoboxitem{Slavery}{outlawed, and criminalise}
      \infoboxitem{Special Laws}{not a signer of the \nameref{sec:Vonir Accord}
      }
      \infoboxitem{Notable Organisations}{\nameref{sec:House Ranian},
        \nameref{sec:Ror-Aram Trading Corporation}
      }
      \infoboxitem{POI}{Dragon teleporter's tower, Amphitheatre, people's opera,
        cathedral of the Order, knights temple of Lor, Naschmarkt (food market),
        biological garden, parks along the banks of the \emph{Al'hari} river
      }
    \end{itemize}
  \end{multicols}
\end{infobox}

\clearpage

\subsection{Avenfjord}
\label{sec:Avenfjord}

The rebuilt kingdom of \emph{Avenfjord} (``from the ashes'' in Enro'ad) of the
elves and halflings sits on the southern shore of the continent of
\nameref{sec:Goltir}. It settled in the vast and fertile river delta of the
\emph{Al'ahri} river. Avenfjord is one of the youngest of all the city
kingdoms, and also the smallest. It is a successor state to
\nameref{sec:Nen-Hilith}.

\subsubsection{Banner}

The banner of Avenfjord depicts two large crowned towers built together by a
bridge, with a backdrop of blue (for the sea) and green (for the fertile farm
land). It was the same banner of Nen-Hilith, and was taken to the new found
city nation without much change. Green, and blue are the royal colours of
Avenfjord.

\subsubsection{History}

It was founded in MI:1920, after the giants, an extra planar race of towering
humanoids, destroyed the previous city of the halflings and elves called
\nameref{sec:Nen-Hilith}. They had invaded the continent of
\nameref{sec:Farlar} thirty years prior, to wage war against the dragons who
also call that continent their home. Nen-Hilith was situated on the northern
shores of Farlar (just across the sea from where Avenfjord stands now).

For many years the elves and halflings were untouched and neutral in the war
between the dragons and the giants. But as the giants had seized most of the
rivers and lakes south of Nen-Hilith, and the fighting had crept north towards
the outlying villages and towns the elves and halflings joined the war on the
side of the dragons in MI:1916. After several failed campaigns to regain
control over the city's water supply, the giants began a devastating siege
against the city in MI:1918. After enduring the siege for more than two years,
with the military aid of various allies, the starved and weakened elves and
halflings were forced to abandon Nen-Hilith and flee across the sea to the
north. The giants then razed the city to the ground.

\subsubsection{Rebuilding}

Many city kingdoms, including \nameref{sec:Forsby}, \nameref{sec:Hraglund},
and \nameref{sec:Fes al-Bashir}, arrived with ships to aid Nen-Hilith during
the evacuation, and as it became clear that a newly founded city kingdom could
not support all the refugees that had been displaced, many sought shelter
within those city kingdoms. The refugees were welcomed warmly, and local
authorities of the city kingdoms and baronies drummed up support from their
population in an effort to help the kingdom of Avenfjord. The people donated
money, materials, or even travelled to Avenfjord to lend the kingdom their
expertise as artisans, artists, traders, labourers, policemen, or even
soldiers. Over the course of many decades almost all citizens of Nen-Hilith
returned to Avenfjord, as the city regained the infrastructure to house and
support its people.

One thing however could not be rebuilt: the dragon teleporter. During the
siege of the city, after it became clear that the city was lost, the mages
were forced to destroy the teleporters or it would have fallen into the hands
of the Giants. To this day the mages of Avenfjord are funding expeditions to
find a replacement, and many attempts have been made to recover pieces of the
original dragon teleporter from the ruins of Nen-Hilith. But for now Avenfjord
remains the only city kingdom that is not connected to the dragon teleporter
network.

\subsubsection{Formian War}
\label{sec:Formian War}

In MI:1926 a war broke out between a hive of formians who objected to the
elves settling in their lands. The kingdom was still in the early stages of
rebuilding, and the new war threatened the very existence of the nation. The
then ruler of Avenfjord, King Ishmael the II., allowed a travelling wizard
named \nameref{sec:Taras} to combat the formians by adapting the
\nameref{sec:Black Blight} to weaken them, in the hopes of driving the further
north. Going far beyond anyone's expectations, the modified plague infected
and killed the vast majority of formians living in the Goban desert north of
Avenfjord. After realising that he had been part of a genocide upon an entire
sentient species King Ishmael II exiled Taras, who claimed that he did not
know the blight would kill the formians. Later in the same year King Ishmael
II committed suicide, and his son King Ishmael III took the throne soon after
his father's passing.

\aren{If he just had enough resolve left to take Taras with him...}

\subsubsection{Culture}

Although one might suspect that the near destruction would shake the culture
of the elves and halflings to the core, you might be wrong. Instead the
formian war, and the destruction of their previous city hardened the nation's
stance of non-interference. The people of Avenfjord prefer not to meddle in
other people's issues and troubles, and prefer diplomatic solutions over
conflict.

Avenfjord are known for their generous patronage for the arts and sciences and
house many theatres, libraries, galleries and workshops. They fund one of the
largest arcane academies on Aror, called the Mage's Guild. The elves and
halflings of Avenfjord are often described as jovial, carefree but creative
and expert diplomats. They are an open society, and especially welcome other
artists and craftsmen into their cities.

The hardships they had to endure did very little to damper their spirits, and
what experience you might have had in Nen-Hilith, still holds true within
Avenfjord.

\subsubsection{Population}

The city of Avenfjord is now home to roughly two million people, yet the city
of \emph{Nen-Hilith} was home to almost eleven million at its peak. Most of
the citizens of Avenfjord are elves (48\%) and halflings (39\%), with dwarves
following third (8\%) and all other races making up the remaining 5\%.

Common male names are: Aelius, Caius, Felix, Florian, Horatio, Julius, Marcus,
Marius, Otho, Ovid, Senec, Tacitus, Varo, Vitus.

Common female names are: Aelia, Aemilia, Aurelia, Caelia, Cassia, Claudia,
Flavia, Flora, Hadriana, Julia, Lucia, Marina, Paula, Sabina, Titiana,
Valentina, Vita.

\subsubsection{Rule}

Avenfjord is a dual monarchy, where both the reigning king or queen of the
elves, and the reigning monarch of the halflings rule together. Neither of the
monarchs has absolute power to reign alone. Although Avenfjord holds houses of
nobility, their power and influence is minimal compared to the houses of other
city kingdoms. Slavery and serfdom is outlawed, and all citizens of Avenfjord
are free people. The city also has a fair court, and almost corrupt free city
guard, that enact the laws the monarch sign into law fairly.

\subsubsection{Relations}

Avenfjord has not signed the \nameref{sec:Vonir Accord}. Although this puts
citizens of Avenfjord at risk of enslavement should they travel to Norbury,
such cases are extremely rare. Neither Norbury, nor Fes al-Bashir wish to
raise diplomatic issues between them and Avenfjord.

After Nen-Hilith received little aid from the warrior nations during its
siege, Avenfjord has sought to rectify the decrepit relations with its
peers. Avenfjord generally holds good relations with all city kingdoms, even
with \nameref{sec:Morkan}.

Though traditionally heavily aligned with the city kingdom of Forsby, the
failure to send military aid that was promised during that city's siege
soured relations between the two kingdoms.
