\cleardoubleevenemptypage

\ifimages
\begin{figure*}[ht!]
  \centering
  \vspace{-4.6cm}
  \centerline{
    \includegraphics[width=\paperwidth,keepaspectratio]{%
      media/hraglund.\imagesuffix
    }
  }
  \par
  Some heroes should never be forgotten.
\end{figure*}
\fi

\begin{infobox}{City Kingdom of Hraglund}
  \ifimages
  \begin{subfigure}[t]{\textwidth}
    \centering
    \includegraphics[width=0.20\linewidth]{media/hraglund-banner.\imagesuffix}
  \end{subfigure}
  \vspace{-1.0cm}
  \fi
  \begin{multicols}{2}
    \begin{itemize}[label={},noitemsep,leftmargin=0.0cm,topsep=0pt]
      \infoboxitem{Location}{Eastern shores of \nameref{sec:Eilean Mor}
      }
      \infoboxitem{Languages}{Teranim, Old Teranim, Enro'ad}
      \infoboxitem{Government}{Absolute Monarchy}
      \infoboxitem{Major Religions}{\nameref{sec:Lor} (city),
        \nameref{sec:Old Ways} (rural areas)
      }
      \infoboxitem{Area}{est. 110,000 $km^2$}
      \infoboxitem{Population}{est. 18 million in total}
      \infoboxitem{Non Grata}{monstrous races (except the royal guard),
        devils, druids, followers of \nameref{sec:Three Kings}, undead,
        Gorgons, Aren Fel
      }
      \infoboxitem{Magic}{divine magic is allowed, arcane magic, and psionic
        powers overseen by the church of \nameref{sec:Lor}, soul magic and is
        outlawed
      }
      \infoboxitem{Slavery}{outlawed, still signer of the \nameref{sec:Vonir
          Accord}
      }
      \infoboxitem{Special Laws}{worship of lesser deities other than
        \nameref{sec:Lor} is heavily policed, and suppressed when it comes
        to evil deities
      }
      \infoboxitem{Notable Organisations}{headquarter of the \nameref{sec:Knight
          Order of Tavos}, \nameref{sec:House Ranian}, headquarter of the
        \nameref{sec:Wayfaerers Guild}
      }
      \infoboxitem{POI}{cathedral to \nameref{sec:Lor}, original founding site
        of the Wayfaerers, central park, castle Hrag built by
        \nameref{sec:Eigyr}
      }
    \end{itemize}
  \end{multicols}
\end{infobox}

\clearpage

\subsection{Hraglund}
\label{sec:Hraglund}

\emph{Hraglund} is a humanoid city kingdom on the eastern shores of
\nameref{sec:Eilean Mor}. It is one of the oldest city kingdoms on Aror,
having been founded in \emph{GT:339}.

Before the plague the city kingdom has roughly 31 million inhabitants, and
after only about 9 million, before rising again towards 18 million. Most of
which most were humans (43\%), elves (25\%), halflings (21\%) and dwarves
(9\%) and other various half races (2\%).

The kingdom flies a green banner with a white shield that contains a five
headed hydra. This is also the banner which had been used by the
\nameref{sec:Wayfaerers Guild} since the foundations of the first hunting camp.

\subsubsection{History}

\emph{Hraglund} grew out of smaller villages that settled around a fortified
hunting camp called \nameref{sec:Wayfaerers Guild}. The hunters proofed to be
effective in deterring the attacks and raids from the monstrous races that
lived in the surrounding area of the guild. They thus attracted more and more
settlers, farmers and workers. Soon the hunting camp grew, first into a small
military outpost and over the span of several centuries into a full kingdom.

Once \emph{Hraglund} was the biggest city on \emph{Eilean Mor}, with over
31 million people living within the kingdom and in the vast outlying lands
that included many smaller and bigger cities, such as the river trading city
of \emph{Braemer}.

\subsubsection{Siege during the Holy Crusade}
\label{sec:Siege of Hraglund}

In \emph{MI:-2} the city was under siege by vast armies that had declared war
against \nameref{sec:Griannar} and his followers, in the name of the
\nameref{sec:Silent Queen}. This siege was part of a decade long war known as
the \nameref{sec:Holy Crusade}. Most of these armies that laid siege to the
city were mercenaries that had been paid by the priests and priestesses of the
Silent Queen. Hraglund had been a major centre for the faith of Griannar, run
by the cardinal of the Holy Church. At first the city defended itself against
the siege for several months, but the majority of the population grew unruly
and attempted to oust the Church of Griannar from the city. After months of
siege civil unrest and even skirmishes broke out between the city guard,
believers on one side, and part of the population that wanted to exile the
church from the city, on the other side. The mob believed that if the church
were ousted from the kingdom, the armies in front of the gates would abort
their siege. During the civil unrest the cardinal fled the city, and was
ultimately betrayed by the royalty of the city. He was captured and executed
by the followers of the Silent Queen. Just as the rebellious elements
predicted, the besieging army left soon after.

\subsubsection{Plague}
\label{sec:Plague}

In \emph{MI:1680} the city was struck with a major outbreak of
the \nameref{sec:Black Blight}, in which roughly 12 million people lost their
lives. The city was crippled, weakened, and lost most of its military power
and influence in the resulting chaos. The city not only lost many of its
inhabitants, but also lost their king, and the last heir of their ruling house
of nobility \emph{Antred}, which had ruled the city for many centuries. The
city was taken over by a council of high ranking officials and dukes that have
been trying to restore order to the kingdom. It was later revealed that a mad
wizard and follower of the \nameref{sec:Aria} had allowed a plague bearer into
the city in an attempt to take revenge against the kingdom's then ruling
religion of Lor. The wizard claimed he had a personal vengeance against the
kingdom and church of Lor, as his brother - a vampire doctor - was executed by
the church for necromancy. He was sentenced and publicly hanged for high
treason.

\subsubsection{War with Terevar}
\label{sec:Terevar}

Many of the surrounding baronies sensed the weakness in Hraglund after the
plague had been defeated in \emph{MI:1682}. After consolidating into a large
alliance called Terevar, these baronies declared war on Hraglund in an attempt
to take control of the city. Although Terevar was inferior in terms of man
power, tactics and resources, the war dragged on for several years. Hraglund
had lost most of its fighting forces but its overwhelming wealth made it
possible for the kingdom to hire mercenaries to fight the war for them. The
war cost both sides countless lives, and displaced even more civilians. The
war was ultimately lost by Terevar when \nameref{sec:Tredegar} joined the war
as allies on the side of Hraglund. Terevar fell apart again into various
smaller baronies in the chaos and aftermath of their defeat. As war
reparations the kingdom annexed most of the border baronies and integrated
them into the kingdom. Afterwards the council of Hraglund was disbanded, a new
line of royalty was founded based upon House \emph{Aralia}. Supported by their
allies from Tredegår, Hraglund claimed on multiple occasions that Norbury had
secretly aided the alliance of Nerevar to weakened Hraglund's power in the
region. Something that Norbury vehemently denied.

\subsubsection{Royal Guard}
\label{sec:Royal Guard of Hraglund}

During the war with Terevar the city hired many mercenaries with its immense
wealth, including a sizeable group of \hyperref[sec:Hobgoblins]{hobgoblins}.
These hobgoblin soldiers impressed the then ruling duchess, \emph{Tirana III
  of Aralia}, with their fighting prowess, unwavering loyalty and fierceness
in battle, that she hired them as her personal guard. After the duchess was
crowned queen these hobgoblins were promoted to royal guards of the crown, and
were allowed to settle in the city. Their descendants live there to this day,
fulfilling their ancestor's pledge to protect the ruling monarch of the
Hraglund.

\subsubsection{Population}

Before the blight the city had roughly 49 million inhabitants, more than
\nameref{sec:Fes al-Bashir}. But after the blight its population dropped
drastically, just barely scratching 15 million in MI:2000. The population
is evenly spread among the core humanoid races, with elves being slightly
ahead (28\%), followed by humans (26\%), halflings (23\%) and dwarves
(22\%) and half races making about 1\%.

Common male names are: Alfgar, Alfred, Alred, Albert, Berengar, Bernard,
Cenric, Dunstan, Eadmund, Edwin, Godrik, Hadbert, Helmfrid, Karl, Norman,
Odo, Raimund, Sigismund, Theobald, Ulfrid, Walther, Yngvarr

Common female names are: Adelaide (Ada, Adela), Alflad, Aenor, Alfhild,
Amelina (Lina), Avelina, Brunhild (Hild, Hilda), Emma, Frida, Helga,
Hilda, Irma, Katla, Miriam, Nele, Sieglind (Lind, Linda), Yngvild

\subsubsection{Culture}

The population of Hraglund has always been very religious, with the city
population following Griannar before the crusade, and then joining the holy
church of Lor afterwards as many saw similarities in terms of dogma and
practice. The country population, with its close proximity to the
\nameref{sec:Dirgewood} have mostly been followers of the \nameref{sec:Old
  Ways} which has always caused a deep rift between both city and towns folk.

This rift is so deep, that the city itself is often at war with warring tribes
from the Dirgewood, who often take control of the outlying farms and
villages. For the people of the Dirgewood, the city kingdom is an affront to
their beliefs and traditions, as their nobility and Wayfaerer's Guild claim to
be descendants of \nameref{sec:Eigyr}, while restricting worship of the Old
Ways, choosing lesser deities over the spiritual traditions of the holy
mothers, and allowing monsters to protect their monarch. Many of the country
folk often welcome these incursions, protecting or covering for the warriors
of the Dirgewood, when the city's guard and knights inevitably come to retake
the conquered villages.

Generally the city people of Hraglund are seen as stoic, overly religious,
traditional, and keen to judge people by the wrong criteria. Since city, in
conjunction with the dogma of Lor, has outlawed tieflings, soul magick,
unlicenced arcane or psionic magic within the kingdom, many have become
suspicious of creatures and magic as a whole. The country folk of Hraglund are
often the opposite, practising a welcoming, open, and yet deeply spiritual
interpretation of the \nameref{sec:Old Ways} sheer out of spite for their city
overlords. Although tensions are high, and there are militants on both side
which would call for war, no civil war has yet broken out.

\subsubsection{Society}

The council of dukes and government officials rules the kingdom in
administrative matters, yet beneath the current ruling monarch of House
Aralia. The city kingdom is known for its stability, fair laws and for having
fought and defeated many problems and enemies in its past. They have a strong
culture, and are proud of their nation, which their society and culture
embodies. Unlike most of its neighbours the kingdom does not practice slavery,
yet still practices indentured servitude as an alternative to
incarceration. Even though slavery is banned, Hraglund has signed the
\nameref{sec:Vonir Accord}, in attempt to keep good relations with is
neighbour \nameref{sec:Norbury}.

Unlicenced arcane or psionic magic is forbidden within the city, and is
strictly enforced by the clergy of Lor. Any devil (including tieflings),
undead (even vampires), as well as Gorgons are forbidden to enter the kingdom.
The kingdom also puts a death penalty on many magical practices it deems evil,
such as soul magic, necromancy, druidic spell casting, or conjuration of
devils, daemons or demons.

\subsubsection{Industries}

Hraglund is surrounded by ample and fertile land of which most is used to
grow plants, crops, fruits and vegetables. Hraglund produces more food than
it requires itself, and exports food to many other neighbouring city kingdoms,
nations and baronies. It has no direct access to any mines, mountain ranges or
ores, so it imports either ore directly, or purchases metal products from
city kingdoms such as \nameref{sec:Stenheim}. Hraglund is known for its liberal
attitude towards trade, and so many traders of luxury goods have their main
offices in Hraglund to benefit from low interference and taxes.

The city also has no functioning mage's guild, and thus is often seen as
primitive in terms of arcane industry, research and development. The strict
laws surrounding private arcane practice and research do not help that matter.

\subsubsection{Relations}

Hraglund always had an uneasy relationship with neighbouring baronies and
realms, as they are known to aggressively extend their influence by any means
necessary. This often put them in direct conflict with both Helmarnock as well
as Norbury. However the kingdom holds good relations with both Forsby and
Tredegår.
