\cleardoubleevenemptypage

%% TODO: Artwork

\begin{infobox}{City Kingdom of Fes al-Bashir}
  %% TODO: Crest
  \begin{multicols}{2}
    \begin{itemize}[label={},noitemsep,leftmargin=0.0cm,topsep=0pt]
      \infoboxitem{Location}{North eastern shores of \nameref{sec:Arania}
      }
      \infoboxitem{Languages}{Kalest, Teranim}
      \infoboxitem{Government}{Oligarchy}
      \infoboxitem{Major Religions}{\nameref{sec:Forun}, atheism
      }
      \infoboxitem{Area}{est. 330,000 $km^2$}
      \infoboxitem{Population}{est. 33 million in total}
      \infoboxitem{Non Grata}{monstrous races (except slaves), devils, druids
      }
      \infoboxitem{Magic}{all magic must be permitted by the \nameref{sec:Hall
          of Knowledge}
      }
      \infoboxitem{Slavery}{yes, all forms, overseen by the
        \nameref{sec:Velvet Hand}, used as punishment for criminals, signer
        for the \nameref{sec:Vonir Accord}
      }
      \infoboxitem{Special Laws}{-}
      \infoboxitem{Notable Organisations}{\nameref{sec:House Ranian},
        headquarters of \nameref{sec:Velvet Hand}, headquarters of the
        \nameref{sec:Ror-Aram Trading Corporation}
      }
      \infoboxitem{POI}{\nameref{sec:Hall of Knowledge}, including the biggest
        library on all of Aror, grand bazaar, many oasis which have been
        converted into recreational parks, the sea of towers
      }
    \end{itemize}
  \end{multicols}
\end{infobox}

\clearpage

\subsection{Fes al-Bashir}
\label{sec:Fes al-Bashir}

The city kingdom of \emph{Fes al-Bashir} is the oldest city kingdom, and also
the oldest civilisation on Aror. It was founded in \emph{GT:0}. This is not a
coincidence. When the calendars were consolidated by the scholars of the city,
they realigned the years to use the foundation of the city as point zero for
the old calendar.

The banner of Fes al-Bashir is a red shield, featuring a golden eagle beneath
two golden half moons which point downward.

\subsubsection{History}

The city was first founded when many nomadic tribes began to permanently
settle down in the vast delta of the river \emph{Alis} on the eastern shores
of \nameref{sec:Arania}. Throughout its history it was always in conflict with
other nomadic tribes, and gnolls that sought to claim the vast and fertile
oasis for themselves. Although the city was sacked, besieged and even
conquered by gnolls several times throughout its history, the nomadic tribes
always managed to reconquer and retake the city. It had been continuously
inhabited by humanoid species for several thousand years, but had to endure
many sieges and attacks in its vast and long standing history.

The city has a vast network of old and ruined watchtowers, walls and military
camps just outside the main walls. Although these are in various states of
disrepair they are still used as way points, and watchtowers should enemies
lay siege to the city. This area has been nicknamed the \emph{sea of towers},
as the old watchtowers can still be seen dotting the landscape from the city
walls.

\subsubsection{Culture}

The people of \emph{Fes al-Bashir} have learned that the art of war, is as
important as the sciences or the arcane studies. They are known as
imperialistic expanders, and seek to prevent and extinguish threats before
they materialise against the city itself. Although the outlying small towns of
the river banks are sparsely manned by the army, each and every citizen is
encouraged to train in combat or even in the arcane for self defense.

Most of the city's population is atheistic, and the only religion that still
manages to keep a hold onto believers within Fes al-Bashir is the main church
of \nameref{sec:Forun} (known there as Nuit). Although proud, family oriented
and staunch believers in their heritage, they do not see the gods and deities
as the answer to their problems. The city kingdom has the oldest university of
all of Aror, and the belief of the population is that scientific and arcane
research can solve any problem, and does so better than adherence to belief,
or old superstition.

Due to harrowing temperatures in the summer months (around 40 to 50 degrees)
the people of Fes al-Bashir tend to wear long, light clothes with
turbans. Their dark skin - which they share with most of Arania's inhabitants
- also aids them against the unbearable heat of the desert they inhabit.

Within the society the traditional roles of men are women still run strong, as
women are encouraged to seek safer employment, instead of becoming soldiers or
labourers. Although women are not bared from seeking employment in these
fields by the law, the cultural pressure for them not to do so remains high.

People hailing from Fes al-Bashir are often described as arrogant or snobby,
as they see their oldest city as the birth place of modern civilisation, and
often find the other city kingdoms often lacking in culture, or outright
primitive. A trait that is often down played as a jest in good spirit when
brought up by foreigners.

The cuisine of Fes al-Bashir relies heavily on the spices they grow around the
oasis, and is thus fiery hot and spicy. The people of the nation also heavily
drink tea and coffee, and both drinks have become a national and traditional
staple in every day life.

\subsubsection{Architecture}

The city itself has one central area called the grand bazaar, that contains the
main campus and the Hall of Wisdom. The grand bazaar also housed the palace of
the monarchy, which has since been converted into a garrison for the armed
forces of Fes al-Bashir. The former royal palace also serves a meeting hall
for the city's council. While the central plaza has many larger buildings, and
estates, the surrounding city's architecture is vastly different. Families live
small apartments, often just a single room, where each house providing several
apartments on several floors. Each floor is accessible separately from the
outside through wooden staircases, and many houses also interconnect with each
other with wooden bridges. This levelled and modest architecture allows the
city to house a large population of several millions in a small space in the
desert. Most of these houses are built out of sandstone and painted white, while
the estates, palaces, and the House of Wisdom are built out of marble.

The city itself spreads itself across several smaller oases, and thus the city
is interspersed with many green parks that surround small ponds of fresh
water. These small parks are mostly used for fresh water supply, and for
recreation, albeit polluting these ponds is punished harshly. The city itself
has many smaller bazaars spread across the vast network of streets and smaller
settlements, that sell spices, food, luxury goods and slaves.

\subsubsection{Population}

The city itself and its surrounding areas is one of the largest civilisations
on Aror, housing over 33 million people. Most of these are equally spread out
among the four major humanoid races, with humans leading by a small margin
(26\%), followed by elves (25\%) and halflings (22\%) and dwarves (20\%) with
various half races and undead make up the rest (7\%).

Common male names are: Ali, Adam, Amir, Bakar, Fahim, Farouk, Hamid, Gamal,
Hasan, Jaffar, Karim, Musa, Nadeem, Nassim, Raheem, Tarek, Roshua

Common female names are: Amina, Adila, Aida, Amira, Dana, Hagir, Hala, Hannah,
Inas, Jamila, Leyla, Lina, Nada, Raisa, Sarah, Thamina, Yasmin

\subsubsection{Rule and Laws}

Although called a city-kingdom, it is no longer a monarchy. The previous ruling
monarchy - called Malek (king) or Malekha (queen) - has been abolished in
favour of a ruling council comprised of important and powerful artisans,
scientists, mages and scholars from the Hall of Wisdom. By tradition, the high
magus of the Hall of Wisdom oversees the council and acts as its mouthpiece,
and as an external representative.

The city kingdom still practises slavery, which are drawn from the criminals,
the poor, captured enemy warriors and those that the kingdom has bought from
other slavering nations. The city is a signer of the \nameref{sec:Vonir
  Accord}, and is also known for being one of the major slave trading hubs in
the region. The city operates its own slaver's guild called the
\nameref{sec:Velvet Hand}.

Intelligent undead are not banned from the city, and may become citizens of
the city. The city, and its surrounding areas thus has a sizeable community of
\nameref{sec:Umgeher} and \nameref{sec:Vampires} within its borders.

\subsubsection{Hall of Knowledge}
\label{sec:Hall of Knowledge}

The \emph{Hall of Knowledge} is the state run university of Fes al-Bashir. It
is the oldest, biggest and most respected university of the sciences and
arcane arts. Many foreign scientists and wizards come here to study and
learn, and the university has a track record for educating some of the most
prestigious wizards and scientists. Many of the biggest scientific
advancements and discoveries have been made within the Hall of Knowledge: the
true nature of the solar system, advancements in physics, engineering,
mathematics and medicine. Its most accomplished student and leader was none
other than \nameref{sec:Graham Balance}.

As an institution the Hall of Knowledge holds vast political power within
Fes Al-Bashir, and many of its leaders are also powerful, charismatic and highly
intelligent politicians and leaders within the city itself. While the Hall of
Knowledge holds no direct power, it has very close ties to the ruling elite of
the city, and provides them with a means and a place to network amongst each
other.

\subsubsection{Relations}

The city kingdom is known for its scientific prowess and vast riches in
agricultural produce, such as herbs, spices, coffee and chocolate. It is more
than willing to trade these goods to others through the \nameref{sec:Ror-Aram
  Trading Corporation}. This, along with a propensity to lend military and
scientific aid to its allies, has lead to a good diplomatic standing with
almost all other city kingdoms. However the fall of \nameref{sec:Esmayar} and
the kingdom's subsequent defeat there has dampened the power and influence of
the nation on the continent of Aror.
