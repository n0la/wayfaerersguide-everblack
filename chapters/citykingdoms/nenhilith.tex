\subsection{Nen-Hilith}
\label{sec:Nen-Hilith}

\emph{Nen-Hilith} (``city of two'' in Taavid and Enro'ad), was the joint
kingdom of elves and halflings. It was situated on the northern tip of
\nameref{sec:Farlar}, and was destroyed in \emph{MI:1920} after being besieged
by \nameref{sec:Giants}, and various monstrous races.

\subsubsection{Banner}

The banner of Nen-Hilith depicted two large crowned towers, joined together
by a bridge, with a black drop of blue (for the sea), and green (for the
fertile farmland of Farlar).

\subsubsection{History}

The city kingdom was founded by those elves, and halflings that sought to
escape the battles of the \nameref{sec:Strife} by settling in a remote area.
They found a land of peace, on Farlar, a continent just north of the dragon
claimed land of \nameref{sec:Draigynus}. Both elves and halflings settled
the fertile land north of the Lias'wa mountains, and made their riches by
exporting their produce through a huge trading port. Soon this trading port
grew, and grew in size and population, before finally being recognised as
one of the first city kingdoms in \emph{GT:551}.

Although it became rich from agricultural products, it shifted its focus
towards ship building, and the arts. At its prime, Nen-Hilith was known as the
city that produced most of the world-renowned artists, actors, playwrights,
and gold smiths. The artistic focus of the city was also seen in the
architecture, as many buildings were lavishly decorated, embroidered or
already designed to be grand palaces, and estates.

\subsubsection{Fall}

In \emph{MI:1910} the \nameref{sec:Giants} landed on Farlar to fight their
ancient enemies, the \nameref{sec:Dragons} of Draigynus. At first the city
kingdom did not interfere in the war, but the giants built huge damns in
the Lias'wa mountains to support their own armies, severely limiting the
city's water supplies. In \emph{MI:1916} the city formally joined the war
between the giants, and dragons in a desperate attempt to reclaim the city's
water supplies. The city had very few warrior nations as allies, and
relied heavily on its naval fleet which was of no use in a land-based
conflict. It suffered defeat after defeat against the giants. The war reached
its catastrophic height when the giants laid siege to the city in
\emph{MI:1918}, and finally broke through the walls in \emph{MI:1920}. The
city dwellers fled on their ships northward, towards the shores of southern
\nameref{sec:Goltir}, while the giants razed the city of Nen-Hilith to the
ground.

\subsubsection{Culture}

While the main focus of the city was focused around the arts, the city always
saw itself as a neutral, diplomatic force in the world of Aror. The city, and
its citizens abhorred war, only seeing it as the last course of action if all
attempts at diplomacy had failed. Nen-Hilith was often chosen as a neutral
ground to settle diplomatic disputes between rivalling factions, and it was
within Nen-Hilith that the \hyperref[sec:Two Courts]{Court of the Moons} was
founded.

Even though its main focus was on the arts, and diplomacy, the city kingdom
had one of the grandest fleet in all of Aror. Its mere presence was enough to
deter conflict, and the waters of Farlar, and southern Goltir were mostly free
of piracy, smuggling, and slaver ships.

The city itself outlawed slavery early in its history, and was noted as a save
haven that freed, and protected any slave seeking shelter and refuge. This of
course brought it in direct conflict with many slaving nations, who were of
course kept at bay by the city's massive fleet of warships.

The average citizen of Nen-Hilith, was agnostic, creative, interested in arts,
history, the sciences, politics and generally tried to keep good relations with
all sorts of cultures, and backgrounds. Nen-Hilith was a diverse, yet welcoming
place for everyone, no matter their fate or background.

\subsubsection{Population}

Before the fall, the city had roughly 11 million inhabitants, most being
various sorts of elves (33\%), halflings (32\%), humans (13\%), dwarves
(10\%), half races (8\%), and others (4\%).

Common male names were: Aelius, Caius, Felix, Florian, Horatio, Julius, Marcus,
Marius, Otho, Ovid, Senec, Tacitus, Varo, Vitus.

Common female names were: Aelia, Aemilia, Aurelia, Caelia, Cassia, Claudia,
Flavia, Flora, Hadriana, Julia, Lucia, Marina, Paula, Sabina, Titiana,
Valentina, Vita.

The city spoke mostly Taavid (halfling language), and Enro'ad (elvish language),
and both were the official languages of the kingdom.

\subsubsection{Rule}

The city was a totalitarian monarchy, were two monarchs - one elven, one
halfling - ruled the city together. The city was known for its fair, and just
courts, employing the \nameref{sec:Five Holy Orders} to keep order within the
city.

\subsubsection{Relations}

The city held good relations with \nameref{sec:Forsby}, as well as with the
dragons that ruled Draigynus further south. It held poor relations with most
of the slaving nations, including \nameref{sec:Fes al-Bashir} and
\nameref{sec:Norbury}.
