\subsection{El-Fayam}
\label{sec:El-Fayam}

The city kingdom of \emph{El-Fayam} lies on the south western shores of
\nameref{sec:Arania}. It was founded in \emph{MI:1260} from the refugees that
had to flee the sacking of \emph{Esmayar}.

\subsubsection{Banner}

The city's banner features two white sabres crossed at the blades, upon
a light red background. The banner was taken from the old banner of Esmayar.

\subsubsection{History}

The city is one of the youngest on Aror, and was officially recognised as a
city kingdom in MI:1310. It existed as a small fishing village in the oasis of
Nakhmet, but exploded in size after the sacking of Esmayar by the gnolls. Many
of refugees from Esmayar fled to the village and settled there. Soon the small
fishing village became a small town, and ultimately a small kingdom. In
MI:1310 the small kingdom was officially recognised as a city kingdom and
successor in spirit to Esmayar. An act that enraged the gnolls of the northern
neighbours. In MI:1630 the gnolls of Esmayar attempted to besiege and conquer
the city but where ultimately repelled by an alliance of humanoid city
kingdoms.

Many of the new arrivals and refugees were highly skilled labourers and city
builders that helped raise the small village to the status of a prosperous
city and kingdom within a few generations. Over the course of many years the
city expanded into the outlying lands, and thus now owns the vast majority of
the Nakhmet oasis.

\subsubsection{Population}

The city kingdom is one of the smallest, being home to roughly 900000
people all in all. The diversity among the humanoid species is high, with
elves leading slightly (30\%), followed by humans (28\%) and dwarves (22\%)
and then halflings (18\%). Half races and various others make up the remaining
2\%.

Common male names are: Abar, Ahmes, Amosis, Cleo, Hori, Iamesu, Menes,
Merenor, Nedjem, Seneb, Seth, Takar, Turo, Yna, Zamon

Common female names are: Ahmose, Cleo, Dia, Herneith, Kasha, Lysandra, Maia
(or Maya), Merita, Nebet, Neth, Pevena, Satiah, Sema, Tabia

\subsubsection{Everblack}

The city kingdom found a large subterranean vein of Everblack in MI:1490
beneath the city while digging and expanding the old village's water drainage
system. An excavation was immediately started, and has now turned into one of
the largest mining operations on Aror. With the new found wealth of selling
the everblack across the world, the city hired skilled labourers, mining
crews, smelters and overseers to continue the massive mining operation.

Although tempted, the city kingdom did not use slave labour to mine beneath
the city, but paid the workers fair wages. But over the course of many
centuries old mining plans were lost, mining shafts collapsed, and the shafts
became home to various subterranean creatures, making the mines beneath the
city's aqueduct a deadly labyrinth. The workers were unwilling to return
there out of fear of being lost or killed. Unable to find any workers willing
to mine the depths beneath the city, the kingdom reinstated slavery in MI:1720
and is now using forced labour to mine the veins in the deep caverns below.

The city kingdom is the main source of everblack on the southern continent,
which made the kingdom unfathomably rich. It has used this wealth to expand
its land, power and influence in the region; as well as continuing to fund
explorations, charting missions and mining operations in the depths beneath
the city in the hopes of finding new veins of Everblack.

\subsubsection{Culture}

At first the people of El-Fayam were determined to retake their old city as
soon as possible, and thus heavily focused on military and arcane studies in
the early decades of the kingdom's foundation. This view has since shifted after
the discovery of everblack, towards trading, bartering and mining. After
generations having a strong focus on military achievements and pride within
the culture, the values shifted towards wealth and mercantile prowess as well
as arcane study of the black crystal.

Since the ``black gold'' (as it is called in the city) has taken over the main
focus of the kingdom, any measure that aids finding, mining and refining the
black gold has become socially desirable within the city. This culture has
earned the people of El-Fayam a reputation of being ruthless dealers and
businessmen, that would not shy away from introducing slavery to become rich.
The truth however, is that their massive amount of wealth has trickled down
to the people, establishing a large, wealthy and socially stable middle class.

Unlike in other city kingdoms the ruling Malek has very little actual power,
and is seen only as a mere puppet of the mining and trading guilds that
rules the city.

\aren{You will find the people of El-Fayam to be friendly, loving, exceptional
  hosts, always ready and willing to barter with you. Just don't look under
  the metaphorical rug that are their Everblack mines.}

\subsubsection{Relations}

Although still officially an enemy of the new gnoll kingdom of Esmayar, the
kingdom concerns itself little with recapturing its former home. Still it
fights small skirmishes against its northern neighbour, bust mostly to
defend its borders from gnollish incursions.

It holds good relations with Fes al-Bashir, going so far as to invite the Ror
Aram-Trading corporation to help with the selling of the everblack crystals,
as well as inviting the Velvet Hand to oversee slavery within the city. It is
a signer of the \nameref{sec:Vonir Accord} and often trades slaves with both
Fes al-Bashir, Norbury and Helmarnock.

The city kingdom also uses the acquired wealth to simply buy itself into
a good diplomatic standing with anyone that the kingdom deems valuable enough
to have as a friend.
