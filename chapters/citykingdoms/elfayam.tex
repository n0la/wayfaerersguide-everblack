\cleardoubleevenemptypage

%% TODO: Artwork

\begin{infobox}{City Kingdom of El-Fayam}
  %% TODO: Crest
  \begin{multicols}{2}
    \begin{itemize}[label={},noitemsep,leftmargin=0.0cm,topsep=0pt]
      \infoboxitem{Location}{South western shores of \nameref{sec:Arania}
      }
      \infoboxitem{Languages}{Kalest, Teranim, Latas (regional dialect)}
      \infoboxitem{Government}{Absolute Monarchy}
      \infoboxitem{Major Religions}{\nameref{sec:Nyddwr}, \nameref{sec:Forun},
        \nameref{sec:Order}, \nameref{sec:Marwaid}
      }
      \infoboxitem{Area}{est. 50,000 $km^2$}
      \infoboxitem{Population}{est. 900 thousand in total}
      \infoboxitem{Non Grata}{monstrous races (except slaves), devils, druids,
        Gorgons
      }
      \infoboxitem{Magic}{all magic allowed, almost impossible to channel
        magical energy within the city
      }
      \infoboxitem{Slavery}{yes, all forms, overseen by the
        \nameref{sec:Velvet Hand}, used as punishment for criminals, signer
        for the \nameref{sec:Vonir Accord}
      }
      \infoboxitem{Special Laws}{-}
      \infoboxitem{Notable Organisations}{\nameref{sec:House Ranian},
        large auction house of the \nameref{sec:Velvet Hand},
        \nameref{sec:Ror-Aram Trading Corporation}
      }
      \infoboxitem{POI}{Church of \nameref{sec:Lor}, cathedral of the
        \nameref{sec:Order}, large recreational sand beach within the
        city walls, central promenade with many restaurants and shops
      }
    \end{itemize}
  \end{multicols}
\end{infobox}

\clearpage

\subsection{El-Fayam}
\label{sec:El-Fayam}

The city kingdom of \emph{El-Fayam} (``young flower'' in Kalest, also known
as ``Lorium'' in local dialect) lies on the south western shores of
\nameref{sec:Arania}. It was founded in \emph{MI:1260} from the refugees that
had to flee the sacking of \nameref{sec:Esmayar}.

The city's banner features two white sabres crossed at the blades, upon a
light yellow background, embedded in a shield upon which rests a crown. The
banner was taken from the old banner of Esmayar.

\subsubsection{History}

The city is one of the youngest on Aror, and was officially recognised as a
city kingdom in MI:1310. It existed as a small fishing village in the oasis of
Nakhmet, but exploded in size after the sacking of Esmayar by the gnolls. Many
of refugees from Esmayar fled to the village and settled there. Soon the small
fishing village became a small town, and ultimately a small kingdom. In
MI:1310 the small kingdom was officially recognised as a city kingdom and
successor in spirit to Esmayar. An act that enraged the gnolls of the northern
neighbours. In MI:1630 the gnolls of Esmayar attempted to besiege and conquer
the city but where ultimately repelled by an alliance of humanoid city
kingdoms.

Many of the new arrivals and refugees were highly skilled labourers and city
builders that helped raise the small village to the status of a prosperous
city and kingdom within a few generations. Over the course of many years the
city expanded into the outlying lands, and thus now owns the vast majority of
the Nakhmet oasis.

\subsubsection{Culture}

The average citizen of El-Fayam enjoys a secure, wealthy live, with many of
the jobs available in the city paying enough to afford a decent living. The
city itself is known for offering many opportunities, both of gifted and skilled
craftsmen, as well as for adventurers. The justice system of the city is fair,
and the guards men pride themselves from keeping order both inside, and outside
of the city. The politicians have learned from the tragedy of Esmayar, and did
not expand the city's sphere of influence beyond what they were able to control,
and secure. Crime in the city is low, although the harbour districts is a hotbed
for smuggling.

The citizens of El-Fayam pray to a wide variety of gods, including
\nameref{sec:Nyddwr}, \nameref{sec:Forun}, \nameref{sec:Marwaid} and the
\nameref{sec:Order}. Although the cultural influence from \nameref{sec:Fes
  al-Bashir} is ever present, the people of El-Fayam are deeply
religious. However tensions from the different believes do arise, they are
very rarely violent.

Many craftsmen from Esmayar took their craft with them, and thus the city is a
mixture of both yellow sandstone, and white brick buildings. The city itself
is known for its delicious food, beautiful scenery, and availability for
luxury goods such as tobacco and coffee.

While the city practices, and allows slavery, the amount of slaves within the
city is low. Most slaves are criminals that toll away their sentences within
the Everblack mines. Most servants to the rich and nobility are serfs, and are
thus not considered property.

The citizens speak a wide variety of languages, including Kalest, Teranim,
Latas (their regional dialect), Enro'ad, and even Rutari.

\subsubsection{Population}

The city kingdom is one of the smallest, being home to roughly 900.000
people all in all. The diversity among the humanoid species is high, with
elves leading slightly (30\%), followed by humans (28\%) and dwarves (22\%)
and then halflings (18\%). Half races and various others make up the remaining
2\%.

Common male names are: Abar, Ahmes, Amosis, Cleo, Hori, Iamesu, Menes,
Merenor, Nedjem, Seneb, Seth, Takar, Turo, Yna, Zamon

Common female names are: Ahmose, Cleo, Dia, Herneith, Kasha, Lysandra, Maia
(or Maya), Merita, Nebet, Neth, Pevena, Satiah, Sema, Tabia

\subsubsection{Everblack}

The city kingdom found a large subterranean vein of Everblack in MI:1490
beneath the city while digging and expanding the old village's water drainage
system. An excavation was immediately started, and has now turned into one of
the largest mining operations on Aror. With the new found wealth of selling
the everblack across the world, the city hired skilled labourers, mining
crews, smelters and overseers to continue the massive mining operation.

Although tempted, the city kingdom did not use slave labour to mine beneath
the city, but paid the workers fair wages. But over the course of many
centuries old mining plans were lost, mining shafts collapsed, and the shafts
became home to various subterranean creatures, making the mines beneath the
city's aqueduct a deadly labyrinth. The workers were unwilling to return
there out of fear of being lost or killed. Unable to find any workers willing
to mine the depths beneath the city, the kingdom reinstated slavery in MI:1720
and is now using forced labour to mine the veins in the deep caverns below.

The city kingdom is the main source of everblack on the southern continent,
which made the kingdom unfathomably rich. It has used this wealth to expand
its land, power and influence in the region; as well as continuing to fund
explorations, charting missions and mining operations in the depths beneath
the city in the hopes of finding new veins of Everblack.

Since the city kingdom sits directly on top of so much Everblack, any form
of magic is disrupted within the city. This works both for the city, as
foreign casters cannot do a lot of damage, but also severely restricts any
form of magic research within the city. The city also has a special prison
for magic wielders, that is in high demand around the world.

\begin{note}
  Any form of magic has a 50\% spell failure chance in El-Fayam, and its
  surrounding lands. The spell failure is 75\% within the Everblack mines,
  as trace dust of Everblack absorbs all sorts of magic.
\end{note}

\subsubsection{Culture}

At first the people of El-Fayam were determined to retake their old city as
soon as possible, and thus heavily focused on military and arcane studies in
the early decades of the kingdom's foundation. This view has since shifted after
the discovery of everblack, towards trading, bartering and mining. After
generations having a strong focus on military achievements and pride within
the culture, the values shifted towards wealth and mercantile prowess as well
as arcane study of the black crystal.

Since the ``black gold'' (as it is called in the city) has taken over the main
focus of the kingdom, any measure that aids finding, mining and refining the
black gold has become socially desirable within the city. This culture has
earned the people of El-Fayam a reputation of being ruthless dealers and
businessmen, that would not shy away from introducing slavery to become rich.
The truth however, is that their massive amount of wealth has trickled down
to the people, establishing a large, wealthy and socially stable middle class.

Unlike in other city kingdoms the ruling Malek has very little actual power,
and is seen only as a mere puppet of the mining and trading guilds that
rules the city.

\aren{You will find the people of El-Fayam to be friendly, loving, exceptional
  hosts, always ready and willing to barter with you. Just don't look under
  the metaphorical rug that are their Everblack mines.}

\subsubsection{Relations}

Although still officially an enemy of the new gnoll kingdom of Esmayar, the
kingdom concerns itself little with recapturing its former home. Still it
fights small skirmishes against its northern neighbour, bust mostly to
defend its borders from gnollish incursions.

It holds good relations with Fes al-Bashir, going so far as to invite the Ror
Aram-Trading corporation to help with the selling of the everblack crystals,
as well as inviting the Velvet Hand to oversee slavery within the city. It is
a signer of the \nameref{sec:Vonir Accord} and often trades slaves with both
Fes al-Bashir, Norbury and Helmarnock.

The city kingdom also uses the acquired wealth to simply buy itself into
a good diplomatic standing with anyone that the kingdom deems valuable enough
to have as a friend.
