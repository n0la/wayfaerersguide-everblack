\cleardoubleevenemptypage

\ifimages
\begin{figure*}[ht!]
  \centering
  \vspace{-5.8cm}
  \centerline{
    \includegraphics[width=\paperwidth,keepaspectratio]{%
      media/esmayar.\imagesuffix
    }
  }
  \captionsetup{labelformat=empty}
  \caption{``Visiting Esmayar as a core race should be avoided at all costs.'' - Aren Fel}
\end{figure*}
\fi

\begin{infobox}{City Kingdom of Esmayar}
  \ifimages
  \begin{subfigure}[t]{\textwidth}
    \centering
    \includegraphics[width=0.22\linewidth]{%
      media/esmayar-banner.\imagesuffix
    }
  \end{subfigure}%
  \fi%
  \begin{multicols}{2}
    \begin{itemize}[label={},noitemsep,leftmargin=0.0cm,topsep=0pt]
      \infoboxitem{Location}{North western shores of \nameref{sec:Arania}
      }
      \infoboxitem{Languages}{Gnoll, Giant}
      \infoboxitem{Government}{Absolute Monarchy, Tyranny}
      \infoboxitem{Major Religions}{\nameref{sec:Three Kings},
        \nameref{sec:Isamir}, \nameref{sec:Forun} and \nameref{sec:Marwaid}
      }
      \infoboxitem{Area}{est. 120,000 $km^2$}
      \infoboxitem{Population}{unknown}
      \infoboxitem{Non Grata}{any non-monstrous races without permit
      }
      \infoboxitem{Magic}{all magic banned for non-citizens
      }
      \infoboxitem{Slavery}{yes, all forms}
      \infoboxitem{Special Laws}{unknown}
      \infoboxitem{Notable Organisations}{unknown}
      \infoboxitem{POI}{slave auctions, gladiatorial arena, shrine to the
        \nameref{sec:Three Kings}
      }
    \end{itemize}
  \end{multicols}
\end{infobox}

\clearpage

\subsection{Esmayar}
\label{sec:Esmayar}

Esmayar (``white jewel'' in Kalest, also known as ``Arcania'' in local
dialect) was one of the older city kingdoms of Aror, with a rife history but
has since fallen to the gnoll raiders in \emph{MI:1213}. It is located on the
north western shores of \nameref{sec:Arania}.

The city's old humanoid banner featured a yellow double headed eagle, upon a
light red background, embedded in a shield upon which rests a crown. Yellow
and red were the city nation's primary colours.

The new banner shows a dark red head of a gnoll upon white background, beneath
a golden crown. Red, gold and white are the primary colours of the monstrous
kingdom of Esmayar.

\subsubsection{History}

It was founded in \emph{GT:592} in the delta, and along the banks of the
\emph{Balran} river on the north western shores of \nameref{sec:Arania}. Much
like Fes al-Bashir, with which the city shared a long historical and cultural
friendship, it got rich by selling agricultural products, such as exotic
fruits, tobacco, chocolate and coffee to the other city kingdoms and
baronies. This wealth attracted raiders, bandits and the gnoll tribes of the
desert who besieged the city constantly over the course of thousands of
years. Although it was always capable of defending against these attacks, the
constant ransacking and pillaging of the outlying farms and plantations
drained the kingdom's resources to the point of bankruptcy. With the military
aid of Fes al-Bashir, and northern city kingdoms it held on to the power within
the city but had long lost the outlying villages. They fragmented into smaller
nomadic tribes and proofed difficult to reintegrate into the waning
empire. Unable to reunite the land, the kingdom fell apart leaving only the
city of around four million inhabitants under the control of the ruling
monarch.

In \emph{MI:1213} the final and last siege of the kingdom began as an army of
gnolls marched upon the walled city. The siege, with the aid of Fes al-Bashir
and the ``devils of the north'' (soldiers and mercenaries from
\nameref{sec:Norbury}) were able to hold off the besieging army for several
months before the city fell in the late months of that year. Many citizens
were evacuated by sea, moving further down south to ultimately found
\nameref{sec:El-Fayam}.

\subsubsection{Culture before the Fall}

The ``white jewel'', as it was known, got its name from the beautiful, and
elaborate buildings its architects build within the city. Most of its
buildings were painted white - hence the name - giving the city an almost
angelic appearance. Its main attraction was the Colosseum, a huge arena
that was used for both musical, theatrical performances, as well as for
sport events. The Colosseum is now used mainly for fights between enslaved
gladiators, price fighters, and monsters by the gnolls.

The kingdom itself was always deemed imperialistic, and expanded its borders
through war, conflict and subjugation. Its army was well trained, equipped,
and matched in ferocity only by the \hyperref[sec:Norbury]{wolves of the
  north}. Its imperialistic nature was also the root causes of its down fall,
as it often conquered more land than it could hold, and defend.

The average Esmayan was proud, family oriented, and followed the
\nameref{sec:Order} for guidance. They had a strong legal system, and were one
of the few city kingdoms that only used slavery for indentured servitude of
criminals. Although they did not sign the \nameref{sec:Vonir Accord}, their
tradition, and culture which focused on warriors, pride, and honour made them
staunch allies of \nameref{sec:Norbury}.

\subsubsection{Gnoll Kingdom}

The gnolls ransacked the city and the surrounding lands, killing all of the
remaining defenders through ritualistic immolation. Many civilians fled
towards Fes al-Bashir by ship with the aid of the retreating army of the
Norbury and Fes al-Bashir. The gnolls further enslaved all civilians that were
unable to flee. Although many kingdoms ignored the self proclaimed gnoll
kingdom, dismissing it as a short-lived and temporary kingdom that would fall
apart on its own due to the gnoll's limited knowledge on how to effectively
run such a vast and big empire.

The retreating humanoid races destroyed the dragon teleporter within the city,
and all other humanoid city kingdoms avoided contact with the new gnoll
empire, isolating it both diplomatically and economically. Most organisations
of power believed that this was sufficient to starve out the gnolls, and make
their kingdom fall within two decades. But to much surprise, the gnollish
kingdom has now lasted for over eight hundred years. They withstood famine,
plague and many attempts of the humanoid races to retake the city. At first
they took and plundered everything their kingdom required, but are now, after
using their slaves to rebuild and tend to farms and plantations, self
sufficient.

At first their laws against humanoids was harsh and unforgiving, enslaving any
and every humanoid they found. But over the course of centuries their society
opened up, and even allowed humanoid species to visit and trade within the
city. The gnolls have shifted the focus of the empire's economy towards
precious stones, silver, and gold, while engaging heavily in the slave trade
of humanoid species. They are not a signer of the \nameref{sec:Vonir Accord},
and thus enslave anyone that misbehaves or commits crimes (real or merely
accused) within their city.

The gnoll pack leader claims the king's throne and is the supreme ruler of the
city, although their state as a city kingdom is not recognised by the other
major humanoid city kingdom. Although they are now one the weakest and
smallest of the city kingdoms, no one dares to besiege or reclaim it, due to
the fear of either a military defeat, or heavy losses such a campaign would
bring. It is widely known that the gnolls have made pacts and alliances with
other monstrous baronies, especially those of \nameref{sec:Eilean Mor} and
\nameref{sec:Iafandir} to bolster their power and influence.

\subsubsection{Relations}

\aren{It is fine if you enslave your own races, but mothers' forbid if
  the monstrous races do it.
}

None of the city kingdoms accept or recognise the gnoll's reign over Esmayar,
and thus the city kingdom has no official allies. In truth however the various
agencies of the other slaver nations (such as the \nameref{sec:Velvet Hand}
and the \nameref{sec:Hunters Guild}) have been known to trade with the gnoll
empire. A tactic that is both lucrative and highly controversial.

The city kingdom of \nameref{sec:Morkan} has often used the kingdom as an
example of what happens when the humanoids show lenience and weakness against
monstrous invaders. Taras has sworn on many occasions to turn his gaze towards
the gnolls of the city once the continent of \nameref{sec:Iafandir} has been
cleansed of the monstrous races.

Esmayar is allied with many smaller monstrous kingdoms, such as hobgoblin
kingdoms and larger ogre tribes or bugbear baronies. It openly seeks to
establish new relations with monstrous villages and tribes, especially with
the other nomadic gnoll tribes that still wander the continent of
\nameref{sec:Arania}.
