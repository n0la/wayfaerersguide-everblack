\cleardoubleevenemptypage

\ifimages
\begin{figure*}[ht!]
  \centering
  \vspace{-3.6cm}
  \centerline{
    \includegraphics[width=\paperwidth,keepaspectratio]{%
      media/norburysm.\imagesuffix
    }
  }
  \par
  Market place in \emph{Norbury}, circa GT:2102
\end{figure*}
\fi

\begin{infobox}{City Kingdom of Norbury}
  \ifimages
  \begin{subfigure}[t]{\textwidth}
    \centering
    \includegraphics[width=0.22\linewidth]{%
      media/norbury-bannersm.\imagesuffix
    }
  \end{subfigure}
  \fi
  \begin{multicols}{2}
    \begin{itemize}[label={},noitemsep,leftmargin=0.0cm,topsep=0pt]
      \infoboxitem{Location}{Island of Völand off the northern coast of
        \nameref{sec:Eilean Mor}
      }
      \infoboxitem{Languages}{Teranim, Danvark (regional dialect), Old Teranim}
      \infoboxitem{Government}{Meritocracy with an absolute ruler}
      \infoboxitem{Major Religions}{\nameref{sec:Old Ways}, \nameref{sec:Forun},
        \nameref{sec:Order}, \nameref{sec:Three Kings}
      }
      \infoboxitem{Area}{est. 200,000 $km^2$}
      \infoboxitem{Population}{est. 4 million in total}
      \infoboxitem{Non Grata}{monstrous races (except slaves), devils, druids,
        priests and knights of \nameref{sec:Lor}
      }
      \infoboxitem{Magic}{all magic allowed}
      \infoboxitem{Slavery}{yes, all forms, overseen by the
        \nameref{sec:Hunters Guild}, used as punishment for criminals
      }
      \infoboxitem{Special Laws}{worship of \nameref{sec:Lor} is banned,
        \nameref{sec:Gorgons} must suppress their powers while in public
      }
      \infoboxitem{Notable Organisations}{\nameref{sec:House Ranian},
        headquarter of the \nameref{sec:Hunters Guild}
      }
      \infoboxitem{POI}{Castle Norr, central market with law steele, auction
        house, church of Forun, Warmage's academy with dragon teleporter,
        library of the \nameref{sec:Second Order}
      }
    \end{itemize}
  \end{multicols}
\end{infobox}

\clearpage

\subsection{Norbury}
\label{sec:Norbury}

\graham{Surely the most vile city kingdom Aror has to offer...}
\aren{Bless thy innocent heart, for you have not lived long enough to see the
  rise of Morkan.}

The second youngest city kingdom, \emph{Norbury} resides on the large island
of \emph{Völand} off the northern coast of \nameref{sec:Eilean Mor}. Norbury
is a large walled city, situated in a fjord in the north-western part of the
island.

\subsubsection{History}

It was founded around \emph{GT:1849} as a joint military outpost of
\nameref{sec:Hraglund}, and other northern baronies of Eilean Mor. It was
originally intended as first line of defence against the many raids of the
beast races that came from the northern most continent of
\nameref{sec:Iafandir}. Quickly the fortress of Norbury grew into a castle
(named castle \emph{Norr}), and more and more people were required to keep the
castle and its army supplied. Armies need smiths, smiths need smelters,
smelters need coal huts and miners, and all of these need food, lodging and
entertainment. Within a few generations Norbury exploded in size and
population, all working towards one goal: keeping the raids and incursion of
the beast races away from the main continent.

By \emph{GT:2041} the city had surpassed baronies in sheer size and population,
and was granted the official status of a \emph{city kingdom}.

\subsubsection{Banner}

The kingdom's banner shows two silver swords crossed at the blade within a
light red shield, topped with a crown. The main colours of the kingdom are
silver and red.

\subsubsection{Districts}

The centre of the city is a large market place, with a big stone steeple in
the middle. The tower both signifies the eternal vigilance, but also houses a
large bell that is rung in case of an attack. Inscribed into the walls of
the steeple are the laws of the kingdom for everybody to read. The market
place is vast circular open courtyard, and all sorts of goods - including
slaves - are sold there.

To the north lies \emph{Norbury} castle, a huge fortified military camp and
seat of the monarch. It oversees the north eastern sea off the island, and
rests upon a roughly one hundred metre high cliff side. The castle district
also houses the richer barons, lords and ladies of the nobility of Norbury.

Further north west and south east lie the two major harbour districts that
the kingdom uses to maintains its vast fleet. Although these harbours are also
major trading hubs, they also house a majority of the city's working class or
poorer slaves.

To the south west lies the ``outreach'', a sprawling suburban district in which
the bulk of the Norbury citizens life. Both rich, poor, worker, artisan and
slave share this vast cityscape and life there together. Even though it is
called the ``outreach'' or ``outskirts'' this suburban area is still within
the city walls.

The surrounding area of the walled city houses farms and smaller villages that
also belong to the city kingdom. These farms are not walled, and thus
exposed to surprise sea raids from smaller vessels. The city provides these
outlying farms with watchtowers, patrols and sometimes even with full army
support are thus rarely defenceless.

In \emph{MI:1910} Norbury expanded its territory onto the mainland of Eilean
Mor. After the coastal barony surrounding castle Rothorn fell into disarray
when their old baroness died, the city kingdom invaded and claimed the land.
Norbury now holds and owns a small patch of costal area at the northern foot
of the \nameref{sec:Great Divide}.

\subsubsection{Sea Raids}

A network of watch towers, light houses, guard towers and scout ships
constantly check the sea north of \nameref{sec:Eilean Mor} for any impending
sea raiders that embark from the continent of \nameref{sec:Iafandir}. If a
suspicious ship or raiding party is discovered, the entire kingdom goes on
high alert and deploys their fleet. Norbury are excellent ship builders,
sailors and warriors on the sea; and so they prefer to capture and destroy
raiding parties before they land on Eilean Mor. Most of the ships manned by
the beast races of \nameref{sec:Iafandir} are often inferior in design, and
thus unfit for prolonged sea battles. Many are sunk off the coast of the
continent. The raiders that survive are fished out of the sea and then brought
back to Norbury.

There is a common misconception that Norbury enslaves all raiders that land
at the shores of Eilean Mor. But the city only enslaves those that actively
attack them, or raid coastal villages and towns as a way to make the raiders
repay the damage they have caused. Those that sail from \nameref{sec:Iafandir}
unarmed, or surrender, are more often than not escorted back to the savage
lands.

Sometimes raiding parties do sneak past the ever vigilant eye of the city,
and then land on the northern shores of Eilean Mor. Most of the baronies
on the shores have increased their armies to repel these raids upon their
lands. Yet some also call the mainstay army of Norbury for aid. Then the
soldiers of the barony attempt to bar the raiders from entering their land,
while the ships and navy of Norbury attack the landing party from the sea.

Defending Eilean Mor (the ``home land'') from sea raids is a cultural goal and
ideal, that is extended to any barony on the main continent. Some warriors of
Norbury even see it as impolite to ask for compensation or gold in exchange,
as there is enough honour already in defending their brethren of the main
land. Still many baronies and smaller earldoms pay Norbury for their aid,
either in gold, everblack or in trade deals highly favouring Norbury. Rivalries
that the city kingdom might have with the neighbouring baronies will not
hinder it to come to their aid in case of a raid from the sea.

\subsubsection{Religious Civil War}
\label{sec:Religious Civil War}

Ever since its foundation, the religions surrounding \nameref{sec:Lor}, the
\nameref{sec:Order} and the \nameref{sec:Three Kings} vied for the position of
dominance within the city. The Order had the most followers, followed by Three
Kings and then Lor. Although the priests and paladins of Order raised issues
with mistreatment of slaves, they were not outright against slavery unlike
the priests and followers of Lor who wished to see the practice banned. The
followers of the Three Kings found both the followers of Lor and the Order to
be weak in the face of their common enemy from the north, and were ready to
expel both religions from the city. Over centuries this conflict hardened and
brewed in the heads of the followers and citizens. Then, in \emph{MI:1480}
when a follower of the Three Kings beat his slave to death with a pavement
stone in the middle of the market place for disobeying him, a paladin of Lor
interfered by killing the slave owner in single combat. This event brought a
spark to the already volatile mix of religious animosity. The followers of the
Three Kings moved in retaliation against the temple and followers of Lor. The
Order, who sided with the mistreated slave, joined the conflict on the side of
the followers of Lor. A religious civil erupted that war lasted for two
months, in which both the Order and the followers of Lor suffered heavy losses,
and ultimately defeat, against a well-trained, and well-equipped force which
was inferior in numbers. To prevent the bloodshed to spilling over to
civilians, the then ruling queen \emph{Arianna of Nordholm}, banned the
religions associated with Lor and the Order, and exiled their followers for
being ``weak in the face of adversity''.

Even though this conflict has long passed, the prevailing attitude in Norbury
is that followers of the Order and Lor were, and are still, weak and not
worthy of honour. Although the religious worship of Lor is still banned, the
ban on he worship of the Order has since been lifted. Followers of Lor may
enter the city, but priests, and knights of Lor are still persona non grata.

\subsubsection{Culture}

Since the incursion and raids of the beast races still occur to this day, and
have grown to be more ferocious and demanding, the culture of Norbury has
grown in response. Norbury is from the lowliest slave and peasant up to the
king or queen herself, a meritocracy. You are worth as much to the city, and
in the eyes of your fellow citizen as you can contribute to the well being of
the entire whole.

Men and women of Norbury pride themselves in the work they are contributing to
the collective defence effort, be it front line combat, creating weapons and
armour for those that do, or aiding the effort in an administrative
fashion. Warrior culture runs strong in the kingdom, their deeds are sung in
taverns, and their likeness is made eternal in art. Many fighters and warriors
of Norbury follow the \emph{Three Kings}, although worship of \emph{Forun} is
also wide spread in the kingdom.

Although arcane and divine magic is still frowned upon in the kingdom, the army
of Norbury runs an academy for battle mages and wizards. Arcane research is
often scrutinized by its potential military application, and wizards are also
required to undergo basic military training in weapons and armour.

All citizens of Norbury (and those that wish to become citizens) must complete
a mandatory civil service of at least five years. Many use this mandatory
service to begin an apprenticeship, while others join the military to counter
raids and incursions. No one is excluded from this service, not even the
children of the reigning monarchs. Avoiding this service is not only illegal,
but is also seen as a major dishonour.

\subsubsection{Society}

Titles within the kingdom's society, such as commoner, earl, baron or even
duke and grand duke can be bought by any citizen of Norbury. There is a rather
unusual twist: These titles are not inherited, which means that the son of a
duke is a commoner upon birth. The common census is, that this child has not
done anything yet to earn the rank of earl for him or herself. There is a high
pressure on children to achieve the same status as their parents, or perhaps
even outperform them. Failure is always an option as well, as it is not
unheard of for a noble son to fall into slavery.

Who reigns as king or queen is decided in a ritual combat between all eligible
arch dukes that wish to rise to the challenge. This combat is not to the
death, although many grand duke's have perished in their claim for the
kingdom. A king or queen reins until his reign is challenged by another, or
until they die or resign. The people of Norbury mostly do not care what race
or social background of their king or queen, but instead judge all their peers
by their honour, combat prowess, and contribution to society as a whole.

\subsubsection{Slavery}
\label{sec:Slavery in Norbury}

Norbury is the foremost kingdom to practice slavery on a massive scale. Both
\hyperref[sec:Indentured Servitude]{indentured servitude} and
\hyperref[sec:Unregulated Slavery]{chattel slavery} are encoded in Norbury
law.

Many of the raiders that are captured, and most of those that commit major
crimes are enslaved. The wizard academy has constructed special
\hyperref[sec:Slave Band]{arcane collars} that bind the slave to their
owners. Slavery is rarely a sentence for life, as there are a few legal ways to
escape slavery. Slaves may be released at any time by their owners, those in
indentured servitude may buy themselves free, and state-owned slaves may be
simply set free once they can no longer contribute in any meaningful way to
society. However Norbury slaves (as compared those in servitude) have no
rights whatsoever, and are marked, colour coded, registered by number, and
tracked down by the \nameref{sec:Hunters Guild} should they decide to
run. Slave collars allow slave owners to track, command and often punish their
wearers.

Foreigners are allowed to purchase slaves, however they have to pay an
additional fee half the slave's value. This was intentionally done, so that
most of the slaves remain within the city and contribute to the city. Slave
ownership is recognised as lawful in all kingdoms and baronies that have
signed the \hyperref[sec:Vonir Accord]{Vonir Accord}.

Although the city has a large slave population (between 20 and 30 percent),
it rarely faces slave revolts or uprisings. There are four main pacifiers at
work that keep the slave population suppressed and obedient. First, for many
slavery is potentially only temporary. Many are forced to join the Norbury
army, the Hunter's Guild or begin work in other positions that would qualify
them for citizenship later on, and thus do not wish to risk their permanent
freedom through revolt. The second is a ever permeating culture of honour and
duty, especially in the still ongoing sea raids from the north. Many slaves
derive a sense of purpose from working toward a common goal. Also many
churches and charitable organisations provide a network of services and goods,
such as food, warmth and medical services to slaves making their lives
bearable. Although gatherings of slaves are forbidden, the Hunter's Guild
often looks the other way in regard to slave bars or inns, as long as no
direct threat stems from these establishment. And the fourth reason is the
ever vigilant \nameref{sec:Hunters Guild}. Their network of informants,
hunters, and agents quench any rebellious flame that might kindle in the dark.

\subsubsection{Spinetails}
\label{sec:Spinetails}

However the city also has a certain societal class of slaves that have slipped
through the crack of the Hunter's Guild. Those are slaves that were deemed
valueless, cannot work because of disease, illness or handicaps, are mentally
unstable or even violent, or may have lost their owner and thus no one lays a
claim on them. Their collars often show the colours ``white'' (valueless) and
``red'' (without owner), and are thus often named ``spinetails'' after the
bird that wears the same colours in its feathers. Spinetails hide within the
vast sprawling suburbia of the city from the guild, and survive by committing
petty crimes, begging or from aid given to them by generous people and
organisations. They have since begin to form their own societies, subculture
and even their own crime organisations. The Hunter's Guild warns anyone from
claiming a spinetail as a slave, and rewards citizens should they turn them in
or give hints that lead to their apprehension. Generally spinetails are viewed
as a problem by most citizens, and even other slaves. Disease, drug abuse, and
crime is rampant within Spinetail subculture, and many see them as free-riders
and slackers.

\subsubsection{Population}

Norbury, and its outlying villages, houses roughly 4 million people. Of which
the vast majority are human and elves (39\%), then halflings (24\%) followed
by dwarves (15\%) and half races (10\%). The rest (12\%) are beast races,
almost all of which are enslaved. Roughly 31\% of the entire population is
either currently enslaved or in servitude.

Common male names are: Agmund, Agnar, Aki, Aleif, Ari, Asbjörn (Björn),
Asvald, Baggi, Bardr, Egil, Einarr, Folki, Gunnar, Halli, Hallthorr, Hallvard,
Ivar, Knut, Magni (Magna), Ragnar, Sigurd, Sven, Thorbjörn, Thorsten, Yngvar

Common female names are: Alfhild, Asa, Asdis, Aslaug, Brynja, Edda, Frida,
Gudrun, Gyda, Helga, Inga, Idunn, Katla, Runa, Sigrid, Sigrun, Thora,
Thordis, Thorhild, Thorun, Thorvi, Thyri, Ynvgild
