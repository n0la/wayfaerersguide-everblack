\subsection{Tredegår}
\label{sec:Tredegar}

The city kingdom of Tredegår (old-high Teranim for ``three courtyards'')
lies on the south western shores of \nameref{sec:Eilean Mor} west of the
\hyperref[sec:Great Divide]{great divide}. It is encased by two large rivers:
The Morre river to the north and west, and the Moy river to the
east and south. It's close proximity to the mineral wealth of the great
divide, the easy access to two rivers and the sea made Tredegår one of the
richest city kingdoms of all of Aror.

\subsubsection{An'Rath}
\label{sec:AnRath}

In the centre of the city is the vast palace of An'Rath, which houses roughly
a thousand people. Workers, nobles, clergymen, bureaucrats, judges, advisers,
as well as the ruling family live within the palace. The palace by itself is
an impressive architectural feat, and is constantly being extended with
additional towers and buildings. The palace has its own army, and its own
inner stone wall for protection, and even inner draw bridges and gates to
separate individual parts from one another in case of a siege. The outer ring
of the castle is open to the public, and houses a library, a school that
offers various courses, its own brewery and tavern the ``The Drunken Trader''.

\subsubsection{Banner}

The banner of Tredegår features a white tree with three thick branches resting
within a shield. A golden crown rests above the shield. Gold, white and blue -
representing the gold of the great divide, the white snow up on their peaks,
and the blue rivers and the sea - are the colours of Tredegår.

\subsubsection{Population}

Among all the city kingdoms Tredegår it is of the biggest. It houses roughly
32 million inhabitants, when all outlying smaller cities and villages are
included. Most of these are humans (55\%), with dwarves that once lived in the
great divide following close second (21\%) followed by elves (10\%), half
races (12\%) and various others (2\%).

Common male names are: Åke, Albert, Alex, Alfred, Birger, Bjarne, Björn, Danne,
Einar, Felix, Gunnar, Halvar, Hjalmar, Holger, Janne, Kalle, Karl, Magnus,
Mikael, Nils, Ola, Per, Roland, Sigfrid, Sven, Varg, Yngve

Common female names are: Agata, Aina, Alicia, Alva, Amanda, Anja, Åsa, Birgit,
Britta, Carolina, Dina, Eira, Elina, Emma, Hannah, Irene, Janna, Johanna,
Julia, Laila, Linnea, Margareta, Malin, Pia, Runa, Sara, Sofia, Tea, Thora,
Vera, Veronika, Vivian

\subsubsection{Culture}

The immense wealth of the city has trickled down to most of the lower and middle
class citizens. Thus the people of Tredegår live in an unparalleled state of
social security that is only known in that kingdom. Most people of the city
work far less than their contemporaries in other kingdoms. Still they make
enough money to life comfortably. This gives the people of Tredegår time and
opportunity to enhance their education and other skills. The people of Tredegår
are often described as well educated, easy going and relaxed; with strong focus
on self improvement and family. Some have described them as arrogant and snobby,
although those attitudes can often be exaggerated by envy. A typical Tredegår
craftsman for example works slow, meticulously and prefers to deliver quality
over quantity. This culture of ``do it right; or do not do it at all'' has
earned the city a good reputation as a reliable trading partner that delivers
excellent product.

This culture is also present within the kingdom were houses, streets, bridges
are seldom derelict or run down. Instead they are always lavishly decorated with
flowers, and the city maintains a network of arcane street illumination housed
in lamp posts. Public workers keep the streets clean and make sure that the
city is always in good shape. Although the kingdom has no slum or worker
district, poverty does still exist within the city, especially among the sick
and crippled.

\subsubsection{Steel and Smithing}

The Tredegår Steel and Gold Corporation is one a large company that
runs many forges and smithies across the city. It works off the raw ore mined
either in the great divide, or ore that has been brought in from the
\nameref{sec:Silver Isles}. The corporation is known for its excellent
craftsmanship in weapons and armours, and their products are sold all across
the major kingdoms and are renowned for their quality and steep price. It also
has the Tredegår tree as a logo which can be found on many of their armour
and weapons. This logo has become a sign and indication for top quality in
crafted goods, steel, weapons and armour.

\subsubsection{Society}

Due to the immense wealth of the city, slavery has been abolished several
centuries ago. The city has been ruled by the same noble family Gylleborg
for thousands of years. Both female and male heirs may rule. The city kingdom
has a long history of peace and prosperity, and the justice system is known
all around the world as one the best of its kind. Independent judges and
prosecutors work in tandem with private defence attorneys to ensure the
justice system remains fair and independent.

\subsubsection{Relations}

The city and its nobility is closely related to \nameref{sec:House Ranian},
and the house has the main headquarters near the main square of the city.

Due to their status as rich trading partner, the city kingdom of Tredegår is
in good standing with almost all other city kingdoms, except
\nameref{sec:Morkan}.  Their longest standing trading partner and ally is the
city kingdom of \nameref{sec:Fes al-Bashir} to the south. The city kingdom
strengthened their bond with the other city states by providing logistical
advisers, money and their best smiths during the rebuilding of
\nameref{sec:Forsby}. The kingdom is a signer of the \nameref{sec:Vonir
  Accord}.
